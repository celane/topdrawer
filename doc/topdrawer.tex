%
% $Id: titlepag.tex%v 3.38.2.153 1993/07/08 03:43:32 woo Exp woo $
%
%\ifx\undefined\selectfont % DSL 26 May 93
%\documentstyle[toc_entr]{article}
%\else
%\documentstyle[toc_entr,helv]{article}
%\fi
%\setlength{\textwidth}{6.25in}
%\setlength{\oddsidemargin}{0.5cm}
%\setlength{\topmargin}{-0.5in}
%\setlength{\textheight}{9in}
%\setlength{\parskip}{1ex}
%\setlength{\parindent}{0pt}
%\adjustarticle
%\setlength{\threenum}{4.0em} % DSL 26 May 93
\documentstyle[a4j,twoside]{article}
\addtolength{\textheight}{3cm}
\begin{document}

\pagestyle{empty}
   \rule{0in}{3in}
   \begin{center}
   {\huge\bf TOPDRAWER}\\
   \vspace{3ex}
   {\Large An Interactive Plotting Program}\\
   \vspace{4ex}
   \large
   Version 5.12 ( Rice Bonner Lab )\\
   July 1993\\
   \vspace{5ex}
   Fermilab Document {\tt PP0005.1}\\
   Author: John Clement\\
   Support level: 5 (full)\\

   \vfill
   {\small This manual is for TOPDRAWER version 5.12\\
%   \begin{verbatim}
% $Id: titlepag.tex%v 3.38.2.153 1993/07/08 03:43:32 woo Exp woo $
%   \end{verbatim}
   }

   \end{center}
\newpage

\tableofcontents
\newpage

\setcounter{page}{1}
\pagestyle{myheadings}
\markboth{TOPDRAWER 5.12}{TOPDRAWER 5.12}

\section{TOPDRAWER}
TOPDRAWER
Bonner Lab Version 4.0
This is a program primarily designed to produce graphs of scientific data.
It also has some calculation ability.  You may create, modify, fit, combine,
and smooth data sets.  The results may be displayed in a variety of formats.
In the documentation your response is underlined and optional syntax is
enclosed inside square brakets [].  
\section{Setup}
To use Topdrawer it may be necessary to set it up.  To do this:  
\begin{verbatim}
     $ SETUP TOPDRAWER 
\end{verbatim}
If you wish to use the previous old version:  
\begin{verbatim}
     $ SETUP TOPDRAWER.OLD 
\end{verbatim}
If you wish to use the version currently being tested:  
\begin{verbatim}
     $ SETUP TOPDRAWER.TEST 
\end{verbatim}
\section{Introduction}
TOPDRAWER is fairly easy to use if you confine yourself to simple plots.
First invoke TOPDRAWER by typing:  
\begin{verbatim}
     $ TOPDRAWER 
\end{verbatim}
You are then prompted:  
\begin{verbatim}
     TD:  You first enter your data as a set of X,Y,DX,DY values.  The
\end{verbatim}
DX,DY may be omitted if no error bars are needed.  Each data point is
specified on a separate line.  For example a line going from 0,0 to 1,1 is
specified as:  
\begin{verbatim}
     TD:0,0 
     TD:1,1 
\end{verbatim}
Then the data may be plotted as a set of points with error bars (if DX,DY
are not zero) by typing the command:  
\begin{verbatim}
     TD:PLOT 
\end{verbatim}

If you wish to produce a histogram use the command:  
\begin{verbatim}
     TD:HISTOGRAM 
\end{verbatim}
On the other hand if you wish to plot a smooth curve type:  
\begin{verbatim}
     TD:JOIN 
\end{verbatim}
For series of straight lines connecting the points:  
\begin{verbatim}
     TD:JOIN 1 (1 segment per interval) 
\end{verbatim}
To put a title at the top:  
\begin{verbatim}
     TD:TITLE TOP 'This is my title' 
\end{verbatim}
To put a title at the bottom:  
\begin{verbatim}
     TD:TITLE BOTTOM 'This is the X axis' 
\end{verbatim}
To put a title on the left:  
\begin{verbatim}
     TD:TITLE LEFT 'This is the Y axis' 
\end{verbatim}

If you are done, you then type:  
\begin{verbatim}
     TD:EXIT 
\end{verbatim}

Now you liked the plot, but you wish to change a few things.  The commands
you typed are in file TD.TDJ.  You may edit this file to correct any
mistakes you made.  You also should rename it:  
\begin{verbatim}
     $ RENAME TD.TDJ MYPLOT.TOP 
\end{verbatim}
and invoke TOPDRAWER again.  Now instead of typing all the commands in,
just type:  
\begin{verbatim}
     $ SET FILE IN 'MYPLOT' 
\end{verbatim}
TOPDRAWER will re-execute the commands in the file.  Now you wish to get
the plots out on the Talaris printer.  so you type:  
\begin{verbatim}
     TD:SET DEVICE EXCL 
\end{verbatim}
Then replot the data by typing:  
\begin{verbatim}
     TD:SET FILE IN 'MYPLOT' 
\end{verbatim}
Finally type:  
\begin{verbatim}
     TD:EXIT 
\end{verbatim}

Now the plot is printed on the Talaris by:  
\begin{verbatim}
     $ PRINT UGDEVICE.DAT 
\end{verbatim}

More plots may be added to the first plot by entering more data, and then
either PLOT, HISTOGRAM, or JOIN.  Each command may be modified by extra
commands to make the plot more pleasing.  For example if you type:  
\begin{verbatim}
     TD:HISTOGRAM DOTS RED 
\end{verbatim}
Then the histogram will be formed with red dotted lines.  This is useful to
differentiate it from other previously plotted histograms.  
\section{Philosophy}
TOPDRAWER is designed to plot and manipulate data.  It assumes that data
consists of X vs Y with errors DX, DY.  It may be X,Y vs Z with errors
DX,DY,DZ.  The data may be a histogram in which case DX would be the bin
width, or it may be a set of discrete points with unrelated errors.  You
may plot the data accordingly.  

The basic plotting operations are:  
\begin{verbatim}
     1.  PLOT plots the data with optional error bars.  
     2.  JOIN joins the data points with line segments.  
     3.  HISTOGRAM plot the data as a histogram.  
     4.  BAR plots the data as a bar chart (similar to a histogram) 
\end{verbatim}

\section{Operation}
The following command invokes TOPDRAWER:  
\begin{verbatim}
     $ TOPDRAWER [filename] [device] [/options] 
\end{verbatim}
If the filename is omitted you are prompted with 
\begin{verbatim}
     TD:  
\end{verbatim}

You then start typing in commands to TOPDRAWER.  

If you start TOPDRAWER from a batch stream, the commands follow the
\$ TOPDRAWER command.  
\subsection{Filename}
\begin{verbatim}
     TD:SET FILE INPUT filename 
\end{verbatim}

If the optional filename is specified after TOPDRAWER, the commands are
read from the specified file and both the journal and list file are not
produced.  
\subsection{Options}
\begin{verbatim}
     1.  /[NO]COMMAND[=file] - Specifies a file to get commands for
         TOPDRAWER setup.  
         (Default:TDINIT.TOP) 
     2.  /[NO]JOURNAL[=file] - Specifies a file to journal all
         interactive commands.  
         (Default:TD.TDJ) 
     3.  /[NO]LIST[=file] - Specifies a file to list the commands and
         error messages.  
         (Default:TD.LIS) 
\end{verbatim}
\section{Devices}
Normally TD sets the output device according to the logical name PLOT\_TERM
or if it is not defined PLOT\_DEVICE.  If neither is defined the device is
assumed to be a Tektronix 4010.  
For more information see:COMMANDS SET DEVICE 
\subsection{List\_of\_devices}
To set your device use the command:  
\begin{verbatim}
     TD:SET DEVICE options 
\end{verbatim}
Options                  Device 
TEKTRONIX                Tektronix 4010 terminal 
TEKTRONIX ``CIT467''       CIT-467 Color graphics terminal 
TEKTRONIX ``RADM3A''       ADM-3A with Retrographics 
TEKTRONIX ``RVT100''       VT-100 with retrographics 
TEKTRONIX ``RVT100A''      VT-100 with retrographics (Green screen) 
TEKTRONIX ``RVT100,YPIXELS=3055''
\begin{verbatim}
                         
                         Falco infinity series (48 lines) 
\end{verbatim}
TEKTRONIX ``SEL100XL''     Selanar HiREZ 100XL 
TEKTRONIX ``VIS102''       Visual 102 
TEKTRONIX ``TAL1590''      Talaris 1590 laser printer 
TEKTRONIX ``KERMIT''       Kermit simulation of TEK-4010 with color 
TEKTRONIX ``LSI7107''      LSI model 7107 color terminal 
TEKTRONIX ``TEK4207''      Tektronix color terminal 
TEKTRONIX ``TEEMTALK''     Tektronix color terminal 
TEKTRONIX ``VT240''        C.ITOH 328 
TEKTRONIX                ``TEK4010,RVT100,NSEGM''
\begin{verbatim}
                         
                         VersaTerm (Macintosh terminal emulation) 
\end{verbatim}
TEK4027                  Tektronix-4027 color term.  
TEKEMUL                  Tektronix 4010 emulator 
REGIS                    VT-240 
REGIS ``COLOR=WRGWWWWD''   VT-241 color terminal 
EXCL                     Talaris 1590 
VERSATEC                 Versatec printer plotter 
SIXELS ``LN03''            LN-03 Laser printer 
SIXELS ``LN03HI''          LN-03 High resolution 
SIXELS ``LN03LO''          LN-03 Low resolution 
SIXELS ``LA50''            LA-50 or LA-100 with 8.5'' paper 
SIXELS ``LA100''           LA-100 14'' paper 
QMS1200                  QMS Lasergrafix 1200 laser printer 
IMAGEN                   Imagen laser printer 
POSTSCRIPT               Postscript printer 
PRINTRONIX               Printronix MVP printer/plotter.  
XWINDOWS X windows ( Decwindows, MOTIF ) 

\begin{verbatim}
                                 Example
     TD:SET DEVICE TEKTRONIX "RVT100" 
\end{verbatim}
See also:TOPDRAWER command set device 
\section{Input}
Normally  TOPDRAWER  gets  primary  input from either a batch file, or your
terminal (SYS\$INPUT).  You may have secondary  input  files  by  using  the
command:  
\begin{verbatim}
     TD:SET FILE INPUT filename 
\end{verbatim}

If  you run TOPDRAWER interactively, but from an indirect command file, the
input will be from the file that invoked TOPDRAWER.  If you  wish  to  have
input from the terminal, you may assign TT:  to SYS\$INPUT temporarily by:  
\begin{verbatim}
     $ ASSIGN /USER TT:  SYS$INPUT 
     $ TOPDRAWER 
\end{verbatim}

You  should  not  do  this  in BATCH, as TT:  is the default output for the
graphics.  You may get  data  from  output  files  generated  by  histogram
packages.  For more information on this see:Command SET HISTOGRAM.  

Some  input may be made via the cross hair cursor.  For example SHOW CURSOR
brings up the cursor on your terminal.  Move the cursor to the location you
desire,  and  press  any  key  except  ``Return'' The current location of the
cursor will be typed and entered into the journal file as a comment.   Some
other  commands that use the cursor are ARROW, BOX, CIRCLE, ELLIPSE, TITLE,
SET WINDOW, and SET LIMITS.  

The  input  string  may  not be longer than 256 characters.  If you need to
omit columns or set up a special format for  the  commands:   SET CARD  and
SET FORMAT.  

For more information see:TOPDRAWER DATA.  
\section{Output}
TD produces the following output:  
\begin{verbatim}
     1.  Plot - This may be on your terminal, LPA0:, another terminal, or a
         file on disk.  The plot data format and file is  modified  by  the
         command:  
              TD:SET DEVICE 
         Plot  output  for a device other than your terminal is normally in
         the file UGDEVICE.DAT.  
     2.  Listing  - This lists all TD commands you used in a session.  This
         is normally TD.LIS, but the name may be changed by the command:  
              TD:SET FILE OUTPUT 'newfilename' 
         If you do not want a list file:  
              TD:SET MODE LIST OFF 
                   or...  
              TD:SET FILE OUTPUT NL:  
     3.  Journal  -  This file contains all commands entered interactively.
         It is normally file TD.TDJ, but the name may  be  changed  by  the
         command:  
              TD:SET FILE JOURNAL 'newfilename' 
         You may temporarily turn off journaling by:  
              TD:SET MODE JOURNAL OFF 
     4.  Errors - A temporary file TOPDRAWER_ERRORS.DAT is used to save the
         error messages for the current plot.  This file is only opened for
         interactive sessions.  
     5.  Data  -  You may use the LIST command to output data sets suitable
         for input to TOPDRAWER.  If you do not specify the output file  it
         goes to the Listing (TD.LIS).  
\end{verbatim}
\section{Set\_up}
You  may  specify  a  setup  file to initialize TOPDRAWER.  This is done by
specifying the logical name TOPDRAWER\_INIT.  TOPDRAWER\_INIT should point to
a  file  with  TOPDRAWER commands to be executed when TD first starts.  The
default file name is TDINIT.TOP.  If  you  have  a  setup  file  then  then
journal  file  and listing file are suppressed.  You must start them in the
setup file if you want them.  

\begin{verbatim}
                                  example
     $ ASSIGN TDHBOOK TOPDRAWER_INIT 
\end{verbatim}

The file TDHBOOK.TOP contains the commands:  
\begin{verbatim}
     TD:SET HIST HBOOK FILE="hbook.dat" 
     TD:DEFINE KEY PF3 "SET HIST TITLE PREV;NEW;HIST"/TERMINATE 
     TD:DEFINE KEY PF4 "SET HIST TITLE NEXT;NEW;HIST"/TERMINATE 
     TD:SET FILE JOURNAL HISTS.TDJ 
\end{verbatim}

Now  when  you start TOPDRAWER your hbook histograms are already available,
and keypad keys are defined to easily access them.  Your  journal  file  is
HISTS.TDJ and you have no listing file.  

You define:  
\begin{verbatim}
     $ DEFINE TOPDRAWER_INIT SYS$LOGIN 
\end{verbatim}
The  file  TDINIT.TOP  in  your root default login directory is used as the
TOPDRAWER initialization file.  
\section{Data}
TOPDRAWER accepts data in a variety of forms.  
\subsection{Expressions}
\begin{verbatim}
     <expressions>
\end{verbatim}
An  expression is an arithmetic computation enclosed inside angle brakets
''$<$$>$''.  An expression uses similar syntax to FORTRAN.  The syntax  is  not
as  rigid  as  FORTRAN  and  some  syntactical  elements  may be omitted.
Expression evaluation is very slow, so you should use them sparingly.  
\subsubsection{Constants}
Certain constants are built into TOPDRAWER.  They are:  
\begin{verbatim}
     1.  PI - 3.1416...  
                                example
     <2*PI> is 6.2831...  
\end{verbatim}
\subsubsection{Functions}
Functions are expressed as:  
\begin{verbatim}
     function<expression> 
\end{verbatim}
The  parenthesis  around the argument of the function may be omitted if
the argument is a number.   All  functions  may  be  abbreviated.   The
available functions are:  
\begin{verbatim}
     1.  [ARC]SINE - [inverse] Sine of angle in degrees 
     2.  [A]COSINE - [inverse] Cosine of angle in degrees 
     3.  [A]TANGENT - [inverse] Tangent of angle in degrees 
     4.  RADIANS - Converts degrees to radians 
     5.  DEGREES - Converts radians to degrees 
     6.  GAMMA - GAMMA function (GAMMA(n)=(n-1)!) (Cern Library) 
     7.  LOGARITHM - Log to the base 10 
     8.  EXPONENTIAL - Power of e (inverse LN) 
     9.  LN - Logarithm to the base e 
    10.  SQRT - Square root 
    11.  RANDOM  -  Random  number.   RAN(0)  produces a random number.
         RAN(n) sets the seed to n  and  starts  a  new  random  number
         sequence.  
    12.  ABSOLUTE - Absolute value 
    13.  INTEGER - Integer part of a number 
    14.  FRACTION - Fractional part of a number 
    15.  NINTEGER - Nearest integer to a number 
    16.  ERF - Error function (Cern Library) 
    17.  ERFC - Error function (Cern Library) 
    18.  FREQ - Frequency function (Cern Library) 
\end{verbatim}
\subsubsection{Operators}
\begin{verbatim}
     1.  + - Add 
     2.  - - Subtract 
     3.  * - Multiply 
     4.  / - Divide 
     5.  ** - Power 
\end{verbatim}
These follow the same rules of precedence as in FORTRAN.  For example: 
\begin{verbatim}
     1.  2*3/4 is the same as (2*3)/4 or 1.5 
     2.  2*3+4*5 is the same as (2*3)+(4*5) or 26 
     3.  2**2**3 is the same as 2**(2**3) or 256 
     4.  2**2*3 is the same as (2**2)*3 or 12 
     5.  -2**2 is the same as -(2**2) or -4 
\end{verbatim}
Missing  operators  in  the  middle  of an expression are assumed to be
multiply ''*''.  $<$2(3)$>$ is the same as $<$2*(3)$>$.  
\subsubsection{Filenames}
TOPDRAWER will accept either VMS or UNIX style file specifications.  
\begin{verbatim}
                                Example
     VMS                 UNIX
     [dir.subdir]file    /dir/subdir/file
     [.subdir]file       subdir/file
     [-.subdir]file      ../subdir/file
\end{verbatim}
\subsubsection{Examples}
\begin{verbatim}
     <LOG(V_SUM)> 
\end{verbatim}
the logarithm of the SUM.  
\begin{verbatim}
     <SINE(V_SUM)> 
          or...  
     <SIN(V_SUM)> 
\end{verbatim}
the sine of the SUM.  
\begin{verbatim}
     <ARCSIN(V_SUM)> 
          or...  
     <ASIN(V_SUM)> 
\end{verbatim}
the arcsine of the SUM.  
\begin{verbatim}
     <4 SINE 30) 
\end{verbatim}
is the number 2.  
\begin{verbatim}
     <4(2+3)> 
\end{verbatim}
is the number 24.  
\subsection{Order}
Data by default is entered in the following order.  
\begin{verbatim}
     X Y DX DY SYMBOL 
\end{verbatim}

Where:  
DX is the error or 1/2 bin width for X+-DX.  
DY is the error on Y+-DY.  
SYMBOL is the symbol to plot.  

You  may  change the actual order that data is entered with the SET ORDER
command.  
\subsection{Numbers}
Data may be expressed in any legal FORTRAN format.  For example 
\begin{verbatim}
     1 = 1.0 = +1.0E0 
\end{verbatim}
\subsection{Points}
Data  is  entered  in a series of lines one point to a line.  Each number
may be separated by spaces, tabs, or commas'',''.  Numbers may also be  run
together  without separators, provided they begin with either plus ''+'' or
minus ''-''.  If you wish to enter several data points  on  a  single  line
then  they  may  be separated by semicolons '';''.  Instead of a number you
may also use expressions by enclosing it inside angle brakets ''$<$$>$''.  

If you omit data the last value is used.  See: Command READ POINTS.  
\subsection{Symbols}
Symbols  are  0O  to  9O  or  any  legal UGSYS duplex character set pair.
``NONE'' indicates no symbol.  ''  '' or ``DOT'' produces a dot.  If the symbol
is  omitted  it is assumed to be NONE and the datum is plotted by default
as a point rather than a symbol.  This may be modified by SET SYMBOL.  If
you wish to use the number 1 as a symbol it must be enclosed in quotes or
apostrophes (``1'').  
\subsection{Time-Angles-Dates}
Time  may  be  entered in hours in the format hh:mm:ss.nnn.  You may also
enter angles as dd:mm:ss.nn in degrees, minutes, seconds.  Time or  angle
data  must  begin  with  a number.  0::1.5 is legal for 1.5 seconds after
midnight, while ::1.5 is not legal.  

Dates  may  be  entered  in hours as YYYY$\backslash$MM$\backslash$DD.  If omitted MM or DD are
assumed to be 1.  If you wish to enter both date and time the  format  is
YYYY$\backslash$MM$\backslash$DD$\backslash$HH:mm:ss.ss.  See:Command SET DATE.  

If  time is preceded by a sign, then all 3 HOURS,MINUTES, and SECONDS are
considered negative.  If a date or date$\backslash$time is preceded  by  a  sign  it
applies only to the YYYY (Year).  Signs (+ -) may not be imbedding inside
a date$\backslash$time.  

\begin{verbatim}
                                Examples
     10 (10 am) 
     10:00 (10 am) 
     15:30:25.2 (25.2 seconds past 3:30 pm) 
     0.5 (Half past midnight) 
     0:30 (Half past midnight) 
     24:30 (Half past midnight) 
     1987\12\25\12:30 (12:30 on Dec 25, 1987) 
     -1:30:00 (10:30) 
\end{verbatim}
\subsection{Types}
Data  may  either  be  X,Y,Z points or mesh data.  A point consists of an
X,Y,[Z] value with optional errors DX,DY,[DZ] Z and DZ are  optional  and
may  not necessarily be present.  MESH data is a data set consisting of Z
values as a function of X and Y.  A mesh is essentially a square grid  of
data  with X,Y vs Z.  A mesh may also have optional DX, DY, and DZ values
attached to it.  

Data  may  be  subdivided  into  data  sets.   Each  data set is numbered
consecutively starting with 1.  Within  each  data  set  the  points  are
numbered from 1 to n.  To create a new data set use the DATA SET command.
Some commands such as FIT allow automatic creation of a new data  set  if
the option APPEND is used.  

When  plotting  you  may specify which data set you wish to plot with the
option SET=n.  For example PLOT SET=1 plots the first data set.  You  may
also  use  FIRST,LAST,NEXT,PREVIOUS,  or  CURRENT  to specify a data set.
FIRST is always set number 1 while LAST  is  the  number  of  data  sets.
CURRENT  is  the  last  set  specified.  NEXT is the current data set +1,
while PREVIOUS is the current set -1.  

For example you may plot 2 sets by:  
\begin{verbatim}
     TD:PLOT SET=FIRST 
     TD:PLOT SET=NEXT 
\end{verbatim}
You may also specify a range of sets.  
\begin{verbatim}
     TD:PLOT SET=2 TO LAST 
\end{verbatim}
plots all data sets except for number 1.  
\begin{verbatim}
                                 Warning
\end{verbatim}
When  plotting  multiple  data sets, the format of the plot is taken from
the first data set.  If the first set is a mesh, the plot will be 3-D.  
\subsection{ON$|$OFF}
A number of options use ON or OFF to modify them.  For example:  
\begin{verbatim}
     TD:SET MODE VECTOR=OFF 
\end{verbatim}
Uses  hardware  characters whenever possible.  Instead of ON, OFF you may
also use T,F or TRUE,FALSE or YES,NO.  
\subsection{Examples}
The   following  will  enter  a  series  of  data  points  in  the  order
x,dx,y,dy,symbol.  
\begin{verbatim}
     TD:SET ORDER X DX Y DY SYMBOL 
     TD:+1+.5+10+.5'0o' 
     TD:2 .5 25 .5 1o 
     TD:3 .25 10 .01 2o 
     TD:4 .1 5 1e-1 3o 
\end{verbatim}
\section{Coordinate\_frame}
There are 2 different sets of frames used in creating plots.  They are DATA
and TEXT frames.  TEXT is always specified in  inches  Unless  modified  by
either a SET SIZE or SET UNITS command.  The character and symbol sizes and
are all specified in tenths of inches.  Both of these may be changed by the
SET UNITS  command.   The actual data is in the DATA frame, which is mapped
to the physical (TEXT) frame at the time of plotting.  The  plots  are  all
scaled  automatically, unless you use the command SET LIMITS to specify the
scale.  All commands which plot ``things'' work in either  the  DATA  or  the
TEXT  frame.   You  may  select the frame with the option DATA or TEXT.  In
addition if 3d coordinates are used the frame is automatically DATA.  

\begin{verbatim}
     The  DATA  frame  is  not  mapped  to  the TEXT frame until a plotting
\end{verbatim}
command or a command which uses the data frame is used.  Plotting  commands
are JOIN, HISTOGRAM, PLOT and so on.  Commands which use the data frame are
ARROW DATA, BOX DATA, TITLE DATA and so on.  If you wish to have  TOPDRAWER
automatically  set  the  data  frame, you should supply the data before any
commands which reference the data frame.  

Several commands modify the data frame, once it has been defined:  
\begin{verbatim}
     SET WINDOW 
     SET LIMITS 
     SET THREE 
     SET SIZE 
     SET SCALE 
\end{verbatim}
These  commands allow you to overlay different plots with different scales.
Just reading in new data does not change the current data frame.  
\section{Fonts}
There  are  several  fonts  available  in  the unified graphics.  They are:
EXTENDED and DUPLEX.  

The EXTENDED and DUPLEX support many sets of characters.  The DUPLEX set is
slow compared to the  extended  set,  but  it  looks  almost  like  printed
characters  if  you  use a high resolution device.  The alternate character
sets are determined by a character pair.  The  first  character  determines
the actual character printer, the second one determines the case.  
\subsection{List\_of\_fonts}
\begin{verbatim}
         Case      Character set 
     1.  blank     Roman (default) 
     2.  L         Roman lower case 
     3.  F+G       Greek 
     4.  B+C       Cyrillic 
     5.  P         Punctuation 
     6.  S         Typographicl symbols 
     7.  M         Math symbols 
     8.  T         Theoretic symbols 
     9.  W         Arrows 
    10.  K         Physics symbols 
    11.  A         Astronomical symbols 
    12.  O         Markers (Centered symbols) 
    13.  D         Drawing symbols (underscore,overscore) 
    14.  U+V       Movement control 
    15.  X         Subscript/Superscript 
    16.  Y         Character size 
    17.  Z         Position save/restore 
\end{verbatim}

\subsection{Roman}
Case character:  
blank = upper case, L= Lower case 

This  is the normal ASCII character set.  It is not necessary to use L to
shift to lowercase if your computer accepts lowercase characters.  
\subsection{Greek}
Case character:  
F = upper case, G= Lower case 
\begin{verbatim}
  A Alpha     B Beta      G Gamma     D Delta
  E Epsilon   Z Zeta      H Eta       Q Theta
  I Iota      K Kappa     L Lambda    M Mu
  N Nu        X Xi        O Omicron   P Pi
  R Rho       S Sigma     T Tau       U Upsilon
  F Phi       C Chi       Y Psi       W Omega
\end{verbatim}
\subsection{Cyrillic}
Case character:  
B=Upper case, C=Lower case 
\begin{verbatim}
  A Ah        B Beh       V Veh       G Geh
  D Deh       E Yeh       X Zheh      Z Zeh
  I Ee        1 Ee S Kratkoy
  K Kah       L El        M Em        N En
  O Oh        P Peh       R Err       S Ess
  T Teh       U Ooh       F Ef        H Kha
  C Tseh      2 Cheh      3 Shah      4 Shchan
  Q Tvyordy Znak
  Y Yery      5 Myakhki Znak
  6 Eh Oborotnoye
  W Yoo       J Yah
\end{verbatim}
\subsection{Punctuation}
Case character:  
P=Punctuation 
\begin{verbatim}
     1.  .    Colon 
     2.  ,    semi-colon 
     3.  E    Exclamation mark 
     4.  U    question mark 
     5.  I    Interrobang 
     6.  A    Apostrophe 
     7.  Q    Quotation marks 
     8.  P    New paragraph 
     9.  D    Dagger 
    10.  F    Double dagger 
\end{verbatim}
\subsection{Math symbols}
Case character:  
M = Mathematical symbols 
\begin{verbatim}
     1.  .    Dot product 
     2.  X    Cross product 
     3.  /    Division sign 
     4.  P    Group plus 
     5.  M    Group multiply 
     6.  +    Plus or minus 
     7.  -    Minus or plus 
     8.  L    Less than 
     9.  G    Greater than 
    10.  M    Less than or equal 
    11.  H    Greater than or equal 
    12.  N    Not equal 
    13.  =    Identically equal 
    14.  A    Approximately equal 
    15.  C    Congruent to 
    16.  S    Similar to 
    17.  R    Proportional to 
    18.  T    Perpendicular to 
    19.  2    Square root 
    20.  D    Degrees 
    21.  I    Integral sign 
    22.  J    Line integral 
    23.  Y    Partial derivative 
    24.  Z    Del.  
    25.  (    Left floor braket 
    26.  )    Right floor braket 
    27.  B    Left ceiling braket 
    28.  E    Right ceiling braket 
    29.  0    Infinity 
\end{verbatim}
\subsection{Theoretic}
Case character:  
T = Theoretic special symbols 
\begin{verbatim}
     1.  E    Existential quantifier 
     2.  A    Universal quantifier 
     3.  M    Membership symbol 
     4.  N    Membership negation 
     5.  I    Intersection 
     6.  U    Union 
     7.  L    Contained in 
     8.  G    Contains 
     9.  K    Contained in or equals 
    10.  F    Contains or equals 
\end{verbatim}
\subsection{Arrows}
Case character:  
W = arrows 
\begin{verbatim}
     1.  U    Up 
     2.  D    Down 
     3.  L    Left (<--) 
     4.  R    Right (-->) 
     5.  B    Left/right (<-->) 
\end{verbatim}
\subsection{Physics}
Case character:  
K = Physics symbols 
\begin{verbatim}
     1.  H    H-bar 
     2.  L    Lambda bar 
\end{verbatim}
\subsection{Astronomical}
Case character:  
A=Astronomical symbols 
\begin{verbatim}
     1.  H    Sun 
     2.  M    Mercury 
     3.  V    Venus 
     4.  E    Earth 
     5.  W    Mars 
     6.  J    Jupiter 
     7.  S    Saturn 
     8.  U    Uranus 
     9.  N    Neptune 
    10.  P    Pluto 
    11.  O    Moon 
    12.  C    Comet 
    13.  *    Star 
    14.  X    Ascending mode 
    15.  Y    Descending mode 
    16.  K    Conjunction 
    17.  Q    Quadrature 
    18.  T    Opposition 
    19.  0    Aries 
    20.  1    Taurus 
    21.  2    Gemini 
    22.  3    Cancer 
    23.  4    Leo 
    24.  5    Virgo 
    25.  6    Libra 
    26.  7    Scorpius 
    27.  8    Sagittarius 
    28.  9    Capricornus 
    29.  A    Aquarius 
    30.  B    Pisces 
\end{verbatim}
\subsection{Markers}
Case character:  
O= Plotting symbols 
\begin{verbatim}
          0   Vertical cross
          1   Diagonal cross
          2   Diamond
          3   Square
          4   Fancy diamond
          5   Fancy square
          6   Fancy vertical cross
          7   Fancy diagonal cross
          8   Star burst
          9   Octagon
\end{verbatim}
\subsection{Drawing}
Case character:  
D= Drawing characters 
\begin{verbatim}
          U   Underscore
          O   Overscore
\end{verbatim}
\subsection{Movement}
This  moves the current position by the specified amount.  This is useful
for overstriking 2 characters to create a new character.  
Case character:  
U=Horizontal movement 
\begin{verbatim}
          0   1 space back
          1   Half space
          2   Half space back
          3   Third space
          4   Third space back
          5   Sixth space
          6   Sixth space back
\end{verbatim}
V=Vertical movement
\begin{verbatim}
          1   Half  up
          2   Half  down
          3   Third  up
          4   Third  down
          5   Sixth  up
          6   Sixth  down
\end{verbatim}
\subsection{Subscript}
Case character:  
X=Subscript control 
\begin{verbatim}
          0   Enter subscript
          1   Leave subscript
          2   Enter superscript
          3   Leave superscript
\end{verbatim}
\subsection{Character\_size}
Case character:  
Y= Character size control 
\begin{verbatim}
          0   Increase by One-half 
          1   Decrease by One-half 
\end{verbatim}
\subsection{Position}
Case character:  
Z=Position control 
\begin{verbatim}
          0   Save current state in area 1
          1   Restore state in area 1
          2   Save current state in area 2
          3   Restore state in area 2
          4   Save current state in area 3
          5   Restore state in area 3
          6   Save current state in area 4
          7   Restore state in area 4
\end{verbatim}
\subsection{Examples}
To draw the letter A with a bar over it.  You first draw an ``A'' then move
back 1 space, then draw an Overscore.  
\begin{verbatim}
     TITLE "A0O" CASE " UD" 
\end{verbatim}
To draw a lowercase greek Lambda with a ``hat or circumflex'' over it.  
\begin{verbatim}
     TITLE "L012" CASE "GUV V" 
\end{verbatim}
\subsection{Setting\_fonts}
The fonts are selected in 2 different ways for plot symbols and titles.  

For  plot  symbols  you use a 2 character identifier for the symbol.  For
example 0O selects the cross as a plot symbol.  See commands:  
\begin{verbatim}
     1.  SET SYMBOL 
     2.  PLOT 
     3.  SET ORDER 
\end{verbatim}

For  titles  the special symbols are selected with the CASE command.  The
case command specifies a string with a font selection character for  each
character  in  the  title.   For  example  you  wish  to  write  in Greek
Alpha,Beta,Gamma followed by ABC.  
\begin{verbatim}
     TD:TITLE "ABGABC" 
     TD:CASE  "GGG   " 
\end{verbatim}
\section{3-Dimensional\_plots}
TOPDRAWER  will  do  a  variety  of  3-d  plots.   You can plot a series of
separate graphs superimposed on a third axis, or you can enter an array  of
data  to be plotted with hidden line removal.  The array of data is entered
by the READ MESH command.  The mesh is then plotted by the HISTOGRAM  ,JOIN
or PLOT commands.  
\begin{verbatim}
     1.  HISTOGRAM BLOCK produces a block or "Lego" plot of mesh data.  
     2.  JOIN produces a mesh surface.  
     3.  SET THREE OFF;PLOT produces a scatter plot of mesh data.  
     4.  CONTOUR produces a contour plot.  
\end{verbatim}

\begin{verbatim}
     Axes  are  not  automatically  added to 3-d graphs so you must use the
\end{verbatim}
command PLOT AXES.  You may turn each axis on separately using the SET AXIS
command  and  then  PLOT  AXES  X,Y,Z  to  place  each axis in a convenient
location.  

\begin{verbatim}
     The  command  SET  THREE  is  used to specify the "viewpoint" of a 3-d
\end{verbatim}
plot.  SET SCALE is also used to specify the range and type of scale to use
along each of the X,Y,Z axes.  

\begin{verbatim}
     3-d  plots  may  also be built from a series of 2-d plots.  To do this
\end{verbatim}
you must provide z values and:  
\begin{verbatim}
     TD:SET THREE ON 
\end{verbatim}
Then  either  JOIN, HISTOGRAM or PLOT is used to plot the data.  Normally Z
is the dependent variable, and X,Y are the independent variables.  If  this
rule  is violated, 3-d histograms will not be correctly plotted.  Your data
may be moved with the SWAP command if you need to get it into  the  correct
variables.  
\subsection{Axes\_example}
Assume  you are plotting a mesh which extends x from 0 to 10, y from 0 to
20, z from 0 to 1.  The axes will normally appear behind the  plot  which
may  be  hard to read.  You can move the axes to the front for X,Y and to
the right for Z with the following commands.  
\begin{verbatim}
     TD:SET AXES ALL OFF X ON Y ON 
     TD:SET THREE ORIGIN 10 20 0 
     TD:PLOT AXES 
     TD:TITLE X CENTER "X title" 
     TD:TITLE Y CENTER "Y title" 
     TD:SET AXES ALL OFF Z ON 
     TD:SET THREE ORIGIN 0 20 0 
     TD:PLOT AXES 
     TD:TITLE Z CENTER "Z title" 
\end{verbatim}
\section{Histograms}
TOPDRAWER can read files written by other histogram packages, and then plot
them.  Support is provided for  the  HBOOK  histogram  package.   For  more
information see:  
\begin{verbatim}
     SET HISTOGRAM 
\end{verbatim}
\section{COMMANDS}
\subsection{Format\_of\_commands}
Normally  1  command is placed on a single line.  If you wish to put more
than 1 command per line, they must be separated by a semicolon '';'' 

Optional  syntax  in a command is shown enclosed in square brackets ''[]''.
If options are mutually exclusive they are separated by vertical bars ''$|$''
Options within a command may appear in any order.  Parameters must follow
the option they modify.  Options in a command may be  separated  by  tabs
spaces  * , or =.  For example SIZE=5 is the same as SIZE 5 or /SIZE=5 or
,SIZE 5.  

Commands  and options may be abbreviated to the shortest number of unique
characters or a  minimum  of  2  characters.   For  example  the  command
BARGRAPH may be abbreviated to BARGR, BAR or even BA.  SET FILE INPUT may
be abbreviated to SE FI  IN.   All  commands  may  be  abbreviated  to  3
characters.  The minimum abbreviation for each option is underlined.  

Commands  are  read from left to right.  If conflicting options are given
for the same command,  the  right  most  option  is  used.   For  example
SET AXES ON OFF turns the axes off.  

Many  commands have synonyms.  For example AXES or AXIS, STOP or END, NEW
FRAME or NEW PLOT or just NEW.  

Comments may be added to a command by enclosing them in parentheses ''()''.
If the line begins with a double slash, ''//'' the rest of the  line  is  a
comment.  
\subsection{?}
The  question  mark  may be used after a command to get immediate help on
the options.  TOPDRAWER  will  list  the  various  options  and  what  is
expected  after  each option.  If the option expects more key words, they
are not listed.  
\subsection{List\_of\_commands}
\begin{verbatim}
                       3-d plot info.  is in parentheses ()
     1.  ADD - Adds together 2 data sets 
     2.  ARROW - Draws an arrow.  
     3.  BARGRAPH - Produces a bar graph.  
     4.  BIN - This transforms the data to a frequency distribution
         (histogram).  This may also be used to add data sets together.  
     5.  BOX - Draws a box.  
     6.  CIRCLE - Draws a circle or ELLIPSE.  
     7.  CLEAR - Clears the screen the next plot will use same parameters
         (WARNING:  usually you want command NEW) 
     8.  CONTOUR - Draws a contour plot of MESH data.  
     9.  CONVOLUTE - Calculates the convolution of 2 data sets.  
    10.  CREATE MESH - Creates mesh data.  
    11.  DATA SET - Starts a new data set.  
    12.  DELETE - Deletes data points.  
    13.  DEFINE COMMAND/KEY - Defines commands or keys on your keypad.  
    14.  DEFINE HISTOGRAM - Defines histograms (HBOOK ) 
    15.  DIAMOND - Draws a diamond.  
    16.  DIVIDE - Divides 2 data sets (histograms) 
    17.  DX/DY/DZ - Enters a sequence of DX,DY, or DZ data values.  
    18.  ELLIPSE - Draws a circle or ELLIPSE.  
    19.  ENDREPEAT - Ends a repeat loop.  
    20.  ENDFILE - Ends a file.  
    21.  FFT - Calculates a "fast" fourier transform of a data set.  
    22.  FIT - Fits functions to the data and/or generates a curve.  
    23.  FLUSH - Flush out all remaining data.  
    24.  HISTOGRAM - Draws a histogram or ("Block") plot.  
    25.  HELP - Get help on TOPDRAWER.  
    26.  IF/ELSE/ENDIF - Tests status 
    27.  JOIN - Joins the data points with a line or (Mesh) plot.  
    28.  LIST - List the current data points.  
    29.  LIST HISTOGRAM - List histograms (HBOOK ) 
    30.  MERGE - Merges the contents of data sets.  
    31.  MULTIPLY - Multiplies 2 data sets (histograms) 
    32.  NEW FRAME - Starts a completely new plot.  
    33.  PAUSE - The program pauses.  
    34.  PLOT - Plots data as points or symbols (Scatter plot).  
          *  OUTLINE - Plots an outline around the plot.  
          *  AXES - Plots axes (Ticks and labels).  
          *  TABLE - Plots a table (list of numbers).  
    35.  READ - Reads in data points.  
    36.  REPEAT - Repeats a series of commands.  
    37.  RESTORE - Restores things 
         A.  DATA - Restores data sets from a file.  
         B.  HISTOGRAM - Restores a set of histograms (HBOOK ) 
         C.  FIT - Restores a fit.  
    38.  SAVE - Saves things See:RESTORE 
    39.  SHOW - Shows current options as set by the SET command.  
    40.  SMOOTH - The data is smoothed.  
    41.  SORT - Sorts the contents of a data set.  
    42.  SPAWN - Spawns out of TOPDRAWER.  
    43.  STOP - Stops TOPDRAWER.  
    44.  SUBTRACT - Subtracts 2 data sets.  
    45.  SWAP - Swaps data for 2 axes.  
    46.  SYMBOL - Enters symbols for existing data.  
    47.  TITLE - Puts a title on the plot.  
    48.  TYPE - Types a line on your terminal.  
    49.  CASE - Controls the format of the title.  
    50.  MORE - Adds more text under the current title.  
    51.  SET - Sets options for plots (defaults).  
    52.  X/Y/Z - Enters a sequence of X,Y, or Z values.  
\end{verbatim}
\subsection{Options}
There  are  a  number of options that are common to many commands.  These
select either the data to use, or the  plot  attributes.   Commands  that
modify data have some common output options.  
\begin{verbatim}
                                Data selection
\end{verbatim}
POINTS$|$COLUMNS=[FROM] n1 [TO] [n2] 
\begin{verbatim}
     Selects the data points or columns of a mesh by number 
\end{verbatim}
LINES$|$ROWS=[FROM] n1 [TO] [n2] 
\begin{verbatim}
     Selects the lines of a mesh 
\end{verbatim}
SETS=[FROM] n1 [TO] [n2] 
\begin{verbatim}
     Selects the data sets by number 
\end{verbatim}
SELECT=``name'' 
\begin{verbatim}
     Selects the data sets by name 
\end{verbatim}
LIMITED  [VLOG[=ON$|$OFF]] [[FROM]$|$TO [[X=]nx,[[Y=]ny[,[Z=]nz]]] [RECURSOR]
\begin{verbatim}
     [CURSOR] 
     Selects the data by specifying a range or value for each point 
\end{verbatim}
SLICES [X$|$Y$|$Z] [FROM] v1 [[TO] v2] 
\begin{verbatim}
     Selects  the data by specifying a range and selects the axis to view
     a mesh.  
\end{verbatim}
[TITLE[=ON$|$OFF]] 
\begin{verbatim}
     Uses the name of the current histogram to put a title on the plot.  
                                  Plot limits
\end{verbatim}
EXPAND[=ON$|$OFF] 
\begin{verbatim}
                                Plot attributes
\end{verbatim}
CYCLE[=ON$|$OFF] 
\begin{verbatim}
     Each data set is plotted with a different intensity, color ...  
\end{verbatim}
INTENSITY$|$WIDTH=n 
\begin{verbatim}
     Selects the line width for the plotted data 
\end{verbatim}
WHITE$|$RED$|$GREEN$|$BLUE$|$YELLOW$|$MAGENTA$|$CYAN 
\begin{verbatim}
     Selects the color for the plotted data 
\end{verbatim}
SOLID$|$DOTS$|$DASHES$|$DAASHES$|$DOTDASH$|$PATTERNED$|$FUNNY$|$SPACE 
\begin{verbatim}
     Selects the texture of the lines for plotted data.  
\end{verbatim}
HIDE[=ON$|$OFF] 
\begin{verbatim}
     Hides the current data if behind a previous hidden set.  
\end{verbatim}
FILL[=ON$|$OFF] 
\begin{verbatim}
     Fills the data with a pattern.  
                               Output selection.
\end{verbatim}
APPEND[=ON$|$OFF] 
\begin{verbatim}
     The  resulting "new" data set is appended to the current sets rather
     than replacing it.  
\end{verbatim}
NAME=``name'' 
\begin{verbatim}
     Specifies the name of the "new" data set.  
                                 Miscellaneous
\end{verbatim}
LOG[=ON$|$OFF] 
\subsubsection{APPEND}
For  commands  that  transform  data either the specified data sets are
replaced by the new data set, or all data sets are replaced by the  new
data.   If  you wish to retain the old data the option APPEND creates a
new data set with the transformed data.  
(Default:OFF) 

See:Command SET MODE 
\subsubsection{CYCLE}
When  plotting  more  than  1  slice or data set, CYCLE causes the line
texture, color, and width to vary from one plot to the next.  The order
of  cycling  is  set  by  the SET CYCLE command.  If you have specified
either texture, color or width it overrides the cycle value.  
(Default:OFF) 
\subsubsection{EXPAND}
Sets  the  limits so that the selected data expands to fill the window.
This has  no  effect  unless  LIMITED,  POINTS,  SETS,  or  SLICES  are
specified.   Normally  the  limits of the plot are set according to all
data.  This may inconvenient  when  the  plotted  subset  of  the  data
occupies only a small portion of the window.  EXPAND will make the data
fill the window.  
(Default:OFF) 

See:Command SET MODE 
\subsubsection{FILL}
Generates a fill pattern.  To set the fill pattern see:SET FILL.  
(Default:OFF) 
\subsubsection{HIDE}
Controls  whether hidden lines are drawn for a 3-d mesh, or whether one
plot hides the next.  You must specify HIDE for both  the  plot  to  be
hidden  and the plot that is to hide it.  The plot which is drawn first
will hide the second one.   HIDE=OFF  draws  the  hidden  lines,  while
HIDE=ON omits them.  
(Default:HIDE=ON for mesh, HIDE=OFF for data) 

\begin{verbatim}
                                example
\end{verbatim}
Assume  you  have  2 data sets where you want set 1 to hide elements in
set 2.  Also you would like to fill data set 2.  
\begin{verbatim}
     TD:JOIN SET=1 HIDE 
     TD:JOINS SET=2 HIDE FILL 
\end{verbatim}

You  can  produce a drawing of a mesh with hidden lines as dotted lines
by doing the histogram twice:  
\begin{verbatim}
     TD:HISTOGRAM 
     TD:HISTOGRAM HIDE=OFF DOTTED 
\end{verbatim}
\subsubsection{LOG}
This  logs  the operation by typing on your terminal a line telling you
what was done.  
(Default:OFF) 
\subsubsection{NAME}
The  new  set  will  usually have a name consisting of a transformation
name followed by the old set name.  If you specify a  new  name  it  is
applied to the new data set.  If the name ends in ''\%'' then the old name
is appended to the new name.  See option:APPEND 
\subsubsection{POINTS$|$COLUMNS}
This  selects the points or the colums of a mesh to plot or modify.  If
not selected all points are used.  If n2 is omitted it is assumed to be
n1.  You may specify the point number or the options FIRST,LAST.  
\subsubsection{SELECT}
SELECT=``name''
selects  the  data  sets to plot or modify by name.  You may use ``wild''
characters.  ''*'' means any  characters,  while  ''\%''  means  any  single
character.  Normally the match is case blind unless you:  
\begin{verbatim}
     TD:SET EXACT 
\end{verbatim}
\subsubsection{SETS}
This  selects  the data sets to plot or modify.  If n2 is omitted it is
assumed to be n1.  You may specify the data set number or  the  options
FIRST,LAST.  
\subsubsection{SOLID...}
This determines the texture of the line.  By stringing together several
texture commands you may create very complicated textures.  For example
DOT  SPACE  DOT  DASH  will produce a dot-dot-dast pattern with a large
space between the dots.  

\begin{verbatim}
     1.  SOLID  produces  a  totally  solid  line.   This overrides all
         previous specifications.  
     2.  DOT produces a dot 
     3.  DASH produces a short dash 
     4.  DAASH produces a long dash 
     5.  SPACE produces extra space between previous elements 
     6.  FUNNY produces random distances between elements 
     7.  PATTERNED  makes  the  pattern  according  to  the current SET
         pattern if it is specified by itself.  If specified with other
         options it produces the patterns by generating a pattern.  The
         length of the elements is determined by the size  in  the  SET
         PATTERN command.  
     8.  Some  combinations  will produce the patterns by hardware when
         available.  This makes for quick,  but  not  necessarily  nice
         plots.  The following generally use hardware:  
         A.  DOT 
         B.  DASH 
         C.  DOT DASH 
         D.  DAASH 
\end{verbatim}

See:Command SET TEXTURE 
\subsubsection{SLICES}
\begin{verbatim}
          [SLICES [X|Y|Z] [FROM] v1 [[TO] v2]] [CYCLE[=ON|OFF]] 
               or...  
          [SLICE X|Y|Z=v1] [CYCLE[=ON|OFF]] 
\end{verbatim}

This  plots  or  joins  points  sliced from mesh data.  This command is
illegal for normal data.  If v2 is omitted then v2=v1.  If a  range  is
selected then all slices within the range are histogrammed.  IF you set
THREE=OFF then the slices are histogrammed according to the first X,Y,Z
Mentioned.   If  X  is first then the slices are taken perpendicular to
the X axis.  You only need select X$|$Y$|$Z the first time you slice  after
a NEW FRAME command.  
\subsection{TITLE}
This takes the name of the data set you are plotting and uses it to write
a title on the plot.  The name is  divided  into  4  parts  separated  by
semicolons '';''.  The 4 parts are placed at the TOP,X,Y,Z locations of the
plot.  If the default title color, width  are  none  then  the  title  is
plotted with the same color, width as the the data set.  

This can be very helpful when overlaying several data sets, as the titles
correspond to the data set.  
\subsection{WHITE...}
This sets the color of the plot.  

See:Command SET COLOR 
\subsection{ADD}
See:  SUBTRACT 
\subsection{SUBTRACT}
\{ADD$|$SUBTRACT\} [Y$|$Z] [FROM$|$TO] [EWEIGHT=n] [WEEIGHT=n] \{n1$|$``name1''\} [BY]
\begin{verbatim}
          {n2|"name2"|FIT} 
     [AVERAGE|EFFICIENCY[=ON|OFF]] [APPEND[=ON|OFF]] [NAME="name"]
          [CHECK[=ON|OFF]] [ERROR[=ON|OFF]] [POINTS|
          COLUMNS=[FROM] n1 [TO] [n2]] [LINES|ROWS=[FROM] n1 [TO] [n2]]
          [LIMITED [VLOG[=ON|OFF]] [[FROM]|TO [[X=]nx,[[Y=]ny[,[Z=]nz]]]
          [RECURSOR] [CURSOR] ] [LOG[=ON|OFF]] [VECTOR[=ON|OFF]] 
\end{verbatim}
This adds or subtracts the (Y/Z) values in data set n2 (from,to) data set
n1.  The result is a modified set n1.  
\subsubsection{APPEND}
If  APPEND is specified, then a new data set is created, containing the
result and n1 is unchanged.  
\subsubsection{AVERAGE}
This  option  will  not  work  for the SUBTRACT command.  If AVERAGE is
specified, the data is averaged together according to the Y statistics.
If  Y  is  non zero but DY is 0 then DY is assumed to be 1.  If both Y1
and DY1 are zero then Y3,DY3 = Y2,DY2.  
\begin{verbatim}
     Y3=(Y1/DY1**2+Y2/DY2**2)/(1/DY1**2+1/DY2**2) 
     DY3=SQRT(2*DY1**2*DY2**2/(DY1**2+DY2**2)) 
\end{verbatim}
If  you  wish to average several data sets with statistics, this should
give you the correct result.  Zero data with zero statistics  indicates
the  absence of data.  If you wish to average several data sets without
statistics then you should always set the statistics to a  ``reasonable''
value.  
\subsubsection{CHECK}
CHECK=OFF  turns  off  data set checking.  When CHECK=ON both data sets
must have identical X (and Y if mesh)  values,  but  data  set  n2  may
contain  more  points than n1.  If (DX/DY) is non zero then the both DX
and X must be identical within 1\% of DX.  If you have  data  sets  with
non  identical  values  of X you may create a set with identical values
using the BIN command.  you may also add together data  sets  with  the
BIN  command.   If the X values of 2 data sets are not quite identical.
You may force TOPDRAWER to add them by setting  DX  for  the  data  100
times greater than the difference in X or by setting CHECK=OFF.  
\subsubsection{ERROR}
ERROR=OFF excludes the errors from the FIT in the computation.  
(Default:ERROR=ON) 
\subsubsection{FIT}
Specifies  that the last FIT is to be added or subtracted from the data
set.  See:Command FIT.  
\subsubsection{LIMITED}
You  may  specify the limits over which the histograms are to be added.
If you specify the Y or Z limits, then all data  that  contains  values
inside these limits will be added.  For example if you 
\begin{verbatim}
     ADD 1 to 2 LIMITED FROM Y=10 to Y=11 
\end{verbatim}
But data set 1 contains the followind data:  
1,0;  2,10;  3,0;  4,11;  5,0 
Points  2 to 4 inclusive will be added, and only points 1 and 5 will be
omitted.  
\subsubsection{LOG}
If  LOG  is  specified  then the result of the command is typed on your
terminal.  
\subsubsection{VECTOR}
The  DX,DY,DZ's will be added together assuming they are a vector.  The
X,Y,Z are not changed, and they should match for both input data sets. 
\subsubsection{WEIGHT-EWEIGHT}
WEIGHT=n  specifies  a  weighting factor to multiply the data by before
adding or subtracting it.  EWEIGHT=n specifies  the  error  on  WEIGHT.
This option must precede the data set number.  
\subsubsection{X$|$Y}
Selects  which  coordinate  of  regular data is to be added.  Mesh data
always adds the dependent coordinate which is usually z.  (Default:Y) 
\subsubsection{Notes}
For  mesh  data the independent coordinate must be Z.  For 2-d data the
independent coordinate is always Y.  If you add  normal  data  to  mesh
data  each  normal  Y value is added to the corresponding mesh Z value.
You may not add mesh data to normal 2-d data.  

If  EFFICIENCY is not specified then the new value is Y3=Y1(+-)Y2.  The
errors  (DY)  are  treated  as  if  the  data  sets  are   uncorrelated
DY3=SQRT(DY1**2+DY2**2).  

The ADD command with CHECK=OFF allows you to combine X,Y,Z...  data 
\begin{verbatim}
     TD:SWAP  X  Y  SET=1  from  1  data  set with mismatched data from
\end{verbatim}
another set.  For example you wish to plot Y of set 1 vs Y of set 2.  
\begin{verbatim}
     TD:SWAP X Y SET=1 
     TD:ADD 1,1 APPEND (Produces set 3=set 1) 
     TD:Y=0 SET=LAST (Set 3 has only X) 
     TD:ADD 2,3 CHECK=OFF (Set 3 has Y(1),Y(2)) 
\end{verbatim}
Set 3 now has X=Y(set 1) and Y=Y(set 2) 
\subsubsection{Example}
\begin{verbatim}
     TD:ADD 1 TO 2 
          or...  
     TD:ADD 2 1 
          or...  
     TD:ADD TO 2 1 
\end{verbatim}
Data set 1 is added to data set 2.  Data set 1 is unchanged.  
\begin{verbatim}
     TD:ADD 1 TO 2 APPEND 
\end{verbatim}
Data  set  1  and  2 are added.  The result in put into a new data set.
Both sets 1 and 2 are unchanged.  
\begin{verbatim}
     TD:SUBTRACT 1 FROM 2 
\end{verbatim}
Data set 1 is subtracted from data set 2.  Data set 1 is unchanged.  
See:Command DIVIDE,MULTIPLY 
\subsection{ARROW}
Draws an arrow from 1 point to another.  
ARROW [FROM] [x,y[,z]$|$CURSOR] [DATA] [LESS=n] 
\begin{verbatim}
     TO[x,y[,z]|CURSOR] [DATA] [LESS=n] 
     [SIZE=n] [FLARE|FLAIR=n] 
     [INTENSITY|WIDTH=n] 
     [WHITE|RED|GREEN|BLUE|YELLOW|MAGENTA|CYAN] 
     [SOLID|DOTS|DASHES|DAASHES|DOTDASH|PATTERNED|FUNNY|SPACE] 
     [FILL[=ON|OFF]] [HIDE[=ON|OFF]] 
\end{verbatim}
\subsubsection{Options}
\begin{verbatim}
     1.  x,y,z  -  specify the location of the ends of the arrow.  If z
         is specified the end point is assumed to be specified in  data
         coordinates.   If  CURSOR  is  specified you may enter the end
         point with the cross hair cursor.  The cursor appears  on  the
         screen.   Just  move  it  to the end point and press the space
         bar.  If you press  any  other  key,  the  end  point  is  not
         entered.  
     2.  DATA  -  specifies X,Y are in data coordinate frame.  Normally
         they are in the TEXT coordinate frame.  
     3.  LESS - Shortens the arrow by n so it does not touch x,y.  
     4.  SIZE - length of arrow head in tenths of an inch.  
     5.  FLARE  -  is ratio of the base to the height of the arrow head
         (fatness).  
     6.  INTENSITY...  - Selects the intensity of the arrow.  
     7.  WHITE...  - Selects the color of the arrow 
     8.  SOLID...   -  Selects  the  texture  of the body of the arrow.
         See:Command SET TEXTURE.  
     9.  FILL - Fills in the arrow head 
\end{verbatim}
\subsubsection{Example}
\begin{verbatim}
     TD:ARROW From 1,1 TO DATA 25,100 LESS 0.1 
\end{verbatim}
Draws an arrow from 1,1 in the text frame to data location 25,100.  The
arrow is shortened by 0.1 inches so it does not touch the data.  
\subsection{BARGRAPH}
Produces  a  simple  bar  graph.   Bars are centered on the x values with
heights given by Y.  Half widths are determined by DX and DY is not used. 

BARGRAPH [EXPAND] 
\begin{verbatim}
     [SETS=[FROM] n1 [TO] [n2]] 
     [SELeCT="name"] 
     [POINTS=[FROM] n1 [TO] [n2]] 
     [SLICES [X|Y|Z] [FROM] v1 [[TO] v2]] [CYCLE[=ON|OFF]] 
     [INTENSITY|WIDTH=n] 
     [WHITE|RED|GREEN|BLUE|YELLOW|MAGENTA|CYAN] 
     [SOLID|DOTS|DASHES|DAASHES|DOTDASH|PATTERNED|FUNNY|SPACE] 
                         3-d options 
     [X] [Y] [Z] [BLOCK|LEGO] [HIDE[=ON|OFF]] [FRAME[=ON|OFF]]
          [DEPTH[=ON|OFF]] [XY|YZ|ZX] [CROSS|RANDOM] 
\end{verbatim}

\begin{verbatim}
     If  the points are not specified all data points are used.  The line
\end{verbatim}
texture is determined  by  the  options  specified  or  by  the  default.
See:Command HISTOGRAM for more information.  
\subsection{BIN}
BIN$|$FREQUENCY 
\begin{verbatim}
     [MESH] [ERROR[=ON|OFF]] [[X|Y] [FROM=n] [TO=n] [BY|WIDTH|STEP=n]
          [BINS|N=n]] 
     [SETS=[FROM] n1 [TO] [n2]] 
     [SELeCT="name"] 
     [POINTS=[FROM] n1 [TO] [n2]] 
     [APPEND[=ON|OFF]] [NAME="name"] [AVERAGE[=ON|OFF]] [VALUES] 
     [LIMITED [VLOG[=ON|OFF]] [[FROM]|TO [[X=]nx,[[Y=]ny[,[Z=]nz]]]
          [RECURSOR] [CURSOR] ] 
     [NORMAL[=ON|OFF]] 
     [LOG[=ON|OFF]] [MONITOR[=ON|OFF]] 
\end{verbatim}
This transforms data to a frequency distribution (histogram) with equally
spaced X/Y values.  
\subsubsection{Introduction}
For  each  X entered, the corresponding Y is added/averaged to a bin in
the histogram.  If all Ys are zero Y and DY  are  assumed  to  be  1.0.
When  averaging  any  DYs  which  are  zero are assumed to be 1.0.  The
result is an evenly spaced series of X values  with  the  corresponding
distribution  in  Y  and  the  standard deviation of each Y in DY.  The
symbol for each bin is set to the  default.   You  may  find  that  the
resulting  data is not in the correct storage locations especially if a
3-d plot is to be produced.  You may change  locations  with  the  SWAP
command.  If the data has a non zero DX then it may be distributed over
more than 1 bin.  The DX error distribution is assumed to be  gaussian,
and the data is distributed with a cut off at 4 sigma.  
\subsubsection{Options}
\begin{verbatim}
     1.  MESH - Bins data into a MESH 
     2.  ERRORS - Bins data into a MESH with errors if MESH is
         selected.  
     3.  X,Y - Select the axes to specify bins.  
     4.  BINS|N - The number of bins in the histogram.  
     5.  VALUES - You are specifying the bin values rather than the bin
         edge.  
     6.  FROM - Sets the lowest bin value (low edge) 
     7.  TO - Sets the highest bin value.  
     8.  BY - Sets the bin width.  
     9.  NORMAL - Selects either a normal (Gaussian) distribution or a
         flat (histogram) distribution.  
    10.  NAME - Selects the name of the appended data set 
    11.  POINTS - Specifies the range of points within a data set to
         use.  
    12.  SETS - Specifies the range of data sets to use.  
    13.  APPEND - The new data is appended to the current data as a new
         set rather than replacing it.  
    14.  AVERAGE - Produces a weighted average of the data 
    15.  LIMITED - Selects the range of data to bin.  
    16.  LOG - Types the result of the command 
    17.  MONITOR - Plots the result.  
\end{verbatim}

\subsubsection{AVERAGE}
A  weighted  average  is produced of the data.  The data is weighted by
1/DY**2 if DY is non zero.  Otherwise a  weight  of  1  is  used.   The
result  is  X,Y,DY.   If  the original DYs are all zero (or 1) then the
final DY is 1/SQRT(points).  Where  points  is  the  number  of  points
averaged to form a bin.  
\subsubsection{APPEND}
If  APPEND=ON  the resulting data sets are appended to the current data
sets.  If OFF then the current data is replaced by the new binned data. 
\subsubsection{ERRORS}
If this option and MESH are selected, then the mesh will have errors on
its data.  
\subsubsection{FROM...}
[[X$|$Y] [FROM=n] [TO=n] [BY$|$WIDTH$|$STEP=n] [BINS$|$N=n]] 
Specifies  the  range  of  the  histogram.   If  FROM,TO  and NBINS are
specified BY may not be specified.  FROM is the lowes bin edge.  TO  is
the  highest  bin  edge.   BY  is the bin width, And N is the number of
bins.  

If VALUES is specified then the FROM and TO are the center of the first
bin and the center of the last bin.  

If these are omitted then the data is scanned and the bins are arranged
according to the current scale values.  (See:SET SCALE) If the data  to
bin  is  arranged  as  a set of matching histograms, the final data set
will have the same x values as the initial.  A  histogram  has  equally
spaced  x  values  with identical DX.  Data sets match if the initial X
and DX values are the same.  

The following are equivalent commands:  
\begin{verbatim}
     TD:BIN FROM 0 TO 10 BY 1 
     TD:BIN FROM 0 TO 10 N=10 
     TD:BIN FROM 0 BY 1 N=10 
     TD:BIN VALUES FROM .5 TO 9.5 BY 1 
\end{verbatim}
\subsubsection{LIMITED}
Specifies  which  data  are to be binned.  If limits are not specified,
the default is the current plot limits.  See:Command SET LIMITS.  
\begin{verbatim}
     1.  X - Specifies X limit 
     2.  Y - Specifies Y limit 
     3.  Z - Specifies the Z limit 
     4.  CURSOR  -  Brings up the cursor.  You move it to the X,Y value
         you wish then press the space bar to enter both X,Y  or  X  to
         enter X or Y to enter Y.  
     5.  RECURSOR - The cursor enters both limits.  
\end{verbatim}

\begin{verbatim}
                                Example
     TD:BIN LIMITED FROM X=1 to X=5 
\end{verbatim}
Sets limits on the range of X data to bin.  
\begin{verbatim}
     TD:BIN LIMITED FROM 1,5.5 TO 5,50.8 
\end{verbatim}
Sets limits on the range of X,Y to bin.  This does not set the range of
the final data sets.  
\subsubsection{MESH}
Bins  data into a mesh of Z vs X,Y.  The Z values are added or averaged
together.  If Z does not exist you are given an error message.   APPEND
is ignored for a MESH.  
\begin{verbatim}
                                example
     TD:BIN MESH X FROM 1 to 10 by 1 Y FROM 10 to 100 by 10 
\end{verbatim}
\subsubsection{MONITOR     }
If selected it histograms the result.  
See:  SET MODE MONITOR 
\subsubsection{NORMAL}
NORMAL=ON  selects  a  normal distribution with a cutoff at 4 sigma for
the error DX.  NORMAL=OFF selects a flat distribution from -DX to  +DX.
This is probably preferable for rebinning another histogram.  
(Default:NORMAL=ON) 
\subsubsection{X...}
This  specifies  which  axes  to  bin.   X  is  always binned.  If Y is
specified then both X and Y are  binned.   After  X  or  Y  you  should
specify  the  FROM,TO,BY  options  if  you  wish to bin into a specific
range, or step size.  If you specify Y without MESH then a set of  data
sets are produced, each for a different Y.  
\subsubsection{VALUES}
Normally  FROM,TO specify the edges of the minimum, maximum bins.  When
VALUES=ON then they specify the center of the bins.  
\subsubsection{Histograms}
You may add together 2 histograms or average together 2 histograms with
different bin widths.  You should specify the correct  DX  and  DY  for
each  data  set  you  wish  to  add.   You  probably  also  wish to use
NORMAL=OFF 

If  you do not specify FROM, TO ...  a set of values is assumed.  First
if you are binning 2 histograms with equally spaced equivalent  values,
then  the  same  X  values  are  assumed.  Otherwise the current X tick
spacing is used.  
\subsubsection{Example}
Say  you  wish  to  find the frequency distribution of a set of student
grades in 5 point steps.  The following will do the job:  
\begin{verbatim}
     TD:SET ORDER X 
          (Enter grades) 
     TD:Grade1;  Grade2;  Grade3.....  
     TD:Graden;.......  
     TD:BIN BY 5 (now do the binning) 
     TD:HISTOGRAM 
\end{verbatim}
Suppose  you  wish  to add together 2 HBOOK histograms.  Each histogram
contains counts, but the X spacing is different.  
\begin{verbatim}
     TD:SET HIST HBOOK ID=n1 
     TD:SET HIST ID=n2 APPEND 
          You now have both hists in memory.  
     TD:DY=POISSON (set the errors) 
     TD:BIN NORMAL=OFF 
\end{verbatim}
You  now  have  both histograms added together.  If both histograms had
the same X values/bin then the result will have the same  X  values  as
the original.  
\subsection{BOX}
Draws a box centered on the x,y with the sides parallel to the axes.  

BOX [AT$|$FROM$|$TO]\{x,y[,z]$|$CURSOR\} [DATA] [SIZE=dx[,dy]]
\begin{verbatim}
          [DSIZE=dx[,dy[,dz]]] [ROTATE=n] [SQUARE|SYMMETRIC[=ON|OFF]]
          [FILL[=ON|OFF]] [HIDE[=ON|OFF]] 
     [SOLID|DOTS|DASHES|DAASHES|DOTDASH|PATTERNED|FUNNY|SPACE] 
     [INTENSITY|WIDTH=n] 
     [WHITE|RED|GREEN|BLUE|YELLOW|MAGENTA|CYAN] 
\end{verbatim}
\subsubsection{AT}
x,y,z  specify  the center of the box.  If z is specified then they are
assumed to be data units.  Instead of x,y,z you may specify CURSOR.  If
x,y,z are specified without FROM,TO, then AT is assumed.  
\subsubsection{FROM-TO}
You  may also specify the diagonally opposite corners of the box.  If z
is specified then they are assumed to be data units.  Instead of  x,y,z
you may specify CURSOR.  
\subsubsection{FILL}
Fills in the box using the current fill pattern.  See:SET FILL 
\subsubsection{CURSOR}
The  cross  hair  cursor is used to specify the limits or center of the
box.  First move the cursor to a corner of the box.   Press  the  space
bar.   Next  move  the  cursor the diagonally opposite corner and press
space again to draw the box.  If you decide not to draw the box,  press
any other key other than space or ``Return''.  
\subsubsection{DATA}
Specifies  X,Y  are in data coordinate frame.  Normally they are in the
TEXT coordinate frame.  This option is not necessary if z or DSIZE  are
specified.  
\subsubsection{DSIZE}
Specifies the size in data units.  If the size is in data units the x,y
are assumed to be data units also.  
\subsubsection{ROTATE}
Specifies  a  rotation  angle  in degrees.  Rotation is right handed or
counterclockwise.  
\subsubsection{SIZE}
Specifies  the  size  in inches DX is the width of the box.  If omitted
the width is the default see:Command SET BOX SIZE.  DY is the height of
the box.  If omitted DY=DX.  
\subsubsection{SQUARE}
The  option  SQUARE  or  SYMMETRIC  produces a square box.  The SIZE is
treated as a diagonal through the center of a square, and FROM, TO  are
treated as the coordinates of 2 opposite sides of a square.  
\subsubsection{SOLID...}
Sets the texture of the line.  See:Command SET TEXTURE.  
\subsubsection{INTENSITY}
Sets the intensity (n=1-5) 
\subsubsection{WHITE...}
Sets the color of the line.  
\subsubsection{Example}
\begin{verbatim}
     TD:BOX AT 6.5,5 size=6.5,5 
          or...  
     TD:BOX 6.5,5 size 6.5,5 
          or...  
     TD:BOX FROM 3.25,2.5 TO 9.75,7.5 
          or...  
     TD:BOX FROM 3.25,2.5 size=6.5,5 
\end{verbatim}
Draws  a  centered box half the size of the screen assuming the default
size of 13x10.  
\begin{verbatim}
     TD:Set limits X -100,100 Y -100,100 
     TD:BOX 0,0 DATA size 1 
\end{verbatim}
Draws a box centered on the data 1 inch square 
\begin{verbatim}
     TD:Set limits X -100,100 Y -100,100 
     TD:BOX 50,50 Dsize 100,100 
\end{verbatim}
Draws a box around the upper right quadrant of the data.  
\begin{verbatim}
     TD:BOX AT CURSOR TO CURSOR 
\end{verbatim}
Draws  a box centered at the first cursor position with a corner at the
second cursor.  
\begin{verbatim}
     TD:BOX FROM CURSOR SIZE=2 
\end{verbatim}
Draws  a box with the lower left corner at the cursor position 2 inches
square.  
\subsection{CASE}
See:Command TITLE 
\subsection{CIRCLE}
Draws a circle or ellipse.  See:Command ELLIPSE.  
\subsection{CONTOUR}
Draws a contour plot from mesh data.  
CONTOUR [n1,n2,n3.....] 
\begin{verbatim}
                                 Contour selection
     [FROM n TO n BY n N n] 
     [PRIMARY|SECONDARY] [LABEL[=ON|OFF]] 
                                  Data selection
     [POINTS|COLUMNS=[FROM] n1 [TO] [n2]] 
     [LINES|ROWS=[FROM] n1 [TO] [n2]] 
     [LIMITED [VLOG[=ON|OFF]] [[FROM]|TO [[X=]nx,[[Y=]ny[,[Z=]nz]]]
          [RECURSOR] [CURSOR] ] [SELeCT="name"] 
                                  Label selection
          [INSIDE[=ON|OFF]] [OUTSIDE[=ON|OFF]] [PEPPENDICULAR[=ON|OFF]]
          [PARALLEL[=ON|OFF]] [DOUBLE[=ON|OFF]] 
                                    Attributes
     [EXPAND[=ON|OFF]] 
     [INTENSITY|WIDTH=n] 
     [WHITE|RED|GREEN|BLUE|YELLOW|MAGENTA|CYAN] 
     [SOLID|DOTS|DASHES|DAASHES|DOTDASH|PATTERNED|FUNNY|SPACE] 
     [CYCLE[=ON|OFF]] 
                                       Misc
     [TITLE[=ON|OFF]] 
     [SAVE[=ON|OFF]] [NAME='name'] 
\end{verbatim}
\subsubsection{N1...}
Contour lines are drawn for the specified values n1,n2...  

If  no values are specified, then lines are drawn according to the tick
locations selected by the SET SCALE command.  Primary contour lines are
determined  by  the long tick locations, and the secondary lines by the
short ticks.  

If  the  scale of the Z axis is set to be logarithmic, then the primary
lines are at powers of 10.  
\subsubsection{COLOR}
Sets the color of primary contour lines.  
\subsubsection{CYCLE}
CYCLE causes the line texture, color, and width attributes to vary from
one set of secondary contour lines to the next.  The order  of  cycling
is  set  by the SET CYCLE command.  The attributes specified by the SET
SECONDARY command override those specified by the  SET  CYCLE  command.
Primary  contour  lines  also vary according to cycles.  If you specify
any attributes they are used for the primary contours, and override the
the cycles.  
\subsubsection{EXPAND}
Sets  the  limits so that the selected data expands to fill the window.
This has no effect unless POINTS, or SETS are specified or  the  limits
on MESH data have been set by a SET LIMITS command.  

See:SET MODE 
\subsubsection{FROM$|$TO$|$BY$|$N}
You  may  also  specify a range of contours with FROM,TO,BY,N.  You may
specify any 3.  The maximum number of contours you may specify at  once
is 100.  

\begin{verbatim}
                                example
     TD:CONTOUR FROM 1 TO 10 BY 2 
          or...  
     TD:CONTOUR FROM 1 BY 2 N=5 
          or..  
     TD:CONTOUR 1,3,5,7,9 
\end{verbatim}
Draws contour lines at 1,3,5,7, and 9.  
\subsubsection{INTENSITY}
Sets the line width or intensity of primary contour lines.  
\subsubsection{LABEL}
If  LABEL=ON is specified then any contours after the LABEL option will
be labelled.  If no  contours  are  specified  LABEL=OFF  produces  the
default contours without any labels.  
\subsubsection{OUTSIDE}
This turns on labels along the outside boundary of the contour plot for
all contours that terminate at the boundary.  (Default) 
\subsubsection{INSIDE}
This  turns  on  labels  in  the  interior  of the contour plot for all
contours.  (Default) 
\subsubsection{DOUBLE}
This turns on double labels on the outside boundary of the contour plot
for all contours that terminate at the boundary.  This labels both ends
of each terminal contour.  
(Default:DOUBLE=OFF) 
\subsubsection{PARALLEL}
Produces  label  which  are parallel to inside contour lines.  Normally
labels are horizontal parallel to the X axis.  
(Default:PARALLEL=OFF) 
\subsubsection{PERPENDICULAR}
Produces  label  which  are  perpendicular  to  inside  contour  lines.
Normally labels are horizontal parallel to the X axis.  
(Default:PERPENDICULAR=OFF) 
\subsubsection{PRIMARY$|$SECONDARY}
Sets  the  type  of  contour  line.   Primary  lines use the default or
selected color, intensity, and texture.  The  secondary  contour  lines
are determined by the SET SECONDARY command.  The default for secondary
lines is DOTTED.  
\subsubsection{SAVE}
This  saves  the  contour as a series of data sets rather than plotting
it.  The new data sets have the name 'Contour ...' where  ...   is  the
old data set name.  You may select the new data set with the NAME='...'
option.  
\subsubsection{SIZE}
SIZE=n  determines  the size of the contour labels in tenths of inches.
If unspecified the size is determined by the  SET LABEL  command.   The
color,  and  intensity  of the contour labels is also determined by the
SET LABEL command.  
\subsubsection{SOLID...}
Sets the texture of primary contour lines.  
\subsubsection{Example}
\begin{verbatim}
     TD:CONTOUR 
\end{verbatim}
Draws a series of contour lines according to the Z scale.  

\begin{verbatim}
     TD:CONTOUR SECONDARY 1,2,3,4,6,7,8,9,PRIMARY,20,LABEL,5,10 
          or...  
\end{verbatim}

\begin{verbatim}
     TD:CONTOUR SECONDARY FROM 1 TO 9 BY 1 PRIMARY,20,LABEL,5,10 
\end{verbatim}
Draws  contours for Z=1 to 20.  Contour 5 and 10 are labeled.  Contours
5,10, and 20 will be SOLID, all others are dotted.  
\begin{verbatim}
     TD:CONTOUR SAVE 10 
\end{verbatim}
Saves  the  contours  for  Z=10  as  separate  data  sets.   No plot is
produced.  
\subsection{CONVOLUTE}
CONVOLUTE \&BY=\{nset$|$``name''\} [APPEND[=ON$|$OFF]] [NAME=``name''] [CHECK[=ON$|$
\begin{verbatim}
          OFF]] [ERROR[=ON|OFF]] [INVERT[=ON|OFF]] [LOG[=ON|OFF]]
          [MONITOR[=ON|OFF]] [NORMALIZE[=ON|OFF]]
          [SETS=[FROM] n1 [TO] [n2]] [SELeCT="name"] 
\end{verbatim}
Convolutes nset with the specified data sets.  

The specified data sets and nset must be histograms with the same spacing
between points.  If nset is a mesh, then the data sets must also be mesh
data.  You may convolute a mesh by a normal data set.  

NOTE:  The inverse convolution of 2 mesh data sets may not be correct.
This problem may be fixed in the next version.  
\subsubsection{Options}
\begin{verbatim}
     1.  BY selects the data set to convolute by.  You may specify
         either the name or the data set number.  
     2.  CHECK=OFF Turns off the error checking.  Normally the 2 data
         sets must have matching channel widths.  And each data set
         must have equally spaced x values and y values if a mesh.  
         (Default:ON) 
     3.  ERROR=OFF Errors are not calculated for the result.  
         (Default:ON) 
     4.  INVERT=ON Convolutes by the inverted function.  This is not an
         exact calculation.  The resulting data may exhibit ripples,
         due to the inversion process.  Inversion uses CERN library
         routine CFT.  
         (Default:OFF) 
     5.  NORMALIZE=ON Convolutes the data set with a normalized nset.  
         (Default:OFF) 
     6.  APPEND=ON Creates a new data set containing the entire result.
         The appended set is longer than the original set to contain
         the entire result.  If you do not select APPEND some data
         might be lost.  
         (Default:OFF) 
     7.  MONITOR Histograms both the original data and the result in 2
         windows.  
         See:SET MONITOR 
     8.  NAME selects the name for the new appended data set.  
     9.  SELECT selects data sets to convolute by name.  
    10.  LOG=ON Logs the operation on your terminal.  
         (Default:OFF) 
\end{verbatim}
\subsection{CREATE}
\begin{verbatim}
     CREATE MESH 
     X|Y|Z [FROM n][TOn][BYn][N=n] [ERRORS] [APPEND[=ON|OFF]]
          [NAME="name"] 
\end{verbatim}
Creates a mesh.  You may fill the mesh using the Z= command.  
ERRORS creates a mesh with errors.  Storage is allocated for DX,DY,DZ.  

\begin{verbatim}
                                 example
     TD:CREATE MESH X FROM 1 TO 10 BY 1 Y FROM 10 TO 100 BY 10 
     TD:Z="V_XV+V_YV" 
\end{verbatim}
creates some mesh data.  
\subsection{DATA}
\begin{verbatim}
     DATA [SET] [NAME="name"] This starts a new data set.  If the data is
\end{verbatim}
broken up into sets, you may specify which data set to plot.  Also JOIN
and HISTOGRAM will plot each set independently.  The Data sets are
numbered in order 1,2,3...  When a data set is deleted the remaining data
sets are renumbered.  You may specify data sets either by number or name. 

You may intermix MESH and regular data sets.  
\subsubsection{Example}
To add a new data set to the existing data, and enter 5 values into it: 
\begin{verbatim}
     TD:DATA SET name="New set" 
     -1.5,-1;-1,1;0,2;1,1;-2,-1.5 
\end{verbatim}
\subsection{DELETE}
\begin{verbatim}
     DELETE [HISTOGRAMS|DATA] 
\end{verbatim}
\subsubsection{DATA}
DELETE [ALL] 
\begin{verbatim}
     [SETS=[FROM] n1 [TO] [n2]] 
     [SELeCT="name"] 
     [POINTS=[FROM] n1 [TO] [n2]] 
     [LIMITED [VLOG[=ON|OFF]] [[FROM]|TO [[X=]nx,[[Y=]ny[,[Z=]nz]]]
          [RECURSOR] [CURSOR] ] 
     [CONFIRM[=ON|OFF]] [LOG[=ON|OFF]] 
\end{verbatim}

Deletes the specified data points.  If all the points in a data set are
deleted, the entire set is deleted.  If an entire data set is  deleted,
all  the  following  sets are renumbered.  You must specify either ALL,
POINTS, SETS or LIMITS.  POINTS or SETS may be combined with LIMITS  to
select  only  certain points within a range.  For mesh data LIMITED and
POINTS are not legal options.  
\paragraph{ALL}
Deletes all data points 
\paragraph{CONFIRM}
If  ON  you are asked to confirm whether you wish each point deleted.
If ON or OFF are omitted ON is assumed.  You reply:  
\begin{verbatim}
  *  Y - Yes delete it.  
  *  N - No, do not delete it (Default).  
  *  Q - Quit and delete no more points.  
  *  A - All delete it and all other specified points, without asking
     any more.  
\end{verbatim}
\paragraph{LIMITED}
Specifies  deletion  of data within specific limts on X,Y, and Z.  If
limits are not specified, the default is  the  current  plot  limits.
See:Command SET LIMITS.  
\begin{verbatim}
     1.  X - Specifies X limit 
     2.  Y - Specifies Y limit 
     3.  Z - Specifies the Z limit 
     4.  CURSOR - Brings up the cursor.  You move it to the X,Y value
         you wish then press the space bar to enter both X,Y or X  to
         enter X or Y to enter Y.  
     5.  RECURSOR - The cursor enters both limits.  
              TD:DELETE LIMITED RECURSOR 
                   is the same as...  
              TD:DELETE LIMITED FROM CURSOR TO CURSOR 
     6.  VLOG - Draws a cross when you press the space bar, and draws
         a dotted line around the final limits.  
                               example
     TD:DELETE LIMITED FROM 1,1 to 2,5 
          is the same as 
     TD:DELETE LIMITED FROM X=1 Y=1 TO X=2 Y=5 
\end{verbatim}
Deletes  all points that with X between 1 and 2 and Y between 2 and 5
inclusive.  
\begin{verbatim}
     TD:DELETE LIMITED FROM Y=1 TO Y=5 
\end{verbatim}
Deletes all data points with Y values between 1 and 5 inclusive.  
\begin{verbatim}
     TD:DELETE LIMITED FROM CURSOR TO CURSOR 
\end{verbatim}
Deletes all data points within a box defined by the cursor.  
\begin{verbatim}
     TD:DELETE LIMITED FROM X=CURSOR TO X=CURSOR 
\end{verbatim}
Deletes all data points within X values defined by the cursor.  
\paragraph{LOG}
LOG=ON  Lists  all  points  deleted.  Note, that pressing Ctrl\_C will
abort the deletions.  If you do not specify ON or OFF, ON is assumed. 
\paragraph{POINTS}
Specifies  the  range  of  points  to  delete.  n1,n2= the data point
number or LAST or FIRST.  
\paragraph{SETS}
Specifies  the  range  of  data  sets to delete.  n1,n2= the data set
number or LAST or FIRST.  
\subsubsection{HISTOGRAMS}
DELETE [ALL] 
\begin{verbatim}
     [IDENT=[FROM] nmin [TO] [nmax]] 
     [EXACT[=ON|OFF]] 
     [SELECT|NAME='hist_name'] 
     [AREA|DIRECTORY="dir/subdir..."] 
     [TReE[=ON|OFF]] 
     [ENTRIES[=ON|OFF]] 
     [HISTOGRAM[=ON|OFF]] 
     [MESH[=ON|OFF]] 
     [ARRAY[=ON|OFF]] 
     [NTUPLES[=ON|OFF]] 
     [CONFIRM[=ON|OFF]] [LOG[=ON|OFF]] 
\end{verbatim}

Deletes  the  specified  histograms.  You must specify either ALL, or a
specific range of IDENTSs.  You may  select  histograms  to  delete  by
NAME,   type   (ARRAY,HISTOGRAM,MESH,NTUPLES),  or  whether  they  have
entries.  In addition AREA, TREE allow selection  of  portions  of  the
directory tree.  
See also commands:  SHOW HISTOGRAM, SET HISTOGRAM 
\paragraph{ALL}
Deletes all histograms.  
\paragraph{IDENT}
Selects the range of IDs to delete.  
\paragraph{CONFIRM}
If  ON  you are asked to confirm whether you wish each point deleted.
If ON or OFF are omitted ON is assumed.  You reply:  
\begin{verbatim}
  *  Y - Yes delete it.  
  *  N - No, do not delete it (Default).  
  *  Q - Quit and delete no more points.  
  *  A - All delete it and all other specified points, without asking
     any more.  
\end{verbatim}
\paragraph{LOG}
LOG=ON  Lists  all  points  deleted.  Note, that pressing Ctrl\_C will
abort the deletions.  If you do not specify ON or OFF, ON is assumed. 
\subsection{DEFINE}
This  defines  various  things  such  as  new commands, or the keys for a
VT-100 keypad.  
\subsubsection{COMMAND}
\begin{verbatim}
     DEFINE COMMAND new_command "command string" 
\end{verbatim}

This defines a new command as the command string.  New commands must be
different from existing commands, and are  only  used  if  an  existing
command is not found.  There is a limit of 3 command substitutions in a
row, and 1000 total substitutions per input line.   The  limit  on  the
number of substitutions per line may be modified by SET COMMAND.  

\begin{verbatim}
                                example
     DEFINE COMMAND EDT "SPAWN EDT" 
\end{verbatim}
Defines the command EDT so you may edit a file while using TOPDRAWER.  
\begin{verbatim}
     DEFINE COMMAND MAIL "SPAWN MAIL" 
\end{verbatim}
Defines  the  command  MAIL so you may use the mail utility while using
TOPDRAWER.  
\begin{verbatim}
     DEFINE COMMAND @ "SET FILE INPUT=" 
\end{verbatim}
Defines  @  to work similarly to @ in VMS.  @filename will get commands
from the specified file.  
\begin{verbatim}
     DEFINE  COMMAND  EDAT  "LIST  DATA FILE=DAT.TMP;EDT DAT.TMP;DELETE
\end{verbatim}
ALL;  SET FILE IN=DAT.TMP'' 
Defines the command EDAT to allow you to edit the current data.  
\paragraph{Initialization}
You  may also define commands before running TOPDRAWER.  You make DCL
symbols of the form:  
\begin{verbatim}
     TD_C_new_command :==the_command 
\end{verbatim}
When  you  type  in new\_command the\_command is executed.  For example
the DCL definition:  
\begin{verbatim}
     $ TD_C_DIR*ECTORY :=SPAWN DIRECTORY 
\end{verbatim}
is the same as the TOPDRAWER command:  
\begin{verbatim}
     TD:DEFINE COMMAND "DIR*ECTORY" "SPAWN DIR" 
\end{verbatim}
\paragraph{Examples}
For  example  you  wish to define some DCL commands such as DIRECTORY
for convenience.  
\begin{verbatim}
     TD:DEFINE COMMAND DIR "SPAWN DIR" 
          or...  
     TD:DEFINE COMMAND "DIR*ECTORY" "SPAWN DIR" 
\end{verbatim}
These  both define the command DIR.  The second definition allows any
abbreviation from DIR to DIRECTORY to be used.  
\begin{verbatim}
     TD:DEFINE COMMAND SC "SHOW HIST CURRENT FULL" 
\end{verbatim}
Defines  the  command  SC  to show you the full status of the current
histogram.  (See:Command SET HIST and SHOW HIST) 
\begin{verbatim}
     TD:DEFINE COMMAND SD "SHOW DATA STATISTICS" 
\end{verbatim}
Defines a command to show you the statistics of the data set.  
\begin{verbatim}
     TD:DEFINE COMMAND SD1 "SHOW DATA STATISTICS SET=1" 
\end{verbatim}
Defines  a  command to show you the statistics of the data set number
1.  
\subsubsection{HISTOGRAMS}
This   creates   HBOOK   histograms   or   modifies   the   histograms.
See:Command SET HISTOGRAM.  You may  add  together  histograms  or  add
ntuples  to  histograms.   The  variables  in  NTUPLES  may be added to
several histograms all in one operation.  
DEFINE   HISTOGRAMS   \{IDENT[=FROM n][TOn]$|$  APPEND=n$|$  CURRENT$|$  NEXT$|$
\begin{verbatim}
     PREVIOUS| FIRST| LAST| ALL} [LOG[=ON|OFF]] 
                          Histogram selection
     [AREA|DIRECTORY="name"]                     [ENTRIES[=ON|OFF]] 
     [HISTOGRAM[=ON|OFF]]  [MESH[=ON|OFF]]       [ARRAY[=ON|OFF]] 
     [NTUPLES[=ON|OFF]]    [SELECT|NAME="hist name"] 
     [CONFIRM[=ON|OFF]] 
                          Histogram creation
     [[[X|Y][FROM n][TO n][BY n][BINS=n]]| 
     [SETS=[FROM] n1 [TO] [n2]]]                 [CHECK[=ON|OFF]] 
     [RANGE=n]             [PROFILE[=ON|OFF]] 
                        Histogram modification
     [BINSZ[=ON|OFF]]      [DEFNME='name']       [ERRORS[=ON|OFF]] 
     [STATISTICS[=ON|OFF]]                       [ZERO] [SECTION]] 
                           Histogram Filling
     [[ADD|SUBTRACT|MULTIPLY|DIVIDE=n1[,n2]]     [FACTOR=f1[,f2]] 
     [CYCLE[=ON|OFF]]] 
                             Print options
     [DEFAULT[=ON|OFF]]    [PRAUTOMATIC]         [PRBIGBI=n] 
     [PRBLACK]             [PRCHANNELS[=ON|OFF]] 
     [PRCONTENTS[=ON|OFF]]                       [PRERRORS[=ON|OFF]] 
     [PRHISTOGRAM[=ON|OFF]]                      [PRINTEGRAL[=ON|OFF]] 
     [PRLOGARITHMIC|PRLINEAR[=ON|OFF]]           [PRLOW[=ON|OFF]] 
     [PRMAXIMUM=n]         [PRMINIMUM=n]         [PRROTATE[=ON|OFF]] 
     [PRSTAR]              [PRSTATISTICS[=ON|OFF]] 
     [PRSQUEEZE[=ON|OFF]]  [PR1PAGE[=ON|OFF]]    [PR2PAGE[=ON|OFF]] 
     [PRPAGE=n]            [TITLE="general_title"] 
\end{verbatim}
\paragraph{Options      }
\begin{verbatim}
                              Histogram selection
     1.  APPEND=n  - Selects the IDENT as 1 more than the last one to
         last one+n.  
     2.  CURRENT,NEXT,PREVIOUS,ALL,FIRST,LAST - Selects a histogram. 
     3.  ENTRIES - Selects only histograms with entries 
     4.  HISTOGRAMS - Selects only histograms 
     5.  SELECT - The name of histograms to modify.  
     6.  IDENT - Selects the histograms by number.  
     7.  MESH - Selects only mesh histograms 
     8.  NTUPL - Selects only ntuples 
     9.  SECTION - Selects the Global Section (applies to ZERO only) 
                                  Operations
    10.  ADD,SUBTRACT,MULTIPY,DIVIDE  -  Combines 2 histograms into a
         third.  
    11.  CYCLE - Cycles through hists.  
                             Modifying parameters
    12.  DISK  -  Enables  histogram storage on DISK, with n records.
         It is stored using unit 87.  
    13.  DEFNAME - The new name to assign to histograms.  
    14.  ERRORS - Enable/Disable error computations.  This slows down
         histogram filling.  
    15.  TITLE  -  Defines  the  title  as  the  header  to each hist
         printout.  
    16.  PR...  - Modifies the printed output from the LIST HISTOGRAM
         command.  
    17.  BINSZ  -  YES sets bin size to be set to a "resonable value"
         when a histogram  is  defined.   This  is  defined  for  all
         histograms.  
    18.  STATISTICS   -   Enable/Disable   "real   time"   statistics
         calculations.  
    19.  ZERO - Clears or sets the histogram to zero 
                                   Creation
    20.  X,Y FROM n TO...  - Specifies the hist scale.  
    21.  SETS - Specifies the data sets to put into the histograms.  
    22.  RANGE - Selects the maximum range of the histogram (HBOOK) 
                                 Miscellaneous
    23.  LOG - Types the results of the definition 
\end{verbatim}
\paragraph{APPEND       }
Selects the histogram ID after the last one for creating a histogram.
If the option n is specified  it  is  the  number  of  histograms  to
create.  
\begin{verbatim}
                               Example
     DEFINE HIST X fROM 1 to 10 by 1 APPEND=5 
\end{verbatim}
Creates 5 histograms with net ID's of size 10.  
\paragraph{ADD...       }
ADD,SUBTRACT,MULTIPLY,DIVIDE      [IDENT=n1[,n2]]     [FACTOR=f1[,f2]
[CYCLE=[ON$|$OFF] 
Takes   histograms  IDENT=n1  and  n2  and  combines  them  into  the
previously specified histograms  n3=nmin  to  nmax.   The  number  of
channels in n1,n2, and n3 must all be the same.  
\begin{verbatim}
     n3=n2*f2 operation n1*f1 
\end{verbatim}
If  n1  or  n2  are  omitted  they are assumed to be n3.  If f1 or is
omitted it is assumed to be 1.0.  If n2 is specified and n3 does  not
exist,  it  is  created.   CYCLE  increments  N1,N2  each time.  This
combines n1,n2 into nmin and n1+1,n2+1 into nmin+1 and so  on.   Only
histograms  of  the same type with the same number of channels can be
added.  

If  n1  is  an NTUPL then it is binned into n3=nmin to nmax using the
parameters  specified  in  previous  SET  HISTOGRAM  commands.    The
possible options the may be set are X,DX,...,EVENTS,NLIMIT,NMASK.  If
the target histogram does not exist it is automatically created  with
100  channels  using the NTUPL name.  If CYCLE is selected then the X
specification is incremented.  
\begin{verbatim}
                               example
     TD:DEFINE HISTOGRAM IDENT=25 DIVIDE IDENT=1,2 FACTOR=.25,.75 
\end{verbatim}
HIST(25)=HIST(1)*0.25 / HIST(2)*0.75 
\begin{verbatim}
     TD:DEFINE HISTOGRAM IDENT=25 ADD IDENT=1,2 
\end{verbatim}
HIST(25)=HIST(1)+HIST(2) 
\begin{verbatim}
     TD:DEFINE HISTOGRAM IDENT=25 ADD IDENT=1 FACTOR=0.5 
\end{verbatim}
HIST(25)=HIST(1)*0.5 
\begin{verbatim}
     TD:DEFINE HISTOGRAM IDENT 25 TO 30 ADD IDENT=1 CYCLE 
\end{verbatim}
N-tupl value 1 is added to histogram 25 
N-tupl value 2 is added to histogram 26 
...  
N-tupl value 6 is added to histogram 30 
\paragraph{AREA$|$DIRECTORY}
\begin{verbatim}
     TD:DEFINE HISTOGRAM AREA=directory/subdir...  
\end{verbatim}
For HBOOK4 this creates the specified directory, and sets it.  If the
specified area does not  begin  with  ''/''  it  is  assumed  to  be  a
subdirectory  of  the current directory.  If it begins with ''/'' it is
assumed to be a subdirectory of the current root.  
\begin{verbatim}
                               example
\end{verbatim}
Assume you have no subdirectories.  
\begin{verbatim}
     TD:DEFINE HIST AREA SUBA 
\end{verbatim}
creates directory //PAWC/SUBA 
\begin{verbatim}
     TD:DEFINE HIST AREA Q 
\end{verbatim}
then creates directory //PAWC/SUBA/Q 
\begin{verbatim}
     TD:DEFINE HIST AREA /SUBB/X/Y 
\end{verbatim}
creates subdirectories:  
\begin{verbatim}
          "//PAWC/SUBB" 
          "//PAWC/SUBB/X" 
          "//PAWC/SUBB/X/Y" 
\end{verbatim}
\paragraph{CHECK        }
Turns  off  bin  checking for creating histograms from Topdrawer data
sets.  
\paragraph{CONFIRM      }
Each  histogram  that  matches the specifications is logged, then you
are asked whether you want to modify it.  You reply:  
\begin{verbatim}
  *  YES - to modify it.  
  *  NO - do not modify it (Default).  
  *  QUIT - do not modify all the rest 
  *  ALL - Stop asking questions and Modify all the rest.  
\end{verbatim}
If you omit ON or OFF then ON is assumed.  
(Default:CONFIRM=OFF) 
\paragraph{IDENT...    }
You must specify the IDENT of the histogram to modify or create.  
\begin{verbatim}
     1.  IDENT=FROM n TO n - where n is the ID to modify 
     2.  APPEND - The last histogram ID+1 in the current area 
     3.  CURRENT - The current histogram if any 
     4.  NEXT - The current histogram ID +1 
     5.  PREVIOUS - The current histogram ID -1 
     6.  ALL - All histograms 
\end{verbatim}

\paragraph{LOG        }
Type on your terminal a log of the definition.  
\paragraph{SELECT     }
The name of the histograms you wish to modify.  
\paragraph{SETS       }
Creates histograms from TOPDRAWER data sets.  It selects the range of
data sets to use in creating some histograms.  The  MESH  and  SELECT
options  may be used to select the data set.  This allows you to take
Topdrawer data and put it into the HBOOK structures.  The  data  used
must  be strictly monotonic in x.  Mesh data must have equally spaced
bins.  If you wish to store Mesh data  without  equally  spaced  bins
then  specify  CHECK=OFF  to  turn off the bin checking.  The PROFILE
option is ignored.  

If you specify SETS then you may not specify X or Y ranges or ZERO.  
\begin{verbatim}
                               EXAMPLE
     DEFINE HIST SETS 5 APPEND PR 
\end{verbatim}
Puts data set number 5 into an HBOOK histogram.  
\begin{verbatim}
     DEFINE HIST SETS APPEND 
\end{verbatim}
Puts all data sets into HBOOK histograms.  
\paragraph{DEFNAME}
The  name  you  wish  to assign to a histogram.  HBOOK histograms may
only have a new name when they are zeroed.  
\paragraph{DEFAULT      }
Resets all options to their default values.  
\paragraph{HISTOGRAMS   }
Selects only histograms (not mesh) to modify.  
\paragraph{MESH         }
Selects only MESH histograms to modify.  
\paragraph{NTUPL        }
Selects  only  NTUPLES  to modify.  NTUPLE=OFF selects all histograms
but NTUPLES.  
\paragraph{PR...        }
These  options  modify  the  printed output for LIST HISTOGRAM.  They
either take a numeric parameter or ON/OFF.  If ON or OFF are  omitted
ON is assumed.  
\begin{verbatim}
     1.  MIN=n - Selects the minimum Y for print plot.  
     2.  MAX=n - Selects the maximum Y for print plot.  
     3.  BIGBI=n  -  Selects  the  number  of columns per channel for
         histogram print.  
     4.  AUTOMATIC 
     5.  BLACK - Fills the area under the hist with "X" 
     6.  STAR - Prints an asterisk for the histogram 
     7.  ERROR - Puts on error bars 
     8.  CONTENTS - Prints the histogram contents 
     9.  INTEGRAL - Prints the integrated contents 
    10.  HISTOGRAM - Prints the histogram shape 
    11.  LINEAR - Prints on a linear scale in Y 
    12.  LOGARITHIC - Prints on a log scale in Y 
    13.  LOW - Prints the lower limits of the bins 
    14.  MAXIMUM=n - Sets the maximum on the contents 
    15.  MINIMUM=n - Sets the minimum on the contents 
    16.  PAGE=n - Sets the number of lines/page (all histograms) 
    17.  ROTATE - Rotates the hist 
    18.  STATISTICS - Includes hist statistics in output 
    19.  SQUEEZE  -  Squeezes  pages  together  to  save paper.  (all
         histograms) 
    20.  1PAGE - hist takes only 1 page 
    21.  2PAGE - Spills hist over 2 pages 
    22.  TABLE - Prints 2-d hists as a table 
    23.  SCATTER - Prints 2-d hists as a scatter plot.  
\end{verbatim}
\paragraph{RANGE       }
Specifies  the  maximum range for a new histogram.  (Default:RANGE=0)
See:  HBOOK documentation.  

\begin{verbatim}
                               example
     TD:DEFINE HIST ID=5 RANGE=500 FROM 10 TO 10 BY 0.1 
\end{verbatim}
\paragraph{STATISTICS   }
Turns on or off statistics accumulation.  This applies to HBOOK.  
\paragraph{X$|$Y...       }
If  you  specify  X$|$Y  FROM,TO  and  BY or N= then a new histogram is
created, or an existing histogram is rescaled.  When rescaled  it  is
recreated,  and  the  existing  scales,  and  title  are kept, if not
respecified.  If you specify this  for  both  X  and  Y  then  a  3-d
histrogram  is  created.   If  unspecified  FROM/TO=0 and N=100.  The
option PROFILE creates a profile histogram using  the  specified  X,Y
limits.  

\begin{verbatim}
                              examples
     TD:DEFINE HIST ID=5 X FROM 10 TO 20 BY 0.5 DEFNAME='test' 
     TD:DEFINE HIST ID=5 FROM 10 TO 20 BY 0.5 DEFNAME='test' 
     TD:DEFINE HIST ID=5 FROM 10 TO 20 N=20 DEFNAME='test' 
\end{verbatim}
all create a histogram with 20 bins from 10 to 20 in steps 0f 0.5 
\begin{verbatim}
     TD:DEFINE  HIST  NEXT  X  FROM 10 to 20 BY .5 Y FROM 1 TO 5 BY 1
\end{verbatim}
DEFNAME='3-d test' 
Creates a 3-d histogram of size 5 by 20.  
\begin{verbatim}
     TD:DEFINE HIST NEXT X N=50 
\end{verbatim}
Creates a histogram with 50 bins.  
\begin{verbatim}
     TD:DEFINE HIST NEXT X 
\end{verbatim}
Creates a histogram with 100 bins.  
\begin{verbatim}
     TD:DEFINE HIST IDENT=2 X FROM 1 TO 100 Y FROM 0 TO 1.5 PROFILE 
\end{verbatim}
Creates  a profile histogram with 100 bins from 1 to 100 for Y values
from 0 to 1.5.  
\paragraph{ZERO        }
Sets  the contents of the specified histogram to zero.  If SECTION is
specified then the histogram in  a  global  section  is  zeroed.   If
DEFNAME  is  specified  then a new name is assigned to the histogram.
Note:  To zero a histogram in a global section you must be the  owner
of  the  section,  or  before  running  the  program that creates the
section you must:  
\begin{verbatim}
     $ SET PROT=G:RWE/DEFAULT 
\end{verbatim}
\subsubsection{KEY}
\begin{verbatim}
     DEFINE KEY keyname "key definition" options 
\end{verbatim}

This  defines a key on your keypad so you can abbreviate a command to a
single keystroke.  The syntax is similar to the VMS DEFINE/KEY command.
Since  the  syntax is VMS the key definition must be enclosed in quotes
('') and not in apostrophes (').  For more information See:DEFINE/KEY in
the VAX/VMS DCL Dictionary.  
\paragraph{Keyname}
The PF2 key is already defined as the HELP key.  
The PF1 key is the gold key.  

\begin{verbatim}
                            Vt-100 keypad
     PF1,PF2,PF3,PF4 
     PERIOD,COMMA,MINUS,ENTER 
     KP0,KP1...,KP9 
                             Vt-2xx keys
          Upper row keys:  
     F6,...F20 
          Named keys:  
     E1    E2    E3    E4    E5    E6 
     Find  Ins.  Rem.  Sel   Prev. Nex.  
     HELP,DO 
\end{verbatim}


\begin{verbatim}
                                NOTE
\end{verbatim}

\begin{verbatim}
     F1-F5  can  not  be  redefined.  F6-F14 are used for keypad
     editing and may not be redefined unless Ctrl_V is  pressed.
     See HELP TERMINALS KEYS.  
\end{verbatim}


\paragraph{Options}
In general the only options you will frequently use are:  /TERMINATE,
/NOECHO, and possibly /IF\_STATE.  
\begin{verbatim}
     1.  /[NO]TERMINATE  -  When  the key is pressed the line is also
         terminated.  
         (Default:/NOTERMINATE) 
     2.  /[NO]ECHO - The command is [not] echoed.  
         (Default:/ECHO) 
     3.  /[NO]IF_STATE=gold - The definition is for this key prefaced
         by gold key (PF1).  
         (Default:/NOIF_STATE) 
     4.  /[NO]LOCK_STATE - The state is defined till NOLOCK.  
         (Default:NOLOCK_STATE) 
     5.  /[NO]SET_STATE=name 
         (Default:/NOSET_STATE) 
\end{verbatim}
\paragraph{Example}
The PF1 and PF2 keys are already defined:  
\begin{verbatim}
     TD:DEFINE KEY PF1 " " /SET_STATE=gold 
     TD:DEFINE KEY PF2 "HELP "/TERMINATE 
\end{verbatim}

\begin{verbatim}
                              examples
     TD:DEFINE KEY PF3 "SET HIST PREV;  NEW;HIST" /TERMINATE 
     TD:DEFINE KEY PF4 "SET HIST NEXT;  NEW;HIST" /TERMINATE 
\end{verbatim}
This  sets  up  2  keys  so  that  you  can  step  through  a list of
histograms.  

\begin{verbatim}
     TD:DEFINE KEY MINUS "SET HIST ID=" 
\end{verbatim}
This  sets  up  key  to  give the command.  When the minus key on the
keypad is pressed the command is entered, and you  enter  the  number
followed by the ``RETURN'' key.  

\begin{verbatim}
     TD:DEFINE KEY KP7 "HISTOGRAM"/TERMINATE 
\end{verbatim}
This sets up the 7 key on the keypad to histogram.  

\begin{verbatim}
     TD:DEFINE KEY COMMA "SET FILE INPUT 'myfile'"/term 
\end{verbatim}
This sets up the comma (,) key on the keypad to get commands from the
file MYFILE.TOP.  

\begin{verbatim}
     TD:DEFINE KEY PF4 "SET THREE ON"/term 
     TD:DEFINE KEY PF4 "SET THREE OFF"/term/ifstate=gold 
\end{verbatim}
This  sets up PF4 as a key to turn THREE ON or OFF.  To turn on three
press PF4.  To turn THREE OFF press PF1 PF4.  
\subsubsection{STRING}
You may define strings for ``Lexicals'' 
\begin{verbatim}
     TD:DEFINE STRING name 'string' 
\end{verbatim}
A lexical of the form S\_name is give the string 

\begin{verbatim}
                                Example
     TD:DEFINE STRING NAME 'George Washington 
     TD:TITLE TOP 'The first president was ' S_NAME 
     TD:TYPE 'NAME="' S_NAME '"' 
\end{verbatim}

You  may  show  all  defined strings with the SHOW STRINGS command.  An
individual string may be examined by the TYPE command.  
\begin{verbatim}
     TYPE S_name 
\end{verbatim}
\paragraph{Variable}
This defines a local variable of the form 
\begin{verbatim}
     TD_S_name 
\end{verbatim}
which  contains  the  string  that  you have defined.  You may define
variables before entering TOPDRAWER so that they  are  available  for
use.  You do this by:  
\begin{verbatim}
     $ TD_S_NAME := "String to use" 
\end{verbatim}
\subsubsection{VALUE}
You may define values for ``Lexicals'' 
\begin{verbatim}
     DEFINE VALUE name=value [LOG=[ON|OFF]] [FAST[=ON|OFF]] 
\end{verbatim}
A  lexical  of  the  form V\_name is give the value n If you specify LOG
then the value is typed on your terminal.  When FAST=ON  the  value  is
defined  as  a  local symbol in non readable form.  This will speeds up
symbol definitions, but it has little effect on the symbol usage.   You
should  use  FAST  when  you  are  repeatedly  redefining symbols.  Non
readable symbols do not make sense for a SHOW VALUE  command.   If  you
need speed, single letter names are the most efficient.  
(Default:FAST=OFF) 

\begin{verbatim}
                                Example
     TD:DEFINE VALUE NORM=125.5 
     TD:Y=Y TIMES V_NORM 
\end{verbatim}
Multiplies all data by 125.5.  
\begin{verbatim}
     TD:DEFINE VALUE NORM=<100/V_SUM> 
     TD:Y=Y TIMES V_NORM 
\end{verbatim}
Normalizes the data to SUM=100.  

You   may   show  all  defined  values  with  the  SHOW VALUE  command.
Individual  values  may  be  shown   by   the   SHOW LEXICAL   command.
See:TOPDRAWER DATA 
\paragraph{Variable}
This defines a local variable of the form 
\begin{verbatim}
     TD_V_name 
\end{verbatim}
which  contains  the  value  that  you  have defined.  You may define
variables before entering TOPDRAWER so that they  are  available  for
use.  You do this by:  
\begin{verbatim}
     $ TD_V_NAME := "number" 
                               example
     $ TD_V_A := "25.8E5" 
     $ TOPDRAWER 
     $ TITLE TOP 'A=',T_A 
\end{verbatim}
\subsection{DIAMOND}
Draws a diamond.  

\begin{verbatim}
     DIAMOND [AT|FROM|TO]{x,y[,z]|CURSOR} [DATA] [SIZE=dx[,dy]]
               [DSIZE=dx[,dy[,dz]]] [ROTATE=n] [FILL[=ON|OFF]] [HIDE[=ON|
               OFF]] 
          [SOLID|DOTS|DASHES|DAASHES|DOTDASH|PATTERNED|FUNNY|SPACE] 
          [INTENSITY|WIDTH=n] 
          [WHITE|RED|GREEN|BLUE|YELLOW|MAGENTA|CYAN] 
\end{verbatim}
See also:Command BOX 
\subsubsection{AT}
x,y,z  specify  the center of the diamond.  If z is specified then they
are assumed to be data units.  
\subsubsection{FROM-TO}
You  may  specify  the diagonally opposite corners of box enclosing the
diamond.  
\subsubsection{FILL}
Fills in the diamond using the current fill pattern.  See:SET FILL 
\subsubsection{CURSOR}
The  cross  hair  cursor  is used to specify the limits of the diamond.
Simply move the cross hairs so they define the lower-left limits of the
diamond  and  press  the  space  bar.   Next  move  the  cursor  to the
upper-right limits of the diamond an press space again.  If you press a
key  other than space, the cursor information is not used.  Alternately
you may specify the upper-left then lower-right extension, or any other
combination that defines the diamond.  
\subsubsection{DATA}
Specifies  X,Y  are in data coordinate frame.  Normally they are in the
TEXT coordinate frame.  This option is not necessary if z or DSIZE  are
specified.  
\subsubsection{DSIZE}
DSIZE=dx[,dy] Specifies the size in data units.  If the size is in data
units the x,y are assumed to be data units also.  DY is the height.  If
omitted DY=DX.  
\subsubsection{ROTATE}
Specifies  a  rotation  angle  in degrees.  Rotation is right handed or
counterclockwise.  
\subsubsection{SIZE}
SIZE=dx[,dy]  specifies the size in inches DX is the width.  If omitted
the width is the default  See:Command SET  DIAMOND  SIZE.   DY  is  the
height.  If omitted DY=DX.  
\subsubsection{SOLID...}
Sets the texture of the line.  See:Command SET TEXTURE.  
\subsubsection{INTENSITY}
Sets the intensity (1-5) 
\subsubsection{WHITE...}
Sets the color of the line.  
\subsection{DIVIDE}
DIVIDE [Y$|$Z] [FROM$|$TO] [EWEIGHT=n] [WEEIGHT=n] \{n1$|$``name1''\} [BY] \{n2$|$
\begin{verbatim}
          "name2"|FIT} 
     [AVERAGE|EFFICIENCY[=ON|OFF]] [APPEND[=ON|OFF]] [NAME="name"]
          [CHECK[=ON|OFF]] [ERROR[=ON|OFF]] [POINTS|
          COLUMNS=[FROM] n1 [TO] [n2]] [LINES|ROWS=[FROM] n1 [TO] [n2]]
          [LIMITED [VLOG[=ON|OFF]] [[FROM]|TO [[X=]nx,[[Y=]ny[,[Z=]nz]]]
          [RECURSOR] [CURSOR] ] [LOG[=ON|OFF]] 
\end{verbatim}
This divides the Y values in data set n1 by data set n2.  The result is a
modified set n1.  You may not divide normal 2-d data by mesh data.  
\subsubsection{APPEND}
If  APPEND is specified, then a new data set is created, containing the
result and n1 is unchanged.  
\subsubsection{CHECK}
CHECK=OFF  turns  off  data set checking.  When CHECK=ON both data sets
must have identical X (and Y if mesh)  values,  but  data  set  n2  may
contain  more  points than n1.  If (DX/DY) is non zero then the both DX
and X must be identical within 1\% of DX.  If you have  data  sets  with
non  identical  values  of X you may create a set with identical values
using the BIN command.  If the X values of 2 data sets  are  not  quite
identical.   You  may  force TOPDRAWER to divide them by setting DX for
the data 100 times greater than the  difference  in  X  or  by  setting
CHECK=OFF 
\subsubsection{EFFICIENCY}
If EFFICIENCY is specified, the data is treated as an efficiency.  That
is set n1 contains the number  of  successes,  while  n2  contains  the
number  of  trials.   The  result  has  error  bars (DY) which follow a
binomial distribution.  See:Command ADD,MULTIPLY.  
\subsubsection{ERROR}
ERROR=OFF excludes the errors from the FIT in the computation.  
(Default:ERROR=ON) 
\subsubsection{FIT}
Specifies  that  the  data  set  is  to  be  divided  by  the last FIT.
See:Command FIT.  
\subsubsection{LIMITED}
You may specify the limits over which the histograms are to be divided.
If you specify the Y or Z limits, then all data  that  contains  values
inside these limits will be divided.  For example if you 
\begin{verbatim}
     DIVIDE 1 by 2 LIMITED FROM Y=10 to Y=11 
\end{verbatim}
But data set 1 contains the followind data:  
1,0;  2,10;  3,0;  4,11;  5,0 
Points  2  to 4 inclusive will be divided, and only points 1 and 5 will
be omitted.  
\subsubsection{LOG}
If  LOG  is  specified  then the result of the command is typed on your
terminal.  
\subsubsection{WEIGHT-EWEIGHT}
WEIGHT=n  specifies  a  weighting factor to multiply the data by before
adding or subtracting it.  EWEIGHT=n specifies  the  error  on  WEIGHT.
This option must precede the data set number.  
\subsubsection{Notes}
If  EFFICIENCY  is not specified then the new value is Y3=Y1/Y2.  If Y2
is zero then Y3 is set to 0.0.  The errors (DY) are treated as  if  the
data sets are uncorrelated, if BINOMIAL is not specified.  
\subsection{DX}
See:Command X 
\subsection{DY}
See:Command X 
\subsection{DZ}
See:Command X 
\subsection{ELLIPSE}
Draws a circle or ellipse.  

\{CIRCLE$|$ELLIPSE\} [AT$|$FROM$|$TO]\{x,y[,z]$|$CURSOR\} [DATA] [SIZE=dx[,dy]]
\begin{verbatim}
          [DSIZE=dx[,dy[,dz]]] [ROTATE=n] [SQUARE|SYMMETRIC[=ON|OFF]]
          [FILL[=ON|OFF]] [HIDE[=ON|OFF]] 
     [ANGLE [FROM] A1 [TO] [A2]] 
     [INTENSITY|WIDTH=n] 
     [WHITE|RED|GREEN|BLUE|YELLOW|MAGENTA|CYAN] 
     [SOLID|DOTS|DASHES|DAASHES|DOTDASH|PATTERNED|FUNNY|SPACE] 
\end{verbatim}
See also:Command BOX 
\subsubsection{AT}
x,y,z  specify  the center of the ellipse.  If z is specified then they
are assumed to be data units.  If x,y,z are specified without FROM, TO,
or AT then AT is assumed.  
\subsubsection{FROM-TO}
Specifies the diagonally opposite corners of box enclosing the ellipse. 
\subsubsection{FILL}
Fills in the ellipse using the current fill pattern.  See:SET FILL 
\subsubsection{CURSOR}
The  cross  hair  cursor  is used to specify the limits of the ellipse.
Simply move the cross hairs so they define the lower-left limits of the
ellipse  and  press  the  space  bar.   Next  move  the  cursor  to the
upper-right limits of the ellipse an press space again.  If you press a
key  other than space, the cursor information is not used.  Alternately
you may specify the upper-left then lower-right extension, or any other
combination that defines the ellipse.  
\subsubsection{DSIZE}
specifies the size in data units.  If the size is in data units the x,y
are assumed to be data units also.  You should note that  if  DSIZE  is
specified, you will usually get an ellipse.  
\subsubsection{ROTATE}
Specifies  a  rotation  angle  in degrees.  Rotation is right handed or
counterclockwise.  
\subsubsection{SIZE}
specifies  the size in inches DX is the width.  If omitted the width is
the  default  See:Command SET ELLIPSE SIZE.   DY  is  the  height.   If
omitted the DY=DX.  
\subsubsection{SQUARE}
The  option SQUARE or SYMMETRIC produces a circle.  The SIZE is treated
as the X,Y of a diameter from the center of a circle, and FROM, TO  are
treated as the ends of diameter of the circle.  
\subsubsection{DATA}
Specifies  X,Y  are in data coordinate frame.  Normally they are in the
TEXT coordinate frame.  This option is not necessary if z or DSIZE  are
specified.  
\subsubsection{ANGLE}
The  ellipse  is  drawn  from the angle A1 to A2.  A1,A2 must be in the
range -360 to 360 degrees.  
\subsubsection{SOLID...}
Sets the texture of the line.  See:Command SET TEXTURE.  
\subsubsection{INTENSITY}
Sets the intensity (1-5) 
\subsubsection{WHITE...}
Sets the color of the line.  
\subsection{ELSE}
\begin{verbatim}
     TD:ELSE 
\end{verbatim}
Starts  or  stops  execution  of  TOPDRAWER  commands  depending  on  the
preceding IF.  See:Command IF.  
\subsection{ENDFILE}
\begin{verbatim}
     TD:ENDFILE "Comment to type" 
\end{verbatim}
Ends  the current file.  If this command is given from the terminal or is
included in the command file used to run TOPDRAWER then it is the same as
EXIT.  
\subsection{ENDIF}
\begin{verbatim}
     TD:ENDIF 
\end{verbatim}
Ends the current if block.  See:Command IF 
\subsection{ENDREPEAT}
\begin{verbatim}
     TD:ENDREPEAT 
\end{verbatim}
Ends the current repeat block.  See:Command REPEAT 
\subsection{CLEAR}
\{CLEAR$|$ERASE\} [FROM] [x,y[,z]$|$CURSOR] [DATA] TO[x,y[,z]$|$CURSOR] [DATA] 
This  clears  the  screen  without  resetting  the buffers or the default
values.  This is useful if you wish to view several different things with
the  same  parameters.   If  you  specify  a  pair  of  X,Y values only a
rectangle inside those values is cleared.  This only works  on  terminals
that  support  selective  erase.  Each X,Y[,Z] may be specified in either
Text or Data coordinates.  Text coordinates are assumed if  a  Z  is  not
specified.  



\begin{verbatim}
                                 WARNING
\end{verbatim}

\begin{verbatim}
     After  CLEAR  the  current  scale  settings  are  retained, and
     subsequent plots will not automatically have  axes  or  scales.
     In general you should use command NEW.  
\end{verbatim}


\subsection{FFT}
FFT [SETS=[FROM] n1 [TO] [n2]] [SELeCT=``name''] [CHECK[=ON$|$OFF]] [LOG[=ON$|$
\begin{verbatim}
          OFF]] [INVERT[=ON|OFF]] [APPEND[=ON|OFF]] [NAME="name"]
          [POLAR[=ON|OFF]] [MONITOR[=ON|OFF]] 
\end{verbatim}
This command produces a complex fast fourier transformation of the
selected data sets.  For 2-d data the real part of the coefficients is
stored in Y, and the imaginary part in Z if it exists.  If not then DY
contains the imaginary part.  X contains the frequency value.  No error
analysis is performed by this operation.  For mesh data the real part is
stored in Z and the imaginary part in DZ.  
\begin{verbatim}
     1.  APPEND=ON Creates a new data set containing the result.  
     2.  CHECK=OFF Turns off the checking to see if data is a histogram. 
     3.  INVERT=ON inverts the forier series contained in Y,Z/DY.  
     4.  SETS selects the data sets to to transform.  
     5.  LOG=ON Logs the operation on your terminal.  
     6.  MONITOR If selected it histograms the original data set and the
         result.  
         See:  SET MODE MONITOR 
     7.  POLAR=ON selects polar coordinate output where Y is the
         magnitude and Z/DY is the angle in degrees.  
\end{verbatim}

The selected data must be a histogram, that is the X values must be
equally spaced and strictly monotonic, and there must be at least 2
points in each data set.  
\subsubsection{Restrictions}
The fast fourier transform is done by a modified version of the CERN
library routine CFT.  The data is assumed to be periodic with an EVEN
number of points.  For example, suppots you have NPT points, then we
assume y[I]=y[I+NPT] Also, data sets are assumed to have lengths (say
NPT) where NPT's largest prime factor is not more than 23.  The speed
is enhanced by using small prime factors.  Thus, the routine fails for
vector lengths 
NPT=  29  31  37  41  43  47  53  61  67  71  73  79  83  89  97
\begin{verbatim}
     101 103 107 109 113 127 131 137 139 149 151 157 163 167 173
     179 181 191 193 197 199 211 223 227 229 233 239 241 251 257
     263 269 271 277 281 283 293 307 311 313 317 331 337 347 349
     353 359 367 373 379 383 389 397 401 409 419 421 431 433 439
     443 449 457 461 463 467 479 487 491 499 503 509 521 523 541
     547 557 563 569 571 577 587 593 599 601 607 613 617 619 631
     641 643 647 653 659 661 673 677 683 691 701 709 719 727 733
     739 743 751 757 761 769 773 787 797 809 811 821 823 827 829
     839 853 857 859 863 877 881 883 887 907 911 919 929 937 941
     947 953 967 971 977 983 991 997 etc.
\end{verbatim}
\subsubsection{Interpretation}
Assuming you have periodic data, the spectrum will exhibit 2 peaks.
The first one will be at f the frequency of the structure.  The second
one will be at a frequency of 1/(chan width)-f.  The peak will appear
both in the real and imaginary parts of the spectrum.  
\subsection{FILL}
FILL BY=\{nset$|$``name''\} [LEVEL=n] [ADD=[n$|$Z]] 
\begin{verbatim}
     [LIMITED [VLOG[=ON|OFF]] [[FROM]|TO [[X=]nx,[[Y=]ny[,[Z=]nz]]]
          [RECURSOR] [CURSOR] ] 
     [POINTS|COLUMNS=[FROM] n1 [TO] [n2]] 
     [LINES|ROWS=[FROM] n1 [TO] [n2]] 
     [SETS=[FROM] n1 [TO] [n2]] 
     [SELeCT="name"] 
\end{verbatim}
This  takes the data set nset and assumes it is a closed curve.  It fills
the curve by adding 1 to each Z value of the specified  data  sets  which
lie inside the curve.  For mesh data Z must be the dependent variable.  

If  the  filled  data  set  has  finite  DX then it is incremented by the
fraction of the curve that lies within the DX of X.  

If you specify LEVEL=n and DY is finite then n steps are used to find the
part of the curve that lies within each channel +-DY.  
\subsubsection{Options}
\begin{verbatim}
     1.  ADD - Selects the value to fill by.  If ADD=Z then the Z value
         of the fill source is used in the fill.  
     2.  BY=n  - Selects the data set to use in filling other data set.
         You may specify the set number or the name enclosed in quotes. 
     3.  LEVEL - Selects the number of passes to test.  0=no fractions. 
     4.  LIMITED - Selects data by range of X,Y,Z 
     5.  LINES - Selects the range of lines or rows of a mesh to fill. 
     6.  LOG - Tells you how many points have been filled.  
     7.  POINTS - Selects the range of points or columns to fill.  
     8.  SETS  -  Selects  the  data  sets to fill.  The fill source is
         never filled.  
     9.  SELECT - Selects the data sets to fill by name.  
\end{verbatim}
\subsection{FIT}
FIT [AGAIN[=ON$|$OFF]] 
\begin{verbatim}
                                 Coordinate to fit
          [Y|Z] 
                                    Type of fit
          [POLYNOMIAL| LEGENDRE| SINE| COSINE| EXPONENTIAL| GAUSSIAN|
          INNVERSE| USER=n| EQUATION "term1;term2;..."] 
     [SCALE] [ORDER|TERMS=n] [OFFSET=n] 
     [NONLINEAR[=ON|OFF]] 
     [COEFFICIENTS|ECOEFFICIENTS|CMINIMUM|CMAXIMUM [INDEX=n] N1,N2,N3...]
          [CONSTRAIN[=ON|OFF]] 
     [INCLUDE|EXCLUDE [NONE|EVEN|ODD] [n1,n2...] ] 
                                  Data generation
     [CURVE [APPEND[=ON|OFF]] [[X|Y] [FROM xmin] [TO xmax] [BY dx]
          [N=steps]] [BINS|VALUES]] [NAME="name"] 
                                  Data selection
     [LIMITED [VLOG[=ON|OFF]] [[FROM]|TO [[X=]nx,[[Y=]ny[,[Z=]nz]]]
          [RECURSOR] [CURSOR] ] 
     [POINTS=[FROM] n1 [TO] [n2]] 
     [SETS=[FROM] n1 [TO] [n2]] 
     [SELeCT="name"] 
     [MESH[=ON|OFF]] 
                                 Output selection
     [LOG[=ON|OFF]] [FULL[=ON|OFF]] [MONITOR[=ON|OFF]] 
                               NONLINEAR=ON options
     [ECHISQ=n] [MINUIT] [GRAD[=ON|OFF]] [PRINTLEVEL=n] [REPEAT=n]
          [TOLERANCE=n] 
\end{verbatim}

Fits  data  and  optionally  produces  a  curve  containing the fit.  The
coefficients are typed on the terminal and placed as comments in the  log
file.   Data  points  with zero or negative values of DY are omitted from
the fit.  If DY does not exist or is 0.0 for all specified data  then  DY
is assumed to be 1.0.  If all coefficients are constrained, no fitting is
done, but you may generate the curve for  those  coefficients.   You  may
subtract the FIT from a data set with the SUBTRACT command.  
\subsubsection{Options     }
\begin{verbatim}
     1.  AGAIN - Redo the last fit.  
     2.  Y|Z selects which coordinate is to be fit.  
                                   Type of fit
     3.  POLYNOMIAL,LEGENDRE,SINE,COSINE,EXPONENTIAL,GAUSSIAN,INVERSE -
         Is the function 
     4.  SCALE  -  Does  fit to LOG(X) or LOG(Y) if the scale is set to
         LOG.  
     5.  ORDER|TERMS=n - The max number of terms used in the fit.  
     6.  OFFSET=n  - An offset for the X value to set the center of the
         fit 
     7.  COEFFICIENTS - Selects constant coefficients.  
     8.  ECOEFFICIENTS|ERRORS - Selects constant coefficient errors.  
         A.  CONSTRAIN selects whether coefficient is constrained.  
         B.  INDEX selects the coefficient number.  
     9.  INCLUDE|EXCLUDE  -  Selects  coef.  to fit or exclude from the
         fit.  
                                Curve generation
    10.  CURVE - Produces a data set containing the fitted curve.  
         A.  APPEND appends the new curve to the data sets.  
         B.  NAME selects the new data set name.  
         C.  [X|Y]  from n1 to n2 by n3 selects the range and bin width
             for the curve.  
                                 Data selection
    11.  LIMITED - Selects data by range of X,Y,Z 
    12.  NAME - Selects the name of the appended data set 
    13.  POINTS - Selects the range of points to use in the fit 
    14.  SETS - Selects the data sets to use in the fit 
    15.  MESH - Selects mesh data (normally regular data is used) 
                                Nonlinear options
    16.  MINUIT - Does nonlinear fits using interactive MINUIT.  
    17.  NONLINEAR   -   Selects  nonlinear  or  linear  fitting.   For
         NONLINEAR=on Uses linear fitting techniques  if  on.   If  off
         uses MINUIT.  
         A.  REPEAT selects the maximum number of tries.  
         B.  PRINTELEVEL select the amount of log output.  
         C.  GRADIENT  selects  whether  the calculated derivitives are
             use in the fitting.  
         D.  TOLERENCE selects the tolerence for good fit 
         E.  ECHISQ  selects  the  error in chisq for finding errors in
             coef.  
             (Default:1.0) 
    18.  Output options 
    19.  LOG  -  Enables/Disable the terminal listing of the results of
         the fit.  
    20.  FULL - Enables output of error matrix as part of LOG.  
    21.  MONITOR - Plots the original data and the fit.  
\end{verbatim}
\subsubsection{AGAIN       }
Redo  the  last fit.  This option must be the first option on the line.
After the fit command.  When you redo  the  fit  you  may  specify  new
options  to  modify  the  fit.   If you want to set fit options without
doing any fits use the command 
\begin{verbatim}
     SET FIT [options] 
\end{verbatim}
This sets fitting options to be performed by the next fit command.  The
options on SET FIT are identical to the options on FIT.  You may use  a
series of SET FIT commands to setup the next fit.  
\subsubsection{EQUATION    }
\begin{verbatim}
     LINEAR  This  allows  you  to  enter  equations for each term of a
\end{verbatim}
series to fit.  Each term is separated by a semicolon '';''.  The zeroeth
order  term  is  always  assumed  to  be  a constant so only the first,
second...  order terms are  entered.   For  example  you  wish  to  fit
Y=C0+C1*X**1.5+C2*EXP(X) 
\begin{verbatim}
     TD:FIT EQUATION "XV**1.5;EXP(XV)" 
\end{verbatim}

If you have previously fit using an equation:  
\begin{verbatim}
     TD:FIT EQUATION "" 
\end{verbatim}
Uses  the  same  equation  for a new fit.  All other fit options may be
used with EQUATION, but any options that modify the number of terms  to
fit must appear after the equation definition.  The X offset is ignored
by the equation.  

\begin{verbatim}
     NONLINEAR If you wish to perform a fit to an arbitrary function of
\end{verbatim}
X, then you must enter the coefficients in the the equation as  COE[n],
and  you  must  specify  the NONLINEAR option.  Remember that each term
automatically has a coefficient, so you should use coefficients in  the
equation  that have not been already used.  For example you wish to fit
the equation:  
\begin{verbatim}
     C0+C1*/(1+C2*(X-C3)**2) 
\end{verbatim}
Recalling that C0, and C1 will already be used you enter:  
\begin{verbatim}
     TD:FIT EQUATION "1/(1+COE[2]*(XV-COE[3])**2)" NONLINEAR TERMS=3 
\end{verbatim}
You  may  also  need to specify initial values for the coefficients and
inital errors for the first step.  
\begin{verbatim}
    CONSTRAIN=OFF COEF n0,n1,n2,n3 ERROR n0,n1,n2,n3 
\end{verbatim}

Note:   The constant term COE[0] is fit by default.  Likewise the C1 is
automatically included as the coefficient of the first  term.   To  get
rid of the constant term you use the option:  EXCLUDE 0 
\subsubsection{POLYNOMIAL  }
Does a fit to:  
\begin{verbatim}
     SUM( an*(X-Offset)**n );  n=0,TERMS-1 
\end{verbatim}
This is the default if no function is specified.  
\subsubsection{INVERSE     }
Does a fit to an inverse poser series:  
\begin{verbatim}
     Y=SUM( an*(X-Offset)**(-n) );  n=0,TERMS-1 
\end{verbatim}
\subsubsection{SINE        }
Does a fit to:  
\begin{verbatim}
     Y=a0+SUM( an*SINE((n)*(THETA-Offset)) );  n=1,TERMS-1 
\end{verbatim}
\subsubsection{COSINE      }
Does a fit to:  
\begin{verbatim}
     Y=a0+SUM( an*COS((n*(THETA-Offset)) );  n=1,TERMS-1 
\end{verbatim}

The angle THETA is normally in degrees.  If you wish to modify this see
SET POLAR.  
\subsubsection{LEGENDRE    }
Does a fit to:  
\begin{verbatim}
     Y=SUM  of  a legendre polynomial in (X-Offset).  ,br;If you select
\end{verbatim}
both LEGENDRE and COSINE or the data is POLAR then the Legendre  series
is a function of COS((THETA-Offset)).  
\subsubsection{EXPONENTIAL }
Does a fit to:  
\begin{verbatim}
     Y=EXP( a0+a1*(x-offset)) + Polynomial when NONLINEAR=ON (Default) 
          or ...  
     LN(Y)=series(x-offset) 
\end{verbatim}
for data points with Y$>$0.0.  
Where series is either POLY,SINE,COSINE or LEGENDRE.  
If  you  wish  to  fit data which has negative or zero data points, you
must either SMOOTH the data to get rid of them, or use the BIN  command
to  rebin  the  data into large bins.  If all y data is less than zero,
you may make it positive with the command:  
\begin{verbatim}
     TD:Y=Y TIMES -1 
\end{verbatim}

When  a  NONLINEAR  fit  is  used,  and  NTERMS=n  is  greater than 2 a
polynomial background of n-2 terms is added to the exponential.  
\subsubsection{GAUSSIAN    }
Does a fit to:  
\begin{verbatim}
     Y=a0*EXP(-0.5*((X-a1)/a2)**2)[+a3+a4*X+...]    for    NONLINEAR=ON
\end{verbatim}
(Default) or 
\begin{verbatim}
     LN(Y)=LN(a0)+-0.5*((X-a1)/a2)**2 for NONLINEAR=OFF 
     for data points with Y,DY not 0.0.  
\end{verbatim}

Since  only  data  with  both  Y,DY  nonzero are fit, there may be some
inaccuracy for histogram data with zero in a large number of bins,  and
small  numbers in a few bins.  Such data should be smoothed or rebinned
to remove the bins with DY=0.  

For  linear  fits if you specify the a0 or a1 coefficient you must also
specify a3.  If you wish to fit data which has negative  or  zero  data
points,  you must either SMOOTH the data to get rid of them, or use the
BIN command to rebin the data into large bins.  If all y data  is  less
than zero, you may make it positive with the command:  
\begin{verbatim}
     TD:Y=Y * -1 
\end{verbatim}

When  a  NONLINEAR  fit  is  used,  and  NTERMS=n  is  greater than 3 a
polynomial background of n-3  terms  is  added  to  the  gaussian.   If
NTERMS=4,5,6  a  constant, straignht line, or parabola are used for the
background.  Larger values are possible, but not recommended.  
For  LINEAR fits the OFFSET is used as the initial guess for the center
of the gaussian.  When done, it is approximately equal to a1.  

WARNING  If  you get the warning message that negative data points have
beem omitted in a Gaussian fit, the fit may be incorrect.  
\subsubsection{DGAUSSIAN   }
Does a fit to:  
\begin{verbatim}
     Z=a0*EXP(-0.5*(((X-a1)/a2)**2)+((Y-a3)/a4)**2))[+a5+a6*X+a7*Y] for
\end{verbatim}
NONLINEAR=ON (Default) or 
\begin{verbatim}
     for data points with Y>0.0.  
\end{verbatim}

If  you  specify  NTERMS=n  where  n  is  larger  than  5 a Dpolynomial
background is added to the  gaussian.   The  number  of  terms  in  the
background  is  n-5.  If NTERMS=6,8 a constant, or sloped plane is used
for the background.  
\subsubsection{DPOLYNOMAIL }
Does a double polynomial fit to:  
\begin{verbatim}
     Z=a0+a1*X+a2*Y+a3*X**2+a4*X*Y+a5*Y**2+z6*X**3....  
\end{verbatim}
\subsubsection{USER        }
\begin{verbatim}
     USER=n 
\end{verbatim}
Does  a  fit to a user supplied function type.  N must be between 1 and
99 with a default of 1.  If none then a polynomial is used.   For  each
coefficient of the linear fit the user supplied function is called:  
\begin{verbatim}
     Y=TDUFUN(n,NORD,X) 
\end{verbatim}
Where  n  is  the function type.  NORD is the number of the coefficient
(1-19) and X is the x value.  The user function must return the y value
of the term.  

For  example  if  you want USER=1 to select a polynomial, and USER=2 to
select  an  inverse  polynomial.   You  would   write   the   following
subroutine:  
\begin{verbatim}
     FUNCTION TDUFUN(n,NORD,X)
     IF (n.eq.1) THEN
       TDUFUN=X**NORD
     IF (n.eq.2) THEN
       TDUFUN=X**(-NORD)
     ELSE
     END
\end{verbatim}
\subsubsection{COEFFICIENTS$|$ERRORS}
Sets individual coefficients$|$errors.  If coefficients are set then they
are not included in the calculation of the  fit  unless  CONSTRAIN=OFF.
If  CONSTRAIN=OFF then you are merely settting the initial value of the
coefficient.  This is only useful for nonlinear fits.  If you  set  the
error  it  is  used as the initial step for nonlinear fits.  INDEX sets
the index of the coefficient$|$error.  The index may range from 0 to  19.
Undefined errors are assumed to be 0.0.  
\begin{verbatim}
                                EXAMPLE
     TD:COEFFICIENTS 1,2.5,.5 ERRORS .1,.5,.2 
\end{verbatim}
Sets - a0=1.0+-.1, a1=2.5+-.5, a3=0.5+-.2 
\begin{verbatim}
     TD:COEFFICIENTS INDEX=3 .5 ERRORS I=3 .1 
\end{verbatim}
Sets - a3=0.5+-.1 
\begin{verbatim}
     TD:COEFFICIENTS 1.0 I=3 2.0 I=5 .5,-.7 
\end{verbatim}
Sets - A0=1.0, A3=2.0, A5=0.5, A6=-0.7 
\subsubsection{CURVE       }
Generates  the  fitted curve, (Y,DY) as a function of X, which replaces
the current data.  DY is calculated using the error matrix.  
\begin{verbatim}
     1.  X                 Selects the X coordinate.  (Default) 
     2.  Y                 Selects  the  Y coordinate.  This is used to
         generate a mesh.  
     3.  APPEND - The new data is appended to the current data as a new
         set rather than replacing it.  
     4.  N - Sets the number of bins in the resulting curve.  
     5.  FROM - Sets the lowest value for the fitted curve.  
     6.  TO - Sets the highest value for the fitted curve.  
     7.  BY  -  Sets the width between data points in the fitted curve.
         You may not specify all four FROM,TO,BY, and N.  
     8.  BINS  - Specifies that FROM, TO refer to the lower edge of the
         first bin and the upper edge of the last bin.  
     9.  VALUES  -  Specifies  that FROM, TO refer to the center of the
         first and last bins.  (Default:VALUES) 
\end{verbatim}

\begin{verbatim}
                                     Example
         The following all produce points at X=0.5,1.5,...9.5 
              CURVE VALUES FROM .5 TO 9.5 BY 1 
              CURVE BINS FROM 0 TO 10 BY 1 
              CURVE BINS FROM 0 TO BY 1 N=10 
              CURVE BINS FROM 0 TO BY 1 N=10 Y FROM 100 to 120 by 2 
\end{verbatim}

If  you do not specify FROM, TO ...  a set of values is assumed.  First
if you are fitting a histogram with equally  spaced  values,  then  the
same  X  values  are  assumed.  Otherwise the current X tick spacing is
used.  
\subsubsection{EXCLUDE/INCLUDE}
Selects  terms  to  include  or exclude from the fit.  n1...  may range
from 0 to 19.  If you specify INCLUDE NONE then no terms are fit.  This
is  useful  if you wish to generate a curve by specifying coefficients,
without actually performing a fit.  
(Default:EXCLUDE NONE) 
\subsubsection{LIMITED     }
Fits points between the specified limits.  If limits are not specified,
the default is the current plot  limit.   See:Command SET LIMITS.   You
may  specify  up  to 5 sets of limits.  This allows you to fit the data
using up to 5 regions, and omitting the rest of the data.  
\begin{verbatim}
     1.  X - Specifies X limit 
     2.  Y - Specifies Y limit 
     3.  Z - Specifies the Z limit 
     4.  CURSOR  -  Brings up the cursor.  You move it to the X,Y value
         you wish then press the space bar to enter both X,Y  or  X  to
         enter X or Y to enter Y.  
     5.  RECURSOR - The cursor enters all regions from the current one,
         until you type "Q".  
     6.  VLOG - Draws a cross when you press the space bar, and draws a
         dotted line around the final limits.  
\end{verbatim}

\begin{verbatim}
                                example
     TD:FIT POLY TERMS=3 LIMITED FROM 1,1 to 2,5 
          or...  
     TD:FIT POLY TERMS=3 LIMITED FROM X=1 Y=1 TO X=2 Y=5 
\end{verbatim}
Fits  all  points  that  with  X  between 1 and 2 and Y between 2 and 5
inclusive.  
\begin{verbatim}
     TD:FIT POLY TERMS=3 LIMITED FROM Y=1 TO Y=5 
\end{verbatim}
Fits all data points with Y values between 1 and 5 inclusive.  
\begin{verbatim}
     TD:FIT POLY TERMS=3 LIMITED FROM CURSOR TO CURSOR 
\end{verbatim}
Fits all data points as defined by the cursor limits.  
\begin{verbatim}
     TD:FIT POLY TERMS=3 LIMITED FROM X=CURSOR TO X=CURSOR 
\end{verbatim}
Fits all data points according to X values defined by the cursor.  
TD:FIT POLY TERMS=3 LIMITED FROM 1 to 20 FROM 40 TO 80 
Fits a parabola to data in 2 regions, and skips the X values 20 to 40. 
\begin{verbatim}
     TD:FIT POLY TERMS=3 LIMITED VLOG RECURSOR 
\end{verbatim}
Fits  a  parabola  to  the data for the regions selected by the cursor.
Both X and Y values are determinded by the cursor.  Each  cursor  value
is entered by pressing the space bar.  After the last region, Press the
``Q'' key to quit.  
\begin{verbatim}
     TD:FIT POLY TERMS=3 LIMITED VLOG X=RECURSOR 
\end{verbatim}
This  is  the  same  as above, except that only X values are entered by
cursor.  
\subsubsection{NONLINEAR   }
Selects  nonlinear  fitting techniques.  If ON MINUIT is used to do the
fit.  The ECHISQ, GRADIENT, TOLERENCE, REPEAT, and PRINTLEVEL are  used
to  modify  the  MINUIT fit.  They are ignored for linear fits.  If you
specify the option MINUIT you will enter interactive MINUIT mode.   You
must  then  guide  the  fit using MINUIT commands.  In addition you may
specify the maximum and minimum value for  each  coeffiecint  with  the
CMINIMUM  and CMAXIMUM options.  You might also wish to pick an initial
error for each coefficinent, and possibly an initial value.  

The  nonlinear  errors  agree  with  the  linear  values  if you select
ECHISQ=1.0.  Note:  Though Topdrawer  and  PAW  both  use  MINUIT,  the
resulting  CHISQ disagrees by about a factor of 3.  The TOPDRAWER CHISQ
seems to agree with the correct value by subtracting the  fitted  curve
from the data and calculating SUM((Y/DY)**2)/(Points-Params) 

The  linear methods are much faster and convergence is guaranteed if no
numerical  errors  occurr.   Unfortunately  many  curves  can  not   be
expressed as a linear combination of terms.  

Most  linear  fits  may  be  performed  either  by  linear or nonlinear
methods.  This may be useful in checking the accuracy  of  the  fitting
techniques.  
(Default:OFF, ON if GAUSSIAN or EXPONENTAIL) 
\paragraph{ECHISQ      }
Selects  the  difference  in  CHISQ  for calculating the error in the
coefficients.  
(Default:1) 
\paragraph{GRADIENT    }
If ON use the calculated gradient.  
(Default:GRADIENT=ON, OFF for EQUATION) 
\paragraph{PRINTLEVEL  }
Selects  the MINUIT printlevel (-1 to 3).  Higher levels produce more
detailed reporting.  
(Default:PRINTLEVEL=-1) 
\paragraph{REPEAT      }
The number of times to repeat before giving up.  
(Default:REPEAT=25) 
\paragraph{TOLERENCE   }
Selects  the  tolerence  for  fitting.   Fit stops when the estimated
distance from the min chisq is 0.001*TOLERENCE*ECHISQ*CHISQ.  
(Default:0.1) 
\subsubsection{LOG/FULL    }
Types  the  result  of the fit on your terminal.  If you select LOG=OFF
then the results of the fit are not typed on  your  terminal.   If  you
select  FULL  then the error matrix as well as all result of the fit is
typed on your terminal.  
(Default:LOG=ON,FULL=OFF).  
\subsubsection{OFFSET      }
\begin{verbatim}
     OFFSET=n  If  non  zero the data is fitted as a function of (X-n).
\end{verbatim}
For non orthogonal functions such  as  plynomials,  you  should  always
specify  the  offset  as the center of the region that you wish to fit.
This will produce a better fit with lower errors on  the  coefficients.
The  offset  is included in the a1 coefficent for GAUSSIAN fits, so you
do not need to specify it.  
\subsubsection{MESH        }
Selects  mesh  data.   Normally  mesh data is ignored when performing a
fit.  
\subsubsection{NAME        }
The  new  set  will  usually have a name consisting of a transformation
name followed by the old set name.  If you specify a  new  name  it  is
applied to the new data set.  If the name ends in ''\%'' then the old name
is appended to the new name.  See option:APPEND 
\subsubsection{MONITOR     }
If selected it plots the original data and joins the fitted curve.  
See:  SET MODE MONITOR 
The  fitted  curve is joined using the SECONDARY attributes.  to change
them use the SET SECONDARY command.  
\subsubsection{POINTS      }
Specifies  the  range of points to use.  If n2 is omitted it is assumed
to be n1.  You may specify the point number or the options FIRST,LAST. 
\subsubsection{SCALE       }
Fits  the  LOG(X or Y)  to  the  base  e if a log scale is selected for
X or Y.  see the command:  SET SCALE.  Similarly the option EXPONENTIAL
fits LOG(Y)=series.  

\begin{verbatim}
                                example
     TD:SET SCALE Y LOG 
     TD:FIT TERMS=4 
\end{verbatim}
Fits a polynomial LOG(Y)=a0+a1*X+a2*X**2+a3*X**3 
\begin{verbatim}
                                example
     TD:SET SCALE X LOG 
     TD:FIT TERMS=3 
\end{verbatim}
Fits a polynomial Y=a0+a1*LOG(X)+a2*LOG(X)**2 
\subsubsection{SETS        }
Specifies  the  range  of  data  sets  to  use.  If n2 is omitted it is
assumed to be n1.  You may specify the data set number or  the  options
FIRST,LAST.  
\subsubsection{ORDER/TERMS }
Sets  the  number of terms in a series or the order of the series.  The
number of terms must be between 1 and 20.  The order (TERMS-1) must  be
between 0 and 19.  
(Default:TERMS=2) 
\subsubsection{Y$|$Z         }
This selects the coordinate to fit.  By default Z is fit for MESH data,
DPOLYNOMIAL, or DGAUSS.  Otherwise Y is fit.  
\subsubsection{Warning     }
A fit may fail with floating point overflow when the number of terms is
large.  Sometimes you may avoid this be rescaling the data to a smaller
range,  and  fitting  the rescaled data.  Selecting an offset for X may
also prevent errors.  

For  example you have data with X ranging from 1 to 100 and Y from 0 to
10.  When you try to fit the data  with  more  than  9  terms  you  get
floating  overflow.   This may be prevented by rescaling X to be in the
range .01 to 1.  
\begin{verbatim}
     TD:X=X TIMES 0.01 
\end{verbatim}
Then try performing the fit again.  

Even if there is no overflow, a fit may fail if a large number of terms
are selected, and the limit of the computer  accuracy  is  reached  (14
digits).   Polynomials  are  particularly  sensative  to  this  type of
problem.  In general polynomials should not be fitted to  more  than  9
terms.  
\subsubsection{Yerrors     }
The  error  bars  on the Y values are used as weighting factors for the
fit.  If they are all 0.0 they are assumed to be  1.0.   The  resulting
fit  has  error bars on the coefficients according to the assumed error
bars on the Y values.  X errors are not used in the fit.  

When  a  curve is generated, it will have error bars on the Y values if
the fitted curve had non zero error bars.  For some functions the final
errors  depend  on the initial center of the expansion.  This is set by
specifying an offset (OFFSET=n).  Generally best results  are  obtained
with an offset in the middle of the data set.  
\subsubsection{Examples    }
\begin{verbatim}
     TD:FIT POLY TERMS=10 INCLUDE EVEN 
\end{verbatim}
Fits  a  polynomial  using  the  even powers of X.  (1.0,X**2...X**8) A
curve with the resulting fit replaces the current data.  
\begin{verbatim}
     TD:FIT LEGEN TERMS=3 
\end{verbatim}
Fits  a  legendre  polynomial  of  3 terms.  (1,L1,L2) A curve with the
resulting fit replaces the current data.  
\begin{verbatim}
     TD:FIT POLY INCLUDE 2,3,5 CURVE APPEND FROM 0 to 10 by .1 
\end{verbatim}
Fits a polynomial using X**2,X**3,X**5 and appends the histogram at the
end of the current data.  The curve contains 101 data  points  spanning
the range of X from 0.0 to 10.0 
\begin{verbatim}
     TD:FIT POLY INCLUDE NONE COEF 1.0,.5,-.25 CURVE FROM 0 TO 10 BY .5 
\end{verbatim}
Creates  a parabola Y=1.0+0.5*X-0.25*X**2 from 0 to 10 in steps of 0.5.
Note:  that no fit is performed since no coefficints were included.  
\subsection{FLUSH}
\begin{verbatim}
     FLUSH  This  flushes  any  information  in  the buffers.  If you are
\end{verbatim}
plotting on an interactive device, from the terminal you can do the  same
thing by hitting ``return''.  
\subsection{FREQUENCY}
This is a synonym for BIN.  See the BIN command.  
\subsection{HISTOGRAM}
Draws  a histogram from the specified data points.  Bin edges are halfway
between the ends of adjacent error bars (DX), or halfway  between  points
if no error bars.  

HISTOGRAM [EXPAND[=ON$|$OFF]] 
\begin{verbatim}
     [POINTS|COLUMNS=[FROM] n1 [TO] [n2]] 
     [LINES|ROWS=[FROM] n1 [TO] [n2]] 
     [SETS=[FROM] n1 [TO] [n2]] 
     [SELeCT="name"] 
     [LIMITED [VLOG[=ON|OFF]] [[FROM]|TO [[X=]nx,[[Y=]ny[,[Z=]nz]]]
          [RECURSOR] [CURSOR] ] 
     [SLICES [X|Y|Z] [FROM] v1 [[TO] v2]] 
     [TITLE[=ON|OFF]] 
     [CYCLE[=ON|OFF]] 
     [INTENSITY|WIDTH=n] 
     [WHITE|RED|GREEN|BLUE|YELLOW|MAGENTA|CYAN] 
     [SOLID|DOTS|DASHES|DAASHES|DOTDASH|PATTERNED|FUNNY|SPACE] [HIDE[=ON|
          OFF]] [FILL[=ON|OFF]] 
                         3-d options 
     [X] [Y] [Z] [BLOCK|LEGO] [FRAME[=ON|OFF]] [DEPTH[=ON|OFF]] [XY|YZ|
          ZX] [CROSS|RANDOM] 
\end{verbatim}
\subsubsection{CYCLE}
When  plotting  more  than  1  slice or data set, CYCLE causes the line
texture, color, and width to vary from one plot to the next.  The order
of  cycling  is  set  by  the SET CYCLE command.  If you have specified
either texture, color or width it overrides the cycle value.  
\subsubsection{FILL}
Generates a fill pattern under the histogram, or bar chart.  
\subsubsection{FRAME}
This  option  specifies  that  the edge of the 3-d plot is to be always
drawn.  If turned off  the  edge  is  not  drawn  when  the  data=ZMIN.
Essentially this histograms only the ``important'' interior data.  
(Default:FRAME=ON) 
\subsubsection{EXPAND}
Sets  the  limits so that the selected data expands to fill the window.
This has no effect unless POINTS, SETS, SLICES  are  specified  or  the
limits on MESH data have been set by a SET LIMITS command.  

See:SET MODE 
\subsubsection{DEPTH}
DEPTH=OFF plots the mesh as a series of disconnected histograms, rather
than as a solid.  Each histogram is drawn as a flat 2 dimensional hist,
but  one  after  the  other.   Necessary  lines  to  delineate edges of
different planes are included when a plot with depth is requested.   By
omitting some lines the plot is much faster.  
(Default:DEPTH=ON) 

If  you wish to pick which axes is histogrammed you may use the options
X,Y,Z 
\subsubsection{HIDE}
Controls  whether hidden lines are drawn for a 3-d mesh, or whether one
histogram hides the next.  You must specify HIDE for both the histogram
to be hidden and the histogram that is to hide it.  The histogram which
is drawn first will hide the second one.   HIDE=OFF  draws  the  hidden
lines, while HIDE=ON omits them.  
(Default:HIDE=ON for mesh, HIDE=OFF for data) 

\begin{verbatim}
                                example
\end{verbatim}
Assume  you  have  2 data sets where you want set 1 to hide elements in
set 2.  Also you would like to fill data set 2.  
\begin{verbatim}
     TD:JOIN SET=1 HIDE 
     TD:JOINS SET=2 HIDE FILL 
\end{verbatim}

You  can  produce a drawing of a mesh with hidden lines as dotted lines
by doing the histogram twice:  
\begin{verbatim}
     TD:HISTOGRAM 
     TD:HISTOGRAM HIDE=OFF DOTTED 
\end{verbatim}
\subsubsection{POINTS$|$COLUMNS}
Specifies  the  range of points or the colums of a mesh to use.  If not
selected all points are plotted.  If n2 is omitted it is assumed to  be
n1.  You may specify the point number or the options FIRST,LAST.  
\subsubsection{SETS}
This  selects  the  data  set to plot.  The data must be broken up into
sets using the data set command, for this option to work properly.   If
n2  is  omitted  it  is assumed to be n1.  You may specify the data set
number or the options FIRST,LAST.  
\subsubsection{SOLID}
This determines the texture of the line.  See:Command SET TEXTURE 
\subsubsection{X$|$Y$|$Z$|$BLOCK}
For normal data X,Y,Z pick the dependent axis of the plot.  

Mesh  data  produces  a  ``solid''  plot.   The  limits  are set from the
beginning of the first X,Y data point to the last.   This  may  not  be
satisfactory if the bins at the edges are large.  

The options X...  modify the format of the plot.  
\begin{verbatim}
     1.  X - Draws all bin edges in the X direction.  
     2.  Y - Draws all bin edges in the Y direction.  
     3.  Z - Draws all bin edges in the Z direction.  
     4.  BLOCK  or  LEGO  -  Draws  all bin edges.  It is equivalent to
         X Y Z.  
\end{verbatim}
This produces a block or ``Lego'' plot.  
\subsubsection{XY}
XY$|$YZ$|$ZX  determine  which  faces are ``shaded''.  The shading is done by
either drawing parallel lines, a cross from corner to corner or  random
dot patterns.  The default is parallel lines.  
\subsubsection{CROSS$|$RANDOM}
CROSS$|$RANDOM produce either a diagonal cross or a set of random dots as
shading patterns on the selected faces.   DX  controls  the  number  of
dots/face.  DX is the total number of dots to pruduce.  
\subsubsection{DX}
DX$|$DY$|$DZ  are  the  separation between parallel lines used to shade the
faces.  If zero parallel lines are not drawn.  

\begin{verbatim}
                                Example
\end{verbatim}
You  wish  to  shade  face  YZ,ZX  with  lines  parallel  to the Y axis
separated by 0.1 units.  Essentially this draws lines at even multiples
of  0.1  units.  This may be used in a similar fashion to grid lines on
1-d plots.  
\begin{verbatim}
     TD:HISTOGRAM YZ ZX DX=0.1 
\end{verbatim}
Note:  you specify DX the distance along the X axis.  Since DY is 0, no
lines parallel to the X axis are drawn.  When selecting RANDOM  shading
DX specifies the number of dots/face.  
\subsubsection{SLICES}
\begin{verbatim}
          [SLICES [X|Y|Z] [FROM] v1 [[TO] v2]] [CYCLE[=ON|OFF]] 
               or...  
          [SLICE X|Y|Z=v1] [CYCLE[=ON|OFF]] 
\end{verbatim}

This  plots a histogram sliced from mesh data.  This command is illegal
for normal data.  If v2 is omitted then v2=v1.  If a range is  selected
then  all  slices  within  the  range  are  histogrammed.   IF  you set
THREE=OFF then the slices are histogrammed according to the first X,Y,Z
Mentioned.   If  X  is first then the slices are taken perpendicular to
the X axis.  You only need select X$|$Y$|$Z the first time you slice  after
a NEW FRAME command.  
\paragraph{Operation}
If THREE is ON then the histogram is sliced into a section between v1
and v2.  The data outside of the range is omitted from the plot.  

If THREE is OFF then each line/column of the histogram between v1 and
v2 is plotted as a  separate  histogram.   If  both  v2  and  v1  are
omitted, then all slices are plotted.  

If  you  wish  to  add together a series of slices and plot them as a
single histogram see:Command PROJECT.  
\paragraph{Warning}
If  you  intend  to  use  options X,Y,or Z to control the format of a
histogram they should not follow a SLICE command, or they  should  be
separated from it by some other option.  
\paragraph{Example}
\begin{verbatim}
                             2-d Example
\end{verbatim}
You  read  in  a  mesh of data x,y vs Z.  X ranges from 0.0 to 1.0 in
steps of 0.1 and Y from 100 to 200 in steps of 1.  Then you:  
\begin{verbatim}
     TD:SET THREE OFF 
\end{verbatim}
You wish to view a histogram of x vs Z for Y=140 and expand the scale
so that the final plot fills the window.  
\begin{verbatim}
     TD:HISTOGRAM SLICE Y=140 EXPAND 
\end{verbatim}

You now wish to view y vs z for values Y=.2 to .5:  
\begin{verbatim}
     TD:HISTOGRAM SLICE X FROM .2 to 0.5 
     (This plots 4 histograms) 
\end{verbatim}

Now you wish to view all slices along the X axis:  
\begin{verbatim}
     TD:HISTOGRAM SLICES Y 
     (This plots 100 histograms!) 
     TD:HISTOGRAM SLICE X FROM 0 to 0.5 Y FROM 105.5 to 110.0 
\end{verbatim}
Plots  the  mesh  data as a series of slices looking along the X axis
for a limited range of Y.  
\begin{verbatim}
     TD:HISTOGRAM SLICE Y=105 X FROM .25 TO .5 
\end{verbatim}
Plots  the mesh data as a single slice looking along the Y axis for a
limited range of X.  

\begin{verbatim}
                             3-d Example
\end{verbatim}
You enter mesh data in steps of 1 for Y=1 to 10, X=0 to 20.  
\begin{verbatim}
     TD:HIST SLICE X=10.  
\end{verbatim}
You are looking at X=10.  
\begin{verbatim}
     TD:HIST SLICE X=10.  TO 20 
\end{verbatim}
You are looking at X=10 to 20.  
\begin{verbatim}
     TD:HIST SLICE X=10.  TO 20 Y=5 TO 7 
\end{verbatim}
You are looking at X=10 to 20, Y=5 to 7.  
\subsubsection{3-d\_histograms}
A 3-d histogram can be either a plot of mesh data or a plot of ``normal''
x,y,z data.  Mesh data is normally plotted as a ``solid''.  XYZ  data  is
plotted as a histogram of X vs Z in the plane determined by Y.  For all
3-d  histograms  Z  must  be  the  dependent  variable,  and  X,Y   are
independent.   Repeated  XYZ  plots may be used to build up a series of
histograms.  

\begin{verbatim}
                                 NOTE
\end{verbatim}
Some  lines may be missing from the plot, because of the viewpoint.  If
this happens you may need to:  
\begin{verbatim}
     TD:SET THREE RDIST 1000 
\end{verbatim}
This  moves  the  viewpoint  closer  to infinity, but maintans the plot
size.  

If  the  plot  is too large you may need to adjust the screen or window
distance.  
\begin{verbatim}
     TD:SET THREE DIST 1000 SCRD 300 
\end{verbatim}
Will  reduce  the  size  of  the plot.  The size of the plot is roughly
proportional to the ratio of SCRD to DIST.  

If you specify color, intensity, or texture it is used for the top side
of the mesh, while the default is used for the bottom.  

\begin{verbatim}
                                example
     TD:SET COLOR RED 
     TD:SET TEXTURE DOTS 
     TD:HISTOGRAM SOLID WHITE 
\end{verbatim}
Produces  a  mesh  plot  with bottom in dotted red lines, while the top
side is drawen in white solid lines.  
\subsubsection{Variables}
Normally  for  2-d  plots  X  is  the independent variable and Y is the
dependent one.  

Normally  for  3-d plots X,Y are the independent variables and Z is the
dependent variable.  

The  bin  width is half way between the X or X,Y values if DX,DY are 0.
If DX,DY are specified then they determine the bin width.  
\subsubsection{Limits}
The  limits  of  the  data  for  histograms is normally set so that the
lowerlimit of the dependent axis with a linear  scale  is  0.0,  unless
some data points are less than zero.  If you wish a different value you
can:  
\begin{verbatim}
     TD:SET LIMIT YMIN=n 
          or for 3-d....  
     TD:SET LIMIT ZMIN=n 
\end{verbatim}


\begin{verbatim}
                                WARNING
\end{verbatim}

\begin{verbatim}
     If  this command is issued in between 2 plots overlaying each
     other, they will not have the same scales.  
\end{verbatim}


\subsubsection{Shading\_example}
Shading is produced by the options:  
\begin{verbatim}
     XY YZ ZX DX DY DZ CROSS RANDOM 
\end{verbatim}
If  you  wish  to  shade  the 3 faces differently, you must histogram 3
times with each pattern you wish.  For example you wish parallel  lines
on the YZ face, dots on the ZX face, and crosses on the XY face:  
\begin{verbatim}
     TD:HISTOGRAM XY CROSS 
     TD:HISTOGRAM YZ DX=0.1 DY=0.1 
     TD:HISTOGRAM ZX RANDOM DX=1000 
\end{verbatim}
\subsection{HELP}
This gives you help on TOPDRAWER commands.  This command is not journaled
unless the MODE ABORT=OFF is set.  
\begin{verbatim}
     TD:HELP topic subtopic sub-subtopic...  
\end{verbatim}

For example to get the list of set commands:  
\begin{verbatim}
     TD:HELP COMMAND SET list 
\end{verbatim}
To get help on the JOIN command.  
\begin{verbatim}
     TD:HELP COMMAND JOIN 
\end{verbatim}
\subsection{IF}
IF$|$IFNOT [.AND.$|$.OR.] [.NOT.]
\begin{verbatim}
          [DATA] [ERROR] [INTERACTIVE] [MESH]
          [THREE|3D] [value {.LT.|.EQ.|.GT.|.LE.|.NE.|.GE.} value]
\end{verbatim}

The  condition  is tested and if satisfied the rest of the line up to the
semicolon '';'' is executed.  If not satisfied it is skipped.  If .AND.  or
.OR.  are omitted .AND.  is assumed.  

IF$|$IFNOT [conditions] THEN
\begin{verbatim}
          . . .
\end{verbatim}
ELSE
\begin{verbatim}
          . . .
\end{verbatim}
ENDIF

If  the  tested  condition is satisfied lines up to the ENDIF or ELSE are
executed.  Otherwise they are skipped.  
\begin{verbatim}
     1.  DATA - True if any data points are available.  
     2.  ERROR - This condition is true if the last line had an error.  
     3.  INTERACTIVE  -  True  if  the  program is running interactively,
         False if it is run in BATCH mode.  The device you are using  may
         or may not be interactive independent of the mode of operation. 
     4.  MESH - True if the data is (3D) mesh data.  
     5.  3D  - True if 3D is turned on.  You should notice that mesh data
         may be plotted as flat or 3D data.  
     6.  value - Any number or lexical.  
\end{verbatim}

\begin{verbatim}
                                 Example
     TD:IF MESH PLOT;IFNOT MESH HISTOGRAM;  
          or...  
     TD:IF MESH THEN;PLOT;ELSE;HISTOGRAM;ENDIF;  
\end{verbatim}
If  the  current  data  is  mesh  data  it  is  plotted as a scatter plot
otherwise it is histogrammed.  
\begin{verbatim}
     TD:IF 3D .AND.  V_SUM .GT.  0 HISTOGRAM 
\end{verbatim}
Histograms 3-d data if it is greater than zero.  
\begin{verbatim}
     TD:IF V_SUM .GT.  0 .OR.  V_SUM .LT.  -1 .AND.  3D JOIN 
\end{verbatim}
Joins mesh data if the sum over the data is not between -1 and 0.  
\subsection{INTERPOLATE}
INTERPOLATE [LEVEL=n] [\{SPLINE$|$GENERAL\}] [CLOSED[=ON$|$OFF]] 
\begin{verbatim}
     [POINTS|COLUMNS=[FROM] n1 [TO] [n2]] 
     [SETS=[FROM] n1 [TO] [n2]] 
     [SELeCT="name"] 
     [LIMITED [VLOG[=ON|OFF]] [[FROM]|TO [[X=]nx,[[Y=]ny[,[Z=]nz]]]
          [RECURSOR] [CURSOR] ] 
     [APPEND[=ON|OFF]] [NAME="name"] [LOG[=ON|OFF]] 
\end{verbatim}
Interpolates  a  data set and puts the result into a new data set.  LEVEL
is the number of intervals per pair of points.  If 2 a  single  point  is
interpolated  in  between  each existing point.  If it is 5 then 4 points
are interpolated.  X, Y, and Z if available are all interpolated.  

The  data  must  be  strictly  monotonic  in  either  Y or X for a SPLINE
interplation.  

If  the  first  and  last  points  of  a curve are the same, the curve is
assumed to be closed, and the interpolation is done accordingly.  If  the
data  is POLAR or SPHERICAL then the angles must vary smoothly for proper
closure.  
\subsubsection{Options}
\begin{verbatim}
     1.  LEVEL - Selects the number of subintervals from 2 to 10.  
         (Default:LEVEL=2) 
     2.  SPLINE|GENERAL   -   Selects   the  type  of  fit  to  use  in
         interpolation.  
         (Default:GENERAL) 
     3.  CLOSED  forces  the  interpolation  to  consider the data as a
         closed curve.  CLOSED=OFF will fil an unclosed curve even when
         the first and last points are identical.  
         (Default:data determines closure) 
     4.  POINTS - Selects the data points to use by number.  
     5.  SETS - Selects the range of data sets to interpolate.  
     6.  SELECT="name" - Selects the data sets by name 
     7.  LIMITED - Selects the points to use by a range of values.  The
         actual data selected will include  all  data  that  meets  the
         selection.   Some data outside of the range may be included to
         avoid breaking up a data set into multiple pieces.  
     8.  APPEND  -  Appends  the new data sets.  Otherwise the new data
         replaces the old set.  
         (Default:APPEND=OFF) 
     9.  NAME - Specifies the name for the new data set.  
         (Default:"Spline %" 
    10.  LOG - Summarizes the operations done.  
         (Default:LOG=OFF) 
\end{verbatim}
\subsubsection{Notes}
The  spline  fit  is  based  on  the spline fitter from FXB\$CG from the
Stanford SPSS package.  

The  general  fit  was  originally  adapted from ACM Algorithm \#433, by
Hiroshi Akima.  

The  interpolation  by  n will not give the same results as joining the
data by n, since the joininig is done after the data  coordinates  have
been transformed into text coordinates.  
\subsection{JOIN}
Joins the data points with a series of line segments.  
JOIN [[LEVEL=]n] [\{SPLINE$|$GENERAL\}] [TEXT] [EXPAND[=ON$|$OFF]] 
\begin{verbatim}
     [POINTS|COLUMNS=[FROM] n1 [TO] [n2]] 
     [LINES|ROWS=[FROM] n1 [TO] [n2]] 
     [SETS=[FROM] n1 [TO] [n2]] 
     [SELeCT="name"] 
     [LIMITED [VLOG[=ON|OFF]] [[FROM]|TO [[X=]nx,[[Y=]ny[,[Z=]nz]]]
          [RECURSOR] [CURSOR] ] 
     [SLICES [X|Y|Z] [FROM] v1 [[TO] v2]] 
     [TITLE[=ON|OFF]] 
     [CYCLE[=ON|OFF]] 
     [INTENSITY|WIDTH=n] 
     [WHITE|RED|GREEN|BLUE|YELLOW|MAGENTA|CYAN] 
     [SOLID|DOTS|DASHES|DAASHES|DOTDASH|PATTERNED|FUNNY|SPACE] 
     [HIDE[=ON|OFF]] 
                         3-d options 
               [X|Y|Z] [ERROR[=ON|OFF]] 
\end{verbatim}
\subsubsection{Level}
This is the number of line segments/point to use in joining the points.
If omitted TOPDRAWER picks a suitable value 
\subsubsection{CYCLE}
When  plotting  more  than  1  slice or data set, CYCLE causes the line
texture, color, and width to vary from one plot to the next.  The order
of  cycling  is  set  by  the SET CYCLE command.  If you have specified
either texture, color or width it overrides the cycle value.  
\subsubsection{ERRORS}
This puts error bars as vertical lines on a mesh plot.  
\subsubsection{EXPAND}
Sets  the  limits so that the selected data expands to fill the window.
This has no effect unless POINTS, SETS, SLICES  are  specified  or  the
limits on MESH data have been set by a SET LIMITS command.  

See:SET MODE 
\subsubsection{FILL}
Generates  a  fill  pattern  inside  the  curve.   The  pattern will be
generated so that alternate areas witin  the  curve  are  shaded.   The
curve  is  assumed  to be closed with the last pont connecticted to the
first one.  

No  spline  or general fits are done to the data with fill.  The number
of segements per point or level is assumed to be 1 with fill.  
\subsubsection{HIDE}
Controls  whether hidden lines are drawn for a 3-d mesh, or whether one
histogram hides the next.   HIDE=OFF  draws  the  hidden  lines,  while
HIDE=ON omits them.  
(Default:HIDE=ON for mesh, HIDE=OFF for data) 

This  works  very  well  for curves which are functions of x, or curves
that have at most 2 y values for a given x.  Curves with more than 2  y
values  per  X  will  hide  the  entire  region from the minimum to the
maximum y.  
\subsubsection{SPLINE$|$GENERAL}
SPLINE  uses  a  natural cubic spline to fit the points.  Either x or y
must be strictly increasing.  A GENERAL curve is  calculated  using  an
algorithm  which  allows multiple-valued functions and repeated points.
(Default:GENERAL) 
\subsubsection{TEXT}
Use the TEXT coordinate frame instead of the DATA frame.  
\subsubsection{LINES}
Specify which lines or rows of a mesh are to be used in the plot.  
\subsubsection{POINTS$|$COLUMNS}
Specifies  the  range of points or the colums of a mesh to use.  If not
specified all data points are used.  If n2 is omitted it is assumed  to
be  n1.   You  may  specify the point number or the options FIRST,LAST.
For 3-d plots this specifies the columns to use in the plot.  
\subsubsection{SETS}
This  selects  the  data  set to plot.  The data must be broken up into
sets using the data set command, for this option to work properly.   If
n2  is  omitted  it  is assumed to be n1.  You may specify the data set
number or the options FIRST,LAST.  
\subsubsection{INTENSITY}
Sets line intensity or width (1-5).  
\subsubsection{WHITE...}
Sets the line color.  See:Command SET COLOR.  
\subsubsection{SOLID...}
Set the line texture.  See:Command SET TEXTURE 
\subsubsection{X$|$Y$|$Z}
This  specifies  that mesh data is to be joined only in X,Y, or Z.  The
default is to join in both.  
\subsubsection{SLICES}
This plots data sliced from a mesh.  For more information see HISTOGRAM
SLICE.  
\subsubsection{3-d\_plot}
This produces a mesh plot of the data.  Essentially the data points are
joined in X,Y.  If you specify color, intensity, or texture it is  used
for the top side of the mesh, while the default is used for the bottom. 

\begin{verbatim}
                                example
     TD:SET COLOR RED 
     TD:SET TEXTURE DOTS 
     TD:JOIN SOLID WHITE 
\end{verbatim}
Produces  a  mesh  plot  with bottom in dotted red lines, while the top
side is drawen in white solid lines.  
\subsection{LIST}
\subsubsection{DATA}
LIST [DATA] [APPEND[=ON$|$OFF]] [FILE$|$OUTPUT=filename] [LOG[=ON$|$OFF]] 
\begin{verbatim}
     [SETS=[FROM] n1 [TO] [n2]] 
     [SELeCT="name"] 
     [POINTS|COLUMNS=[FROM] n1 [TO] n2]] 
     [LINES|ROWS=[FROM] n1 [TO] n2]] 
     [LIMITED [VLOG[=ON|OFF]] [[FROM]|TO [[X=]nx,[[Y=]ny[,[Z=]nz]]]
          [RECURSOR] [CURSOR] ] 
     [STATISTICS[=ON|OFF]] 
     [COMPRESS[=ON|OFF]] 
\end{verbatim}

lists  the current data in the buffer.  The data values are listed in a
format that may be used as input to TOPDRAWER.   For  more  information
See:Command SHOW DATA.  
\paragraph{APPEND}
Adds the listing to the end of the specified file.  
(Default:APPEND=OFF) 
\paragraph{COMPRESS}
Selects  compressed output.  This saves space on disk, at the expense
of readability.  If you do not specify ON or OFF, ON is assumed.  
(Default:COMPRESS=OFF) 
\paragraph{FILE}
Selects  the  filename  for  output.   If omitted the listing file is
used.  If the filename is NONE then no file name is selected.  If you
wish  to  specify a file name containing either semicolons or blanks,
it must be enclosed in parenthesis.  
\paragraph{LIMITED}
Limits  the  data  listed  to  the specified range.  For 3-d data the
limit on the dependent  variable  is  ignored.   If  limits  are  not
specified,    the    default    is    the    current   plot   limits.
See:Command SET LIMITS.  
\begin{verbatim}
     1.  X - Specifies X limit 
     2.  Y - Specifies Y limit 
     3.  Z - Specifies the Z limit 
     4.  CURSOR - Brings up the cursor.  You move it to the X,Y value
         you wish then press the space bar to enter both X,Y or X  to
         enter X or Y to enter Y.  
     5.  RECURSOR - The cursor enters both limits.  
     6.  VLOG - Draws a cross when you press the space bar, and draws
         a dotted line around the final limits.  
     7.  LOG - Type on the terminal a summary of the data listing.  
\end{verbatim}

\begin{verbatim}
                               example
     TD:LIST DATA LIMITED FROM 1,1 to 2,5 
          or...  
     TD:LIST DATA LIMITED FROM X=1 Y=1 TO X=2 Y=5 
\end{verbatim}
Lists  all  points  that with X between 1 and 2 and Y between 2 and 5
inclusive.  
\begin{verbatim}
     TD:LIST DATA LIMITED FROM Y=1 TO Y=5 
\end{verbatim}
Lists all data points with Y values between 1 and 5 inclusive.  
\begin{verbatim}
     TD:LIST DATA LIMITED FROM CURSOR TO CURSOR 
\end{verbatim}
Lists all data points as defined by the cursor limits.  
\begin{verbatim}
     TD:LIST DATA LIMITED FROM X=CURSOR TO X=CURSOR 
\end{verbatim}
Lists all data points according to X values defined by the cursor.  
\paragraph{LINES$|$ROWS}
For 3-d plots specifies which lines or rows of the mesh data to list. 
\paragraph{POINTS$|$COLUMNS}
This  specifies  which  points to list.  For 3-d this is the range of
column numbers.  
\paragraph{SETS}
This specifies which data set to list.  
\paragraph{STATISTICS}
This  displays  only  the statistics on the data, and the actual data
values are suppressed.  
\subsubsection{FIT}
LIST FIT [APPEND[=ON$|$OFF]] [FILE$|$OUTPUT=filename] [FULL[=ON$|$OFF]] 
Produces a listing on disk of the current fit parameters.  
\begin{verbatim}
     Options:  
     1.  APPEND adds the listing to the end of the specified file 
         (Default:APPEND=OFF) 
     2.  FULL outputs all information.  
         (Default:FULL=OFF) 
     3.  FILE specifies the file name 
         (Default:tfit.lis) 
\end{verbatim}

\subsubsection{HISTOGRAMS}

LIST HISTOGRAMS 
\begin{verbatim}
     [APPEND[=ON|OFF]] 
     [FILE|OUTPUT=filename] 
     [IDENT=[FROM] n1 [TO] [n2]] 
     [SELECT|NAME='hist_name'] 
     [EXACT[=ON|OFF]] 
     [AREA|DIRECTORY="dir/subdir..."] 
     [TReE[=ON|OFF]] 
     [ENTRIES[=ON|OFF]] 
     [HISTOGRAM[=ON|OFF]] 
     [MESH[=ON|OFF]] 
     [ARRAY[=ON|OFF]] 
     [NTUPLES[=ON|OFF]] 
     [INdEX[=ON|OFF]] 
     [LOG[=ON|OFF]] 
\end{verbatim}
This produces a line printer listing of the current histogram data.
The options ENTRIES,MESH,NTUPLES,EXACT,NAME, and IDENT select which
histograms are printed.  TREE lists all histograms in the directory
tree.  You may specify the output file name FILE=name.  The option LOG
types the file name of the histogram listing on your terminal.  The
histogram data is listed in a form suitable to be printed.  Some
histogram packages allow you to specify the format of the output.  For
more information see command:  SET HISTOGRAM.  

An index precedes all of the histograms.  If you specify INDEX=OFF then
no index is listed.  
(Default:FILE=histo.lis,TREE=OFF,INDEX=ON,APPEND=OFF) 
\subsection{MERGE}
MERGE [ALL] 
\begin{verbatim}
     [SETS=[FROM] n1 [TO] [n2]] 
     [SELeCT="name"] 
     [NAMe="name"] 
     [POINTS=[FROM] n1 [TO] [n2]] 
     [APPEND[=ON|OFF]] [NAME="name"] [LOG[=ON|OFF]] 
     [LIMITED [VLOG[=ON|OFF]] [[FROM]|TO [[X=]nx,[[Y=]ny[,[Z=]nz]]]
          [CURSOR] ] 
\end{verbatim}

Merges the specified data points into a single data set.  
\begin{verbatim}
     1.  ALL merges all data 
     2.  APPEND  creates  a  new data set and copies all of the specified
         sets into it.  If not specified then the specified range of data
         sets is deleted.  
     3.  LIMITED allows you to specify which points to retain, all others
         are deleted or omitted in the case of APPEND.  
     4.  LOG shows which sets are merged.  
     5.  NAME specifies the name of the new data set.  
     6.  POINTS selects the data points by number 
     7.  SETS selects the range of data sets to merge.  
     8.  SELECT selects the data sets by name 
\end{verbatim}
You  must  specify  either  SETS, POINTS, LIMITED, or ALL If the range of
data sets contains a mesh data set and APPEND is not specified, the  data
can not be merged.  

After merging several data sets you may need to SORT the data if you wish
to HISTOGRAM, JOIN, or BARCHART it.  
\subsection{MONITOR}
\begin{verbatim}
     MONITOR HISTOGRAMS [AVAILABLE|FILE=name|SECTION=name]
          [IDENT[=[FROM] n1 [TO] [n2]] 
          [EXACT[=ON|OFF]] [EXPAND[=ON|OFF]] [SELECT|NAME='hist_name']
          [INTERVAL=n] [ENTRIES[=ON|OFF]] [HISTOGRAM[=ON|OFF]] [LOG[=ON|
          OFF]] [MESH[=ON|OFF]] [OVERFLOWS[=ON|OFF]] [PAUSE[=ON|OFF]]
          [REPEAT=n] [RESCALE[=ON|OFF]] [SHOW[=ON|OFF]] [SCAN[=ON|OFF]]
          [SKIP=n] [STATISTICS[=ON|OFF]] [TIME[=ON|OFF]] [TREE[=ON|OFF]]
          [WINDOWS=n] 
\end{verbatim}
Repeatedly displays the selected histograms, or the last one selected by
a SET HISTOGRAM command.  If more than 1 histogram are selected only the
first 20 are displayed.  SEE:TOPDRAWER LINK and SET MONITOR 
\subsubsection{AVAILABLE}
Selects currently available histograms rather than histograms in a
global section or direct access file.  
\subsubsection{EXACT}
Histogram names are treated as exact strings, and they are not searched
in a case independent manner.  
(Default:OFF) 
\subsubsection{EXPAND}
Expands the histogram scales by omitting leading and trailing zero
channels.  The histogram is also centered in the window.  You may also
set limits on the Y scales of individual histograms by:  
\begin{verbatim}
     DEFINE HIST ID=n PRMIN=nlow PRMAX=nhigh 
\end{verbatim}
\subsubsection{ENTRIES}
YES selects only histograms with entries, while NO selects histograms
without entries.  
\subsubsection{FILE}
Selects the direct access file to monitor.  If FILE='''' then the current
one is used.  
\subsubsection{IDENT}
Specifies the IDs of the histograms to display.  If this qualifier is
not used, then the current histogram is displayed.  If IDENT is used
without any parameters then all histograms are candidates for
monitoring.  
\subsubsection{INTERVAL}
Specifies the interval in seconds between updates.  
(Default:10) 
\subsubsection{LOG}
Types on your terminal the name of each histogram monitored.  If
STATISTICS is selected then the statistics are also typed.  
\subsubsection{MESH}
YES selects only mesh data (scatter plots), while NO selects only non
mesh data.  Normally both mesh and nonmesh data are displayed.  
\subsubsection{NAME}
Selects the histograms by name.  See SET or SHOW HISTOGRAM.  
\subsubsection{OVERFLOWS}
Displays the overflow data in the first/last channels.  
(Default:ON) 
\subsubsection{PAUSE}
Selects pause mode rather than wait mode.  After each plot Topdrawer
pauses until you press the ``Return'' key.  
(Default:OFF) 
\subsubsection{REPEAT}
Is the number of times to repeat.  It must be in the range of 1 to
2147483647.  
(Default:9999) 
\subsubsection{RESCALE}
Selects whether hist is rescaled when it overflows.  Normally the
TOPDRAWER automatically picks the scale of the displayed histogram.
Then when it is too big to fit on the screen, it is erased and
rescaled.  If you set limits before the MONITOR command, they are used
to determine the scale of the histogram.  The histogram is not
automatically rescaled, if the limits are picked or if RESCALE=OFF.  
(Default:ON) 
\subsubsection{SECTION}
Selects the global section to monitor.  If SECTION='''' then the current
one is used.  If neither FILE nor SECTION are specified then section is
assumed unless FILE has been specified in a previous MONITOR HISTOGRAM
or SET HISTOGRAM command.  SEE:TOPDRAWER LINK 
\subsubsection{SCAN}
Displays all selected histograms, in order.  The screen is cleared
between each update.  This allows you to look at more than 20
histograms.  
(Default:OFF) 
\subsubsection{SKIP}
\begin{verbatim}
     SKIP=n Skips the first n histograms which meet the selection
\end{verbatim}
criteria.  
\subsubsection{STATISTICS}
Puts histogram overflow/underflow and sum on the line below the top
title.  
\subsubsection{SHOW}
Adds the hist ID to the lower right corner of each window below the X
title.  
(Default:ON) 
\subsubsection{TIME}
Adds the time to the lower left corner of the display.  
(Default:ON) 
\subsubsection{TREE}
Search the directory tree from the current directory for histograms to
monitor, that match the selection criterions.  
(Default:OFF) 
\subsubsection{WINDOWS}
Selects the maximum number of windows to use.  You may specify any
number from 1 to 20.  
(Default:20) 
\subsection{MORE}
Adds more text after the last title with the same parameters as the last
title.  It may also be modified by a matching CASE command, see TITLE.  
\begin{verbatim}
     MORE 
\end{verbatim}
\subsection{MULTIPLY}
MULTIPLY [Y$|$Z] [FROM$|$TO] [EWEIGHT=n] [WEEIGHT=n] \{n1$|$``name1''\} [BY] \{n2$|$
\begin{verbatim}
          "name2"|FIT} 
     [APPEND[=ON|OFF]] [NAME="name"] [CHECK[=ON|OFF]] [ERROR[=ON|OFF]]
          [POINTS|COLUMNS=[FROM] n1 [TO] [n2]] [LINES|
          ROWS=[FROM] n1 [TO] [n2]] [LIMITED [VLOG[=ON|OFF]] [[FROM]|TO
          [[X=]nx,[[Y=]ny[,[Z=]nz]]] [RECURSOR] [CURSOR] ] [LOG[=ON|OFF]]
          [VECTOR[=ON|OFF]] 
\end{verbatim}
This  multiplies  the  (Y/Z)  values  in data set n1 by data set n2.  The
result is a modified set n1.  You may not multiply mesh  data  by  normal
2-d data.  
\subsubsection{APPEND}
If  APPEND is specified, then a new data set is created, containing the
result and set n1 is unchanged.  
\subsubsection{CHECK}
CHECK=OFF  turns  off  data set checking.  When CHECK=ON both data sets
must have identical X (and Y if mesh)  values,  but  data  set  n2  may
contain  more  points than n1.  If (DX/DY) is non zero then the both DX
and X must be identical within 1\% of DX.  If you have  data  sets  with
non  identical  values  of X you may create a set with identical values
using the BIN command.  If the X values of 2 data sets  are  not  quite
identical.   You may force TOPDRAWER to multiply them by setting DX for
the data 100 times greater than the  difference  in  X  or  by  setting
CHECK=OFF 
\subsubsection{ERROR}
ERROR=OFF excludes the errors from the FIT in the computation.  
(Default:ERROR=ON) 
\subsubsection{FIT}
Specifies  that  the  data  set  is  to  be multiplied by the last FIT.
See:Command FIT.  
\subsubsection{LIMITED}
You  may  specify  the  limits  over  which  the  histograms  are to be
multiplied.  If you specify the Y or  Z  limits,  then  all  data  that
contains values inside these limits will be multiplied.  For example if
you 
\begin{verbatim}
     MULTIPLY 1 to 2 LIMITED FROM Y=10 to Y=11 
\end{verbatim}
But data set 1 contains the followind data:  
1,0;  2,10;  3,0;  4,11;  5,0 
Points  2  to  4  inclusive will be multiplied, and only points 1 and 5
will be omitted.  
\subsubsection{LOG}
If  LOG  is  specified  then the result of the command is typed on your
terminal.  
\subsubsection{VECTOR}
The  DX,DY,DZ's  will be cross multiplied together assuming they form a
vector.  The X,Y,Z are not changed, and  they  should  match  for  both
input data sets.  
\subsubsection{WEIGHT-EWEIGHT}
WEIGHT=n  specifies  a  weighting factor to multiply the data by before
adding or subtracting it.  EWEIGHT=n specifies  the  error  on  WEIGHT.
This option must precede the data set number.  
\subsubsection{Notes}
The  new value is Y3=Y1*Y2.  The errors (DY) are treated as if the data
sets    are     uncorrelated.      DY3=Y3*SQRT((DY1/Y1)**2+(DY2/Y2)**2)
See:Command ADD,DIVIDE.  
\subsection{NEW\_FRAME}
\begin{verbatim}
     NEW [FRAME|PLOT] [RESET] [ALIAS=alias] "Comment about plot" 
\end{verbatim}
Starts a new picture.  Untreated points are PLOTTED before going on.  The
old data points are retained until new ones  are  read.   Parameters  are
returned  to  the  default values.  If the plot device is interactive and
you are executing commands inside a file the program prompts:  
\begin{verbatim}
     PAUSE:  
\end{verbatim}
You  press  the  ``Return''  key  to  start  the  next plot.  If instead of
``return'' you type in STOP, TOPDRAWER will stop executing the  input  file
and return to command level.  You can also type EXIT, QUIT, HALT, END, or
Ctrl\_Z instead of STOP.  

If  you  press  Ctrl\_C  then TOPDRAWER will skip input until another STOP
command is encountered, or the file ends.  Pressing Ctrl\_C twice  rapidly
in  succession  will abort all file input.  Pressing Ctrl\_C twice rapidly
in succession again will cause the program to ask  you  if  you  wish  to
abort.  
\subsubsection{ALIAS}
Each  picture  is given the name PICT001, PICT002 etc., but they may be
given an alias name.  If the name is '''' then name is Plot0001,  Plot002
...  
\subsubsection{Comment}
The  ``Comment on plot'' is typed on your terminal, or written to the log
file.  
\subsubsection{RESET}
This  option  resets  all PERMANENT options to their original defaults.
For example you have:  
\begin{verbatim}
     TD:SET LIM FROM 1,2 TO 3,4 PERMANENT 
\end{verbatim}
The RESET option will set them back to the default of no limits.  
\begin{verbatim}
                                 note
\end{verbatim}
The  STORAGE is not reset back to the default, and the existing data is
not modified.  
\subsection{PAUSE}
\begin{verbatim}
     {PAUSE|WAIT} ['String'] [FOR n] 
\end{verbatim}

This  causes  TOPDRAWER  type the string and PAUSE.  This only works in a
file or inside a repeat block.  You press the ``Return'' key  to  continue.
If  instead  of  ``return'' you type in STOP, TOPDRAWER will stop executing
the input file or repeat block and return to command level.  You can also
type  EXIT, QUIT, HALT, END, or Ctrl\_Z instead of STOP.  You may also use
SET and SHOW commands while paused.  
\subsubsection{FOR}
If  FOR  n  is  specified  then TOPDRAWER will wait for n seconds, then
resume normal execution.  n may be in the range 0 to 200.  (Default:10) 

example 
\begin{verbatim}
     TD:REPEAT "NEW;SET HIST NEXT;HIST;WAIT FOR 10" 10 
\end{verbatim}
Gets  the  next  histogram,  plots it, then waits for 10 seconds before
plotting the next one.  
\subsection{PLOT}
Plots the data as a series of points or symbols 

PLOT [EXPAND[=ON$|$OFF]] [AXES$|$GRID] [OUTLINE] [TABLE] 
\begin{verbatim}
     [POINTS|COLUMNS=[FROM] n1 [TO] [n2]] 
     [LINES|ROWS=[FROM] n1 [TO] [n2]] 
     [SETS=[FROM] n1 [TO] [n2]] 
     [SELeCT="name"] 
     [SLICES [X|Y|Z] [FROM] v1 [[TO] v2]] 
     [LIMITED [VLOG[=ON|OFF]] [[FROM]|TO [[X=]nx,[[Y=]ny[,[Z=]nz]]]
          [CURSOR] ] 
     [TITLE[=ON|OFF]] 
     [INTENSITY|WIDTH=n] 
     [WHITE|RED|GREEN|BLUE|YELLOW|MAGENTA|CYAN] 
     [SOLID|DOTS|DASHES|DAASHES|DOTDASH|PATTERNED|FUNNY|SPACE] 
     [CYCLE[=ON|OFF]] 
     [[NO]SYMBOL[=sym]] [SIZE=n] 
     [[X|Y|X]ERRORS[=ON|OFF]] 
     [FILL[=ON|OFF]] 
                         Data plot options 
     [VECTOR[=ON|OFF]] 
                         Mesh (scatter) plot options 
     [[NO]RANDOM[=n]] [VARIABLE[=ON|OFF]] [COUNTS] 
                         Non data options 
     [OUTLINE] [AXES] [GRID] [GRID] 
\end{verbatim}

The  data points are plotted as a series of symbols with error bars.  The
options AXES, OUTLINE, and GRID allow you to add specified elements to  a
plot.  TABLE plots the actual values for the data as a set of numbers.  
\subsubsection{Options}
\begin{verbatim}
     1.  AXES - Plots axes (if enabled).  You may specify where to draw
         the ticks.  
     2.  GRID - Plots a grid with specified locations.  
     3.  COUNTS - Controls the number of symbols/bin for a scatter plot 
     4.  EXPAND - Sets the limits so the specified set fills the window 
     5.  FILL - Fills in the symbols that are plotted.  
     6.  INTENSITY - Sets the intensity of the plot 
     7.  LIMITED - Limits the plot to a specified range of X,Y,Z 
     8.  [X|Y|Z]ERRORS=OFF - Disables error bars.  
     9.  RANDOM - Controls the number of symbols/bin for a scatter plot 
    10.  OUTLINE - Draws an outline around the plot (if enabled) 
    11.  SETS - Selects the data sets to plot 
    12.  SLICES - Selects the mesh data points to plot in 2-d.  
    13.  [NO]SYMBOL - Selects the SYMBOL to use when plotting.  
    14.  SIZE - Selects the symbol size.  
    15.  TABLE - Plots the values of the data points as numbers.  
    16.  VARIABLE  - Enables symbol sizes proportional to the value for
         scatter plots.  
    17.  VECTOR - Plots data as a vector field.  (DX,DY,DZ)=vector 
    18.  WHITE...  - Selects the color of the plot.  
    19.  SOLID...  - Selects the texture of the plot (error bars) 
\end{verbatim}
\subsubsection{AXES$|$GRID}
\begin{verbatim}
     PLOT  {AXES|GRID}  [AT [x [y [z]]]] [[TOP|BOTTOM|RIGHT|LEFT|X|Y|Z]
\end{verbatim}
[[LABELS$|$TICKS]  [SHORT$|$LONG]  [n1,n2,n3.....]   [FROM n TO n BY n N n]
[FFORMAT$|$GFORMAT$|$EFORMAT$|$YERS $|$MONTHS$|$DAYS$|$TIME]] 
Puts Axes and grid on the plot, but does not plot points.  This command
is not usually necessary as axes are added  to  the  picture  when  the
first  PLOT,  HISTOGRAM,  or JOIN command is executed.  If you use this
command before a PLOT, HISTOGRAM or JOIN command,  then  the  automatic
axes  are  not  produced.  The limits of the axes are determined either
from the existing data or from the the SET LIMITS command,  or  by  the
LIMITED  options.  The format of the axes is determined by the SET AXES
command.  The origin of the axes may be specified after the option AT. 

\begin{verbatim}
     This  command  is mainly useful for special graphs where 2 sets of
\end{verbatim}
axes with different scales are needed.  This is also used to plot  axes
on  a  3  dimensional  plot where x,y,z specify the origin of the axes.
3-d plots do not automatically produce axes, so you must use  the  PLOT
AXES command.  

\begin{verbatim}
     The origin may also be specified by the command SET THREE.  If not
\end{verbatim}
already specified by SET AXES, LABELS, or TICKS  commands,  the  color,
texture, and intensity options will be used in plotting the axes.  
\paragraph{AT}
This allows you specify the data X,Y,Z at which the axis is plotted. 
\paragraph{FROM$|$TO$|$BY$|$N}
You  may specify a range of ticks with FROM,TO,BY,N.  You may specify
any 3.  

\begin{verbatim}
                               example
     TD:PLOT AXES X FROM 1 TO 10 BY 2 
          or...  
     TD:PLOT AXES X FROM 1 BY 2 N=5 
\end{verbatim}
Draws ticks at 1,3,5,7, and 9.  
\paragraph{GRID}
If  specified  without  the  option AXES then a GRID is plotted.  The
grid must be ON to be plotted.  See SET GRID.  If AXES is  specified,
then  both  are plotted if the GRID is ON.  Grid lines or symbols are
plotted only for long ticks.  

\begin{verbatim}
                               example
     TD:SET GRID ON 
     TD:PLOT AXES Y 0 X 2,4,6,8 ticks 1,3,5,7,9 
\end{verbatim}
The  grid  is  plotted  at ticks 2,4,6,8 on the X and 0 on the Y.  An
axes is also plotted with labels at X=2,4,6,8 Y=0, and short ticks at
X=1,3,5,7,9.  
\begin{verbatim}
     TD:PLOT GRID Y 0 X 2,4,6,8 
\end{verbatim}
Only a grid is plotted.  
\paragraph{HIDE}
Controls  whether  hidden  lines  are drawn for a 3-d axes.  HIDE=OFF
draws the hidden lines, while HIDE=ON omits them.  This assumes  that
the  axes  are  drawn in the normal position behind the plot.  It may
not work properly for other positions.  You must specify  which  axes
are to be plotted.  
(Default:HIDE=OFF) 
\begin{verbatim}
                               example
\end{verbatim}
For example histogram a mesh with hidden axes:  
\begin{verbatim}
     TD:HISTOGRAM ( Hist mesh data) 
     TD:PLOT AXES X Y Z HIDE ( Now plot only visible portion of axes) 
\end{verbatim}
\paragraph{TOP...}
Produces  only  a  single axes at the TOP according to n1,n2....  You
may also produce the axes to the LEFT,RIGHT, or BOTTOM.  For 1-d plot
X  produces both TOP and BOTTOM axes, Y produces both LEFT and RIGHT,
while Z produces only TOP.  On 3-d plots they produce X,Y, or Z axes.
The axes are not produced if disabled by the SET AXES command.  

After this option the tick values are assumed to be LABELS LONG.  
\paragraph{LABEL$|$TICK}
LABEL  selects  long  ticks  with  labels.  TICKS selects short ticks
without labels.  Labels or ticks are not produced if disabled by  the
SET LABELS or SET TICKS command.  
\paragraph{LIMITED}
If  this  option  is used it may be used to specify the length of the
axis, with respect to the data.  
\paragraph{LONG$|$SHORT}
Selects  either  long  or  short  ticks.  This must be used after the
LABEL or TICK command it modifies.  
\paragraph{FORMATS...}
These  determine  the  number  format.  They are analogous to FORTRAN
formats.  
\begin{verbatim}
     1.  FFORMAT  - Numbers are written as nn.nnn or nn The number of
         digits to  the  right  or  left  of  the  decimal  point  is
         determined  by the data.  All labels for a selected axis are
         written with the same number of digits to the right  of  the
         decimal point.  
     2.  GFORMAT  -  Numbers  are  written  as  either  F or E format
         depending on the range.  For example if you  specify  labels
         from 0 to 1 by .25 they are written as.  0,0.25,0.5,1 
     3.  EFORMAT  - Labels are written as n.nn with an exponent power
         of 10.  
     4.  YEARS - Only the year is written 
     5.  MONTHS - The month is written 
     6.  DAYS - The day of the month is written 
     7.  TIME - The time is written.  
     8.  DATE - The date is written in the format dd-mmm-yyyy.  
\end{verbatim}
(Default:GFORMAT) 

\begin{verbatim}
                                NOTE
\end{verbatim}
If the format is not G,F,or E then SMALL ticks are not reproduced for
labels.  This allows you to place the label in any  location  without
any ticks.  
\paragraph{N1...}
These  select  the  tick/label  locations.   If omitted then a set of
default ticks and labels are chosen.  These are selected according to
the SET LABELS and SET TICKS commands.  
\paragraph{Restrictions}
You  may  not  specify more than 200 total ticks at one time.  If you
need more than 200 ticks you need to plot AXES several times.  

\begin{verbatim}
                               example
     TD:PLOT AXES X FROM 1 TO 10 by 0.002 
\end{verbatim}
generates 500 ticks, and is not allowed.  
\paragraph{Examples}

\begin{verbatim}
                               example
     TD:PLOT AXES 
\end{verbatim}
Plot all axes with the current defaults.  
\begin{verbatim}
     TD:PLOT AXES X 
\end{verbatim}
Produces default TOP and BOTTOM axes.  
\begin{verbatim}
     TD:PLOT AXES X LABELS 1,2,3,4 TICKS 1.5,2.5,3.5 
\end{verbatim}
Produces labels at 1,2..  and short ticks at 1.5,...  
\begin{verbatim}
     TD:PLOT AXES RIGHT EFORMAT LABELS 1,2,3,4 TICKS LONG 1.5,2.5,3.5 
\end{verbatim}
Produces  the  same labels, but in exponential format with long ticks
in between the labels.  
\begin{verbatim}
     TD:PLOT AXES BOTTOM LABELS 1,2,3 TOP 1.5,2.5 
\end{verbatim}
Produces  labels  with  long ticks on the top and bottom of the plot.
The labels are at different locations.  
\begin{verbatim}
     TD:PLOT AXES X LIMITED FROM X=1 Y=5 TO X=11 LABELS 1,5,11 
\end{verbatim}
This plots an X axis from X=1 to 11 Y=5 with labeled ticks at 1,5,11. 
\subsubsection{COUNTS}
This sets the number of DOTS=ZMAX-ZMIN.  This is for data that consists
of counts versus X,Y.  ZMIN,ZMAX are either automatically  set  by  the
data set, or by the SET LIMITS command.  
\subsubsection{CYCLE}
When  plotting  more  than  1  data  set, CYCLE causes the line symbol,
texture, color, and width to vary from one plot to the next.  The cycle
table  is  set  by the SET CYCLE command.  If you have specified either
symbol, texture, color or width it overrides the cycle values.  

When  doing  a  scatter  plot  of a mesh, CYCLE causes the symbol, line
color and width to vary according to the cycle table.  

When plotting a mesh table the color and width of each number will vary
according to the CYCLE TABLE.  
\subsubsection{EXPAND}
Sets  the  limits so that the selected data expands to fill the window.
This has no effect unless POINTS, SETS, SLICES  are  specified  or  the
limits on MESH data have been set by a SET LIMITS command.  

See:SET MODE 
\subsubsection{FILL}
Fills  the symbols to make them ``more solid''.  You will have to use the
command SET FILL to achieve the desired effect.  Many symbols  can  not
be  filled properly, so you must just try until you find a satisfactory
one.  

Of  the  normal  ASCII character set the ``0O8SZ\$'' look pretty good when
filled.  The centered symbol 1o, 2o, 6o, 8o look good when filled.   3o
and 4o look bad when filled, while the rest to not fill at all.  
\subsubsection{INTENSITY}
INTENSITY=n sets line intensity or width (1-5).  
\subsubsection{LIMITED}
This  plots only those data points that fall within the specified X,Y,Z
limits.  If no limits are specified, then the limits used are those set
by  the SET LIMIT command.  This option has no effect on the density of
dots or symbol size.  If a singe  axis  is  plotted,  then  the  limits
determine the location and length of the axis.  

\begin{verbatim}
                                example
     TD:PLOT LIMITED FROM 1,2,3 TO 4,5,6 
\end{verbatim}
Only points where 1$<$=X$<$=4 2$<$=Y$<$=5 3$<$=X$<$=6 are plotted.  

\begin{verbatim}
                                example
     TD:PLOT LIMITED FROM Z=.5 to Z=10 
\end{verbatim}
Only points where 0.5$<$=Z$<$=10 are plotted.  
\begin{verbatim}
     TD:PLOT LIMITED FROM Y=.01 
\end{verbatim}
Only points where 0.01$<$=Y are plotted.  

\begin{verbatim}
                                example
     TD:PLOT LIMITED FROM CURSOR TO CURSOR 
\end{verbatim}
The  cross hair cursor appears on the screen.  You move it to the lower
X,Y and press the space bar.  Then move it to the upper X,Y  and  press
the space bar again.  The points plotted will lie within the limits you
set.  
\subsubsection{ERRORS}
This  selects  plots  with  or  without  error  bars.  ERRORS=OFF plots
without any errors, while XERRORS=OFF only  omits  the  X  error  bars.
Similarly YERRORS=OFF or ZERRORS=OFF omit the Y or Z error bars.  
(Default:ERRORS=ON) 

If you wish to modify the appearance of the error bars see the command:
SET BAR.  This modifies the size of the bar at the end of the errors.  
\subsubsection{RANDOM}
RANDOM=n - Sets the maximum number of points per bin for a scatter plot
(3-d) If RANDOM is not specified The number of points in a bin  is  the
value  in the bin-Zmin+1.  The default RANDOM is limited to the range 1
to 10.  If Zmax is less than 1.0 then the default RANDOM is set  to  5.
No  points  are produced when the data equals ZMIN.  Zmin for a scatter
plot is zero unless set by a SET LIMITS command.  

If  the  contents  of  the bins range from 0 to 100 and RANDOM=10, Then
each dot represents the value 10.  Any bin with  values  less  than  or
equal  to  0  will  have no points plotted.  0.00001 to 10.0 will get 1
dot, 10.00001 to 20 will have 2 dots and so on.  

If  the  Z  axis  is  logarithmic  then  the  number  of points will be
proportional to ALOG10(Z)-ALOG10(ZMIN).  

NORANDOM  modifies  a scatter plot so that only one point is plotted in
the center of each bin.  This option should be  used  with  either  the
option LIMITED or VARIABLE.  
\subsubsection{SOLID...}
Selects  the  texture  of  the plot.  This doesn not change the plotted
symbol.  Error bars may be drawn as dotted or dashed.  
\subsubsection{LINES$|$ROWS}
For 3-d plots specifies which lines or rows of the mesh data to list.  
\subsubsection{OUTLINE}
Plots  the  outline around the plot.  Sections of the outline which are
disabled by the SET OUTLINE are not plotted.  If not already  specified
by  SET AXES,  or  OUTLINE  commands,  the  color, and intensity may be
specified by the PLOT OUTLINE command.  
\subsubsection{POINTS$|$COLUMNS}
Specifies  the range of points or the colums of a mesh to use.  For 3-d
this is the range of column numbers.  
\subsubsection{SETS}
This  selects  the  data  set to plot.  The data must be broken up into
sets using the data set command, for this option to work properly.   If
n2  is  omitted  it  is assumed to be n1.  You may specify the data set
number or the options FIRST,LAST.  
\subsubsection{SLICES}
This  scatter  plots  either a single slice or several slices from mesh
data.  If THREE is OFF the first X,Y,Z determines the view along either
the X,Y, or Z axis.  See HISTOGRAM for more details.  
\subsubsection{SYMBOL}
SYMBOL=sym  specifies  the  symbol to use for plotting data.  Any legal
UGSYS symbol may be used.  Points that do not already have a symbol are
normally  plotted  using  the  default symbol (See:Command SET SYMBOL).
The symbol you specify is used instead of the default.  You may  change
the  color  and  intensity  of  a symbol, but it may not have a changed
texture.  You can not plot symbols dotted or dashed.  

NOSYMBOL specifies that all points are plotted with the default symbol,
or the selected SYMBOL instead of the symbol assigned to each point.  

If  you  want a centered symbol use 0o, 1o, 2o, 3o, 4o, 5o, 6o, 7o, 8o,
or 9o.  Some characters from the ASCII character set are also suitable.
``08OXNSYZ\$*''  are  all  possible  characters  which  are  centered  and
reasonably symmetrical.  Lower case characters are too  far  below  the
center of the point.  
\subsubsection{SIZE}
SIZE=n  is  the size of the symbol to plot.  If unspecified the default
size is used.  See:Command SET SYMBOL.  
\subsubsection{TABLE}
\begin{verbatim}
                                Options
          PLOT  TABLE  [AT]  [ABSOLUTE] [DATA] [VALUE] [TOP] [OFFSET=n|
          [CURSOR]] 
\end{verbatim}
For  regular  data  points,  the value of the dependent variable (Z or Y) 
is drawn along the X axis as a number at the angle -90 degrees.  

Mesh data is plotted as a table of numbers centered on the mesh points,
The number is plotted horizontally.  

The  size  of the numbers is adjusted to fit within the available space
with a maximum size equal to the label size.  You may select  the  size
by using the SIZE=n option.  
\begin{verbatim}
                                Options
     1.  OFFSET=n  moves  the  table up or down by the selected offset.
         Normally it is specified in inches from the bottom data axis. 
     2.  DATA - The offset is specified in data units.  
     3.  VALUE - The offset is from the value of the data point.  
     4.  TOP - The offset is down from the top data axis.  
     5.  CURSOR - The cursor is used to select the offset.  
     6.  ABSOLUTE  - The offset is an absolute value from the bottom of
         the window.  
\end{verbatim}
\subsubsection{VARIABLE}
This  modifies  scatter  plots  so  that  the  size  of  the  symbol is
proportional to Z-ZMIN.  Normally this is used with the  options  SIZE,
NORANDOM, and VARIABLE.  
\subsubsection{VECTOR}
This  plots regular data as a vector field.  Each data point is plotted
as an arrow with  the  tail  located  as  (X,Y[,Z])  and  the  head  at
(X+DX,Y+DY[,Z+DZ]).  The arrow head has a width equal to the BAR size. 
\subsubsection{WHITE...}
WHITE, RED, BLUE, GREEN, YELLOW, CYAN, or MAGENTA sets the line color. 
\subsubsection{3-d}
Mesh  data  produces  a  scatter plot if THREE is off and the dependent
axis is Z.  If the dependent axis is Y or X then a series of slices are
plotted of mesh data.  The scatter plot is truncated with zero dots for
data equal to ZMIN, and the  maximum  number  of  dots  at  ZMAX.   The
options  SYMBOL,  DOTS,  COUNTS,  NORANDOM,  and  VARIABLE  are used to
control the look of the scatter plot.  In addition  by  setting  the  Z
limits  and  using  the  LIMITED  option  you can control which data is
plotted.  If the dependent axis is not correct for the desired type  of
plot use the SWAP command to change them.  

\begin{verbatim}
                                Example
     TD:PLOT SYMBOL=1O SIZE=10 NORANDOM VARIABLE 
\end{verbatim}
Plots  the  selected  symbol  in varying sizes.  It is 1.0'' at ZMAX and
nothing is plotted at ZMIN.  
\begin{verbatim}
     TD:SET LIM ZMAX=4;  PLOT BLUE LIMITED 
          or...  
     TD:PLOT BLUE LIMITED TO Z=4 
     TD:SET LIM ZMIN=4;  PLOT RED LIMITED 
          or...  
     TD:PLOT RED LIMITED FROM Z=4.0001 
\end{verbatim}
Points Z=4 and lower are plotted in BLUE.  Points above 4 are RED.  

The default for PLOT is THREE off.  
\subsection{PROJECT}
PROJECT [ADD$|$AVERAGE] [INTEGRATE[=ON$|$OFF]] [SPLIT[=ON$|$OFF]]
\begin{verbatim}
          [SETS=[FROM] n1 [TO] [n2]] [SELeCT="name"] {X|Y|Z}=[v|
          FROM v1TOv2[BYv3|N=n]]} [...] [APPEND[=ON|OFF]] [NAME="name"]
          [LOG[=ON|OFF]] 
\end{verbatim}

Forms 1 or more projections of mesh data parallel to one of the axes.  DX
is set to half of the bin width and Y is either the sum or  average  over
the  mesh  values.   If  available Z=the mean over the summed coordinate.
This forms the projections from all selected mesh data.  
\subsubsection{ADD$|$AVERAGE}
ADD$|$AVERAGE specifies whether the data is added or averaged to form the
projection.  The default is to ADD the data.  This option must  precede
the X,Y,Z that it modifies.  
\subsubsection{APPEND}
If  APPEND is specified, then a new data set is created, containing the
result and the original data is unchanged.  
\subsubsection{INTEGRATE}
This produces a projection of the area under the curve.  The projection
is the SUM( DY*Z ) if Y is specified.  If  AVERAGE  is  also  specified
then  the  projections  is  divided by the SUM( DY ).  This option must
precede the X,Y,Z that it modifies.  
\subsubsection{SPLIT}
SPLIT=ON   splits   channels   to  preserve  the  total  sum  over  the
projections.  For example X channels run from 0 to 1, 1 to 2 ...  9  to
10.  
\begin{verbatim}
     TD:PROJECT  X  FROM 0 TO 10 BY 2.5 Will produce 4 projections, but
\end{verbatim}
the sum over all projections will not be equal to the original sum.  
\begin{verbatim}
     TD:PROJECT SPLIT X FROM 0 TO 10 BY 2.5 
\end{verbatim}
Produces a set of projections and preserves the sum.  
(Default:SPLIT=OFF) 
\subsubsection{X$|$Y$|$Z}
This  specifies  the axis to project along.  If you specify v then only
the row/column corresponding to v is projected.  If specified v1 and v2
may  select  a  range of values to project.  If no values are specified
then the entire mesh is projected.  If BY=v3 is specified then a series
of  projections  are  made from v1 to v2 with a width of v3.  If N=n is
specified then the range from v1 to v2 is split into n equal intervals.
You  need  not  use this command if you wish to histogram only a single
slice.  The SLICE modifier on the HISTOGRAM, or JOIN  commands  may  be
used.   If X,Y, or Z are repeated then extra data sets are defined with
projections.  
\subsubsection{LOG}
Each projection is listed on your terminal as it is formed.  
\subsubsection{SETS}
This  selects the data sets to project.  If n2 is omitted it is assumed
to be n1.   You  may  specify  the  data  set  number  or  the  options
FIRST,LAST.  All mesh data between n1 and n2 are projected.  
\subsubsection{Examples}
You read in a mesh of X,Y vs Z.  
\begin{verbatim}
     TD:PROJECT Y FROM 5 TO 10 
\end{verbatim}
Projects  the  Y slices corresponding to Y=5 to 10 onto the X axis, and
2-d array of values is stored.  The new array of values is the  sum  of
all  Z values corresponding to the range of Y values.  The mesh data is
destroyed.  
\begin{verbatim}
     TD:PROJECT Y FROM 5 TO 10 BY 2 
\end{verbatim}
Makes 3 projections from 5 to 7, 7 to 9, and 9 to 10 

\begin{verbatim}
     TD:PROJECT X Y APPEND 
     TD:SET WINDOW X 1 of 2;HIST SET <SETS-1>;TITLE BOTTOM 'X axis' 
     TD:SET WINDOW X 2 of 2;HIST SET LAST;TITLE BOTTOM 'Y axis' 
     TD:DELETE SETS FROM <SET-1> 
\end{verbatim}
Projects both X and Y.  The projection along the X axis is in data next
to last data set while along Y is in the last  set.   Both  projections
are then histogrammed in separate windows.  Finally the projections are
deleted.  
Note:   This assumes only 1 mesh data set exists.  If n mesh sets exist
then 2n projections will be formed.  

\begin{verbatim}
     TD:PROJECT X 1 2.5 X 3 5 X 6 10;HIST 
\end{verbatim}
Projects 3 different bands of data along the X axis.  All 3 projections
are then histogrammed.  
\subsection{READ}
\begin{verbatim}
     READ POINTS - Reads in data to be plotted 
     READ MESH - Reads in 3-d data to be plotted 
\end{verbatim}
\subsubsection{POINTS}
\begin{verbatim}
     READ POINTS [APPEND=ON|OFF]] [SET] 
\end{verbatim}

This  reads  in  data to be plotted.  Once data is read, it is retained
until it has been used to generate a plot.  It is then  discarded  when
new data is read.  Normally you plot the data immediately after reading
it.  When you start  a  new  plot,  the  data  last  plotted  is  still
available  for replotting.  It is not discarded until new data is read.
The option APPEND allows you to append more data to the already plotted
data.   The  option SET starts a new data set.  If you do not specify a
symbol when reading points, the default is determined by the  last  SET
SYMBOL command.  

This  command  is  not necessary, since TOPDRAWER assumes that a number
not preceded by a command is automatically a point.   If  you  wish  to
read data with the symbol as the first item, you must use the READ DATA
command, and terminate the data with a blank line.  

Values  that  are  omitted are initially set to 0.0, or to the previous
value for that field.  

When reading in data $<$XVALUE$>$, $<$DXVALUE$>$, $<$YVALUE$>$, ...  are the X, DX,
Y, ...  values from the previous point.  
\paragraph{Examples}
\begin{verbatim}
     TD:SET ORDER X Y SYMBOL DX DY 
     TD:0 2 
     TD:1 4 1o 0.5 0.1 
     TD:2 8 
\end{verbatim}
Produces a data set with The following data:  
\begin{verbatim}
          X=0 Y=2 DX=0.0 DY=0.0 Symbol="" 
          X=1 Y=4 DX=0.5 DY=0.1 Symbol="1O" 
          X=2 Y=8 DX=0.5 DY=0.1 Symbol="1O" 
\end{verbatim}

\begin{verbatim}
     TD:SET ORDER SYMBOL X Y 
     TD:READ POINTS 1O 1,1;  
     TD:2O 2.5,3.8 
     TD:(blank line to terminate) 
\end{verbatim}
\subsubsection{MESH}
READ MESH [BINS] [APPEND[=ON$|$OFF]] [WITH] [ERRORS[=ON$|$OFF]] 
\begin{verbatim}
     [FOR] {X|Y|Z}[=] N1,...,Nn [nn+1 ENDBINS] 
\end{verbatim}
[FOR] \{X$|$Y$|$Z\}[=]nn \{X$|$Y$|$Z\}[=] N1,...,Nn 
[FOR] \{X$|$Y$|$Z\}[=]nn \{X$|$Y$|$Z\}[=] N1,...,Nn 
................  
[[FOR] \{X$|$Y$|$Z\}[=]nn ENDBINS] 
\begin{verbatim}
               or instead of N1,...Nn 
          [[FROM=n] [TO=n] [BY|WIDTH|STEP=n] [N=n]] 
\end{verbatim}

You  must  specify  the  points  as  an  array of X,Y,Z values.  If the
attribute BINS is selected you specify the edge of the bins rather than
the center of the bins and the optional END marks the end of the data. 
\paragraph{APPEND}
This  creates  a new data set containing mesh data.  If not specified
or APPEND=OFF, all existing data will be deleted.  
\paragraph{ERRORS}
Storage  is  allocated  for  DX,DY,DZ.   You  must  enter  the errors
following the values for the dependent variable.  The errors  on  the
independent variables are equal to the bin width.  

\begin{verbatim}
                               example
\end{verbatim}
READ MESH WITH ERRORS
FOR X=1,2,3
FOR Y=5
\begin{verbatim}
    Z=5,.5,6,.2,7,.1
\end{verbatim}
FOR Y=8
\begin{verbatim}
    Z=9,.1,8,.1,7,.1
\end{verbatim}
\paragraph{Bin\_spacing}
The  points/bins must be strictly monotonically spaced along the axes
for good results.  

If  a  histogram  has  unequal size bins, the last bin edge should be
specified with the option END.  If this is not done, the bins may not
be  properly  represented.   If  bin edges are not specified, The bin
edges are half way between the points.  Mesh points are assumed to be
centered within each bin.  
\paragraph{Point\_example}
To Read in an array of 3 dimensional data 
\begin{verbatim}
     READ MESH 
               Y  y1 y2 .... yn 
          X x1 Z z11 z21 ... zn1 
          X x2 Z z12 z22 ... zn2 
          ...............  
          X xm Z z1m z2m ... znm 
\end{verbatim}

or...  
\begin{verbatim}
     READ MESH 
          FOR      X= x1 x2 .... xn 
          FOR Y=y1 Z=z11 z21 ... zn1 
          FOR Y=y2 Z=z12 z22 ... zn2 
          ...............  
          FOR Y=ym Z=z1m z2m ... znm 
\end{verbatim}
\paragraph{Bin\_example}
To Read in a 4 by 4 array of 3 dimensional data 
\begin{verbatim}
     READ MESH BINS 
          FOR   Y=0.0 1.0 2.0 3.0 
          X=0.0 Z= 0 0 0 0 
          X=1.0 Z= 0 1 2 3 
          X=2.0 Z= 1 2 3 4 
          X=3.0 Z= 2 3 4 5 
\end{verbatim}

Is the same as:  
\begin{verbatim}
     READ MESH 
          FOR   Y=0.5 1.5 2.5 3.5 
          X=0.5 Z= 0 0 0 0 
          X=1.5 Z= 0 1 2 3 
          X=2.5 Z= 1 2 3 4 
          X=3.5 Z= 2 3 4 5 
\end{verbatim}
or......  
\begin{verbatim}
     READ MESH BINS 
          FOR   Y=0.0 1.0 2.0 3.0 4.0 ENDBINS 
          X=0.0 Z= 0 0 0 0 
          X=1.0 Z= 0 1 2 3 
          X=2.0 Z= 1 2 3 4 
          X=3.0 Z= 2 3 4 5 
          X=4.0 ENDBINS 
\end{verbatim}
\subsection{REPEAT}
\begin{verbatim}
     TD:REPEAT "commands" [n] [FAST[=ON|OFF]] 
\end{verbatim}
This  repeats  the  specified  commands  n  times.   The commands must be
specified as a string.  If n is omitted is assumed to be 9999.  ENDREPEAT
terminates  all  current  repeats,  while  RETURN  terminates the current
repeat.  You may nest up to 5 repeat commands.  If FAST is specified only
the  repeat  command  is  journaled  otherwise  each  repeat is journaled
separately.  If you abort the repeat with  CNTRL\_C,  or  use  the  cursor
inside  a  repeat  and  it  is in FAST mode, the journal file will not be
correct.  The lexical V\_REPEAT[n] is the repeat number 
NOTE:   The  command  SET  FILE  INPUT should not be used inside a repeat
block.  The lines from the new file will not be executed until after  the
repeat.  
(Default:FAST=ON) 
\subsubsection{Examples}
\begin{verbatim}
     TD:SET WINDOW 1 of V_SETS 
     TD:REPEAT "HIST SET=V_WINDOW;SET WINDOW NEXT;" V_SETS 
\end{verbatim}
produces histograms of all currently available data sets.  
\begin{verbatim}
     TD:DEFINE VALUE THETA=0 
     TD:DEFINE COMMAND INCR="DEV VAL THETA=<V_THETA +1>;" 
     TD:DEFINE COMMAND LEND="IF THETA .GT.  180.1 ENDREPEAT" 
     TD:REPEAT "V_THETA,<SIN(V_THETA)>;INCR;LEND" 
          or...  
     TD:REPEAT "V_THETA,<SIN(V_THETA)>;INCR" 181 
\end{verbatim}
Produces a SINE function from 0 to 180 degrees in 1 degree steps.  This
example breaks the command into small separate commands.  This  is  not
really  necessary  since  the  total  length  of  a  command may be 256
characters.  
\begin{verbatim}
     TD:REPEAT "<V_REPEAT-1>,<SIN<V_REPEAT-1>>" 181 
\end{verbatim}
Also  generates a sine function from 0 to 180 degrees, in a much faster
fashion.  
\subsubsection{Warning}


\begin{verbatim}
                                 NOTE
\end{verbatim}

\begin{verbatim}
     This  command  is  very  slow  if  FAST=OFF and JOURNALING is
     enabled.  
\end{verbatim}


\subsection{RESTORE}
\begin{verbatim}
     RESTORE {DATA|FIT|HITOGRAMS} [options] 
\end{verbatim}
Restores the selected data, fit, or histograms.  
\subsubsection{DATA}
RESTORE   DATA  [FILE=file\_name]  [APPEND[=ON$|$OFF]]  [CONFIRM[=ON$|$OFF]]
\begin{verbatim}
          [LOG[=ON|OFF]] [SETS=[FROM] n1 [TO] [n2]] [SELeCT="name"] 
\end{verbatim}
Restores  data  from  the  specified  file.   If  no  file is specified
TDSAVE.TDB is assumed.  If you have unplotted data then the new data is
automatically  appended  to  the  end of the current data as a new data
set.  The LOG option types the  number  of  points  and  sets  of  data
restored.  When restored the data name is in lexical S\_DATA\_NAME.  
\paragraph{APPEND}
Normall  the current data is deleted.  If APPEND is selected, the new
data is sets are appended to the current ones.  
\paragraph{CONFIRM}
You are asked whether you wish to restore each set.  
\paragraph{SELECT}
If the data sets are named, you may select the data set by name.  
\paragraph{SETS}
If  the  data file contains several sets of data you may select which
one you wish to restore.  
\begin{verbatim}
     TD:RESTORE DATA SET=5 
\end{verbatim}
Restores the 5'th data set from the file.  
\begin{verbatim}
     TD:RESTORE DATA SET 5 TO 8 
\end{verbatim}
Restores the 5'th through the 8'th data set from the file.  
\subsubsection{FIT}
RESTORE FIT [FILE=file\_name] 
\begin{verbatim}
     [LOG[=ON|OFF]] 
\end{verbatim}
Restore the current Topdrawer fit from a file.  If no file is specified
TDFIT.TFB is assumed.  The SAVE command is used to save a fit.  
\subsubsection{HISTOGRAMS}
RESTORE HISTOGRAMS 
\begin{verbatim}
     [IDENT=[FROM] [n] [TO n]] 
     [FILE|SECTION=name] 
     [{FETCH|READ|INPUT]}] 
     [ADD|SUBTRACT|MULTIPLY|DIVIDE] 
     [AREA|DIRECTORY[="directory"]] 
     [RELATIVE=[=ON|OFF]] [SHARE=[=ON|OFF]] 
     [TReE[=ON|OFF]] 
     [IOFSET=n] 
     [APPEND] 
     [VERSION=n] 
\end{verbatim}
Restores histograms from files create by the HBOOK package.  FETCH,READ
get histograms in the old HBOOK3 format.  Normally you do not  need  to
specify  FETCH or READ as TOPDRAWER uses the file type to determine the
operation needed to restore it.  
\paragraph{Options}
\begin{verbatim}
     1.  AREA  selects the directory of the disk file/global section.
         If  this  is  unspecified  then  the  subdirectory  on   the
         file/global  section  is assumed to be the same as specified
         by the previous SET HIST AREA command.  
     2.  FETCH  is  for histograms stored in machine dependent HBOOK3
         format by HSTORE.  Either all histograms or 1  histogram  ID
         may be fetched.  
     3.  FILE=file_name  selects the file name or global section name
         for the shared histograms.  If FILE is not specified then it
         defaults to HBOOK.BIN.  If the filename is NONE then no file
         name is selected.  
     4.  IDENT   Selects   the  ID  of  the  histogram  to  get.   If
         unspecified all histograms are fetched.  
     5.  INPUT  gets  histograms  from a direct access file in HBOOK4
         format.  This is the default.  
         NOTE  Topdrawer  checks  the file and will try to figure out
         whether to FETCH, READ, or INPUT the file.  
     6.  READ is for the machine independent HBOOK3 format.  
     7.  RELATIVE  saves  histograms  in  a  relative file for faster
         access.  Otherwise a sequential file is used.  
         (Default:RELATIVE=OFF) 
     8.  SECTION  gets  histograms from the specified global section.
         If    SECTION=""    the    current    section    is    used.
         SEE:TOPDRAWER LINK 
     9.  SHARE shares the file with other users.  SHARE=OFF prohibits
         file sharing.  
         (Default:SHARE=ON) 
    10.  VERSION  Selects the cycle (version number) of the histogram
         to get.  (INPUT only) 
    11.  IOFSET is added to the hist ID when fetched.  (INPUT only) 
    12.  APPEND Sets IOFSET=largest hist ID.  
    13.  ADD  adds  the new histograms to the current ones.  Normally
         all in the current directory are deleted.   ADD  works  only
         with INPUT, or with Global sections.  
\end{verbatim}

\begin{verbatim}
         NOTE:   If  you  have a large number of histograms resident,
         you may not have enough memory to RESTORE ADD all  at  once.
         You may force TOPDRAWER to add one at a time by:  
              RESTORE HISTOGRAM ADD ID FROM 1 TO 99999 
    14.  SUBTRACT,MULTIPLY,DIVIDE  works in a similar fashion to ADD.
         SUBTRACT subtracts the new histogram  from  the  old  one...
         These do not work for HBOOK3 histograms.  
    15.  TREE  restores  the  entire  directory tree.  (INPUT/SECTION
         only) If this is selected along with AREA it is possible  to
         graft a file resident directory tree to a new root.  
\end{verbatim}
\subsection{SAVE}
\begin{verbatim}
     SAVE  {DATA|FIT|HITOGRAMS}  [options]  This  command  saves  things.
\end{verbatim}
SAVE DATA  saves  the  Topdrawer  data  in  a  binary  format  on   disk.
SAVE HISTOGRAM saves foreign histograms on disk.  
\subsubsection{FIT}
SAVE FIT [FILE=file\_name] 
\begin{verbatim}
     [LOG[=ON|OFF]] 
     [NAME="name of data"] 
\end{verbatim}
Saves  the  current  Topdrawer  fit in a file.  If no file is specified
TDFIT.TFB is assumed.  The optional NAME is stored in  the  file  as  a
comment.  The RESTORE command is used to get saved fit.  
\subsubsection{DATA}
SAVE DATA [FILE=file\_name] 
\begin{verbatim}
     [POINTS|COLUMNS=[FROM] n1 [TO] [n2]] 
     [LINES|ROWS=[FROM] n1 [TO] [n2]] 
     [SETS=[FROM] n1 [TO] [n2]] 
     [SELeCT="name"] 
     [SLICES [X|Y|Z] [FROM] v1 [[TO] v2]] 
     [NAME="name of data"] 
     [APPEND[=ON|OFF]] 
\end{verbatim}
Saves  Topdrawer data in a file.  If no file is specified TDSAVE.TDB is
assumed.  By specifying the range of points, sets and  so  on  you  may
select which data is saved.  The optional NAME is stored in the file as
a comment.  APPEND adds the data to the end of an existing file.  

The RESTORE command is used to get saved data.  
\subsubsection{HISTOGRAMS}
SAVE HISTOGRAMS 
\begin{verbatim}
     [FILE=file_name] 
     [IDENT=[FROM] [n] [TO n]] 
     [ENTRIES[=ON|OFF]] 
     [HISTOGRAM[=ON|OFF]] 
     [RELATIVE=[=ON|OFF]] [SHARE=[=ON|OFF]] [RECLENGTH[=n]] 
     [SELECT|NAME='hist_name'] 
     [{WRITE|STORE}] 
     [UPDATE[=ON|OFF]] 
     [AREA|DIRECTORY[="directory"]] 
     [TReE[=ON|OFF]] 
\end{verbatim}
Saves  the  foreign  histograms.   Some  options apply only to specific
histogram packages.  
\paragraph{Options}
\begin{verbatim}
     1.  AREA|DIRECTORY  selects the base directory of the disk file.
         If this is unspecified then it is assumed.  to be  the  same
         as specified by the previous SET HIST AREA command.  
     2.  ENTRIES selects only histograms with entries.  
     3.  FILE=file_name  selects  the  file  name.   If  FILE  is not
         specified then it HBOOK.BIN.  If the filename is  NONE  then
         no file name is selected.  
     4.  HISTOGRAM|MESH  selects  only  regular histograms or scatter
         plots.  
     5.  RECLENGTH specifies the record length for an RZ file.  
         (default:1024 Words) 
     6.  RELATIVE  saves  histograms  in  a  relative file for faster
         access.  Otherwise a sequential file is  used.   This  works
         only for HBOOK V4.10 on a VaX.  
         (Default:RELATIVE=OFF) 
     7.  SELECT selects the name of the histograms to store.  
     8.  SHARE shares the file with other users.  SHARE=OFF prohibits
         file sharing.  This works only for HBOOK V4.10.  
         (Default:SHARE=ON) 
     9.  UPDATE=ON saves histograms in an existing file.  
         UPDATE=OFF saves histograms in a new file.  
    10.  TREE saves the entire directory tree.  (OUTPUT,PUT only) 
\end{verbatim}
(Default:STORE,UPDATE=OFF) 
\subsection{SET}
Sets options for plots.  These options usually revert to the default when
a new plot is started.  
\subsubsection{Introduction}
The  set  commands  are used to set options.  Most set commands set the
option for the duration of the current plot.  When  a  NEW  command  is
issued,  the  options  revert to the default.  You may make the current
option the default by using the option PERMANENT.  If  no  options  are
specified for the set command, the options are set back to the original
default value.  

For  example  you  wish to do a series of plots with the same limits on
the X,Y scales.   Normally  TOPDRAWER  sets  the  limits  automatically
according to your data, so you use the command:  
\begin{verbatim}
     TD:SET LIMITS X FROM 0 TO 10 Y FROM 0.5 to 1.5 PERMANENT 
\end{verbatim}
which  sets  the  limits permanently.  When you have done the plots you
wish to revert to the original default:  
\begin{verbatim}
     TD:SET LIMITS 
\end{verbatim}
Now the limits are reset to automatic limits for the current plot only.
The next plot will have the PERMANENT limits.  To have automatic limits
for all subsequent plots you must:  
\begin{verbatim}
     TD:SET LIMITS;SET LIMITS PERMANENT 
\end{verbatim}

After  using a set command you may ``see'' what you have done with a SHOW
command.  For each set command there is a SHOW command which shows  the
current  options.  For example you may see the limits you have modified
by:  
\begin{verbatim}
     TD:SHOW LIMITS 
\end{verbatim}
\subsubsection{Options}
\begin{verbatim}
     1.  ARROW - Sets default arrow format 
     2.  AXES - Sets default axis 
     3.  BAR - Sets size ends of error bars 
     4.  BOX - Sets default box size 
     5.  BLINK - Sets the BLINK attribute.  
     6.  CARD - Maximum length of input lines 
     7.  CHARACTER - The character set used 
     8.  CIRCLE - Same as SET ELLIPSE 
     9.  CLEAR - Selects deferred or immediate screen clearing 
    10.  COLOR - Sets default color or pen number 
    11.  Ctrl_Z - Enables or disables the Ctrl_Z key.  
    12.  CYCLE - The color,width, texture to cycle through.  
    13.  DATE - Set base date for date\time.  
    14.  DEVICE - Chooses I/O device 
    15.  DIAMOND - Sets default diamond size 
    16.  DIGITS - Sets number of digits for show command.  
    17.  ELLIPSE - Sets default ellipse size 
    18.  ERRORS - Changes the output of errors 
    19.  FILE - Selects files for input, output, and journaling.  
    20.  FILL - Selects the fill style.  
    21.  FLUSH - Sets automatic flush mode.  
    22.  FONT - Chooses UGSYS character set 
    23.  FORMAT - Format for input lines (256A1) 
    24.  GRID - Specifies grid marks overlay plot 
    25.  INTENSITY - Sets default intensity (line width) 
    26.  LABELS - Sets defaults for numeric labels on axes 
    27.  LIMITS - Sets X,Y,Z limits for plot (range of data to plot) 
    28.  MODE - Set misc.  functions 
    29.  MONITOR - Set type of monitor plots 
    30.  ORDER - Determines interpretation of input data 
    31.  OUTLINE - Controls outline around plot 
    32.  PATTERN - Sets the pattern for PATTERNED lines 
    33.  PAUSE - Controls pause at end of plot 
    34.  PEN - Selects the default pen or color to use in plotting.  
    35.  POLAR - Selects polar coordinates.  
    36.  PROMPT - Sets the prompt string.  
    37.  REVISION - Sets the revision level (what features TD has) 
    38.  SECONDARY - Controls the attributes of secondary contour
         lines.  
    39.  SEGMENTS - Controls breaking the plot into segments 
    40.  SHIELD - Sets an area to be shielded so that no more plotting
         is done there.  
    41.  SCALE - Controls scale (log/linear...), and the number of
         default labels and ticks.  
    42.  SIZE - Defines size of screen or paper 
    43.  STATISTICS - Sets the range of data to use in subsequent
         lexicals.  
    44.  STORAGE - Defines what is kept in storage 
    45.  SYMBOL - Sets default symbol to plot 
    46.  TEXTURE - Sets default line style (DOTTED,SOLID...) 
    47.  THREE - Sets parameters for 3-dimensional plots 
    48.  TICKS - Controls tick marks on axes 
    49.  TITLE - Sets size of title 
    50.  UNITS - Sets the units of the TEXT coordinates, and character
         sizes.  
    51.  WIDTH - Sets the line width or intensity 
    52.  WINDOW - Defines the plotting area (labels may be outside
         area) 
\end{verbatim}
\subsubsection{AREA}
\begin{verbatim}
     See:Command SET WINDOW 
\end{verbatim}
\subsubsection{ARROW}
SET ARROW [SIZE=n] [FLARE$|$FLAIR=n] [PERMANENT] 
\begin{verbatim}
     1.  SIZE=n - length of arrow head in tenths of an inch.  The units
         may be modified by SET UNITS CHARACTER.  
         (Default:2) 
     2.  FLARE=n - is ratio of the base to the height of the arrow head
         (fatness).  
         (Default:0.2) 
     3.  PERMANENT - Makes the current setting the new default.  
\end{verbatim}

If no options are specified all parameters are set to the default.  
\subsubsection{AXES}
Controls  the  presence or absence of each axes and sets the attributes
of the axis.  
SET AXES$|$AXIS [INHIBIT$|$ENABLE] [ALL$|$TOP$|$BOTTOM$|$RIGHT$|$LEFT$|$X$|$Y$|$Z] [ON$|$
\begin{verbatim}
          OFF] 
     [INTENSITY|WIDTH=n] 
     [NOCOLOR|WHITE|RED|GREEN|BLUE|YELLOW|MAGENTA|CYAN] 
     [NOTEXTURE|SOLID|DOTS|DASHES|DAASHES|DOTDASH|PATTERNED|FUNNY|
          SPACE] 
     [PERMANENT] 
\end{verbatim}

Normally  AXES  are  drawn  at the first HIST, JOIN, BARGRAPH, PLOT, or
PLOT  AXES  command  unless  inhibited  or  turned  off.   To   control
individual elements of the axes use:  
SET TICKS, SET LABELS, SET OUTLINE, SET SCALES 
\paragraph{Options}
\begin{verbatim}
     1.  INHIBIT|ENABLE  -  inhibits  or  enables  the automatic axes
         generation for the current plot or window.  (Default:ENABLE)
         You   may   still   draw  the  outline  and  axes  with  the
         PLOT OUTLINE, PLOT AXES commands.  
     2.  ALL - enables axes for top, bottom, right and left.  
     3.  TOP - enables an axis for the TOP of plot 
     4.  BOTTOM - selects the axis for the bottom of the plot.  
     5.  RIGHT - selects the axis for the right hand side 
     6.  LEFT - selects the axis for the left hand side.  
     7.  X,Y,Z  - selects the axis to plot.  For 2 dimensional plot X
         turns on both top and bottom, Y  turns  on  both  right  and
         left.  
     8.  ON - allows drawing all three 
     9.  OFF - prevents drawing outline, label and ticks.  If an axis
         is off then no  space  is  allocated  for  either  ticks  or
         labels.  
    10.  INTENSITY  - Sets line intensity or width (0-5).  0 gets the
         intensity from the SET INTENSITY command.  
    11.  WHITE...  - Sets the line color.  
    12.  SOLID...    -   Sets   the  line  texture.   (Default:SOLID)
         See:Command SET TEXTURE.  
    13.  PERMANENT  -  Makes the current settings permanent from plot
         to plot.  
\end{verbatim}
This  acts  on  the  current plot only a new SET AXES is required for
each plot or the options need to be made permanent.  The command must
be  followed  by  either a HIST, JOIN, BARGRAPH or PLOT command to be
effective.  
(Default:ALL ON) 
\paragraph{Example}

\begin{verbatim}
     TD:SET AXES ALL ON TOP OFF RIGHT OFF 
\end{verbatim}
This produces axes only on the left and bottom.  

\begin{verbatim}
     TD:SET AXES X ON Y ON Z OFF 
\end{verbatim}
This produces X,Y axes, but no Z axes on a 3 dimensional plot.  
\subsubsection{BAR}
This sets the size of the lines at the ends of error bars.  
SET BAR [X$|$Y$|$Z] [SIZE]=[n] [PERMANENT] 

n=size  in inches (Default:0.1).  If n is negative n is the fraction of
the error.  You may set the bar  length  independetly  for  X,Y,  or  Z
errors.  

PERMANENT - Makes the current setting the new default.  

If the SIZE is not specified it reverts to the default value.  
\begin{verbatim}
                                example
     TD:SET BAR SIZE=0.2 
\end{verbatim}
Sets all bars to 0.2 inches 
\begin{verbatim}
     TD:SET BAR Z SIZE=0.0 
\end{verbatim}
Sets the bar to 0.0 inches for Z errors only.  
\subsubsection{BOX}
This sets the default size for boxes.  
SET BOX [SIZE=dx[,dy]] [PERMANENT] 

PERMANENT - Makes the current setting the new default.  
(Default:1.0) 

If  dx is omitted it is set to the default.  If dy is omitted it is the
same as dx.  
\subsubsection{BLINK}
SET BLINK [ON$|$OFF] 
Text  or data plotted while this is on will blink.  This will only work
with certain devices.  The blink attrubute  is  reset  to  off  at  the
beginning of each new plot.  If ON or OFF are omitted BLINK is set ON. 
(Default:OFF) 
\subsubsection{CARD}
This  sets  the  number  of  columns per input line for non interactive
input.  
SET CARD$|$INPUT [LENGTH]=[length] 

All data past the length is ignored (Default:256) The length must be in
the range 20 to 256.  See also SET FORMAT.  This parameter remains  the
same when a new plot is started.  

You should be careful when using this command.  If you extend the input
line before extending the format, TOPDRAWER will expect multiple  input
lines per command.  

If length is omitted it is reset to the default value.  For interactive
input on the VAX the length is 256 and it is  independent  of  the  set
card command.  
\subsubsection{CHARACTER}
SET CHARACTER \{TERMINATOR$|$SEPARATOR$|$SPECIAL$|$ALPHA$|$ILLEGAL$|$COMMENT $|$
\begin{verbatim}
          ENDCOMMENT} "characters" 
\end{verbatim}
This  redefines  the  input  character  set.   You may not redefine the
letters ``A'' to ``Z'', the numbers ``0'' to ``9'', the period ''.'' or the minus
and plus signs ''-+''.  The current definitions are:  
\begin{verbatim}
     1.  SEPARATORS = ",/=" 
     2.  TERMINATOR = ";" 
     3.  SPECIAL = "@!#$%^&*~`{}|\[]" 
     4.  COMMENT = "(" 
     5.  ENDCOMMENT = "(" 
     6.  ALPHA = "_" 
     7.  ILLEGAL = all non printable characters.  
\end{verbatim}
Special characters may be imbedded inside an option, but may not be the
first character in an option.  
\subsubsection{CIRCLE}
This sets the default size for circles.  
SET CIRCLE [SIZE=dx[,dy]] [PERMANENT] 

This is same as SET ELLIPSE 

PERMANENT - Makes the current setting the new default.  
(Default:1.0) 

If dx is omitted it is set to the default.  If dy is omitted it is the
same as dx.  
\subsubsection{CLEAR}
SET \{CLEAR$|$ERASE\} \{DEFERRED$|$IMMEDIATE\} 

\begin{verbatim}
     1.  DEFERRED - After a CLEAR or NEW PLOT command the screen is not
         cleared until data is plotted.  In  other  words  the  current
         plot remains on the screen until the new plot appears.  
     2.  IMMEDIATE - The screen is cleared immediately after a CLEAR or
         NEW PLOT command.  
\end{verbatim}
\subsubsection{COLOR}
SET COLOR [WHITE$|$RED$|$GREEN$|$BLUE$|$YELLOW$|$MAGENTA$|$CYAN] 
\begin{verbatim}
     [PERMANENT] 
\end{verbatim}

(Default:WHITE)  This  sets the default color of the display or the pen
number.  This is effective only for the current plot, or until  another
SET  COLOR  or  SET PEN command.  If PERMANENT is specified the current
color is the permanent default.  See:Command SET PEN 

If no options are specified it is set to the WHITE.  
\subsubsection{COMMAND}
SET COMMAND [MAXSUBSTITUTION=n] 
This  sets  the  maximum  number  of command substitutions allowed on a
single line.  
(Default:MAXSUBSTITUTIONS=1000) 
\subsubsection{Ctrl\_Z}
\begin{verbatim}
     SE Ctrl_Z [ON|OFF] 
\end{verbatim}

Enables  or  disables  the Ctrl\_Z key as a means of stopping TOPDRAWER.
(Default:ON) 
\subsubsection{CYCLE}
SET CYCLE [NUMBER=n] [ENTRY[=n]] [WIDTH$|$INTENSITY=n] [NOCOLOR$|$WHITE$|$
\begin{verbatim}
          RED|GREEN|BLUE|YELLOW|MAGENTA|CYAN] [NOTEXTURE|SOLID|DOTS|
          DASHES|DAASHES|DOTDASH|PATTERNED|FUNNY|SPACE] [SYMBOL=sym]
          [PERMANENT] 
\end{verbatim}
This  selects a cycle of attributes to be used when the CYCLE option is
used with HISTOGRAM, JOIN, or PLOT.  
\begin{verbatim}
     1.  NUMBER  selects  the  number  of entries up to a maximum of 20
         (Default:5) 
     2.  ENTRY  selects the cycle entry to change.  If n is omitted the
         next entry is selected.  
     3.  PERMANENT saves the current settings as the default.  
     4.  SYMBOL selects the symbol to use.  
\end{verbatim}
Once  a  color,  texture,  symbol, or width are selected they remain in
effect until changed.  

\begin{verbatim}
                                example
     SET CYCLE WIDTH=3 ENTRY=3 RED ENTRY BLUE ENTRY GREEN 
\end{verbatim}
Will set entries 3 to 5 to colors RED, BLUE, GREEN and all 3 are set to
WIDTH=3.  

\begin{verbatim}
                            The default is:
\end{verbatim}
NUMBER=5 
\begin{verbatim}
     1.  BLUE DOTS SYMBOL=0o 
     2.  CYAN DOT DASH SYMBOL=1o 
     3.  GREEN DASH SYMBOL=2o 
     4.  YELLOW DAASH SYMBOL=3o 
     5.  RED SOLID SYMBOL=4o 
     6.  MAGNETA SOLID SYMBOL=5o 
\end{verbatim}
All others are WHITE SOLID SYMBOL=6o 
\subsubsection{DATE}
SET DATE [YYYY$\backslash$MM$\backslash$DD] [MONTH=n] [YEAR=n] [DAY=n] [JANUARY,...,DECEMBER] 
This  sets  the  base date for date$\backslash$time values.  Date$\backslash$time is normally
expressed in hours.  However since it is only single  precision  it  is
accurate  to only 7 digits or about 1 day on the date 1900.  If greater
accuracy is needed you may set the base date  to  a  value  within  the
range  of  time you need to express.  If you do not specify any options
the base date is reset to the original default.  
(Default:0$\backslash$1$\backslash$1) 

Options  YEAR,MONTH, or DAY allow you to reset just the selected value.
Instead of MONTH=3 you may also specify MARCH.  All 3 may  be  modified
using the standard YEAR$\backslash$MONTH$\backslash$DAY notation.  
\subsubsection{DEVICE}
This  selects  the  unified  graphics  device  and sets options back to
default values.  This is equivalent to starting a new plot  on  another
device.  
SET DEVICE [name] [``option\_string''] 
\begin{verbatim}
     [SIDEWAYS|ROTATED|ORIENTATION=n] 
     [COlOR[=ON|OFF] 
     [LANDSCAPE|PORTRAIT] 
     [FILE|OUTPUT|DDNAME=filename|CHANNEL=dev:] 
     [INTERACTIVE|SEQUENTIAL|SLAVE] 
     [ADD|UNIT=n] 
     [IDENT=aaaa] [NUMBER=nnnn] [PERMANENT] [WIDTH=n] [HEIGHT=n]
          [REVERSE] 
\end{verbatim}
This selects the device name.  It is modified by the option\_string.  If
omitted the default is used.  See:PERMANENT.  

(Interactive Default:TEKTRONIX) 

(Batch Default:POSTSCRIPT) 
\paragraph{Names}
\begin{verbatim}
     1.  TEST - null device 
     2.  CALCOMP - Calcomp plotters 
     3.  PRINTRONIX - Printronix printer plotter 
     4.  QMS-1200 - QMS Lasergrafix 1200 laser printer 
     5.  EXCL - Talaris 1590 printer 
     6.  REGIS  -  Dec VT-24x, Dec GIGI, and other devices supporting
         REGIS protocol.  
     7.  SIXEL  - Dec VT-240, LN-03, LA-50, LA-100 , Talaris 1590 and
         any other device using SIXEL graphics.  
     8.  TEKTRONIX  -  Tektronix  4010/4014  terminals  or emulators.
         This includes:  Dec VT-240,  Terminals  with  retrographics,
         Selanar 100XL, Visual 102, CIT-467, Talaris 1590, Falco, and
         Macintosh with VERSATERM.  
     9.  TEKEMUL  -  Tektronix  4010/4014  emulators,  using the SLAC
         driver.  
    10.  TEK4027 - Tektronix 4027 color terminal 
    11.  GKS - Interface To DEC GKS package 
    12.  GRINNELL - Grinnell graphics device 
    13.  IMAGEN - Imagen laser printer 
    14.  POSTSCRIPT - Postscript printer 
    15.  GPOSTSCRIPT - Postscript driver using DEC-GKS 
    16.  EPS - Encapsulate postscript driver using DEC-GKS 
    17.  HPGL - Hewlett packard plotters 
    18.  XWINDOW - Vax station using DECwindows.  
\end{verbatim}
Note:   Some  of these device drivers may not be available.  You will
get an error message if they do not exist.  
\paragraph{COLOR}
Selects the DEC GKS color driver rather than the Monochrome driver.  
\paragraph{PORTRAIT}
\begin{verbatim}
     LANDSCAPE or PORTRAIT 
\end{verbatim}
Select  the  orientation  of  the plot on paper.  This option applies
only to printer drivers.  This applies to Gpostscript, EPS,  QMS1200,
EXCL,  Versatek,  IMAGEN  and  all  DEC-GKS drivers that support page
oreintation.  

It  does  not apply to workstation windows or TEKtronix graphics.  To
rotate them use the SIDEWAYS option, or the ORIENTATION option.  

(Default:LANDSCAPE) 
\paragraph{TEST}
This  is  a  null  device.  You must specify the device driver in the
``option\_string''.  
\paragraph{CALCOMP}
1.  SMALL CALCOMP - 10inch Calcomp 
2.  LARGE CALCOMP 
\paragraph{EXCL}
This  if  for  laser printers using Talaris extended control language
(EXCL).  This includes the Talaris 1590 printstation.  This device is
capable  of  intensity (line width) variation, and erasing data.  The
line width is the intensity/150 inches.  It produces a plot  11.0  by
8.5 inches in size with 0.25 inch margins.  

The  advantage  of  this  driver  over  TEK4010 on the Talaris is the
number of line widths, and the correct reproduction of scatter plots.
The  disadvantage  is  that  the file size is larger, and you can not
send the file to a TEK-4010 emulator, to see the contents.  
\begin{verbatim}
                               Warning
\end{verbatim}
Previously  it  was  stated  that this will work on LN-03+.  Due to a
misunderstanding this is not true.  For LN-03 printers use SIXELS  or
TEK-4010 for LN-03+.  
\paragraph{Options}
The following options may be specified inside the string:  
\begin{verbatim}
     1.  DM=n - The default margin in cm.  (Default:0.25in.) 
     2.  TM=n - The top margin in cm.  (Default:DM) 
     3.  LM=n - The left margin in cm.  (Default:DM) 
     4.  BM=n - The bottom margin in cm.  (Default:DM) 
     5.  RM=n  -  The  right  margin  in  cm.   (Default:DM) If the
         margins are set to  0,  then  no  margin,  or  orientation
         commands are in the output file.  
     6.  PENLIST=n1n2n3n4n5  -  maps penwidths to number of pixels.
         Setting is number from 01 to 99.  1 pixel=0.0033  in.   If
         the   penwidth   is   00  then  the  default  is  assumed.
         (Default:0204060810) 
     7.  WIDTH=size   -   The   width   of   the   paper   in   cm.
         (Default:8.5in.) 
     8.  LENGTH=size   -   The   length   of   the   paper  in  cm.
         (Default:11.0in.) 
     9.  inches  -  The  WIDTH/LENGTH/margins  must be expressed in
         inches rather than cm.  
    10.  ROTAXIS - Rotates the axis from landscape to portrait.  
\end{verbatim}
\paragraph{Examples}
\begin{verbatim}
     TD:SET DEVICE EXCL 
\end{verbatim}
Selects the EXCL driver.  
\begin{verbatim}
     TD:SET DEVICE EXCL "DM=1,LM=1.5,inches,ROTAXIS" 
\end{verbatim}
Selects  the  EXCL driver, and sets 1 inch margins on the left,top,
and bottom, with 1.5in.  margins on the left.   The  plot  will  be
produced on 8.5x11 paper with portrait orientation.  
\begin{verbatim}
     TD:SET DEVICE EXCL "DM=.375,PENLIST=0202020404" 
\end{verbatim}
Produces plots approximately the same as 
\begin{verbatim}
     TD:SET DEVICE TEK "TAL1590" SEQUENTIAL 
\end{verbatim}
\paragraph{GKS}
This is an interface to the DEC GKS package.  
It  allows  access  to a whole class of devices available through DEC
GKS.  
\paragraph{Options}
\begin{verbatim}
                              Options
     1.  WSTYPE=nnn - Selects the DEC GKS workstation If you do not
         specify a WSTYPE then the logical GKS$WSTYPE  is  used  to
         define   the  work  station.   Normally  you  specify  the
         workstation name.  
     2.  ERRLUN  -  Selects an output logical unit for GKS messages
         (Default:6) 
     3.  OUTLUN  -  Selects  an  output logical unit for GKS files.
         You normally do not specify this.  
     4.  WIDTH=n.n - Window width in cm (Default:25.4 (10 inches)) 
     5.  HEIGHT=n.n   -  Window  height  in  cm  (Default:20.32  (8
         inches)) 
     6.  XSCREEN=n.n - X Location of lower left corner of window in
         cm (Default:0) 
     7.  YSCREEN=n.n - Y Location of lower left corner of window in
         cm (Default:0) 
     8.  XSIZE  -  Actual  physical  X  size  of  screen  if  known
         (Default:   size  reported  by  GKS  or  10   inches   for
         metafile).  Normally you do not need to set this.  
     9.  YSIZE  -  Actual  physical  Y  size  of  screen  if  known
         (Default:  size reported by GKS or 8 inches for metafile).
         If XSIZE is picked, but Y is not then the YSIZE is assumed
         to scale as the selected XSIZE 
    10.  REVERSE - Sets background to black with white foreground. 
    11.  color=rrggbb  - Specifies the mixture of red,green,blue to
         mix to obtain the desired color.  Each value may vary from
         0 to 99.  The default colors are:  
         A.  BLACK=008900 
         B.  WHITE=999999 
         C.  RED=992020 
         D.  GREEN=009900 
         E.  BLUE=202099 
         F.  YELLOW=998000 
         G.  MAGENTA=992099 
         H.  CYAN=008000 
    12.  DDNAME=filename - Specifies the output file name.  
    13.  PENLIST=n1n2n3n4n5 - maps penwidths to .01 cm.  from 01 to
         99.  (Default:PENLIST=0103050709) 01 is  the  minimum  pen
         width.  
\end{verbatim}
If both OUTLUN and DDNAME are unspecified then logical GKS\$CONID is
used to specify the output device.  
\paragraph{Devices}
In addition a whole set of devices are defined to interface through
DEC GKS.  You may specify these instead of  GKS.   If  it  is  both
sequential  an  interactive you may specify SEQUENTIAL.  Sequential
output is sent to the default file if you have not specified a file
name (FILE=xxxx).  These are:  
\begin{verbatim}
     1.  METAFILE WSTYPE=2 (SEQUENTIAL) File=UGDEVICE.AGM 
     2.  CGM WSTYPE=7 (SEQUENTIAL) File=UGDEVICE.CGM 
     3.  VT125 WSTYPE=11 (SEQUENTIAL,INTERACTIVE) File=UGDEVICE.SIX 
     4.  VT125BW WSTYPE=12 (INTERACTIVE) 
     5.  VT240 WSTYPE=13 - VT-240 color (INTERACTIVE) 
     6.  VT240BW  WSTYPE=14  -  VT-240  in  Black  and  white  mode
         (INTERACTIVE) 
     7.  LCP01 WSTYPE=15 (SEQUENTIAL) File=UGDEVICE.SIX 
     8.  VT330 WSTYPE=16 (INTERACTIVE) 
     9.  VT340 WSTYPE=17 (INTERACTIVE) 
    10.  VSXXX  WSTYPE=41  -  A  variety  of VAX stations using UIS
         (INTERACTIVE) 
    11.  LVP16A WSTYPE=51 (SEQUENTIAL) File=UGDEVICE.SIX 
    12.  LVP16B WSTYPE=52 (SEQUENTIAL) File=UGDEVICE.SIX 
    13.  HP7550 WSTYPE=53 (SEQUENTIAL) File=UGDEVICE.HP 
    14.  HP7580 WSTYPE=54 (SEQUENTIAL) File=UGDEVICE.HP 
    15.  HP7585 WSTYPE=56 (SEQUENTIAL) File=UGDEVICE.HP 
    16.  GPOSTSCRIPT WSTYPE=61 (SEQUENTIAL) File=UGDEVICE.PS 
    17.  EPS WSTYPE=65 (SEQUENTIAL) File=UGDEVICE.EPS 
    18.  TEK4107         WSTYPE=83         (SEQUENTIAL,INTERACTIVE)
         File=UGDEVICE.TEK 
    19.  TEK4207         WSTYPE=83         (SEQUENTIAL,INTERACTIVE)
         File=UGDEVICE.TEK 
    20.  TEK4128         WSTYPE=85         (SEQUENTIAL,INTERACTIVE)
         File=UGDEVICE.TEK 
    21.  TEK4129         WSTYPE=85         (SEQUENTIAL,INTERACTIVE)
         File=UGDEVICE.TEK 
    22.  LJ250 (SEQUENTIAL) File=UGDEVICE.LJ 
    23.  LJ250_180DPI (SEQUENTIAL) File=UGDEVICE.LJ 
    24.  XWINDOW     WSTYPE=211     -    X    or    (DEC)    window
         (SEQUENTIAL,INTERACTIVE)    This    uses    the    logical
         DECW$DISPLAY  to  specify  the connection.  To do this you
         must:  
              $ DEFINE DECW$DISPLAY nodename::0.0 
\end{verbatim}
For example you may select interactive XWINDOWS (DEC windows) by:  
\begin{verbatim}
     TD:SET  DEVICE  XWINDOW  For  the  same  display rotated by 90
\end{verbatim}
degrees:  
\begin{verbatim}
     TD:SET  DEVICE  "XWINDOW SIDEWAYS" To select VT125 output to a
\end{verbatim}
file named ``VT125.PLOT'' 
\begin{verbatim}
     TD:SET DEVICE VT125 SEQUENTIAL FILE=VT125.PLOT 
\end{verbatim}
\paragraph{Examples}
\begin{verbatim}
     TD:SET DEVICE TEK4014 
\end{verbatim}
For a Tektronix 4014.  
\begin{verbatim}
     TD:SET DEVICE SEQUENTIAL TEK4010 DDNAME=TEK4014.PLOT 
\end{verbatim}
For a Tektronix 4014 plot sent to a file.  
\begin{verbatim}
     TD:SET DEVICE TEK4010 DDNAME=TKA0:  
\end{verbatim}
For  a  Tektronix  4014  plot  sent  to  a  TEK-4014  window  on  a
VAXstation.  
\begin{verbatim}
     TD:SET DEVICE VSX "RED=990000,GREEN=009900,BLUE=009900' 
\end{verbatim}
For a plot on a VAXstation with ``pure'' colors.  
\begin{verbatim}
     TD:SET DEVICE HP7585 
\end{verbatim}
For a Hewlett Packard 7578 plotter 
\paragraph{Assignments}
If you wish to default to GKS then you should 
\begin{verbatim}
     DEFINE PLOT_TERM "UGGKSWM,[options]" 
\end{verbatim}
\paragraph{GRINNELL}
This is a slave raster graphics device only.  Options are:  
\begin{verbatim}
     1.  CHANNEL=<value>   -   Identification   of   I/O   controller
         (Default:GRA0) 
     2.  UNITS=<value> - Selects the monitors to use.  This is sum of
         the display units (1,2,4...).  (Default:1) 
     3.  BWTYPE=<value>  -  Selects  black  and  white  monitor type.
         (Default:1) 
                   1 = 1 memory plane no look-up 
                   2 = 2 memory planes no look-up 
                   3 = 2 memory planew with look-up 
     4.  COLTYPE=<value> - Selects the type of color monitor.  
                   1 = 4 memory planes (7 colors 2 levels) 
                   2 = 4 memory planes (7 colors + blink mode) 
                   3 = 3 memory planes (color, intensity, blink) 
                   4 = Same as 2 but with look up table.  
     5.  XMAX=<value> Maximum raster value (Default:511) 
     6.  YMAX=<value> Maximum raster value (Default:511) 
     7.  CSIZ=<value> Character spacing.  (Default:7) 
     8.  EXTCHR  -  Indicates  the controller produces extended char.
         set.  
     9.  REPCHR - Character data replaces existing data 
    10.  XSIZ=<value> - Size of monitor in cm (Default:25.4) 
    11.  YSIZ=<value> - Size of monitor in cm (Default:25.4) 
\end{verbatim}

\paragraph{HPGL}
This is for a generic Hewlett Packard plotter.  The plotter pauses in
between each plot to wait for a new sheet of paper.  The  colors  you
select are mapped to pen numbers as:  
\begin{verbatim}
     1.  WHITE 
     2.  Red 
     3.  Green 
     4.  Blue 
     5.  Yellow 
     6.  Magenta 
     7.  Cyan 
\end{verbatim}
Options are:  
\begin{verbatim}
     1.  WIDTH=n.n - Paper width in cm (Default:26.0) 
     2.  HEIGHT=n.n - Paper height in cm (Default:20.0) 
     3.  XSCREEN=n.n - X Offset of plot in cm (Default:0.0) 
     4.  YSCREEN=n.n - Y Offset of plot in cm (Default:0.0) 
     5.  UNITS=nn.n - Number of pixels/cm (Default:400) 
     6.  MAXCOLOR=n  - Maximum number of pens (Default:7) This may be
         set within the range of 1 to 7.  All pens  requested  larger
         than this value will revert to to the maximum.  
     7.  PENLIST=n1n2n3n4n5  -  maps penwidths to .01 cm.  from 01 to
         99.  (Default:PENLIST=0103050709)  01  is  the  minimum  pen
         width.  
     8.  GENIL  simulate  intensity  levels  by doing multiple passes
         with the pen.  If not specified then the PENLIST is ignored. 
     9.  PENWIDTH=n.n   -   The   width  of  the  pen  points  in  cm
         (Default:0.03) 
\end{verbatim}
If you wish to default to HP plotter output:  
\begin{verbatim}
     $ DEFINE PLOT_TERM "UGHPGLD[,options]" 
\end{verbatim}

These options may work differently depending on the plotter.  Most of
the big plotters have the origin at the center of tha page, while the
small  ones  use  an  origin  at a corner.  By suitably adjusting the
WIDTH,HEIGHT,XSCREEN,YSCREEN any coordinate system or page  size  may
be accomadated.  MAXCOLOR=1 is useful when you wish to reduce a color
plot to a single color.  

\begin{verbatim}
                               example
     TD:SET DEVICE HPGL "PENLIST=0309152430" 
\end{verbatim}
Sets up to draw with much wider lines.  
\paragraph{IMAGEN}
This  is  for an IMAGEN laser printer using the IMPRESS language.  It
has a resolution of 1/3150 in the horizontal direction.  Options are: 
\begin{verbatim}
     1.  ROTAXIS 
     2.  BIGMAR 
     3.  GENIL 
     4.  EXPA=LETTER or LEGAL 
\end{verbatim}
The default is to send output to UGS\$IMGN300:  
\paragraph{POSTSCRIPT}
This  is  for  output  on a postscript printer It has a resolution of
1/3387 in the horizontal direction.  The default output file name  is
UGDEVICE.PS 
\paragraph{PRINTRONIX}
Printronix model MVP printer/plotter.  
\paragraph{QMS-1200}
This is the QMS Lasergrafix 1200 laser printer The options are:  
\begin{verbatim}
     1.  FLUNIT=n - Logical unit number (Default:29) 
     2.  TRAY=nnnn  -  4  character  code to specify the tray/stacker
         0000=upper tray, 0001=lower tray, XXXX=keep old setting.  
     3.  PENLIST=n1n2n3n4n5  - maps penwidths to setting.  Setting is
         odd number from 03 to 99.  (Default:0303050709) 
     4.  LANDSCAPE,PORTRAIT - Selects orientation 
     5.  LM,RM,BM,TM=n - Selects margins in inches (Default:0.9) 
     6.  NR,NC=n - Selects number of rows/colums of subplots on page.
         For example NR=2,NC=3 will plot 6 pictures on 1 page.  
         (Default:NR=1,NC=1) 
\end{verbatim}
\paragraph{REGIS}
This  is  for  interactive  GIGI,  VT-240,  VT-241  color monitor, or
suitable emulator such as  the  VT-300  emulation  on  a  VAXstation.
(note:VT-241   can   only  reproduce  the  colors  RED,GREEN,BLUE  or
WHITE,RED,GREEN).  The TEKTRONIX  driver  is  slightly  faster  on  a
VT-240  than  the  REGIS  driver.  The physical screen size is set to
10.56 by 6.6 inches or 26.85 by  16.76  cm.   These  devices  support
Hardware  structure  and  hardware  characters in any size.  The GIGI
also supports BLINK mode.  

\paragraph{Options}
\begin{verbatim}
     1.  COLOR=xxxxxxxx    This   specifies   the   color   mapping
         (default:WRGBYMCD)  You  may  modify  this  to  suit  your
         tastes.  This corresponds to the colors White, Red, Green,
         Blue, Yellow,  Magenta,  Cyan,  Dark(black).   The  eighth
         color "D" is the background color.  
     2.  HARDST  or  SOFTST  to  select  hardware, or software line
         structure.   HARDST  is  faster,  but  SOFTST  will   give
         identical results on all devices.  
\end{verbatim}

\begin{verbatim}
                              Example
     TD:SET DEVICE REGIS 
\end{verbatim}
Sets up to plot on a normal REGIS device 
\begin{verbatim}
     TD:SET DEVICE REGIS "HARDST,COLOR=WWWWWWWD" 
\end{verbatim}
Sets up to plot quickly in black and white only.  
\begin{verbatim}
     TD:SET DEVICE REGIS "COLOR=WRGBRGBD" 
\end{verbatim}
Plots only in white,red,green, and blue.  Yellow, magenta, and cyan
are mapped to red, green, and blue respectively.  
\paragraph{Answerback}
You  should  set the answerback to be ``DECGIGI,COLOR=WRGWWWWD,$<$CR$>$''
for a VT241 terminal.  
\paragraph{Windows}
Normally  when  you do plots in REGIS the cursor is left at the top
of the screen.  If your terminal is capable of more than 24  lines,
you  may  wish  to  freeze the top 24 lines, and create a scrolling
region underneath the plot.  Assume your terminal has more than  24
lines  and  less  than  100.  This is possible if it is an emulated
terminal using a windowing system.   To  move  the  cursor  to  the
bottom of the screen:  
\begin{verbatim}
     TD:SET PROMPT BOTTOM,'TD:' 
\end{verbatim}
To freeze or thaw the top few lines:  
\begin{verbatim}
     TD:define            command            freeze           'type
\end{verbatim}
27,``7'',27,''[24;r'',27,``8'',27,``M''' 
\begin{verbatim}
     TD:define command thaw 'type 27,"7",27,"[0;r",27,"8",27,"M"' 
\end{verbatim}
You  issue  the command FREEZE to freeze the top lines, and THAW to
unfreeze them.  You will want to FREEZE when you are  doing  plots,
and  the  thaw when you are showing data, or parameters, so you can
use the whole window.  

For your convenience you may do all of this by:  
\begin{verbatim}
     TD:SET FILE INPUT TOPDRAWER_DIR:FREEZE 
\end{verbatim}

Now  when  you  do plot the top part of your screen is used to view
the plot while the bottom is used to give the commands.  

If  your  screen  has only 24 lines you may wish to reduce the plot
size and freeze the top 22 lines.  This will  give  you  a  2  line
command window.  To do this:  
\begin{verbatim}
     SET FILE INPUT TOPDRAWER_DIR:FREEZE22 
\end{verbatim}
Similarly there is a FREEZE30 and FREEZE40 file.  
\paragraph{SIXELS}
This  is  for  the  LA-50, LA-100, VT-240, LN-03 and any other device
that uses sixel graphics.  The size is about 10 by 8'' except  for  an
LA-100 which is 14 by 10.  To specify this device:  
\begin{verbatim}
     TD:SET DEVICE SIXELS "type,options" 
\end{verbatim}
Where  type  is LA50, LA100, VT240, LN03, LN03HI, LN03LO, or LN03MED.
HI,LO,MED select the resolution for the LN-03 printer.  If you do not
have  lots  of  memory  inside  the  LN-03  you  shoule use LN03LO or
possibly LN03MED.  If you wish to get sixel output on a Talaris  1590
printer  You  specify  option  LN03,  and  then  print  the file with
/SET=LN03.  
\begin{verbatim}
                               example
     TD:SET DEVICE SIXEL "LN03" 
\end{verbatim}
\paragraph{TEKTRONIX}
Tektronix 4010/4014 compatible terminals 
\begin{verbatim}
     TD:SET DEVICE TEKTRONIX "type,options" 
\end{verbatim}
\paragraph{Type}
1.  ``RADM3A'' - For ADM-3A+Retrographics terminal 
2.  ``RVT100'' - A VT-100+Retrographics terminal 
3.  ``RVT100A'' - High resolution Retrographics (Green screen).  
4.  ``SEL100XL'' - For a Selanar HiREZ 100XL 
5.  ``VIS102'' - For a Visual 102 with graphics.  
6.  ``KERMIT'' - Kermit simulation of TEK-4010 with color 
7.  ``LSI7107'' - LSI model 7107 color terminal 
8.  ``TEK4207'' - For Tektronix model 4207 with Dialog box.  
9.  ``TEEMTALK''   -  For  Tektronix  4207  emulators  which  support
simultaneous Tek and terminal windows.  
10.  ``VT240'' - For a VT-240 terminal, or C.ITOH 328 For a VT-240 or
241 terminal you will get better results using the REGIS  protocol.
Both TEK and REGIS are supported on the VT-240.  When using the TEK
protocol on a VT-240 the keypad is disabled in graphics  mode,  and
the screen is erased when you leave graphics mode.  
11.  ``CIT467'' - for a CIT-467 color graphics terminal 
12.  ``TAL1590'' - for a Talaris 1590 laser printer.  This produces a
plot approximately 10.24 by 7.8 inches with 3/8 inch margins.  
If you do not specify a terminal type, then a plain vanilla 4010 is
assumed.  
\paragraph{Terminal\_setup}
It  is important that your terminal be setup properly for Tek-4010.
The following settings must be observed.  
\begin{verbatim}
     1.  <CR> should generate only <CR> 
     2.  <LF> should generate only <LF> 
     3.  GIN  terminator  must  be  <CR>.  In particular you do not
         want <EOT> as a terminator.  
     4.  <DEL>  must be enabled as a valid graphics character or it
         implies Lo Y.  
     5.  You may wish to enable Automatic Tek entry.  
     6.  You probably do not want screen clear when Tek is entered. 
\end{verbatim}
\paragraph{Falco}
The  Falco  infinity  series  can  display up to 49 lines of FBG-II
graphics.  You should calculate the number of available y pixels as 
\begin{verbatim}
  For FBG-II 
     pixels=3120*(lines/49) 
  For FBG-I 
     pixels=3120*(lines/23) 
\end{verbatim}
Then you set:  
\begin{verbatim}
     TD:SET DEVICE TEKTRONIX "RVT100,YPIXELS=pixels" 
\end{verbatim}
\paragraph{Options}
\begin{verbatim}
     1.  "HARDST"   or   "SOFTST"  select  hardware  structure,  or
         software line structure.  HARDST  is  faster,  but  SOFTST
         will give identical results on all devices.  
     2.  "HIRES"  or  "LORES" select the resolution (4096 or 1024).
         HIRES is slower, and will only work on some devices.   For
         example HIRES messes up the plots on a Visual-102.  
     3.  "HARDCH"  or  "SOFTCH"  selects  the  range  of  character
         generation.   HARDCH  generates  4  character   sizes   by
         hardware.  SOFTCH generates only 1.  
     4.  "NOINTEN"  or  "INTEN"  selects intensity variation.  This
         only works on "real" 4010 terminals,  and  VT-240.   There
         are only 2 levels of variation 1,2,3 or 4,5.  
     5.  "HEIGHT=n"   selects   the  screen  height  in  cm.   This
         componsates for emulators without "good" aspect ratios.  
     6.  "YPIXELS=n"  selects  the number of Y pixels for emulators
         with a different number from the standard 3120.  This  may
         be  in  the range of 1024 to 4096.  Xpixels are assumed to
         be 4096.  
     7.  "WIDTH=n" selects the screen width in cm.  
     8.  "SLOW" selects full rather than compressed output for each
         point.  All redundant characters are transmitted.  
     9.  "RECSIZE=nnn" selects the record size (72-512) 
    10.  "RSEGM"  or  "NSEGM" - Selecte [no]return to terminal mode
         at end of each segment.  This is useful for terminals that
         support  simultaneous  ANSI, and Tek graphics on screen at
         the same time.  
    11.  "COLOR=WRGBYMCB"   -   specifies   the   8  color  mapping
         characters for White, Red, Green, Blue,  Yellow,  Magenta,
         Cyan,  Black.   Each  escape  sequence has a unique single
         character to select color, which may be specified  by  the
         color  option.   Spaces  indicate keep the color the same.
         For example KERMIT uses "COLOR=71243560" 
    12.  "PRECOL=sequence"  specifies  the "escape" sequence to use
         before the  color  character.   For  example  KERMIT  uses
         "PRECOL=[3" 
    13.  "POSTCOL=sequence" specifies the sequence to use after the
         color character.  For example KERMIT uses "POSTCOL=m" 
    14.  "ENTER=sequence"  specifies the "escape" sequence to enter
         TEK-4010 mode from terminal mode.  For example VT-240  and
         KERMIT use "ENTER='[?38h'" 
    15.  "EXIT=sequence"  specifies  the  "escape" sequence to Exit
         TEK-4010 mode to terminal mode.  This sequence is executed
         whenever  the  screen  is  erased  or  at  the end of each
         segment if RSEGM is  selected.   For  example  VT-240  and
         KERMIT use "EXIT='[?38l'" 
\end{verbatim}

NOTE:   It  is  assumed  that the escape character ``27'' begins each
escape sequence, and it need  not  be  entered.   Escape  sequences
containing lowercase letters need to be inside quotes.  In addition
escape sequences are limited to  7  characters  and  they  may  not
contain blanks.  
\paragraph{Table\_of\_types}
Type      Width  Height Features
TEK4010   19.97  15.21
TEK4014   19.97  15.21  HIRES,HARDCH,INTEN
TEK4207   19.97  15.21  HIRES,HARDCH,INTEN,EXIT,COLOR
TEEMTALK  19.97  15.21  HIRES,HARDCH,INTEN,EXIT,RSEGM,COLOR
VT240     21.59  16.45  HARDCH,INTEN,EXIT
SEL100XL  22.00  16.76  HIRES,HARDCH,RSEGM
VIS102    21.66  16.50  EXIT,RSEGM
RVT100    21.09  15.76  EXIT,RSEGM
RVT100A   21.09  15.76  EXIT,RSEGM
RADM3A    19.95  15.21  EXIT,RSEGM
CIT467    19.97  15.21  EXIT,RSEGM,COLOR
TAL1590          25.92  19.75 HIRES,HARDCH,INTEN
KERMIT    19.97  15.21  HIRES,HARDCH,INTEN,EXIT,COLOR
LSI7107          19.97  15.21 HARDCH,EXIT,COLOR
All use HARDST.
\paragraph{Windows}
Sone  TEK4010  emulators  are  capable of simultaneous graphics and
text where the graphics scrolls off of the screen.  An  example  of
such  a  terminal is the Visual-630.  For this type of terminal you
may wish to reduce the plot size and freeze the top 22 lines.  This
will give you a 2 line command window with a fixed graphics window.
To do this:  
\begin{verbatim}
     TD:SET FILE INPUT TOPDRAWER_DIR:FREEZE22 
\end{verbatim}
When you wish to use the whole window for listings you type:  
\begin{verbatim}
     TD:THAW 
\end{verbatim}
and then to restore the 2 line command window:  
\begin{verbatim}
     TD:FREEZE 
\end{verbatim}
\paragraph{Escape\_seq}
Each  TEK4010  emulator  uses  slightly different escape sequences.
The following is a listing of the sequences which differ  from  one
device to another.  

\begin{verbatim}
     1.  ENTER Tek mode 
         A.  <ESC>[?38h - VT240,Kermit 
         B.  <ESC>#!1 - TEK4207 
         C.  <GS> - Most other devices 
\end{verbatim}

\begin{verbatim}
     2.  EXIT Tek mode or Resume ANSI terminal mode 
         A.  <ESC>[?38l - VT240,Kermit 
         B.  <ESC>#!0 - TEK4207 
         C.  <ESC>2 - CIT467 
         D.  <CAN> - most other devices 
\end{verbatim}

\begin{verbatim}
     3.  Set color (X=single color character) 
         (White,Red,Green,Blue,Yellow,Magenta,Cyan,Black) 
         A.  <ESC>[3Xm - Kermit X=71243560 
             The KERMIT color emulation uses <ESC>[3F;Bm 
             Where   F  is  the  Foreground  color  and  B  is  the
             background.  
         B.  <ESC>MLX - TEK4207 X=12347650 
             Alternate colors for TEK4207 are:  
             <         ;         :         9 
             Violet    Lt.Blue   Blue-Grn  Yellow-Grn 
             8         =         >         ?  
             Orange    Salmon    Dk.Gray   Lt.Gray 
         C.  <ESC>X - CIT467 X=oiljmknh 
         D.  <ESC>CX  - LSI7107 X=71243560 The LIS7107 additionally
             must return to terminal mode to setup a new color.  
\end{verbatim}

\begin{verbatim}
     4.  Set mode ERASE/DRAW 
         A.  Color terminals use color black 
         B.  <ESC><RUB>/<ESC>a - RADM3A 
         C.  <ESC>1d/<ESC>0d - RVT100,VIS102 
         D.  <ESC>OW b` @@/<ESC>OW `` @@ - SEL100XL 
                          Codes (decimal)
\end{verbatim}
3.  $<$ESC$>$ = 27 
4.  $<$CAN$>$ = 24 
5.  $<$GS$>$ = 29 
6.  $<$RUB$>$ = 127 
\paragraph{Example}
\begin{verbatim}
     TD:SET DEVICE TEK "SEL100XL" 
\end{verbatim}
Selects the SEL100XL terminal protocol.  
\begin{verbatim}
     TD:SET DEVICE TEK "SEL100XL,SOFTST,LORES" 
\end{verbatim}
Selects   the  SEL100XL  but  in  1024  resolution,  with  software
structure.  This produces faster plots that look identical to those
produced by hardcopy devices that do not have hardware structure.  
\paragraph{Hints}
Some  terminals  such  as  the  KERMIT  emulation  do  not  support
simultaneous display of graphics and text.  This may  be  fixed  in
several  ways.   You  can  specify  the  ``RSEGM''  option,  and do a
SHOW CURSOR whenever you wish to look at the graphics page.  

\begin{verbatim}
     Some  terminals  such as the CIT328 , and the VT-240 clear the
\end{verbatim}
display when switching from graphics to text.  For such devices the
``NSEGM'' options should be specified.  This returns to terminal mode
only when the device is closed, or when the screen is erased.  

\begin{verbatim}
     You  may  modify  the  actual  color mapping by specifying the
\end{verbatim}
appropriate escape sequences.  For example for KERMIT you may  draw
black against a white backround by specifying:  
\begin{verbatim}
     SET DEVICE TEKTR "KERMIT,COLOR=01243567,POSTCOL=';7m'" 
\end{verbatim}
You  may  modify the color mapping for TEK4027 to substitute Violet
for blue.  
\begin{verbatim}
     SET DEVICE "COLOR=' <' 
\end{verbatim}
\paragraph{TEK4027}
This is for the Tektronix 4027 color terminal.  
\paragraph{VERSATEC}
Versatec  1200 printer/plotter.  The plotting space is 10.55'' by 7.8''
(26.8cm by 19.81cm) 2110 by 1560 pixels.  Plotting  nominally  begins
0.35''  from  the perforations.  The 7.8'' is nominal due to variations
in the paper feed.  There is always a margin of 0.225'' on  the  short
side of the paper.  The following options may be specified inside the
string.  
\begin{verbatim}
     1.  "LENGTH=n,inches"  - n is the length of the paper in inches.
         This allows you to make long plots on roll paper,  or  plots
         that  bleed across the page boundaries.  You can also make a
         plot smaller using this parameter.  
\end{verbatim}

\paragraph{X-windows}
SET  DEVICE  XWINDOWS  for an X window display.  If you wish to serve
the X-windows to another node on DECnet (HEPNET) you must declare the
node before entering Topdrawer, by entering the line:  
\begin{verbatim}
     $ SET DISPLAY/CREATE/NODE=nodename 
\end{verbatim}
If  you  do  not  have a DECnet connection, but you are connected via
TCP/IP (Arpanet or Internet)  you  must  use  one  of  the  following
commands:  

If you have Multinet TCP/IP software:  
\begin{verbatim}
     $ SET DISPLAY/CREATE/NODE=nodename/TRANSPORT=TCPIP 
\end{verbatim}
If you have Wollengong TCP/IP software (WINTCP):  
\begin{verbatim}
     $ SET DISPLAY/CREATE/NODE=nodename/TRANSPORT=WINTCP 
\end{verbatim}
\paragraph{String}
This  is  passed  to  UGOPEN.   This selects options for the selected
device.  The options are separated by commas.  For  more  information
see:UGSYS.  An example of using the string is:  
\begin{verbatim}
     TD:SET DEVICE SIXELS "LN03HI,NOCLEAR,NORESET,LENGTH=n,inches" 
\end{verbatim}

This selects output for the LN-03 with no imbedded formfeeds or reset
commands.  You might produce a plot in this form so you can add it to
an existing text file.  The result is a plot imbedded in the text.  

LENGTH=n  adjusts  the page length where n is the length of the paper
in inches.  Shortening the plot size is useful  for  imbedded  plots.
You may also lengthen the page length on the LA50 and LA100 printers. 

If the option inches is omitted the length is in centimeters.  
\paragraph{ADD}
This  adds  another  device  so  now you have 2 device open.  You may
select the unit number by using UNIT=n.  
\paragraph{FILE}
FILE=filename  -  This  selects  the  device and filename for output.
This option only works for non-interactive or sequential devices.  If
you  wish  to  specify  a  file  name containing either semicolons or
blanks, it must be enclosed in parenthesis.  

CHANNEL=DEV:   -  This  selects  the  output device.  For interactive
devices or slave devices this is your terminal by default.  

You  either  select  a  device  or  filename.   Interactive and slave
devices generally accept CHANNEL,  while  sequential  devices  accept
FILE.   The  TEKTRONIX  and  REGIS  devices treat FILE and CHANNEL as
synonyms.  
\paragraph{WIDTH$|$HEIGHT}
Sets  the width and height of the display in the current units.  This
only applies to some drivers.  
\paragraph{IDENT}
IDENT=aaaa  -  This  selects  an  identification  to  be  added  to a
sequential output file before each plot.  For TEKTRONIX and REGIS, if
aaaa='NONE'  then  there  is  no  identification  for each plot.  For
VAXstations using NETWINDOW, the ident is  put  at  the  top  of  the
window.   Normall  the  IDENT  is only 4 characters long, but for the
VAXSTATION it may be up to 20 characters.  

This is equivalent to ``PICTID=aaaa'' 
\paragraph{INTERACTIVE}
This  plots  on  your  terminal.  INTERACTIVE turns on both PAUSE and
FLUSH.  When this is selected TOPDRAWER stops  after  each  plot  and
waits  for you to hit the ``Return'' key.  If you wish to stop you type
STOP, EXIT, QUIT, HALT, or END.  In addition data  is  flushed  after
each  line  of  interactive input.  INTERACTIVE devices support extra
features such as cursor input.  INTERACTIVE devices produce output on
device  TT:.   If  you  attempt interactive output to TT:  in a batch
queue, your output will vanish, as TT is assigned to NLA0:.  

SEQUENTIAL - Produces plots in an output file 

SLAVE  -  Produces  the  plots at another device.  This is similar to
INTERACTIVE, but the CURSOR is not supported for a slave device.  

The default is:  
\begin{verbatim}
     INTERACTIVE if you are running interactively 
     SEQUENTIAL in BATCH mode 
\end{verbatim}
\paragraph{NUMBER}
NUMBER=nnnn  - This selects the sequence number for the first plot in
the set.  This is only used by sequential devices.  
\paragraph{ORIENTATION}
This  is  an  integer value from 0 to 3 specifying the orientation of
the plot.  It rotates all plots on all opened devices.  
\begin{verbatim}
     1.  Normal orientation 
     2.  Rotated 90 degrees clockwise.  This is the same as sideways.
         X goes from top to bottom, while Y points to the right.  
     3.  Rotated 180 degrees or Upside down 
     4.  Rotated 270 degrees clockwise 
\end{verbatim}
\paragraph{REVERSE}
This reverses the foreground/background colors on the plot.  
\paragraph{SIDEWAYS}
SIDEWAYS  -  Rotates  the  plot by 90 degrees.  See:Command SET SIZE.
This rotates all plots on all opened devices.  
\paragraph{PERMANENT}
This  sets the currently selected device as the Default.  The Default
may be restored by a SET DEVICE command with no device specified.  
\paragraph{UNIT}
You may select the unit to open UNITS range from 1 to 8.  This allows
you send the output to several devices at once.  
\paragraph{Example}
To get plots on your terminal a VT-100 with retrographics.  
\begin{verbatim}
     TD:SET DEVICE TEKTRONIX "RVT100" 
\end{verbatim}

To get plots on another terminal (TTAn:) a VT-100 with retrographics. 
\begin{verbatim}
     TD:SET DEVICE TEKTRONIX "RVT100" FILE=TTAn:  
\end{verbatim}

To get output on the Talaris printer:  
\begin{verbatim}
     TD:SET DEVICE EXCL 
\end{verbatim}
The  actual  output  will be a file called UGDEVICE.DAT which must be
printed on the Talaris printer.  

To get direct output to the Talaris printer:  
\begin{verbatim}
     TD:SET DEVICE EXCL FILE=LPA0:UGDEVICE.DAT 
\end{verbatim}
\subsubsection{DIAMOND}
This sets the default size for diamonds.  
SET DIAMOND [SIZE=dx[,dy]] [PERMANENT] 

PERMANENT - Makes the current setting the new default.  
(Default:1.0) 

If  dx is omitted it is set to the default.  If dy is omitted it is the
same as dx.  
\subsubsection{DIGITS}
\begin{verbatim}
     SET DIGITS=n 
\end{verbatim}

Sets  the number of significant digits displayed in a SHOW DATA command
and the lexicals.  The number of digits may be set to a value from 0 to
7.   If  n=0 then the number of digits is adjusted to suit the display.
If n is omitted it is assumed to be 0.  (Default:DIGITS=0) 
\subsubsection{ELLIPSE}
This  sets the default size for the major and semi-major ellipse sizes.
If dy is omitted the ellipse becomes a circle.  
SET ELLIPSE [SIZE=dx[,dy]] [PERMANENT] 

PERMANENT - Makes the current setting the new default.  
(Default:1.0) 

If  dx is omitted it is set to the default.  If dy is omitted it is the
same as dx.  
\subsubsection{ERRORS}
SET ERRORS [\{DEFERRED$|$IMMEDIATE$|$LIST\}] [WAIT$|$PAUSE[=ON$|$OFF]] 

\begin{verbatim}
     1.  DEFERRED  -  The  error  messages  appear  after  the  plot is
         finished.  
     2.  IMMEDIATE  -  The  error  messages  appear when the command is
         entered.  
     3.  LIST - The error messages appear only in the listing file.  
     4.  WAIT   -  When  error  messages  are  immediate,  WAIT  causes
         TOPDRAWER to wait after each message.  PAUSE is a synonym  for
         wait.  
\end{verbatim}

This option remains the same until changed by a SET ERRORS command.  
(Default:IMMEDIATE,NOWAIT) 
\subsubsection{EXACT}
\begin{verbatim}
     SET EXACT [ON|OFF] [PERMANENT] 
\end{verbatim}
Sets the character comparison to be exact rather than case blind.  
(Default:OFF) 
\subsubsection{FILE}
This selects the file names for data input, output, and journaling.  
SET FILE 
\begin{verbatim}
     [INPUT=filename] 
     [LIST|OUTPUT=filename] 
     [JOURNAL=filename] 
                                 NOTE
\end{verbatim}
This  command  is  very  different  from  the  original SLAC version of
TOPDRAWER.  
\paragraph{INPUT}
Selects the input file.  Once a file is selected, TOPDRAWER looks for
all commands in that file.  After  the  last  command  in  the  file,
TOPDRAWER    looks    for   more   input   in   the   current   file.
(Default:TOPDRAWER.TOP) 

\begin{verbatim}
                               WARNING
     If   any  commands  follow  SET FILE INPUT,  they  will  be
     executed before the commands in the input file.  
\end{verbatim}
\paragraph{LIST}
Selects  the  listing  output  (Default:TD.LIS).   This lists all the
commands you give TOPDRAWER.  Both commands entered at the  keyboard,
and from an input file are listed.  It is also the default file for a
LIST command.  
\paragraph{JOURNAL}
Selects  the  journal  file  name  (Default:TD.TDJ)  The journal file
contains any commands you type at the keyboard.  Commands from a disk
file  are  not incuded in the journal file.  There is no journal file
when TOPDRAWER is run from a batch job.  
\paragraph{Filename}
The  file  specification  may  include  device and directory.  If the
filename contains a semicolon '';'', or begins with ``V\_'' or ``S\_''  or  a
special  character  such  as a dollar sign ''\$'' it must be enclosed in
quotes or apostrophes '''''.  For example you may get the sample  input
file.  
\begin{verbatim}
     TD:SET FILE INPUT TOPDRAWER_DIR:TDINTRO 
\end{verbatim}

This  is  a  modification  of  the  original  Topdrawer  syntax.  The
original required a Fortran unit number.   By  performing  a  logical
assignment you may simulate the original syntax.  For example:  
\begin{verbatim}
     $ ASSIGN FOR009.DAT "9" 
\end{verbatim}
defines 9 as the usual input on unit 9.  
\paragraph{Example}
\begin{verbatim}
     TD:SET FILE INPUT TEST 
\end{verbatim}
Input will be taken from file TEST.TOP.  
\begin{verbatim}
     TD:SET FILE INPUT myfile.dat 
\end{verbatim}
Input will be taken from file MYFILE.DAT.  
\begin{verbatim}
     TD:SET FILE INPUT 'myfile.dat;1' 
\end{verbatim}
Input  will be taken from file MYFILE.DAT;1.  Notice, that if you use
a semicolon '';'' in the file name you must enclose it  in  apostrophes
or quotes.  

\begin{verbatim}
     TD:SET FILE INPUT="$TEST" LIST=TESTD 
\end{verbatim}
Input will be \$TEST.TOP the output will be TESTD.LIS.  

\begin{verbatim}
     TD:SET FILE INPUT TOPDRAWER_DIR:TDINTRO 
\end{verbatim}
The input will be from TOPDRAWER\_DIR:  file TDINTRO.TOP 

\begin{verbatim}
     TD:SET FILE INPUT DRA0:[USER.SUBDIR]MY_PLOTS.PLT 
\end{verbatim}
The  input  will  be from device DRA0:, directory [USER.SUBDIR], file
MY\_PLOTS.PLT 
\begin{verbatim}
     TD:SET FILE JOURNAL NL:  
\end{verbatim}
This  discards  the  journal  output  a  better  way  to  do this is:
SET MODE JOURNAL OFF.  
\subsubsection{FILL}
SET FILL [ENTRY[=n]] [FULL] [SIZE[=n]] [ANGLE[=n]] [ALTERNATE=[ON$|$OFF]]
\begin{verbatim}
          [WIDTH|INTENSITY=n] [NOCOLOR|WHITE|RED|GREEN|BLUE|YELLOW|
          MAGENTA|CYAN] [NOTEXTURE|SOLID|DOTS|DASHES|DAASHES|DOTDASH|
          PATTERNED|FUNNY|SPACE] 
\end{verbatim}
This  selects a fill attributes to be used when the FILL option is used
with HISTOGRAM, JOIN, or PLOT.  This selects the type  of  cross  hatch
fill.  Up to 4 different lines may be used.  
\begin{verbatim}
     1.  ALTERNATE  selects filling alternate areas.  If off the entire
         area  bounded  by  the  outside  of  the  curve   is   filled.
         (Default:ON) 
     2.  ENTRY  selects  the fill entry to change.  If n is omitted the
         next entry is selected.  
     3.  FULL  sets  the  fill  size  to  be  the  same  as  the screen
         resolution, the angle is set to 90, and the WIDTH is set to 1. 
     4.  SOLID...  selects the line texture to use.  (Default:NONE) 
     5.  WHITE...  selects the line color to use.  (Default:NONE) 
     6.  ANGLE  the  angle  that  the  line  is  to  be drawn at.  0 is
         horizontal,   while   90   is   straight    up    and    down.
         (Default:+-45 degrees) 
     7.  SIZE  the  separation  between the lines.  If negative this is
         specified in pixels.  If positive it  is  specified  in  inch.
         (Default:0.1) 
     8.  WIDTH  selects  the  line width to use (0 to 5).  A width of 0
         indicates no width is selected.  (Default:NONE) 
\end{verbatim}
IF no options are specified, the fill patterns are reset to the default
value.  
\subsubsection{FIT}
\begin{verbatim}
     SET FIT [options] 
\end{verbatim}
Sets the fitting options.  
See:Command FIT.  
\subsubsection{FLUSH}
\begin{verbatim}
     FLUSH [ON|OFF] 
\end{verbatim}
This  selects automatic flush of plot data.  When automatic flushing of
data is on, plot data is displayed immediately after  each  interactive
line  of  input.   When  OFF  plot data is only displayed after a FLUSH
command or when the plot buffer is full.   Pause  is  normally  on  for
interactive devices, and off for non interactive.  
\subsubsection{FONT}
This selects which fonts are used in generating plots.  
SET FONT [BASIC$|$EXTENDED$|$DUPLEX] 
\begin{verbatim}
     1.  BASIC - 64 character uppercase font (not useful) 
     2.  EXTENDED - Full font with Greek and lowercase (Default) 
     3.  DUPLEX  -  Fancy  font.  This one may not look good on devices
         such as Tektronix 4010 with  limited  resolution.   This  also
         requires the most memory and time to generate.  
\end{verbatim}

This option remains in effect until changed by a SET FONT command.  The
EXTENDED and DUPLEX fonts both use proportional spacing.  If  you  wish
to use a fixed spacing font See:Command SET MODE PROPORTIONAL.  
\subsubsection{FORMAT}
Sets the format used by the Fortran READ statement.  
SET FORMAT ``format'' 

This  is  useful  if  you wish to skip lines at the beginning of a line
when you are using a list file as input to TOPDRAWER.  Normally you  do
not  need to use this command, unless the data is in an extremely weird
format.  The format is limited to 64 characters.  (Default:(256A1)  The
number of columns is also limited by SET CARD.  This option remains the
same until changed by a SET FORMAT command.  This option has no  effect
on input you enter interactively at the terminal.  
\begin{verbatim}
                                Example
\end{verbatim}
If you wish to skip the first 10 columns:  
\begin{verbatim}
     TD:SET FORMAT "(10x,256A1)" 
\end{verbatim}
\subsubsection{GRID}
This  causes  a grid to overlay the plot when axes are drawn.  The grid
lines are lined up with the large tick marks.  This acts on the current
plot  only  a  new  SET  GRID  is required for each plot or the options
PERMANENT must be specified.  The command must be followed by either  a
HIST, JOIN, BARGRAPH or PLOT command to be effective.  
SET GRID [ON$|$OFF$|$HORIZONTAL$|$VERTICAL$|$SYMBOL[=sym]] 
\begin{verbatim}
     [SIZE=n] 
     [INTENSITY|WIDTH=n] 
     [NOCOLOR|WHITE|RED|GREEN|BLUE|YELLOW|MAGENTA|CYAN] 
     [NOTEXTURE|SOLID|DOTS|DASHES|DAASHES|DOTDASH|PATTERNED|FUNNY|
          SPACE] 
     [THETA=n] [PHI=n] [ANGLE=n] 
     [PERMANENT] 
\end{verbatim}
\paragraph{Options}

\begin{verbatim}
     1.  ON - Draws both horizontal and vertical lines.  
     2.  OFF - No grid is drawn.  (Default) 
     3.  HORIZONTAL - Draws horizontal lines.  
     4.  VERTICAL - Draws vertical lines.  
     5.  INTENSITY  - Sets line intensity or width (0-5).  0 gets the
         intensity from the SET INTENSITY or SET AXES command.  
     6.  WHITE...  - Sets the line color.  (Default:Same as axes) 
     7.  SOLID...   -  Sets the line texture.  (Default:Same as axes)
         See:Command SET TEXTURE.  
     8.  SYMBOL  - Draws a symbol [sym] to be plotted at intersection
         of vertical and horizontal lines.  (Default:sym=0O) 
     9.  SIZE  - sets the size of the grid symbols If the size is not
         set then it is adjusted automatically by the window size.  
    10.  THETA,PHI  -  specify  the normal to the grid symbol for 3-d
         plots.  
    11.  ANGLE - rotates the symbol around the normal for 3-d plots. 
    12.  PERMANENT  -  Sets  the current grid options permanently for
         all NEW plots.  
\end{verbatim}
If  no  options  are specified, then all grid parameters are reset to
the  original  defaults.   (Default:OFF  SYM='0O'   INTEN=0   NOCOLOR
NOTEXTURE THETA=0 PHI=0 ANGLE=0) 
\paragraph{Example}
\begin{verbatim}
     TD:SET GRID DOTS VERT SYM=1O 
\end{verbatim}
When  the  axes  are  plotted, a grid is drawn consisting of vertical
dotted lines at each large X tick, and symbols at the intersection of
large X and Y ticks.  
\begin{verbatim}
     TD:SET GRID DOTS VERT HORIZ 
          is the same as ...  
     TD:SET GRID DOTS ON 
\end{verbatim}

\begin{verbatim}
     TD:SET GRID RED ON PERMANENT 
\end{verbatim}
All subsequent plots with axes have red grids.  

\begin{verbatim}
     TD:SET GRID SYM=0O THETA=90 PHI=90 
\end{verbatim}
The  grid  will  be plotted with crosses as the grid symbol, oriented
normal to the Y axis.  The symbol lies in the XZ plane.  
\subsubsection{HISTOGRAM}
SET HISTOGRAM [HBOOK$|$HANDYPAK$|$RICE] [APPEND[=ON$|$OFF]] [CHECK[=ON$|$OFF]]
\begin{verbatim}
          [CONFIRM[=ON|OFF]] [ENTRIES[=ON|OFF]] [ERRORS[=ON|OFF]]
          [OVERFLOWS[=ON|OFF]] [LOG[=ON|OFF]] [SECTION=section_name]
          [FILE=filename] [NUMBER=n] [SELECT|NAME='hist_name']
          [IDENT=nnn] [EXACT[=ON|OFF]] [CURRENT|NEXT|PREVIOUS|FIRST|
          LAST] [PROCESS='Process_name'] [PID='nnnn'] [TITLE[=ON|OFF]]
          [WRAP[=ON|OFF] [HISTOGRAM[=ON|OFF]] [MESH[=ON|OFF]] 
                                   HBOOK4 only
          [AREA|DIRECTORY[="directory"]] [TREE[=ON|OFF]] [WILD[=ON|
          OFF]] [SHRE=section_id] [IOFFSET=n] [CYCLE=n] 
                                     NTUPLES
          [NTUPLES[=ON|OFF]] [EVENTS[=[FROM] n [TO] n] [X|DXYDYZDZ=n]
          [NLIMIT[=n [FROM] n [TO] n] [NMASK[=n [IDENT=n] [X=n] [Y=n]] 
\end{verbatim}
This  selects  a  histogram  and  puts  it  into  data storage.  Once a
histogram is  selected,  the  data  is  available  for  plotting.   The
command:  
\begin{verbatim}
     SHOW HISTOGRAMS 
\end{verbatim}
Shows the current list of histograms available.  
\begin{verbatim}
     LIST HISTOGRAMS 
\end{verbatim}
Produces a listing of the current histograms.  
\paragraph{Options}
\begin{verbatim}
     1.  APPEND - The hist is appended as a new data set 
     2.  AREA="xxx/yyy/zzz" - Selects the histogram area.  
     3.  CHECK - Turns on or off duplicate checking.  
     4.  CONFIRM - You are asked before the histogram is set.  
     5.  ENTRIES - Selects only histograms which have entries (data). 
     6.  ERRORS - Sets the errors (DY).  
     7.  EXACT  -  Histogram  names are treated as exact strings, and
         they are not searched in a case  independent  manner.   This
         must precede the option NAME="...".  
     8.  SECTION=sectionname  -  Sets  up a global section to RESTORE
         hists from.  SEE:TOPDRAWER LINK 
     9.  FILE  -  Specifies  a  direct  access file to get histograms
         from.  
    10.  HISTOGRAM - Selects only histogram (non mesh) data.  
    11.  IDENT - The histogram number to set.  
    12.  LOG - Informs you when it gets the data.  
    13.  MESH - Chooses only MESH data.  
    14.  NAME - Selects hist by name.  
    15.  NUMBER - Then number of hists to fetch from the file.  
    16.  NTUPLES - Selects only NTUPLES.  
    17.  OVERFLOWS - Includes the overflows in the data set.  
    18.  CURRENT|NEXT|PREVIOUS|FIRST|LAST - Selects the hist to get. 
    19.  TITLE - Plots the histogram name as a title.  
    20.  TREE - Searches through the entire directory tree.  
    21.  WILD  - Allows wild searches of areas in SHOW, MONITOR, SET,
         SAVE, RESTORE HISTOGRAM.  If WILD=ON then the star  "*"  and
         percent   "%"   signs   are   reserved  for  wild  searches.
         (Default:WILD=ON) 
    22.  WRAP  - Continues or terminates the search at the end of the
         list of histograms.  
\end{verbatim}
\paragraph{APPEND}
The new data is appended to the current data as a new set rather than
replacing  it.   If  you  omit  ON  or  OFF  then  ON   is   assumed.
(Default:APPEND=OFF) 
\paragraph{AREA$|$DIRECTORY}
This selects the histogram AREA or DIRECTORY.  If you specify ''//'' or
''/'' then the ``root'' directory or ''//PAWC'' is selected.  If  the  area
begins  with  ''/''  then  it is considered to be a subdirectory of the
root.  '''' moves down the directory tree.  ``AAAA'' move up the tree  to
subdirectory   AAAA.    and   ``aAAA''   moves  over  to  the  adjacent
subdirectory AAAA.  Note that '''' may need to be enclosed  in  quotes,
unless  you  have SET CHAR ALPHA ''''.  Unfortunately if you modify the
meaning of '''' then dates must be entered enclosed  in  angle  brakets
$<$1989523$>$.    This   is  generally  not  necessary  unless  you  have
directories which consist entirely of numbers.  

If  you  use  ``wild'' characters in an area name then a directory tree
search is automatically done using  the  wild  characters  to  select
subdirectories.   Wild characters are ''*'' which means and string, and
''\%'' which means any character.  

\begin{verbatim}
                               example
\end{verbatim}
Assume you are in directory ''//PAWC/X'' 
\begin{verbatim}
     TD:SET HIST AREA Y 
\end{verbatim}
moves you to //PAWC/X/Y 
\begin{verbatim}
     TD:SET HIST AREA "q" 
\end{verbatim}
moves to //PAWC/X/Q 
\begin{verbatim}
     TD:SET HIST AREA /A 
\end{verbatim}
moves to //PAWC/A 
\begin{verbatim}
     TD:SET HIST AREA // 
\end{verbatim}
moves to //PAWC 
\begin{verbatim}
     TD:SET HIST AREA /E* ID=2 
\end{verbatim}
Gets  the  first  histogram  with  ID=2  from  area  //PAWC/ENERGY or
//PAWC/ERROR...  
\paragraph{CHECK}
If  on  it  TOPDRAWER  issues  a  warining  if  there  is a duplicate
histogram for the selected one.  This is normally on except when  the
options CURRENT, NEXT,PREVIOUS,LAST,FIRST are used.  
\paragraph{CONFIRM}
Each  histogram  that  matches the specifications is logged, then you
are asked whether you want to accept it.  You reply:  
\begin{verbatim}
  *  Y,A - Yes accept it.  
  *  N - No, do not accept it (Default).  
  *  Q - Quit and stop asking questions.  
\end{verbatim}
If you omit ON or OFF then ON is assumed.  (Default:CONFIRM=OFF) 
\paragraph{ENTRIES}
If  ON  it selects only histograms which have entries (contain data).
If OFF it selects only histograms without entries.   The  default  is
ON.  
\paragraph{ERRORS}
Set the error array DY or DZ.  If this is specified for HBOOK and the
histogram contains no errors, garbage may be collected in DY.  For  2
dimensional  or RICE histograms the ERROR=SQRT(Y).  If you omit ON or
OFF then ON is assumed.  (Default:ERROR=OFF) 
\paragraph{FILE}
FILE=filename  - Selects the filename to restore the histograms from.
If IDENT is specified then the selected histogram is fetched from the
file.  If the filename is NONE then no file name is selected.  If the
filename is blank (FILE='') then the last opened file is used, or  if
none has been opened.  
\begin{verbatim}
     HBOOK.BIN - For HBOOK4 direct access files.  
\end{verbatim}

\begin{verbatim}
                               example
     TD:SET HIST FILE=MY_HISTS.BIN 
\end{verbatim}
Gets the histograms from file MY\_HISTS.BIN.  
\begin{verbatim}
     TD:SET HIST FILE="MY_HISTS.BIN;1" 
\end{verbatim}
Gets histograms from the first version of the file.  
\paragraph{HISTOGRAM}
If  HISTOGRAM=ON  selects only histograms.  IDENT IDENT=nnn - Selects
the histogram by id number.  The data in the histogram is loaded into
the  data buffer.  If this is combined with NAME='hist\_name' then the
next histogram that first matches either the ID or NAME is selected. 
\paragraph{LOG}
This  causes  TOPDRAWER  to  type the histogram name when it sets it.
The histogram ID,NAME,size, and range of  values  is  typed  on  your
terminal.    If   you   omit   ON   or   OFF   then  ON  is  assumed.
(Default:LOG=OFF) 
\paragraph{MESH}
If  MESH=ON  only  histograms  with 3-d or mesh data are set.  If you
omit ON or OFF then ON is assumed.  

\begin{verbatim}
                               example
     TD:SET HISTOGRAM NEXT MESH 
\end{verbatim}
Gets the next mesh plot.  
\paragraph{NTUPLES}
You  may  select  the  NTUPLE  data  and put it into a 1-d histogram.
These options remain the same, unless respecified.  If the option  is
used  without  any  parameters it is reset to the initial value.  You
may show the options by:   SHOW HISTOGRAM SELECT  These  options  are
also  used  to control NTUPL usage by:  DEFINE HISTOGRAM ADD/SUBTRACT
commands 
\begin{verbatim}
     1.  NTUPLES selects only NTUPLES.  Normall NTUPLES are excluded,
         so you must use this option to look at NTUPLE data.  
     2.  EVENTS  FROM  n1  TO  N2  selects  the event range The event
         selection remains the same, unless respecified.  
     3.  X/DX/Y/DY/Z/DZ=n selects the value to put into X/DX...  from
         each NTUPLE Initially they are X=1, Y=2, Z=3, DX=-1,  DY=-2,
         DZ=-3.  0 is the event number while positive values are each
         of the variables in an NTUPLE.  
     4.  NLIMIT=n  FROM  n1 TO N2 places limits on the selected value
         n.  Where n is the coordinate in the NTUPLE.   If  specified
         without  a  value n or with n=0 then all limits are set.  If
         all parameters are unspecified then all limits are reset  to
         large values.  
     5.  NMASK=n Sets MASK number n.  If 0 all masks are cleared.  
         A.  IDENT=n Will mask the data by histogram n 
         B.  X=m  Specifies  which Variable to check against the X of
             histogram n Zero is the event number.  
         C.  Y=m  Specifies  which  Variable  to  check  against Y of
             histogram n 
         The  data  is  checked  against  the  histogram  and  if the
         correspoinding bin of the histogram is non zero the NTUPL is
         kept,  and  discarded if not.  These options are also use by
         the DEFINE HISTOGRAM command to ADD or SUBTRACT  NTUPL  data
         from histograms.  
\end{verbatim}

\begin{verbatim}
                               example
\end{verbatim}
You have an NTUPLE with 6 coordinate values A,B,C,D,E,F You may place
all of the C values into X and the F values into Y While limiting the
B values to the range -1 to 1.  
\begin{verbatim}
     TD:SET HIST ID=5 EVENTS X=3 Y=6 NLIMIT=2 FROM -1 TO 1 
\end{verbatim}
The resulting data may then be histogrammed using the BIN command.  
\paragraph{NUMBER}
The  number  of  hists to fetch when specifying FILE=name.  This only
works it the ID was specified along with the FILE name.  

\begin{verbatim}
                               example
     TD:SET HIST FILE=my_file IDENT=200 NUMBER=25 
\end{verbatim}
Fetches  25  histograms from my\_file beginning from number 200.  Hist
number 200 is put into data.  
\paragraph{OVERFLOWS}
Selects  underflow and overflow data in the first and last bin of the
histogram.   If  you  omit  ON   or   OFF   then   ON   is   assumed.
(Default:OVERFLOWS=OFF) 
\paragraph{CURRENT$|$NEXT$|$PREVIOUS$|$FIRST$|$LAST}
Selects  the  current, next, previous, first or last histogram.  This
may be modified by ID=n, NAME='hist\_name',  MESH...   to  select  the
next  histogram  with  the  desired  ID,  NAME,  or type.  When NEXT,
PREVIOUS, FIRST or LAST are specified, no error messages  are  issued
for duplicate histograms selected one.  
\paragraph{SECTION}
SECTION=SECTION\_NAME  maps a global section to Topdrawer, for getting
histograms.  If IDENT is specified then  the  selected  histogram  is
fetched  from  the  global  section.  If the section has a blank name
(SECTION='''') the already mapped section is used.  SEE:TOPDRAWER LINK 

To show available global sections:  
\begin{verbatim}
     $ INSTALL :==$sys$system:install/command 
     $ INSTALL LIST/GLOBALS 
\end{verbatim}
The  available  globals  sections  will  be  near  the  end under the
category Group Globals.  
\paragraph{SELECT}
NAME='hist\_name' Selects the histogram by name or title.  The data in
the histogram is loaded into the data buffer.   The  entire  list  of
histograms  is searched starting at the first one for a name matching
the hist\_name.  If a histogram has more characters in its  name  than
the hist\_name, the extra characters are ignored.  If the option NEXT,
PREVIOUS, FIRST, or LAST is used, the search begins  from  the  NEXT,
PREVIOUS,  FIRST,  or  LAST  histogram.  Percent '\%' and star '*' are
wild characters.  '\%' means any single character while '*' means  any
character string.  For example NAME='*ENERGY' will select a histogram
name containing the word ENERGY.  However NAME='ENERGY' will select a
histogram  beginning  with  the  word ENERGY.  NAME='*PI*ENERGY' will
find all histograms with the name containing the string 'PI' followed
by  the  word ENERGY separated by any number of characters.  The case
of the hist\_name is ignored.  

If  more  than 1 histogram match the selected name, a warning message
is issued.  
\paragraph{TITLE}
This plots the histogram name as a title.  The name may be split into
4 parts, Top, X, Y, and Z by separating them with semicolons.  
\begin{verbatim}
     Top label;X label;Y label;Z label 
\end{verbatim}
If  the  number  of  lines  is  insufficient  for a title, then it is
omitted.  For example you wish to omit the Y title so you:  
\begin{verbatim}
     TD:SET TITLE LEFT LINES=0 
\end{verbatim}

WARNING  For  3-D  histograms  the  title should not be plotted until
after the histogram has been plotted.  If you omit ON or OFF then  ON
is assumed.  
(Default:TITLE=OFF) 
\paragraph{TREE}
Searches  the  entire  directory tree for the histogram starting with
the current area.  This option remains set  from  one  SET  HISTOGRAM
command  to  the  next.   You  may  only search a directory tree in a
forward directions so this does not work for the option  PREVIOUS  or
LAST.   Essentially  the  first  time  you  specify  TREE  the entire
directory tree is setup  for  a  search.   Subsequent  SET  HISTOGRAM
commands  should  not  use  the  TREE  option  unless a new search is
desired.  
\paragraph{WILD}
If  WILD  is on then the characters ''*'' and ''\%'' are ``wild'' and may be
used in AREA names.  A wild character in an area causes  a  directory
tree search, of only the selected subdirectories.  
\paragraph{WRAP}
If  WRAP  is on then when you SET HIST NEXT and the current histogram
is the last one, the histogram number will wrap around to  the  first
number.   Similarly  SET HIST PREVIOUS  will  wrap around to the last
one.  If you specify WRAP=OFF then you will  get  a  warning  message
instead  of  wrapping.  If you omit ON or OFF then ON is assumed.  By
default WRAP is ON when TOPDRAWER is used interactively, but OFF when
it is used in a batch stream.  
\paragraph{Lexicals}
When  a  histogram  is  set  S\_HIST\_NAME  contains  the  the  current
histogram name.  

If  the  histogram package has limits or markers set on the histogram
they are put  into  V\_XMARKER[n]  and  V\_YMARKER[n].   HBOOK  has  no
markers.   The number of markers is limited to n=8.  The lexicals are
set to a very large number 1.0E+30 if they do not contain markers.  
\paragraph{Restrictions}
The maximum number of histograms you may restore is 499.  The maximum
memory for  HBOOK  histograms  is  150,000  words.   (1.2 Mbyte)  The
maximum title length is 256 characters.  If you have a file exceeding
these restrictions you must restore each histogram individually by:  
\begin{verbatim}
     TD:SET HISTOGRAM HBOOK FILE=name ID=n 
\end{verbatim}
\paragraph{Examples}
The following sequence will plot a histogram stored by HSTORE.  
\begin{verbatim}
     TD:RESTORE HIST FETCH FILE=filename (Get the hists) 
     TD:SET HIST ID=5;HIST 
          or...  
     TD:SET HIST NAME='title of hist 5' 
     TD:HISTOGRAM 
\end{verbatim}

\begin{verbatim}
          or...  
     TD:SET HIST HBOOK FILE=filename ID=5 
     TD:HISTOGRAM 
\end{verbatim}

\begin{verbatim}
     TD:SET HIST NAME='ENERGY' 
\end{verbatim}
sets the first histogram named ``ENERGY....'' into the data array.  
\begin{verbatim}
     TD:SET HIST NEXT NAME="*ENERGY" 
\end{verbatim}
sets the next histogram with a title containing the word ``ENERGY''.  
\begin{verbatim}
     TD:SET HIST NEXT NAME='*ENERGY' CONFIRM 
\end{verbatim}
types  the  name of each histogram containing the word ``ENERGY'', then
you are asked if it is acceptable.  You type ``Y'' if it is, and  press
``Return'' if it is not.  
\begin{verbatim}
     TD:SET HIST LAST HIST NAME='ENERGY' 
\end{verbatim}
sets  the  last  1  dimensional histogram named ``ENERGY....'' into the
data array.  
\subsubsection{INPUT}
\begin{verbatim}
     See command:  SET CARD 
\end{verbatim}
\subsubsection{INTENSITY}
SET \{INTENSITY$|$WIDTH\} level [PERMANENT] 
This  sets  the intensity level or line width of a plot.  The permanent
default is restored when a new plot is started.  

level=1  to  5.   (Default:2)  5  is  the  brightest  (widest).   If no
intensity is specified it is set to 2.  
PERMANENT - Sets the current width to be the new default.  
\subsubsection{LABELS}
This  sets  the  size  and enables the numeric labels on each axis.  To
control the default number of ticks see SET SCALE.  
SET LABELS [SIZE=n] [CHARACTERS=n] [INSIDE[=ON$|$OFF] [ALL$|$TOP$|$BOTTOM$|$
\begin{verbatim}
          RIGHT|LEFT|X|Y|Z] [ON|OFF] [SHIFT=n] 
     [INTENSITY|WIDTH=n] 
     [NOCOLOR|WHITE|RED|GREEN|BLUE|YELLOW|MAGENTA|CYAN] 
     [PERMANENT] 
\end{verbatim}
This  acts on the current plot only if PERMANENT not specified.  If the
parameters are omitted they are reset  to  the  original  default.   To
control other individual elements of the axes use:  
SET AXES, SET TICKS, SET OUTLINE, SET SCALES 

(DEFAULT:LEFT ON,BOTTOM ON) 
\paragraph{ALL$|$TOP$|$BOTTOM$|$LEFT$|$RIGHT}
Selects  labels  for top, bottom, right and left.  Top selects labels
for the TOP of plot and so on.  
\paragraph{X$|$Y$|$Z}
selects  the  axis  to  plot.  Y selects both LEFT and RIGHT, while X
selects TOP and BOTTOM.  
\paragraph{CHARACTERS}
CHARACTERS=n  specifies  the  maximum number of characters in left or
right titles.  When setting up the windows room is allocated for  the
specified  number  of  characters.  The room is only allocated if the
corresponding axis and labels are enabled.  If  negative  the  actual
number  of characters needed is used to locate titles.  LEFT or RIGHT
titles are  only  moved  over  according  to  the  number  of  actual
characters if they are plotted after the corresponding axes.  

\begin{verbatim}
                               Example
     TD:SET LABELS CHARACTERS=4 PERMANENT 
\end{verbatim}
All  plots  will allocate room for 4 character labels on the left and
right.  
\begin{verbatim}
     TD:SET LABELS CHARACTERS=-6 
     TD:PLOT AXES 
     TD:TITLE LEFT 'Flush title 
\end{verbatim}
First the window is allocated leaving room for 6 characters, then the
title is located according to the actual number of characters used to
form the labels.  
\paragraph{INSIDE}
INSIDE=OFF  turns  off  labels for adjacent axes of negative windows.
If you wish to plot data in a set of windows with the same scales and
limits,  you may want to have each window flush against the next one.
This is achieved  by  using  negative  windows.   Unfortunately  when
plotting  the  axes,  you  will  get labels inside the windows.  With
INSIDE=OFF you will only get labels on the perimeter.  

\begin{verbatim}
                               Example
\end{verbatim}
Assume you have 6 data sets to plot in separate windows.  
\begin{verbatim}
     TD:SET LABELS INDIDE=OFF 
     TD:SET WINDOW X 1.2 OF -3.5 Y 1.2 OF -2.5 
     SET WINDOW 1;PLOT SET=1 
     SET WINDOW 2;PLOT SET=2 
     SET WINDOW 3;PLOT SET=3 
     ....  
\end{verbatim}
\paragraph{SIZE}
SIZE=n  is  the  character size in tenths of an inch Normally this is
automatically set according to the paper size, and window to give you
pleasing  results.   This  applies  to  all labels.  The units may be
modified by SET UNITS CHARACTER.  
(Default:2) 
\paragraph{SHIFT}
selects  the  amount  to shift the label from the edge of the plot to
the center of the first character.  It is specified as the number  of
characters.  This should be preceded by the option ALL,BOTTOM....  If
the axis is not specified ALL is assumed.  
\paragraph{ON$|$OFF}
Allows  or  prevents  drawing  labels  This should be preceded by the
option ALL,BOTTOM....  If the axis is not specified ALL is assumed.  
\paragraph{INTENSITY}
Sets  line  intensity  or width (0-5).  0 gets the intensity from the
SET INTENSITY command.  
\paragraph{WHITE...}
Sets the line color.  
\paragraph{PERMANENT}
Makes the current settings permanent from plot to plot.  
\paragraph{Examples}
\begin{verbatim}
     TD:SET LABELS ON SHIFT=2 
\end{verbatim}
Selects labels on all axes shifted by 2.  
\begin{verbatim}
     TD:SET LABELS RIGHT ON SHIFT=2 
\end{verbatim}
Selects  labels for the right hand axis shifted by 2.  All other axes
remain the same.  
\begin{verbatim}
     TD:SET LABELS ALL OFF TOP ON RIGHT ON SHIFT=2 
\end{verbatim}
Selects  labels  on the top,right axes only.  The right label will be
shifted by 2.  The shift for the other axes remains the same.  
\subsubsection{LIMITS}
SET LIMITS [CURSOR] [PERMANENT] [SCALE n1 [[TO] n2]] 
\begin{verbatim}
                         either...  
     FROM [X=]nx,[[Y=]ny,[[Z=]nz]]] TO [X=]nx,[[Y=]ny,[[Z=]nz]]] 
                         or...  
     [X [FROM] x [TO] x] 
     [Y [FROM] y [TO] y] 
     [Z [FROM] z [TO] z] 
     [XMIN=x] [XMAX=x] [YMIN=y] [YMAX=y] [ZMIN=z] [ZMAX=z] [PERMANENT] 
\end{verbatim}
Limits define the space to draw the data.  They do not necessarily
limit the data points plotted.  Normally limits are automatically set
from all available data during the first plot.  If you read data, plot
and then read more data and plot again, the first data set determines
the limits.  This command allows you to set the limits of the plot
independently of the data.  This should be used before the first PLOT,
JOIN, HIST, or BARGRAPH command.  The limits are reset when a new plot
is started.  If this command is issued without any limits specified,
all currently set limits are removed.  
\begin{verbatim}
                                WARNING
     If this command is issued in between 2 plots overlaying each
     other, they will not have the same scales.  
\end{verbatim}

See also:SET WINDOW 
\paragraph{SCALE}
This sets a uniform scale for both X,Y.  The window size is n1/n2 of
the displayed X,Y.  If n2 is omitted it is assumed to be 1.0.  If
both n1 and n2 are omitted, both x,y have the same scale as
determined by the limits, and window size.  The plot is centered with
respect to any selected limits.  
\begin{verbatim}
     TD:SET LIMITS X 20,40 Y 40,60 SCALE 1 to 10 
     TD:SET WINDOW X 1 to 4 Y 1 to 2 
\end{verbatim}
The X is displayed from 10 to 50 Y is displayed from 40 to 60.  
\paragraph{CURSOR}
CURSOR puts a cursor on the screen.  Move the cursor to the desired
first limits.  Then hit the space bar.  Next move the cursor to the
desired second limits, and again hit the space bar.  A new set of
limits are now set for the next graph.  If you wish to set only the X
limit hit ``X'' instead of ``space''.  Likewise hit ``Y'' or ``Z'' if you
wish to set only Y or Z limits.  The new limits are entered into the
journal file.  

If both X limits are the same, then they are ignored.  Likewise
identical Y or Z limits are ignored.  For 3-d data the limits are not
changed.  Instead the parameters SCRD, THETA, PHI, and CENTER are
adjusted.  The CURSOR option should be the last option in the command
line for 3-d plots.  

For example the following command will set new limits, and replot the
current data:  
\begin{verbatim}
     TD:SET LIM CUR;NEW;PLOT 
\end{verbatim}
\paragraph{PERMANENT}
PERMANENT sets the current limits permanently or until the next SET
LIMIT command.  If you wish to unset permanent limits, set them
permanently immediately after an NEW PLOT command.  
\paragraph{Example}
The following commands are equivalent:  
\begin{verbatim}
     TD:SET LIMITS X FROM 0 to 10.0 Y 5 6 
     TD:SET LIMITS X 0 10 Y FROM 5.0 to 6.0 
     TD:SET LIMITS XMIN=0.0 XMAX=10.0 YMIN=5 YMAX=6 
     TD:SET LIMITS FROM 0.0,5 TO 10.0,6 
     TD:SET LIMITS FROM Y=5 X=0.0 TO Y=6 X=10.0 
\end{verbatim}
\subsubsection{MODE}
This controls a grab-bag of options.  
\begin{verbatim}
     SET MODE 
          [AUTOPLOT[=ON|OFF]] 
          [ABORT[=ON|OFF]] 
          [APPEND[=ON|OFF]] 
          [CHECK[=ON|OFF]] 
          [CONFIRM[=ON|OFF]] 
          [ERASE[=ON|OFF]] 
          [DEBUG[=ON|OFF]] 
          [ECHO=[n]|NOECHO] 
          [EXPAND[=ON|OFF]] 
          [FILECASE=UPPER|LOWER|MIXED] 
          [HARDTEXTURE[=ON|OFF]] 
          [JOURNALING[=ON|OFF]] 
          [LISTING[=ON|OFF]] 
          [LOG[=ON|OFF]] 
          [MONITOR[=ON|OFF]] 
          [PROPORTIONAL[=ON|OFF]] 
          [QUICK|SLOW] 
          [SHOW[=ON|OFF]] 
          [TITLE[=ON|OFF]] 
          [TRACE[=ON|OFF]] 
          [TREE[=ON|OFF]] 
          [VECTOR[=ON|OFF]] 
          [VLOG[=ON|OFF]] 
\end{verbatim}
If ON$|$OFF is omitted ON is assumed.  Most of these options remain the
same until explicitly reset.  They are generally not reset by a
NEW PLOT command.  
(Default:AUTOPLOT=ON, ABORT=ON, APPEND=OFF, CONFIRM=OFF, ERASE=OFF,
DEBUG=OFF, ECHO=20, HARD=ON, JOURNALING=ON, LISTING=ON, LOG=OFF,
MONITOR=OFF, PROPORTIONAL=ON, SHOW=OFF, TRACE=OFF, TREE=OFF, VECTOR=ON) 
\paragraph{AUTOPLOT}
If an EXIT or CLEAR command occurs and none of the current data set
has been histogrammed, plotted, or joined, then it is plotted.  When
AUTOPLOT is off no plotting is automatically done.  
\paragraph{ABORT}
If an error occurs during a command the command is aborted.  This
applies generally to commands that produce output such as ARROW, BOX,
CIRCLE, HISTOGRAM, PLOT, BAR, JOIN, and LIST as well as SMOOTH, BIN,
X=, and so on.  It does not apply to the SET commands.  In addition
ABORT just prevents journaling of inconvenient commands such as SHOW,
SPAWN, and HELP.  
\paragraph{APPEND}
This is equivalent to using the APPEND option for all commands that
create new data sets.  
\paragraph{CONFIRM}
This is equivalent to using the CONFIRM option for any command that
uses this option.  When this mode is on you must confirm all data
deletions with a YES, NO, QUIT, or ALL answer.  
\paragraph{CHECK}
CHECK=OFF is equivalent to using the CHECK=OFF option for commands
such as ADD, SMOOTH, FFT, DEFINE HIST ....  It turns off the checks
for appropriate data.  Usually this means that the data must be a
histogram with equally spaced bins.  For commands that take 2 data
sets there is the requirement that the data sets must match.  
\paragraph{DATAVECTOR}
This modifies the interpretation of DX,DY,DZ to be vector components.
It modifies only the PLOT, ADD, SUBTRACT, and MULTIPLY commands.  
\paragraph{DEBUG}
Automatic flush at the end of each command.  If you SET FILE LIST=TT:
and DEBUG=ON you can see each command and its result in order.  For
compatability with previous versions NODEBUG is a synonym for
DEBUG=OFF.  
\paragraph{ECHO}
ECHO=n selects n data points to be listed when read.  NOECHO is the
same as ECHO=0 If you do not specify n it is assumed to be 1000.  
\paragraph{ERASE}
Selects erase mode for plotting.  While plotting in this mode The
plotted data erases previous data.  This mode is reset to ERASE=OFF
whenever a new plot is begun.  
\paragraph{EXPAND}
Selects EXPAND=ON whenever you PLOT, JOIN, or HISTOGRAM.  
\paragraph{FILECASE}
Selects the case of file names.  
\begin{verbatim}
     1.  Upper - File names are converted to upper case 
     2.  Lower - File names are converted to lower case 
     3.  Mixed - The case remains the same as typed.  
\end{verbatim}

If a file name is enclosed in quotes, then it is never converted.  
\paragraph{HARDTEXTURE}
Selects hardware or software texture generation.  
\paragraph{JOURNALING}
Turns on or off output to the journal file.  
\paragraph{LISTING}
Turns on or off the output to the listing file.  
\paragraph{LOG}
Turns on or off automatic logging.  Many commands use the option LOG
to tell you what they are doint.  When logging mode is on these
commands automatically log teir actions.  For example with LOG=ON
when you use the MULTIPLY command, it informs you which data sets are
multiplied.  
\paragraph{MONITOR}
This is the same as using the MONITOR option for all commands where
it is legal.  For example if MODE MONITOR=ON then when you do a fit
the result is automatically plotted.  
\paragraph{PROPORTIONAL}
Selects proportional or fixed spacing for the text.  
\paragraph{QUICK}
This turns off certain calculations to make plotting faster.  One
side effect of this is to prevent proper centering or justification
of titles, and labels.  If you wish both speed and proper
justification, you can get proper justification for titles with no
sub/superscripts by:  
\begin{verbatim}
     TD:SET MODE PROPORTIONAL=OFF 
\end{verbatim}
Plots can be made still faster by:  
\begin{verbatim}
     TD:SET MODE VECTOR=OFF 
\end{verbatim}
This only works if your output device has hardware character
generation for the size characters you select.  
\paragraph{SHOW}
Show mode automatically generates a SHOW command after any
SET command is used.  
\paragraph{TITLE}
Automatically plots the title of a data set along with the data when
you use a CONTOUR, HISTOGRAM, JOIN, or PLOT command.  This is
equivalent to specifying TITLE=ON along with the commmand.  
\paragraph{TRACE}
Produces traceback whenever an error occurrs.  For compatability with
previous versions NOTRACE is a synonym for TRACE=OFF.  
\paragraph{TREE}
Setting mode TREE=ON is the same as specifying TREE=ON for any
command that uses this option.  This is used when searching HBOOK4
for histograms.  
\paragraph{VECTOR}
VECTOR=ON requires software drawn characters instead of hardware
chars.  Character generation is slower, but more uniform.  Software
characters are always drawn when a non blank case string is supplied,
or if your output device can not produce hardware characters in the
size selected.  You may request hardware characters by setting the
title/label size to a negative value.  For compatability with
previous versions NOVECTOR is a synonym for VECTOR=OFF.  
\paragraph{VLOG}
Enables Video logging.  When you use the cursor a Cross will be drawn
at the selected cursor location, and a dotted line will be drawn
around the resulting limits.  
\subsubsection{MONITOR}
This sets the type of plot each monitor mode produces.  
SET MONITOR [\{ALL$|$DATA$|$MESH\} \{BIN$|$CONTOUR$|$JOIN$|$HISTOGRAM$|$PLOT
\begin{verbatim}
          "options"}] 
\end{verbatim}
The options may be any legitimate options for the type of plot
selected.  If you do not supply any options then the defaults are used. 
\begin{verbatim}
                                example
     TD:SET MONITOR MESH HISTOGRAM "BLOCK" 
\end{verbatim}
Produces a block or ``Lego'' plot when you monitor a mesh histogram.  
\begin{verbatim}
     TD:SET MONITOR DATA JOIN "1 dots 
\end{verbatim}
Joins normal data sets with single dotted line segments.  
\begin{verbatim}
     TD:SET MONITOR DATA PLOT "SYM=5O" MESH JOIN 
\end{verbatim}
Plots normal data and joins mesh data.  
(Default:MESH PLOT ``NORAN VAR SYM=10'' DATA HISTOGRAM '''') 
\subsubsection{NAME}
SET NAME ``New data set name'' [CONFIRM[=ON$|$OFF]]
\begin{verbatim}
          [SETS=[FROM] n1 [TO] [n2]] [SELeCT="old name"] 
\end{verbatim}

Sets  a  new  name  for  all  the  selected  data sets.  If no sets are
specified, then all data set names are modified.  If the name  ends  in
''\%'' then the old name is appended to the new name.  

Options 
\begin{verbatim}
     1.  CONFIRM  - If selected Topdrawer will prompt you for each data
         set.  You repy YES, NO, ALL, or QUIT.  ALL will  do  the  rest
         without any more prompts, while QUIT skips all the rest.  
     2.  SETS - Selects a range of data sets to rename.  
     3.  SELECT  -  Selects the data sets by name.  WIld characters are
         permitted "*" or "%" See:SET EXACT.  
                                Example
     TD:SET NAME "My data" SET=5 
\end{verbatim}
Data set number 5 has the new name ``My data'' 
\begin{verbatim}
     TD:SET NAME "My %" SELECT="Your*" 
\end{verbatim}
All   data   sets  beginning  with  the  name  ``Your''  are  renamed  to
``My Your...'' 
\subsubsection{ORDER}
This  sets  the  order in which data must be entered.  The order is the
same as specified in the set order command.  This  option  remains  the
same  until the next SET ORDER or NEW PLOT command.  This has no effect
on mesh data input.  
SET ORDER [X$|$THETA [fctr]] [Y$|$RADIUS [fctr]] [Z$|$PHI [fctr]] [DX$|$RY$|$
\begin{verbatim}
          DTHETA|RTHETA [fctr]] [DY|RX|DRADIUS|RRADIUS [fctr]] [DZ|RZ|
          DPHI|RPHI [fctr]] [SYMBOL] [DUMMY[=n]] [PACKED[=ON|OFF]]
          [PERMANENT] 
\end{verbatim}
\paragraph{Options}

\begin{verbatim}
     1.  X,Y,Z - The data to plot 
     2.  DX,DY,DZ - The error on the data as half width for the error
         bars.  
     3.  RX,RY,RZ  -  Errors  given  relative  to  the central value.
         (DX=RX*X) 
     4.  RADIUS,DRADIUS,THETA,DTHETA  -  Polar  coordinates  are also
         enabled for the current plot.  
     5.  PHI,DPHI - Spherical coordinates are enabled for the current
         plot.  
     6.  RRADIUS,RTHETA,RPHI  -  Errors  given  relative to a central
         value.  If RPHI is  used  SPHERICAL  is  enabled,  otherwise
         POLAR is enabled.  
     7.  SYMBOL  -  The symbol to plot.  This should not be the first
         item.  If you need to have the symbol first, then  you  must
         use  the  READ  POINTS  command,  or omit the symbol for the
         first data point.  
     8.  DUMMY  -  Specifies  this  entry  is  a  dummy and should be
         ignored.  If n is specified  then  n  entries  are  ignored.
         Only numbers are ignored.  
     9.  fctr  - Is a scale factor to multiply the data by when it is
         read.  (Default:1.0) 
    10.  PACKED  -  Specifies  that  dats  is  in  packed format with
         several points per line and no semicolons ";" between them. 
    11.  PERMANENT - Specifies that this order is permanent until the
         next SET ORDER command.  
\end{verbatim}

(Default:X,Y,DX,DY,SYMBOL) 

If this command is used without any options other than PERMANENT, the
order is set to the default.  Numeric fields preceding the symbol may
be onitted.  
\paragraph{Examples}
\begin{verbatim}
     TD:SET ORDER X,Y,DX,DY,SYMBOL,Z,DZ 
     TD:1,2 1O 3,4,5,6,7 
\end{verbatim}
Enters  a  point  X=1, Y=2, DX=DY=0, SYMBOL=1O, Z=3, DZ=4 The numbers
5,6, and 7 are ignored.  
\begin{verbatim}
     SET ORDER X Y PACKED 
     1 2 3 4 
\end{verbatim}
Enters 2 points X=1,Y=2 and X=3,Y=4 
\begin{verbatim}
     SET ORDER Y PACKED 
     1 3 5 
     X=0 BY 1 
\end{verbatim}
Enters 3 points (0,1), (1,3) and (2,5) 
\subsubsection{OUTLINE}
This  enables  the outline on each axis.  The outline is the line drawn
at the edge of the plot.  
SET \{OUTLINE$|$FRAME\} [INHIBIT$|$ENABLE] [ALL$|$TOP$|$BOTTOM$|$RIGHT$|$LEFT] [ON$|$
\begin{verbatim}
          OFF] 
     [INTENSITY|WIDTH=n] 
     [NOCOLOR|WHITE|RED|GREEN|BLUE|YELLOW|MAGENTA|CYAN] 
     [NOTEXTURE|SOLID|DOTS|DASHES|DAASHES|DOTDASH|PATTERNED|FUNNY|
          SPACE] 
     [PERMANENT] 
\end{verbatim}
This  acts  on  the  current  plot only if PERMANENT not specified.  To
control other individual elements of the axes use:  
SET AXES, SET TICKS, SET LABELS, SET SCALES 

(Default:ON) 
\paragraph{Options}
\begin{verbatim}
     1.  INHIBIT|ENABLE  -  inhibits or enables the automatic outline
         generation for the current plot or window.  (Default:ENABLE)
         You  may  still  draw  the  outline  with  the  PLOT OUTLINE
         command.  
     2.  ALL - enables outlines for top, bottom, right and left.  
     3.  TOP - enables outline for the TOP of plot 
     4.  BOTTOM - selects outline for the bottom of the plot.  
     5.  RIGHT - selects outline for the right hand side 
     6.  LEFT - selects outline for the left hand side.  
     7.  ON - allows drawing outlines 
     8.  OFF - prevents drawing outlines 
     9.  INTENSITY  - Sets line intensity or width (0-5).  0 gets the
         intensity from the SET INTENSITY or SET AXES command.  
    10.  WHITE...  - Sets the line color.  
    11.  SOLID...    -   Sets   the  line  texture.   (Default:SOLID)
         See:Command SET TEXTURE 
    12.  PERMANENT  -  Makes the current settings permanent from plot
         to plot.  
\end{verbatim}

\subsubsection{PATTERN}
This  selects  the  pattern for patterned lines.  These are produced by
JOIN, HIST, PLOT, or BAR commands.  This option remains the same  until
changed by a SET PATTERN command.  
SET PATTERN [RANDOM$|$FUNNY] [DOT] [DASH] [DAASH] [SPACE] [SIZE=n]
\begin{verbatim}
          [PERMANENT] 
\end{verbatim}
SET PATTERN p1 s1 p2 s2 ....p10 s10 

\begin{verbatim}
     1.  RANDOM - The distance following this is random.  
     2.  DOT - Draws a single dot followed by a 0.1" space.  
     3.  DASH - Draws a short dash 0.1" long followed by 0.1" space.  
     4.  DAASH - Draws a long dash 0.3" followed by 0.1" space.  
     5.  SPACE - Adds 0.05" of blank space.  
     6.  SIZE=n  -  Size of dashes, and the distance between dashes and
         dots.  (Default:0.1) 
     7.  p  =  distance  to  draw  in  inches.   If  negative it varies
         randomly.  
     8.  s  =  distance  to  skip  in  inches.   If  negative it varies
         randomly.  
     9.  PERMANENT - Makes the current SIZE the permanent default.  
\end{verbatim}

You  may  enter  up  to  a  maximum  of  20 values or 10 DOT/DASH/DAASH
specifiers.  Random values vary from zero to the value specified.   You
may also specify patterns as part of any drawing command.  For example: 
\begin{verbatim}
     TD:HISTOGRAM PATTERN DOT DOT DASH 
          is the same as ...  
     TD:SET PATTERN DOT DOT DASH;  HISTOGRAM PATTERN 
          or...  
     TD:HISTOGRAM DOT DOT DASH 
\end{verbatim}
See:Command SET TEXTURE 
\paragraph{Example}
\begin{verbatim}
     TD:SET PATTERN DOT DOT DASH DASH 
          or...  
     TD:SET PATTERN 0.,.1,0.,.1,.1,.1,.1,.1 
\end{verbatim}
is a dot,dot,dash,dash pattern 
\begin{verbatim}
     TD:SET PATTERN DOT DOT SPACE 
          or...  
     TD:SET PATTERN 0,.1,0,.15 
\end{verbatim}
is an assymmetrical dot,dot pattern.  
\begin{verbatim}
     TD:SET PATTERN .3,.1,.1,.1 
          or...  
     TD:SET PATTERN DAASH DASH 
\end{verbatim}
is a long,short dash pattern.  
\begin{verbatim}
     TD:SET PATTERN -.2,.1 
\end{verbatim}
is a dash pattern where the dash varies in length from 0.0 to 0.2 
\begin{verbatim}
     TD:SET PATTERN RANDOM DOT 
          or...  
     TD:SET PATTERN 0.0,-0.1 
\end{verbatim}
is a randomly spaced dot pattern.  
\begin{verbatim}
     TD:SET PATTERN RANDOM DOT DASH 
          or...  
     TD:SET PATTERN 0,-.1,.1,.1 
\end{verbatim}
is a random dot,fixed dash pattern.  
\begin{verbatim}
     TD:SET PATTERN SIZE=.05 DOT DASH 
          or...  
     TD:SET PATTERN 0,.05,.05,.05 
\end{verbatim}
is a small dot dash pattern.  
\subsubsection{PAUSE}
\begin{verbatim}
     SET PAUSE|WAIT [ON|OFF] 
\end{verbatim}
Pause  is  normally  set  on  for  interactive devices, and off for non
interactive.  The Pause occurrs at each  NEW  PLOT  command  inside  an
input file, or when deferred errors must be typed.  
\subsubsection{PEN}
This  selects  the  default Calcomp plotter pens or the color.  This is
effective only for the current plot, or until another SET COLOR or  SET
PEN command.  
\begin{verbatim}
     SET PEN 1|2|3|4|5|6|7 
\end{verbatim}

See:Command SET COLOR 

(Default:1) 

If no pen number is specified it is set to 1.  
\subsubsection{POLAR}
SET POLAR [ON$|$OFF] [DEGREES$|$RADIANS$|$GRAD$|$PERCENT$|$FULLCIRCLE=n]
\begin{verbatim}
          [PERMANENT] 
\end{verbatim}

This sets up to plot in polar coordinates.  Coordinates are (R,THETA,Z)
instead of (X,Y,Z) 
\begin{verbatim}
     1.  ON|OFF turn the polar coordinates on or off.  (Default:ON) 
     2.  PERMANENT - Sets these parameters permanently.  
     3.  DEGREES...  - Selects angle representation (Default:DEGREES) 
         A.  GRAD - (0-400) 
         B.  DEGREES - (0-360) 
         C.  PERCENT - (0-100) 
         D.  RADIANS - (0-6.2831853) 
         E.  FULLCIRCLE=n - (0-n) 
\end{verbatim}

This  feature  is  not  fully implemented.  You still set the limits in
X,Y,Z coordinates, and automatic scaling centers the origin.  When  you
plot, the axes are not automatically generated if polar coordinates are
on.  You must manually plot the axes with the command:  PLOT AXES.   If
you wish the X and Y axes to be scaled the same way you should.  
\begin{verbatim}
     TD:SET LIMITS SCALE 
\end{verbatim}
See commands:X,SET ORDER,SET STORAGE,SET SPHERICAL 

\begin{verbatim}
                                Example
\end{verbatim}

\begin{verbatim}
     TD:SET POLAR FULL=1.0 
\end{verbatim}
Angles are represented by a number from 0 to 1.0 for 0 to 360 degrees. 
\subsubsection{PROMPT}
SET   PROMPT   [MAIN$|$PAUSE]   [ERASE]   [TOP]   [BOTTOM]   [BELL]  [nn]
\begin{verbatim}
          ["prompt_string"] 
\end{verbatim}

This  sets  the  current prompt string with a maximum of 80 characters.
If a null or blank string is used, then  there  is  no  prompt.   If  a
prompt  has  imbedded blanks it must be enclosed in quotes.  The prompt
string remains the same until a new one is selected.  
\begin{verbatim}
     1.  MAIN - Is the main prompt string.  (Default:"TD:") 
     2.  PAUSE  - Is the prompt string used when a NEW PLOT is started.
         (Default:"PAUSE:") 
     3.  ERASE  -  The  prompt erases the screen (ANSI terminal).  This
         erases text from a VT-1xx/2xx compatible terminal.  This  does
         not  erase  TEKTRONIX  graphics,  but  REGIS  graphics will be
         erased.  This must be precede the string.  
     4.  TOP - The prompt is at the top of the screen (ANSI terminal). 
     5.  BOTTOM  -  The  prompt  is  at  the bottom of the screen (ANSI
         terminal).  
     6.  BELL - This rings the bell for each prompt.  
     7.  nn - This inserts a control character (0-255) into the string. 
     8.  
\end{verbatim}

This  command  is useful for controlling the prompt for different types
of terminals.  On some terminals you can control the  position  of  the
prompt,  so  that it will not interfere with the plot, while on others,
you may not want any prompts.  
\paragraph{Examples}
\begin{verbatim}
     TD:SET PROMPT MAIN TOPDRAWER:  PAUSE " " 
\end{verbatim}
Changes the main prompt and sets no prompt for pause.  
\begin{verbatim}
     TD:SET PROMPT MAIN "TD: " 
\end{verbatim}
This puts a space at the end of the prompt.  
\begin{verbatim}
     TD:SET PROMPT MAIN=TOP "TD:" 
          or...  
     TD:SET PROMPT MAIN=27,"[f",27,"[KTD:" 
\end{verbatim}
Puts  the  prompt at the top of an ANSI (VT100 etc.) terminal for the
prompt.  
\subsubsection{REVISION}
\begin{verbatim}
     SET REVISION=n.m 
\end{verbatim}
Sets  the  current  revision  level.   This  affects  the  operation of
TOPDRAWER.  By lowering the  revision  level  you  may  return  to  old
behavior.  Some of the old revision levels are:  
\begin{verbatim}
  *  1.0 - Original version 
  *  2.0 - SET THREE THETA modified to conform to documentation.  
  *  2.2 - X= SET=n FROM=n1 TO=n2...  Worked incorrectly if n>1.  
\end{verbatim}
\subsubsection{SECONDARY}
SET SECONDARY 
\begin{verbatim}
     [INTENSITY|WIDTH=n] 
     [WHITE|RED|GREEN|BLUE|YELLOW|MAGENTA|CYAN] 
     [NOTEXTURE|SOLID|DOTS|DASHES|DAASHES|DOTDASH|PATTERNED|FUNNY|
          SPACE] 
\end{verbatim}
\subsubsection{SEGMENTS}
SET SEGMENTS [ALL $|$AXES $|$LABELS $|$OULINE $|$POINTS $|$PLOTS $|$TICKS $|$TITLES
\begin{verbatim}
          =[ON|OFF]] 
\end{verbatim}
Breaks the selected objects into separate segments.  Each segment is
immediately flushed.  
\subsubsection{SCALE}
This selects the scales to use.  To select the range of data to plot
See:Command SET LIMITS.  
\begin{verbatim}
     SET SCALE {X|Y|Z|ALL} 
          [LINEAR] [MONTHS] [YEARS] [DAYS|TIME|HOURS|MINUTES|SECONDS]
               [POWER[=n]|ROOT[=n]] [USER=[n1[,n2,...n10]]] 
          [LABELS|TICKS=n [LONG|SHORT|NONE]] 
          [NORMAL [MEAN=x] [DEVIATION=s] ] 
          [[-]LOGARITHMIC] 
               [LABELS=n1,n2... [EXPONENTIAL[=ON|OFF]][NONE]] 
               [TICKS=n1,n2... [LONG|SHORT|ALL|NONE]] 
               [DECADES=n] [SUBTICKS=n] 
          [REVERSE[=ON|OFF]] [BASE=n] [PERMANENT] 
\end{verbatim}
To control other individual elements of the axes use:  
SET AXES,   SET TICKS,   SET LABELS,  SET OUTLINE  (Default:ALL  LINEAR
LABELS=6 TICKS=5) 
\paragraph{X$|$Y$|$Z$|$ALL }
Selects  which  axis to set.  You may set all three axes with one set
axes command.  
\paragraph{BASE      }
\begin{verbatim}
     BASE=n - Selects the base for the plot.  
  *  LINEAR  this is useful for non-decimal units.  The labeled ticks
     are put at locations BASE*ROUND*POWER where:  
          POWER = a power of 10 1,10,100 
          ROUND = 1.0,1.0,2.5, or 5.0 
  *  LOG  scaling  large  ticks  are put at powers of the base.  This
     determines where ticks are put.  BASE 2 would put labels  at  1,
     2,  4,  8, 16,....  If exponential notation is selected then the
     base is used in the notation.  
\end{verbatim}
\paragraph{DAYS      }
Values    are    the   number   of   days   from   the   base   date.
See:Command SET SCALE TIME 
\paragraph{LABELS    }
This  selects  the  maximum  number  of  ticks with labels.  For more
information See:LINEAR or LOGARITHMIC.  
\paragraph{LINEAR    }
Linear scale is used 
\begin{verbatim}
  *  LABELS - maximum number of big ticks with labels.  (Default:6) 
  *  TICKS - number of subintervals with unlabeled ticks.  SHORT|LONG
     determine whether the unlabeled ticks are  short  or  long.   If
     unspecified SHORT is assumed.  (Default:5 SHORT) 
\end{verbatim}
If TICKS=1 then there are no unlabeled ticks.  
\paragraph{LOGARITHMIC}
Log  scale  is  used.   Each  decade may be divided into subintervals
corresponding  to  the  leading  digit  of  the   scale   (1,2...,9).
-LOGARITHMIC generates a log scale for negative values.  
\begin{verbatim}
  *  LABELS=n1,n2..  - Specify which subintervals within a decade are
     labeled (2-9).  ALL labels all subintervals while NONE labels no
     subintervals.   If  you  specify  NONE  then no subintervals are
     labelled.  If EXPONENTIAL is specified exponential  notation  is
     used.  
  *  TICKS=n1...  - specifies which subintervals have unlabeled ticks
     (2-9).  ALL puts ticks on all subintervals while NONE  draws  no
     ticks.  SHORT|LONG determine whether labelled ticks are short or
     long.  If unspecified LONG is assumed.  Labelled ticks  at  each
     decade are always long.  
  *  DECADES=n  - (1 to 31) Specifies how often decades are labelled.
     For example DECADES=2 labels every other decade.  
  *  SUBTICKS=n  -  (0-9)  Specifies  how  many  subticks are between
     subinterval ticks.  If n is greater than 0 then long  ticks  are
     drawn  at  for  each  interval and n short subticks are drawn in
     between.  
\end{verbatim}
If  unspecified  labels, and ticks are picked for ``pleasing'' results.
Generally if less than one half  decade  is  plotted  the  rules  for
making  linear scales and ticks are followed.  Otherwise a maximum of
6 or 7 labels are produced.  Exponential notation  is  only  used  if
specified  or when the the label would exceed 6 digits or more than 3
decades are plotted.  
\paragraph{MINUTES   }
Values   are   the   number   of   minutes   from   the   base  date.
See:Command SET SCALE TIME.  
\paragraph{MONTHS    }
This selects labels of the form Jan, Feb etc.  This invokes a special
scale where Noon Jan 1=1.01 An entire day is  from  1.005  to  1.015.
1,13,25  are  labeled  January.  2,14 is February and so on.  No year
titles are generated.  Every 4'th year is a leap year with 366  days.
Year  1,2,3  are non leap years while year 4,6,8 are leap years.  The
current year is number 1.  
\begin{verbatim}
     1.  LABELS - The value n specifies the maximum number of labels.
         SHORT,LONG generate either  1  Letter  labels  or  3  Letter
         labels.  (Default:1) 
     2.  TICKS  -  Specify  the maximum number of small ticks between
         labels.  SHORT,LONG specify  either  long  or  short  ticks.
         (Default:6 SHORT) 
\end{verbatim}
\paragraph{NORMAL    }
The  scale  is  in  units  of standard deviation from the center of a
normal curve.  MEAN, DEVIATION set  the  center,  and  width  of  the
normal curve.  
\paragraph{POWER$|$ROOT}
This  sets  the scale to be the selected power or root.  N must be in
the range 0.1 to 100.  If n is not specified it is assumed to  be  2.
The  sign  of  the  axes  is preserved to avoid multiple values.  The
rules for determining ticks and labels are the  same  as  for  linear
scales.  

\begin{verbatim}
                               Example
     TD:SET SCALE Y ROOT=3 
\end{verbatim}
Y is scaled as Y'=SIGN(ABS(Y)**(1/3),Y) 
\begin{verbatim}
     TD:SET SCALE X POWER=2 
\end{verbatim}
X is scaled as X'=SIGN(ABS(X)**2,X) 
\paragraph{REVERSE   }
If this is specified before any data is plotter it makes the scale go
from large to small rather than small  to  large.   For  example  the
limits  on  the x scale are 0.0 and 1.1.  The left hand side of the X
axis would be 0.0 and the right would be 1.0.  REVERSE makes the left
hand side 1.0 and the right hand side 0.0.  

If  this  is  specified  for  3-d plots after the data is plotted, it
moves the origin of the axes to the other side of the specified axis.
For  example  the  following  will plot axes along the sides of a 3-d
cube.  
\begin{verbatim}
     TD:plot axes 
     TD:set scale x reverse y reverse;plot axes 
     TD:set scale x reverse=off y reverse z reverse;plot axes 
     TD:set scale x reverse=on y reverse=off;plot axes 
\end{verbatim}
\paragraph{SECONDS   }
Values   are   the   number   of   seconds   from   the   base  date.
See:Command SET SCALE TIME 
\paragraph{TICKS     }
This selects the maximum number of unlabeled ticks.  This varies with
the type of  scale  selected.   For  more  inforation  see:LINEAR  or
LOGARITHMIC.  
\paragraph{TIME-HOURS}
Values  are  the  number  of  hours  from  the  base  date.  Normally
date/time may be expressed as YYY$\backslash$MM$\backslash$DD or  YYY$\backslash$MM$\backslash$DD$\backslash$HH:MM:SS.SS  or
HH:MM:SS.SS.   This  is  translated into the number of hours from the
base date.  See:Command SET DATE.  You may have the  axes  labels  in
date/time  format  by  selecting either a time or hour scale.  If the
time is stored as seconds,minutes, or days then you  may  select  the
scale appropriately.  
\paragraph{USER      }
\begin{verbatim}
     USER=n1,n1...   specifies  the type of user scale.  This is only
\end{verbatim}
useful if you supply your own scaling  routines.   n1  is  the  scale
type.   n2...n10  are  values  passed  to the user routine.  The user
function is called:  
\begin{verbatim}
     Value=TDFNCT(X,SCALE) 
\end{verbatim}
X is the value to rescale.  
SCALE(10) is the array of values n1...n10.  
For example a log scale would be:  TDFNCT=ALOG(X) 
\paragraph{YEARS     }
Values are given by year and Julian day.  YEAR+DAY*.001 

\begin{verbatim}
                               Example
\end{verbatim}
Jan 1 1976=1976.001 
\begin{verbatim}
     1.  LABELS  -  Specifies  the  maximum number of labelled years.
         (DEFAULT:6) 
     2.  TICKS  -  Specifies  the  maximum  number of ticks per year.
         SHORT,LONG specify either long or short ticks.  
\end{verbatim}
\paragraph{PERMANENT }
sets the values permanently for the current axis.  Other axes are not
affected.  Any values specified to the right of the option  permanent
are not set permanently.  
\paragraph{Example   }
\begin{verbatim}
     TD:SET SCALE X LOG Y LINEAR 
\end{verbatim}
Sets the X scale to logarithmic, and Y to linear.  
\begin{verbatim}
     TD:SET SCALE X LOG SCALE 1,5 TICKS 2,3,4,6,7,8,9 
\end{verbatim}
This  sets  the log scales on the X axis so that labels will be drawn
at 1,10,100...  and 5,50,500...  All  subintervals  will  have  small
ticks 2,3,4,6,7,8,9,20,30,40,60....  
\begin{verbatim}
     TD:SET SCALE X LINEAR LABELS=10 TICKS=8 LONG 
\end{verbatim}
Selects a linear scale for X with a maximum of 10 labelled ticks, and
a maximum of 8 unlabelled ticks.  All ticks will be the same length. 
\begin{verbatim}
     TD:SET SCALE ALL LOG Z LINEAR 
\end{verbatim}
Sets X,Y scales logarithmic, and Z linear.  
\begin{verbatim}
     TD:SET SCALE ALL LOG 
\end{verbatim}
Sets all scales logarithmic.  
\subsubsection{SHIELD}
Sets  an  area  to  be shielded.  Nothing may be drawn inside this area
while shielded.  
SET SHIELD [=n] [OFF] [DAtA$|$TExT] [FROM]$|$TO [[X=]nx,[Y=]ny,[Z=]nz]]
\begin{verbatim}
          [CURSOR] 
\end{verbatim}

\begin{verbatim}
     1.  n - The shield number between 1 and 4.  If unspecified the
         next available shield is used.  
\end{verbatim}

\begin{verbatim}
     2.  OFF - Turns the n'th shield off.  If n is not specied all
         shielded areas are turned off.  
\end{verbatim}

\begin{verbatim}
     3.  FROM|TO - Allows you to specify the limit of the shielded
         area.  If no limits are specified the current window is
         shielded.  
\end{verbatim}

\begin{verbatim}
     4.  DATA - Specifies the limit in data coordinates.  
\end{verbatim}

\begin{verbatim}
     5.  TEXT - Specifies the limit in text coordinates.  
\end{verbatim}

\begin{verbatim}
     6.  CURSOR - You use the cursor to specify the limits.  
\end{verbatim}

\begin{verbatim}
                                example
     TD:SET SHIELD FROM 1,2 to 3,5 
\end{verbatim}
Sets a shielded area in text coordinates.  
\begin{verbatim}
     TD:SET SHIELD DATA FROM Y=5 
\end{verbatim}
Shield all data from Y=5 for all X values.  
\subsubsection{SIZE}
This sets the physical size of the plot.  Once set it remains the same
until a new SET SIZE command is issued.  This command should only be
used immediately after a NEW PLOT command or a SET DEVICE command.  It
may not be used after any commands that plot data.  
SET SIZE [[X=]x [BY$|$Y=] y] 
\begin{verbatim}
     [UNITS=units] [CM|CENTIMETERS|MM|MILLIMTERS|METERS|INCHES] 
     [MAGNIFY|REDUCE=factor] 
     [SIDEWAYS|ROTATED|NORMAL|ORIENTATION=n] 
     [MARGINS=[nl,[nr,[nb,[nt]]]]] [LEFT=n] [RIGHT=n] [BOTTOM=n]
          [TOP=n] 
\end{verbatim}
If no options are specified the sizes are reset to the defaults.  
(Default:NORMAL 13 by 10 inches MARGINS=0) 
\paragraph{X,y}
Set  the overall size of the plot The final size is the selected size
plus the margins.  The actual size of the plot depends on the  device
you  are  using.   The  x,y  are the size of the TEXT system.  If the
final size plus margins is larger  than  the  physical  size  of  the
display, the plot is automatically reduced to fit.  If the final size
is smaller than the physical  size  of  the  device,  the  plot  plus
margins is centered.  
\paragraph{UNITS}
Sets  the units of x and y.  For example UNITS=2.54 sets the units to
centimeters.  This changes only the position  units.   All  character
sizes  are  still  measured in tenths of an inch.  If you wish to set
the units without changing the size of the plot or you wish to change
the character size measurement use the command SET UNITS 
\paragraph{CM....}
This sets the units to a standard set of units.  
\paragraph{MAGNIFY$|$REDUCE}
Magnifies  or reduces the overall plot by the specified factor.  This
modifies the entire plot  including  the  character  sizes.   If  the
requested   final   size  is  too  big  for  the  device,  REDUCE  is
automatically set to a value smaller than 1.   The  magnification  or
reduction is limited to the range 1.0E-5 to 1.0E+5.  

\begin{verbatim}
                               example
     TD:SET SIZE REDUCE=2 
          or...  
     TD:SET SIZE MAGNIFY=0.5 
\end{verbatim}
reduce the plot from 13 by 10 to 6.5 by 5.0 inches in size.  
\paragraph{ORIENTATION}
This  is  an  integer value from 0 to 3 specifying the orientation of
the plot.  
\begin{verbatim}
     1.  Normal orientaiton 
     2.  Rotated 90 degrees clockwise.  This is the same as sideways.
         X goes from top to bottom, while Y points to the right.  
     3.  Rotated 180 degrees or Upside down 
     4.  Rotated 270 degrees clockwise 
\end{verbatim}
\paragraph{SIDEWAYS}
Rotates  the entire plot by 90 degrees.  If SIDEWAYS was specified by
SET DEVICE, the plot is already rotated, in which  case  NORMAL  will
rotate it back.  
\paragraph{MARGINS}
Sets the left,right,bottom, and top margins.  Any unspecified margins
are set equal to the left margin.  If no margins are  specified  then
all margins are set to zero.  
\paragraph{LEFT$|$RIGHT$|$TOP$|$BOTTOM}
Sets just the specified margin.  
\paragraph{Examples}
\begin{verbatim}
     TD:SET SIZE 10 BY 13 SIDEWAYS 
          or...  
     TD:SET SIZE X=10 Y=13 SIDEWAYS 
\end{verbatim}
This  rotates the normal plot from landscape to portrait and uses the
full SHO LOpage.  
\begin{verbatim}
     TD:SET SIZE Y=13 
\end{verbatim}
Sets Y to 13 while leaving X at the previous settting.  
\begin{verbatim}
     TD:SET SIZE SIDEWAYS 
\end{verbatim}
This rotates the plot and it occupies 13 by 10.  
\begin{verbatim}
     TD:SET SIZE 6.0 BY 9.0 SIDEWAYS MARGINS=1.15,.65,.775,.775 
\end{verbatim}
This  sets the size and margins to produce a 1.5'' margin on the left,
and 1'' on all other sides for a portrait plot on the VERSATEC.  
\begin{verbatim}
     TD:SET SIZE 9.0 BY 13.5 SIDEWAYS MARGINS=1.725,.975 
\end{verbatim}
This  sets the size and margins to produce a 1.5'' margin on the left,
and 1'' on all other sides for a portrait plot on the VERSATEC.   This
example  uses  a  9  by 13.5 drawing space instead of a 6 by 9 space.
There will be an automatic reduction of .666667.  
\subsubsection{SPHERICAL}
SET POLAR [ON$|$OFF] [RADIANS$|$DEGREES$|$FULLCIRCLE=n$|$GRAD] [PEERMANENT] 

This  sets  up to plot in spherical coordinates.  (R,THETA,PHI) instead
of (X,Y,Z) 
\begin{verbatim}
     1.  ON|OFF turn the spherical coordinates on or off.  (Default:ON) 
     2.  PERMANENT - Sets these parameters permanently.  
     3.  DEGREES...  - Selects angle representation (Default:DEGREES) 
         A.  GRAD - (0-400) 
         B.  DEGREES - (0-360) 
         C.  PERCENT - (0-100) 
         D.  RADIANS - (0-6.2831853) 
         E.  FULLCIRCLE=n - (0-n) 
\end{verbatim}

You  still  set  the limits in X,Y,Z coordinates, and automatic scaling
centers the origin.  When you plot,  the  axes  are  not  automatically
generated  if spherical coordinates are on.  You must manually plot the
axes with the command:  PLOT AXES.  
\subsubsection{STATISTICS}
SET STATISTICS 
\begin{verbatim}
     [SETS=[FROM] n1 [TO] [n2]] 
     [SELeCT="name"] 
     [POINTS|COLUMNS=[FROM] n1 [TO] [n2]] 
     [LINES|ROWS=[FROM] n1 [TO] [n2]] 
     [LIMITED [VLOG[=ON|OFF]] [[FROM]|TO [[X=]nx,[[Y=]ny[,[Z=]nz]]]
          [RECURSOR] [CURSOR] ] 
\end{verbatim}
This  sets  the range of data used by the lexicals.  If any changes are
made to the data you need to use this command again.  
\paragraph{LIMITED}
Limits  the  data  to the specified range.  For 3-d data the limit on
the dependent variable is ignored.  If limits are not specified,  the
default is the current plot limits.  See:Command SET LIMITS.  
\begin{verbatim}
     1.  X - Specifies X limit 
     2.  Y - Specifies Y limit 
     3.  Z - Specifies the Z limit 
     4.  CURSOR - Brings up the cursor.  You move it to the X,Y value
         you wish then press the space bar to enter both X,Y or X  to
         enter X or Y to enter Y.  
     5.  RECURSOR - The cursor enters both limits.  
              TD:SET STATISTICS LIMITED RECURSOR 
                   is the same as...  
              TD:SET STATISTICS LIMITED FROM CURSOR TO CURSOR 
     6.  VLOG - Draws a cross when you press the space bar, and draws
         a dotted line around the final limits.  
\end{verbatim}
\paragraph{LINES$|$ROWS}
For 3-d plots specifies which lines or rows of the mesh data to use. 
\paragraph{POINTS$|$COLUMNS}
This  specifies  which  points  to use.  For 3-d this is the range of
column numbers.  
\paragraph{SETS}
This specifies which data set to use.  
\subsubsection{STORAGE}
This  defines  what  is kept in storage.  SET ORDER automatically calls
SET STORAGE if room for extra data is needed.  This option remains  the
same until changed by a SET ORDER or SET STORAGE command.  The VAX uses
virtual memory, so the actual amount of storage is not not limited, but
expands  as  needed.   If you have a virtual memory allocation failure,
you may prevent it by preallocating the needed  memory  with  the  SIZE
option.   If  you still get memory allocation failures, ask your system
manager to increase the available virtual memory.  
SET STORAGE [ALL] [X$|$THETA] [Y$|$RADIUS] [Z$|$PHI] [DX$|$RY$|$DTHETA$|$RTHETA]
\begin{verbatim}
          [DY|RX|DRADIUS|RRADIUS] [DZ|RZ|DPHI|RPHI] [SYMBOL] 
     [SIZE=n] 
\end{verbatim}

(Default:SYMBOL,X,Y,DX,DY) 

ALL  selects  all  coordinates.   If  this  command is used without any
options, the storage is reset to the default.   If  all  3  coordinates
X,Y,Z are included in storage, 3-d plots are turned on.  

If   DX   is  selected  X  is  also  selected.   Likewise  DY,DZ,..DPHI
automatically select Y....PHI.  SYMBOL, X and Y are always included  in
the storage.  All other variables must be declared.  

If  THETA  is  specified polar coordinates are enabled permanently.  If
PHI is specified spherical coordinates are enabled.  

SIZE=n  allows  you  to pick the number of words of storage to allocate
for the VAX.  This is not necessary, but it may save time since you may
preallocate the storage and avoid multiple automatic storage expansion.
Normally the amount of storage needed for regular data is  about  n*d+4
where  n  is  the  number  of points and d is the number of coordinates
available in storage (SYMBOL,X,DX,Y,DY,....).   For  mesh  data  it  is
((n+1)*(m+1)+4)*d+4  where  n  and m are the mesh dimensions and d is 1
for regular mesh data or 2 for data with errors.  If you  have  a  name
for the data set it takes up 1 storage location for every 4 characters. 

\begin{verbatim}
                                Example
     TD:SET STORAGE X Y SYMBOL 
\end{verbatim}
To see the current storage use the command:  SHOW STORAGE.  
\begin{verbatim}
     TD:SET STORAGE SIZE=<4+101*201+20> 
\end{verbatim}
Preallocates storage for a 100 by 200 mesh with an 80 character name.  
\subsubsection{SYMBOL}
This selects the default symbol 
SET SYMBOL ``xx'' [SIZE=n] [THETA=n] [PHI=n] [ANGLE=n] [PERMANENT] 
If  no parameters are specified the symbol and size are set to NONE and
2 and the angles are turned off.  In other words SET SYMBOL is the same
as SET SYMBOL=NONE SIZE=2.  

This  command  determines  the symbol plotted data that have no symbol.
If you SET SYMBOL before reading data the symbol  is  assumed  for  the
data you are reading.  See:Command READ POINTS and PLOT.  
\paragraph{Xx}
is  the symbol to plot 0O to 9O or any duplex character.  DOT or ''  ''
produces a dot.  The second character in the xx pair selects the case
or  font  to  use.   Only  plot symbols 0O to 9O are guaranteed to be
properly centered.  If no symbol is desired specify NONE.  
\paragraph{ANGLE}
Specifies  the orientation angle of the symbol in a plot of 3-d data.
This is the rotation angle around the axis  specified  by  THETA  and
PHI.  
\paragraph{SIZE}
is  the  size  in tenths of an inch If the size is not set then it is
adjusted automatically by the window size.  The units may be modified
by SET UNITS CHARACTER.  
(Default:2) 
\paragraph{THETA}
Specifies  the  polar  angle of the normal to the symbol in a plot of
3-d data.  
\subsubsection{PERMANENT}
Makes  the  current  symbol the permanent default.  Normally a NEW PLOT
command resets the symbol to the permanent default.  
\paragraph{PHI}
Specifies  the  azimuthal angle of the normal to the symbol in a plot
of 3-d data.  A negative value makes the symbols plot with the normal
perpendicular to the screen.  In other words you are looking normally
at the symbol for a negative value.  
\paragraph{3-d}
If  you  plot  3-d  data  and  PHI  is  greater than 0, the symbol is
oriented according to the specified THETA, PHI, and ANGLE.  
\paragraph{Example}
\begin{verbatim}
     TD:SET SYMBOL "$" SIZE=3.5 
\end{verbatim}
plots dollar signs .35 inches in size.  
\begin{verbatim}
     TD:SET SYMBOL 0O SIZE=10 
\end{verbatim}
plots ``X'' 1 inch high.  
\begin{verbatim}
     TD:SET SYMBOL GF 
\end{verbatim}
plots uppercase Greek Gammas.  See:FONTS 
\begin{verbatim}
     TD:SET SYMBOL 0O PHI=0 ANGLE=45 
\end{verbatim}
plot crosses with the arms along the X,Y axes 
\begin{verbatim}
     TD:SET SYMBOL 0O PHI=90 THETA=90 ANGLE=45 
\end{verbatim}
plot crosses with the arms along the X,Z axes 
\subsubsection{TEXTURE}
SET TEXTURE$|$STRUCTURE [SOLID$|$DOTS$|$DASHES$|$DAASHES$|$DOTDASH$|$PATTERNED$|$
\begin{verbatim}
          FUNNY|SPACE] 
     [PERMANENT] 
\end{verbatim}
This  sets  the  default  texture  of  a  line  drawn by a BOX, CIRCLE,
ELLIPSE, DIAMOND, PLOT, BAR, JOIN, or HIST command.  To set the Texture
of  the  axes,  ticks, or outline See:Command SET AXIS, SET OUTLINE, or
SET TICKS.  You may also specify the  texture  with  any  command  that
draws.  For example you may JOIN DOTS or JOIN PATTERN and so on.  

\begin{verbatim}
     1.  SOLID - Solid line 
     2.  DOTS - Dotted line 
     3.  DASHES - Short dashed line 
     4.  DAASHES - Long dashed line (3x) 
     5.  DOTDASH - Dot-dash line 
     6.  SPACE - Adds an extra space between dots/dashes 
     7.  PATTERNED - Line determined by SET PATTERN.  
     8.  FUNNY  -  is  a dot at the ends of each line segment.  This is
         generally only useful for joined and smoothed curves.  
     9.  PERMANENT  -  Sets  the  current  texture  to be the permanent
         default.  
\end{verbatim}

(Default:SOLID) 

If  no  options  are  specified it is set to SOLID.  When a new plot is
started the texture is set to the permanent default.  

If  you  specify more than 1 texture option, or use the option PATTERN,
then a patterned texture is assumed and the pattern is generated.  When
FUNNY  precedes  any  other  texture  specifier  then  it  is generated
randomly.  
\paragraph{Example}
\begin{verbatim}
     TD:SET TEXTURE DAASH 
\end{verbatim}
This generates a ``hardware'' texture or long dashes.  
\begin{verbatim}
     TD:SET TEXTURE DOT DOT DASH 
          or...  
     TD:SET PATTERN DOT DOT DASH;  SET TEXTURE PATTERN 
\end{verbatim}
This  is  a dot-dot-dash texture, which is also set to be the current
pattern.  
\begin{verbatim}
     TD:SET TEXTURE DOT FUNNY DASH 
          or...  
     TD:SET PATTERN DOT RANDOM DASH;  SET TEXTURE PATTERN 
\end{verbatim}
This is a uniformly spaced dot 0.1'' followed by a random length dash. 
\begin{verbatim}
     TD:SET TEXTURE PATTERN DASH;  JOIN 
          or...  
     TD:SET PATTERN DASH;  SET TEXTURE PATTERN;  JOIN 
\end{verbatim}
Sets  the  texture  to be a patterned dash.  This will produce device
independent pictures, but they will be slower than:  
\begin{verbatim}
     TD:SET TEXTURE DASH;  JOIN 
          or...  
     TD:JOIN DASH 
\end{verbatim}
\subsubsection{THREE}
This sets parameters for 3-dimensional plots and turns ON 3-d graphics.
Essentially you must set up the viewpoint for 3-D.  The plot is seen by
an  observer  through a ``screen'' or window.  The location of the screen
and observers eye may be specified.  
SET THREE [ON$|$OFF$|$AUTOMATIC]] 
\begin{verbatim}
     [CENTER=x,y,z] 
     [VERTICAL=xv,yv,zv] 
     [DIRECTION=xv,yv,zv] [THETA=angle] [PHI=angle] 
     [DISTANCE|RDISTANCE=n] [SCRD=n] 
     [REDUCE|MAGNIFY=n] 
     [ORIGIN=x,y,z] 
     [SEPARATION=s|LEFT|RIGHT|MIDDLE] 
     [WORLD=x [BY] y [BY] z] 
     [XAXIS|YAXIS|ZAXIS=angle] 
     [PERMANENT] 
\end{verbatim}

NOTE:   The actual view also depends on the window size selected by SET
WINDOW.  


\begin{verbatim}
                                WARNING
\end{verbatim}

\begin{verbatim}
     If  this command is issued in between 2 plots overlaying each
     other, they will not have the same scales.  
\end{verbatim}


\paragraph{Options}
\begin{verbatim}
     1.  CENTER=x,y,z  -  Is  the center of the 3-d view.  This moves
         the plot around.  
     2.  DIRECTION=xv,yv,zv   -   Controls   the  direction  you  are
         "looking" from.  
     3.  THETA=angle  -  The  direction  you  are looking from (polar
         angle) 
     4.  PHI=angle  -  The  direction you are looking from (Azimuthal
         angle) 
     5.  DISTANCE - Distance from you to the object.  
     6.  SCRD - Distance from you to the viewing window.  
     7.  RDISTANCE - Modifies DISTANCE + SCRD together (DISTANCE from
         you to object) 
     8.  REDUCE|MAGNIFY  -  Modifies  size  of the object relative to
         current size.  
     9.  ON|OFF - Turns 3-d plot on or off.  
    10.  AUTOMATICE - 3-d plots are on if the data is 3-d.  
    11.  ORIGIN=x,y,z - Set origin of axes for a PLOT AXES command.  
    12.  PERMANENT   -   Makes   current   parameters  permanent  for
         subsequent plots.  
    13.  SEPARATION=n  -  Specifies left, right separation for stereo
         plots 
    14.  LEFT - Specifies left eye view.  
    15.  RIGHT - Specifies right eye view.  
    16.  MIDDLE  -  Specifies  center  view  (neither  left nor right
         "Cyclops") 
    17.  VERTICAL=xv,yv,zv - Direction of vertical.  
    18.  WORLD=x BY y BY z - Limits on x,y,z coordinates.  
    19.  X|Y|ZANGLE=n  -  Specifies  orientation of labels and titles
         around axis.  
\end{verbatim}
\paragraph{CENTER}
\begin{verbatim}
     CENTER=x,y,z 
\end{verbatim}
The center of the picture in 3-d WORLD coordinates.  This defines the
center of the object being viewed.   If  unspecified  it  is  at  the
center of the WORLD size.  (Default:6.5,5,5 inches) 
\paragraph{DIRECTION}
\begin{verbatim}
     DIRECTION=xv,yv,zv 
\end{verbatim}
Direction vector from CENTER to eye.  
\begin{verbatim}
     THETA=angle PHI=angle br;Alternate specification of DIRECTION in
\end{verbatim}
polar coordinates.  This is probably an  easier  number  to  specify.
You  should specify either the DIRECTION or THETA and PHI.  These are
specified relative to the VERTICAL vector.  

PHI  is  the  angle in degrees from vertical to line through the eye.
PHI is the azimuthal angle of the viewers eye.  (Default:60.0) 


\begin{verbatim}
                               WARNING
\end{verbatim}

\begin{verbatim}
     The  alogorithm  used  in Histogramming or Joining the data
     requires that the viewpoint can not be "over" the structure
     being  plotted.  The hidden line removal algorithm will not
     work for an improperly placed viewpoint.  You should  avoid
     PHI=0 or PHI=180.  
\end{verbatim}



THETA is the angle between projections in the horizontal plane of the
X-Axis and the line to the viewing  position.   THETA  is  the  polar
angle of the viewers eye.  (Default:30.0) 

\begin{verbatim}
                               Example
\end{verbatim}
You wish to look at the plot from the +X axis:  
\begin{verbatim}
     TD:SET THREE THETA=0 
\end{verbatim}
You wish to lood at the plot from the +Y axis:  
\begin{verbatim}
     TD:SET THREE THETA=90 
\end{verbatim}
\paragraph{DISTANCE,SCRD,RDIST}
\begin{verbatim}
     DISTANCE=nnn 
\end{verbatim}
Distance from CENTER to the EYE (Default:35in.) 
\begin{verbatim}
     SCRD=nnn 
\end{verbatim}
Distance from screen or viewing window to EYE.  (Default:-18in.) 
\begin{verbatim}
     RDISTANCE=nnn 
\end{verbatim}
Same  as  distance, but the SCRD is changed also to keep the view the
same.  (Relative distance) 

If SCRD is negative then the actual SCRD is adjusted so that the view
is the same independent  of  the  current  window  size.   If  $|$SCRD$|$
\begin{verbatim}
 = 0.5*DISTANCE then the plot will fill the current window.  
\end{verbatim}

Essentially  the  SCRD  is  the  distance from you to the screen, and
DISTANCE is the distance to the object being viewed.   The  ratio  of
SCRD/DISTANCE  determines  the  plot size while the absolute distance
determines the perspective of  the  plot.   Large  distances  produce
essentially  a  parallel  projection.  If you wish to modify the plot
size use the MAGNIFY or REDUCE option.  

Sometimes  vertical  lines  are  treated as being hidden when in fact
they are visible.  This may be corrected by moving the view point far
from the screen.  To set a distant viewpoint:  
\begin{verbatim}
     TD:SET THREE RDIST=8000 
\end{verbatim}
RDIST  adjusts  the  viewpoint  without  changing  the  size  of  the
resulting plot.  
\paragraph{REDUCE$|$MAGNIFY}
These  change  the size of the plot without changing the perspective.
REDUCE=0.5 or MAGNIFY=2 both increase the size by a factor of 2.  The
SCRD  is  modified  but  not  the  DISTANCE.   The  magnification  or
reduction is limited to the range 1.0E-5 to 1.0E+5.  
\paragraph{ON$|$OFF$|$AUTOMATIC}
This  turns  on  or  off 3 dimensional graphics.  AUTOMATIC specifies
that 3-d graphics is determined by the data.  3-d is turned  on  only
for  Mesh  data  or  if  all  3  coordinates  X,Y,Z  are  in storage.
(Default:ON) 
\paragraph{ORIGIN}
\begin{verbatim}
     ORIGIN=x,y,z 
\end{verbatim}
Sets  the position in DATA system at which 3 axes will intersect when
drawn.  (For MESH data, PLOT AXES is needed to draw them) The  origin
is normally set to be the minimum X,Y,Z value entered.  When set this
value is used locate the axes and the title.  
\paragraph{PERMANENT}
This  makes  all  currently  set  parameters  the default.  Since the
command line is parsed from left to right, options on the  left  side
of this option are permanent, but options to the right are not.  When
a new plot is started the current options  revert  to  the  permanent
value.  
\paragraph{SEPARATION}
This  specifies  the  distance  between  the  true  viewing point and
viewing   axis.    This   is   used   for   making   stereo    pairs.
(Default:SEPARATION=0) 
\begin{verbatim}
     1.  LEFT - SEPARATION = -1.5 
     2.  RIGHT - SEPARATION = 1.5 
     3.  MIDDLE - SEPARATION = 0 
\end{verbatim}
\paragraph{VERTICAL}
\begin{verbatim}
     VERTICAL=xv,yv,zv 
\end{verbatim}
Direction  vector in WORLD system which projects onto a vertical line
in the window.  (Default:0,0,1 - Z axis) 
\paragraph{WORLD}
\begin{verbatim}
     WORLD x [BY] y [BY] z 
\end{verbatim}
This  sets  the  limits  in inches on x,y,z in the world coordinates.
All data outside of these limits is clipped.  This is not set to  the
default when a new plot is started.  (Default:13,10,10 inches) 
\paragraph{XAXIS}
\begin{verbatim}
     XAXIS|YAXIS|ZAXIS angle 
\end{verbatim}

These  options are used to control how the ticks and labels are drawn
relative to the axes.  If the angle is not specified or it is greater
than  360 degrees the axes are drawn in either the XY,YZ, or ZX plane
for  maximum  visibility.   The  angle  allows  you  to  rotate   the
labels,ticks,  and  title around the axis.  The rotation is specified
by permuting the axes XYZ.  For a given axis n the angle  0  lies  in
the  n,n+1  plane,  and  angle  90  lies in the n,n+2 plane, with the
labels in the negative side.  

\begin{verbatim}
                            Table of axes
\end{verbatim}

This gives the plane and direction of the labels.  The sign indicates
the side the labels are drawn on.  
\begin{verbatim}
     Axis    0       90      180     270 degrees
     X       -XY     -ZX     XY      ZX
     Y       -YZ     -XY     YZ      XY
     Z       -ZX     -YZ     ZX      YZ
\end{verbatim}

\begin{verbatim}
     TD:SET THREE XAXIS=0 
\end{verbatim}
This  draws the X labels and ticks in the XY plane with the labels in
the negative X side of the axis.  

\begin{verbatim}
     TD:SET THREE YAXIS=90 
\end{verbatim}
This  draws the X labels and ticks in the YZ plane with the labels in
the negative Z side of the axis.  
\paragraph{Revisions}
Originally  there was a major problem with TOPDRAWER.  The documented
definition of THETA was not the same as the actual definition.  THETA
was actually measured from the Y axis and positive valued put the eye
closer to the positive  X  axis.   The  actual  definition  has  been
modified  to  conform  to  both  the  documentation  and conventional
notation.  If this is a problem, you may return to the old definition
by:  
\begin{verbatim}
     SET REVISION=1.  
\end{verbatim}
\paragraph{Examples}
There  is a set of sample plots you may view.  They include a drawing
of the various angles and distances.  They are:  
\begin{verbatim}
     TOPDRAWER_DIR:TD3D.TOP 
\end{verbatim}
\subsubsection{TICKS}
This  determines the presence or absence of tick marks and their sizes.
To control the default number of ticks see SET SCALE.  
SET TICKS [SIZE=n] [LONG=n] 
\begin{verbatim}
     [ALL|TOP|BOTTOM|RIGHT|LEFT|X|Y|Z] [ON|OFF] 
     [INTENSITY|WIDTH=n] 
     [NOCOLOR|WHITE|RED|GREEN|BLUE|YELLOW|MAGENTA|CYAN] 
     [NOTEXTURE|SOLID|DOTS|DASHES|DAASHES|DOTDASH|PATTERNED|FUNNY|
          SPACE] 
     [PERMANENT] 
\end{verbatim}
This  acts on the current plot only if PERMANENT not specified.  If the
parameters are omitted they are reset  to  the  original  default.   To
control other individual elements of the axes use:  
SET AXES, SET TICKS, SET LABELS, SET OUTLINE 

(Default:ALL ON) 
\paragraph{Options}

\begin{verbatim}
     1.  SIZE  -  Sets  the size of the smaller ticks in inches.  The
         larger ones  are  normally  3  times  the  small  ones.   If
         unspecified this is automatically set according to the paper
         size,   and   window   to   give   you   pleasing   results.
         (Default:0.1) 
     2.  LONG  -  Sets  the ratio of long ticks to short ticks.  This
         may be varied from 0.1 to 10.  (Default:3) 
     3.  ALL - enables ticks for top, bottom, right and left.  
     4.  TOP - enables tick for the TOP of plot 
     5.  BOTTOM - selects tick for the bottom of the plot.  
     6.  RIGHT - selects tick for the right hand side 
     7.  LEFT - selects tick for the left hand side.  
     8.  X,Y,Z - selects the axis to plot.  
     9.  ON - allows drawing ticks 
    10.  OFF - prevents drawing ticks 
    11.  INTENSITY  - Sets line intensity or width (0-5).  0 gets the
         intensity from the SET INTENSITY command.  
    12.  WHITE...  - Sets the line color.  
    13.  SOLID...    -   Sets   the  line  texture.   (Default:SOLID)
         See:Command SET TEXTURE 
    14.  PERMANENT  -  Makes the current settings permanent from plot
         to plot.  
\end{verbatim}

\paragraph{Examples}
\begin{verbatim}
     TD:SET TICKS ALL OFF BOTTOM ON LEFT ON 
\end{verbatim}
specifies ticks only on the bottom and left of the plot.  
\begin{verbatim}
     TD:SET TICKS RED DOTS SIZE=0.2 LONG=2 
\end{verbatim}
specifies  dotted red ticks with short ticks 0.2 inches long and long
ticks 0.4 inches.  
\subsubsection{TITLES}
This sets default title values.  
SET TITLES [ALL$|$TOP$|$BOTTOM$|$RIGHT$|$LEFT$|$X$|$Y] 
\begin{verbatim}
     [INDEX=n] [LINES=n] [MARGIN=n] [SCALE=n] [SHIFT=n] [SIZE=n] 
     [INTENSITY|WIDTH=n] 
     [NOCOLOR|WHITE|RED|GREEN|BLUE|YELLOW|MAGENTA|CYAN] 
     [PERMANENT] 
     [ESCAPE="char"] [SUBSTITUTE="char1char2"] 
\end{verbatim}
If  the  parameters are omitted they are reset to the original default.
You should make modifications to the title options before any  plotting
is  done.   The  MARGIN,  SIZE,  SCALE, INDEX, and LINES all modify the
location and size of relative windows.  See:Command SET WINDOWS.  
\paragraph{Options}
\begin{verbatim}
     1.  ALL,TOP,BOTTOM,RIGHT,LEFT,X,Y - Selects title to modify 
     2.  INDEX=n  -  The separation between consecutive lines for all
         titles 
     3.  INTENSITY=n - The intensity (width) of all titles 
     4.  LINES=n - The maximum number of lines to reserve for a title 
     5.  MARGIN=n - Extra margin around the window for titles 
     6.  PERMANENT - Makes values permanent 
     7.  SCALE=n - Multiply title size by n 
     8.  SIZE=n - Character size for all titles 
     9.  WHITE...  - Title color for all titles 
    10.  ESCAPE - Sets the escape character for imbedded text/case 
    11.  SUBSTITUTE  -  Sets  the begin, end substititute characters.
         This allows you to include predefined text strings.  
\end{verbatim}
\paragraph{ALL$|$TOP$|$BOTTOM$|$LEFT$|$RIGHT$|$X$|$Y}
Selects the title you wish to change.  (Default:ALL) 
\paragraph{INDEX}
Sets  the height of a title line.  The actual character height is the
same as the size.  This determines the separation between 2 lines  of
text  for  the  titles.   The INDEX applies to all titles LEFT...TOP.
(Default:2) 
\paragraph{INTENSITY}
Sets  the title intensity or line width.  (0-5) (Default:NONE) 0 gets
the intensity from the SET INTENSITY command.  
\paragraph{LINES}
Sets  the number of lines to reserve for a title.  For example if you
wish to have a 2 line title at the top of the plot you should:  
\begin{verbatim}
     TD:SET TITLE TOP LINES=2 
\end{verbatim}

Negative lines puts the title inside the window.  
\begin{verbatim}
     TD:SET TITLE BOTTOM LINES=-2 
\end{verbatim}
puts  the  bottom  title  inside  the window and allocates room for 2
lines of title.  (Default:1.2 BOTTOM 2) 
\paragraph{MARGIN}
This  sets  a fractional margin around each window.  Normally this is
0.025 or 2.5\% of the total window size.  You may set it to any  value
from  0.0  to 0.4.  If you use relative or numbered windows the final
window size is reduced by the margin as well as the number  of  title
lines,  and  labels.   The margin is not absulute.  You may still put
titles  inside  the  margin,  by  specifying  the   title   location.
(Default:0.025) 
\paragraph{PERMANENT}
makes the current values the permanent values.  
\paragraph{SCALE}
This sets a scale factor on the title size.  Normally LEFT,RIGHT, and
BOTTOM are 1.0 while TOP is 1.5.  This makes the Top title 50\% bigger
than  other  titles.   You may modify this to range from 0.1 to 10.0.
(Default:1.0 TOP 1.5) 
\paragraph{SIZE}
SIZE=n  gives the approximate spacing between chars.  in tenths of an
inch.  If n  is  negative  hardware  characters  will  be  used  when
available.   If  n  is positive software characters will be used when
mode  VECTOR=OFF.   This  size  determines  the  title  size  when  a
TOP,BOTTOM,...   title  is  specified.  The TOP title is normally 1.5
times this value.  Normally the size is automatically  set  according
to the paper size, and window to give you pleasing results.  The SIZE
applies to all titles LEFT...TOP.   The  units  may  be  modified  by
SET UNITS CHARACTER.  
(Default:2) 
\paragraph{WHITE...}
sets the title color.  (Default:NONE) 
\paragraph{Windows}
When  you  specify  a relative or numbered window the window is first
reduced by  the  margin  fraction.   Then  each  title  is  allocated
INDEX*LINES*SIZE*SCALE  around  the  window.   Finally  the window is
reduced to make room for labels.  Each title except  for  the  bottom
will  be  positioned INDEX*SIZE*SCALE*(LINES-.5) from the edge of the
window+labels.  The bottom is positioned INDEX*SIZE*SCALE/2 from  the
window+labels.  
\paragraph{ESCAPE}
This  defines a character to use as an ``escape'' character.  After the
escape character you enter a ``text'',``case'' pair.  This allows you  to
imbed  text  case  pairs  in  the text, without using a separate case
line.  When used with the substitution  you  can  define  macros  for
expressions.   To output the ESCAPE character you must enter it twice
followed by a space.  
\begin{verbatim}
                              example 
     TD:SET TITLE ESCAPE="@" 
     TD:TITLE TOP "This is a Greek alpha @AG" 
\end{verbatim}
Writes a title with an imbedded alpha.  
\begin{verbatim}
     TD:TITLE TOP "This an at symbol @@ ." 
\end{verbatim}
Writes the title:``This an at symbol @.'' 
\paragraph{SUBSTITUTE}
This  defines  a  pair  of  characters  to  delimit  a  substitution.
Substitutions may not be nested.  You define a string,  and  then  it
will  be  included  in  your  text  when the name is bracketed by the
substitute pair.  When used with the escape character, you can  build
complicated  expressions  and include them in your titles in a simple
fashion.  If you need to output the begin  substitute  character  you
should define a substitution for it.  
\begin{verbatim}
                               example
     TD:SET TITLE SUBSTITUTE="{}" 
     TD:DEFINE STRING curly "{" 
     TD:DEFINE STRING SEASONS "spring,summer,winter,fall" 
     TD:TITLE TOP "The seasons are {SEASONS}." 
\end{verbatim}
Writes the title:  
``The seasons are spring,summer,winter,fall.'' 
\begin{verbatim}
     TD:TITLE TOP "A left curly bracket is {curly}" 
\end{verbatim}
Writes the title:  
``A left curly bracket is \{'' 
\begin{verbatim}
     TD:SET TITLE SUBSTITUTE="" 
     td:title top "the seasons are seasons." 
\end{verbatim}
writes the title:  
``the seasons are spring,summer,winter,fall.'' 
\begin{verbatim}
     td:set title substitute="" 
     td:title top "the seasons are seasons ." 
\end{verbatim}
writes the title:  
``the seasons are spring,summer,winter,fall.'' 
\paragraph{Examples}
\begin{verbatim}
     td:set title index=3 
\end{verbatim}
increases the space per line to make room for sub/superscripts.  
\begin{verbatim}
     td:set title top line=2.5 
\end{verbatim}
allocates  room  for  2  lines  of title at the top, with a half line
margin between the titles and the window.  
\begin{verbatim}
     td:set title scale=1.0 right margin=0 bottom margin=0 line=-1 
\end{verbatim}
sets  all titles to the same size and removes the margin at the right
and bottom.  the bottom title is allocated 1 line of space inside the
window.  
\subsubsection{Units}
this sets the units that you may use to specify sizes and locations.  
set units 
\begin{verbatim}
     [reduce=units] [all|text|characters] [cm|centimeters| dm|
          decimeters| mm|millimters|meters|inches| points|mils] 
\end{verbatim}
if no options are specified the sizes are reset to the defaults.  
(default:text inches characters reduce=10) 
\paragraph{Cm....}
this sets the units to a ``standard'' set of units.  
\begin{verbatim}
     1.  mm       = cm/10 
     2.  dm       = 10 cm 
     3.  meters   = 100 cm 
     4.  inches   = 1.54 cm 
     5.  mils     = inches/1000 
     6.  points   = inches/72 
     7.  feet     = 12 inches 
     8.  yards    = 3 feet 
     9.  fathoms  = 6 feet 
    10.  rods     = 16.5 feet 
    11.  furlongs = 40 rods 
    12.  miles    = 8 furlongs 
    13.  leagues  = 3 miles 
\end{verbatim}
\paragraph{Reduce}
sets  the  units by the specified value.  this is the number of units
per standard unit.  
\paragraph{All...}
this  selects  which  set  of  units  to modify.  there are 2 sets of
units.  one is the text units which  specify  the  location  in  text
space,  while  the  other is used to specify the character and symbol
sizes.  the character units are also used to speicify the fill sizes.
normall  text  sizes  are in inches while the character units are one
tenth of the text units.  
\subsubsection{Wait}
see:command set pause 
\subsubsection{Width}
this sets the default line width or intensity of a plot.  
set width level [permanent] 
level=1 to 5.  (default:2) 5 is the widest (brightest).  
permanent - sets the current width to be the new default.  
if no options are specified it is set to 2.  
\subsubsection{Windows}
there are 3 ways to set windows, absolute, numbered, or relative.  

\begin{verbatim}
                           absolute windows
\end{verbatim}
set window$|$area [cursor] \{x$|$y\}[from] n1 [to] n2] [at n1 [size] n2]
\begin{verbatim}
          [level=n] 
          or...  
\end{verbatim}
set window$|$area [from] nx,ny [to] nx,ny] [at nx,ny [size] nx,ny] 
sets the x,y axes to run between the specified limits in inches.  

\begin{verbatim}
                               numbered
     set window [{first|last|next|previous|n1 [of n2]}] [level=n|
          inside|outside] 
\end{verbatim}
divides  the  entire screen into n2 windows of equal size, and sets the
current window to number n1.  the windows are as  close  to  square  as
possible.  

\begin{verbatim}
                               relative
     set window [{x|y} n1 [of n2]] [level=n] 
\end{verbatim}
divides  one  axis  of  the  screen  into  n2 windows of equal size and
selects one of them.  


\begin{verbatim}
                                warning
\end{verbatim}

\begin{verbatim}
     if  this command is issued in between 2 plots overlaying each
     other, they will not have the same scales.  
\end{verbatim}


\paragraph{Absolute}
the    window    may    not    exceed    the   current   plot   size.
see:command set size.  if cursor is specified, the cursor may be used
to  set  windows.  simply move the cursor to one corner of the window
and hit the space bar.   them  move  the  cursor  to  the  diagonally
oppostte  corner  and  hit the space bar.  the new limits are entered
into the journal file.  

\begin{verbatim}
                               example
     td:set window x from 1 to 6 y from 1.5 to 9.0 
          or...  
     td:set window from 1,1.5 to 6,9.0 
          or...  
     td:set window at 3.5,5.25 size=5,7.5 
          or...  
     td:set window x at 3.5 size=5 y at 5.25,7.5 
\end{verbatim}
all produce the same window in inches.  
\begin{verbatim}
     td:set window cursor 
\end{verbatim}
you  move the cursor to diagonally opposite corners of the window and
press the space bar to enter the cursor location.  
\paragraph{Numbered}
a  numbered  window  is  a special type of relative window.  if n2 is
omitted then the n2 is assumed to be the last n2.   both  n1  and  n2
must  be  greater  than  or equal to 1.0.  n2 must be smaller than or
equal to n2.  you may initially set up  a  window  as  relative,  and
specify it as a set of numbered windows.  when you do this the window
number modifies only the integer part of the  relative  window.   for
example:  
\begin{verbatim}
     td:set window 1 of 4 
\end{verbatim}
this sets the lower left window out of 6.  
\begin{verbatim}
     td:set window 6 of 6 
\end{verbatim}
sets the upper right window out of 6.  
\begin{verbatim}
     td:set window x 1.1 of 3.2 y 1.2 of 2.4 
     td:set window 5 
          is the same as 
     td:set window x 2.1 of 3.2 y 2.2 of 2.4 
\end{verbatim}
\paragraph{Relative}
if  n2  is  omitted then the last n2 for the current plot is assumed.
both n1 and n2 must be larger than 0.8 and n1 may not be larger  than
1.2*n2.  if both n1 and n2 are positive the window is reduced in size
to allow for labels and titles.  if negative the window occupies  the
entire  space.   if  you  do not specify a fixed symbol, grid, label,
title, or tick size then they are adjusted to be in proportion to the
window  size.   absolute  windows  do not adjust the sizes.  if n2 is
omitted then the current n2 is assumed.  

for  windows which do not occupy the entire space, the actual size of
the window depends on the location and position of labels and titles.
the size is set to allow a 5\% border with room for titles and labels.
the default is a window with approximately 8.5\% left on the right and
top for titles, and 13\% on the bottom and left for labels and titles.
for 3-d plots it is assumed that no  labels  will  be  produced.   to
change the size available for titles and labels use the set title and
set labels commands.  

non  integral  n1  or n2 may be used for fine adjustment.  if you use
negative windows you  may  also  need  to  turn  off  inside  labels.
see:command set labels inside 
\paragraph{Example}
\begin{verbatim}
     td:set window y 1.2 of 2.5 
\end{verbatim}
divides  y  into  2.5  windows  and sets the current window to 1.2.
assuming the y size is the default 10 inches, the total window runs
from  0.8  inches to 4.8 inches.  since the window is positive, the
actual window area is reduced to make room for labels.  the  actual
plot area would be approximately 1.6 inches to 4.3 inches.  
\begin{verbatim}
     td:set window y 1.2 of -2.5 
\end{verbatim}
divides  y  in  the  same  fashion as above, except that the actual
window occupies the entire specified  area  from  y=0.8  inches  to
y=4.8 inches.  if a second window is then specified:  
\begin{verbatim}
     td:set window y=2.2 
\end{verbatim}
it  is  adjacent  to  the first one, with the top axes of the first
window in the same location  as  the  bottom  axes  of  the  second
window.  
\paragraph{Level}
inside,ouside,  or  level=n  sets the window level.  levels may range
from 1 to 4.  inside increases the level by 1 and  outside  decreases
the level by 1.  using this option you may inset windows inside other
windows.  when you increase the level by 1 the current window is used
to  set the maximum limits of the new window.  you may increase level
by 1 or decrease it to any number from 1 to the current level.   when
you  increase the level, you have effectively decreased the plot size
to the current window size.  

\begin{verbatim}
                               example
\end{verbatim}
as  an  example  you  wish  to  create  a set of six windows within a
definite location.  
\begin{verbatim}
     td:set window x from 1 to 12 y from 1 to 7 
\end{verbatim}
set the outer level 
\begin{verbatim}
     td:set window 1 of 6 inside 
          or...  
     td:set window 1 of 6 level=2 
\end{verbatim}
now you have 6 equal size windows bounded by 1,1 and 12,7.  to return
to having only 1 window 
\begin{verbatim}
     td:set window level=1 
                               example
     td:set window x from 1 to 12 y from 5 to 7 
     td:set window level=2 from 1,1 to 2,2 
\end{verbatim}
insets  a window within another.  the actual physical location of the
new window is x=2 to x=3 and y=6 to y=7.  
\paragraph{Labels}
normally space for labels is allocated according to the current axes,
and label settings.  for example if you set labels left off, no space
will  be allocated for labels on the left side, so the window will be
larger.  if you need to have  several  plots  all  sharing  the  same
scale,  use either negative windows or absolute windows.  for example
2 plots which share the same x axis may be plotted.  
\begin{verbatim}
     td:set window y 1.2 of -2.6 
     td:plot set=1 
     td:set labels bottom off 
     td:set window y 2.2 
     td:plot set=2 
\end{verbatim}
only 1 bottom axes is plotted for both windows.  

it is assumed that left or right labels will occupy 6 characters.  if
this is incorrect you may adjust this value when you set  the  window
with the characters parameter.  
\begin{verbatim}
     td:set label characters=4 
\end{verbatim}
allocates  only  4  characters  for  the labels.  this gives you more
usable plotting area.  
\paragraph{Examples}
assuming  you  have  a plot size of 13 by 10 you may setup 6 windows,
and start with the window in the middle, bottom:  
\begin{verbatim}
     td:set window 2 of 6 
          or...  
     td:set window x 2 of 3 y 1 of 2 
\end{verbatim}
you may select the next window (right, bottom) by:  
\begin{verbatim}
     td:set window x next 
          or...  
     td:set window x 3 
          or...  
     td:set window 3 
\end{verbatim}
you may set the previous window by:  
\begin{verbatim}
     td:set window x previous 
          or...  
     td:set window x 1 
\end{verbatim}
you may select the last window (right, upper corner) by:  
\begin{verbatim}
     td:set window 6 
          or...  
     td:set window x 3 y 2 
\end{verbatim}
\paragraph{3-d}
when  setting the windows for 3-d plots there is compensation for the
viewing distance from the object if scrd is negative.  for  the  same
distance  and  positive  scrd  a  window  which is half the size of a
normal window will only show half of the view.   you  may  compensate
for  this effect by changing distance or scrd.  the effect is similar
to what you would see if you, and the window  remained  in  the  same
position  and  only the window size changed.  the following will plot
approximately the data with the same view.  
\begin{verbatim}
     td:set three distance 100 scrd 50 
     td:set window x 1 of 1 y 1 of 1 
     td:histogram 
          or ...  
     td:set three distance 100 scrd -50 
     td:set window x 1 of 2 y 1 of 2 
     td:histogram 
          or ...  
     td:set three distance 100 scrd 25 
     td:set window x 1 of 2 y 1 of 2 
     td:histogram 
\end{verbatim}

the  following  will  show  a  similar  view, but from a more distant
perspective:  
\begin{verbatim}
     td:set three distance 200 scrd 50 
     td:set window x 1 of 2 y 1 of 2 
     td:histogram 
\end{verbatim}

since  the size of 2 windows is less than half of a single window the
example with 2 windows shows slightly less than 1 window.  
\subsection{Show}
\begin{verbatim}
     show opt1 opt2 ,....  optn 
\end{verbatim}
shows options for the current plot.  most options for set may be used for
show.  the options are:  
\begin{verbatim}
     all       arrow     axis      bar       box       blink     card
     character cycle     circle    clear     color     command   ctrl_z
     cursor    data      date      device    diamond   digits    ellipse
     errors    exact     fit       file      flush     font      format 
     grid      histogram intensity keys      labels    lexicals  limits 
     mode      monitor   order     outline   pattern   pause     pen
     plots     polar     revision  scale     secondary shield    size
     statistics          storage   strings   symbol    texture   three
     ticks     time      title     units     width     values    version
     window 
\end{verbatim}
the  ``default''  values  are the values that are automatically set for the
next plot.  these are controlled by the ``permanent''  option  on  the  set
commands.  the show commands are not usually journaled with the exception
of show cursor.  
\subsubsection{Options}
\begin{verbatim}
     1.  all - shows all options except for cursor, flags, and
         histogram.  
     2.  arrow - default arrow format 
     3.  axis - default axis 
     4.  bar - ends of error bars 
     5.  box - default box size 
     6.  blink - whether plots are in blink mode 
     7.  card - maximum length of input lines 
     8.  cycle - the color,width, texture to cycle through.  
     9.  characters - shows character definitions 
    10.  circle - same as show ellipse 
    11.  clear - clear setting is deferred or immediate.  
    12.  color - default color or pen number 
    13.  command - shows the currently defined commands.  
    14.  ctrl_z - whether ctrl_z stops topdrawer.  
    15.  cursor - shows the location of the cursor.  
    16.  data [options] - show current data.  
    17.  date - show current base date.  
    18.  device - show current device being used 
    19.  diamond - default diamond size 
    20.  digits - the number of digits to use for output 
    21.  ellipse - default ellipse size 
    22.  errors - shows status of error messages, and any errors for
         the current plot are displayed.  
    23.  file - shows current files open, and line number.  
    24.  fill - the current fill patterns 
    25.  fit [full] - shows the current fit [with error matrix].  
    26.  flush - controls automatic flushing of plot data 
    27.  font - the current character set 
    28.  format - format for input lines 
    29.  grid - grid marks that overlay plot 
    30.  histogram - shows the list of histograms available.  
    31.  intensity - default intensity (line width) 
    32.  keys [key names] - the defined keypad keys 
    33.  labels - numeric labels on axes 
    34.  lexicals [lexical] - exams the current values of lexicals.  
    35.  limits - limits for plot axis 
    36.  mode - misc.  options.  these change the mode of operation.  
    37.  monitor - plot options for monitor command.  
    38.  order - determines interpretation of input data 
    39.  outline - outline around plot 
    40.  pattern - the pattern for patterned lines 
    41.  pause - controls whether topdrawr pauses at new plot 
    42.  pen - the pen or color to use in plotting.  
    43.  plots - the number of plots produced 
    44.  polar - whether mode is polar or spherical plots 
    45.  revision - the current revision level.  
    46.  secondary - secondary contour line options.  
    47.  scale - scale (log/linear) and units.  
    48.  shield - the location and sizes of shielded areas.  
    49.  size - size and orientation of screen or paper 
    50.  statistics - the statistics set by the last set statistics or
         show data command.  
    51.  storage - what is kept in storage 
    52.  strings - strings defined by define string 
    53.  symbol - default symbol to plot 
    54.  texture - default line style (dotted,solid...) 
    55.  three - parameters for 3-dimensional plots 
    56.  ticks - tick marks on axes 
    57.  title - size of title 
    58.  units - the units of measurement 
    59.  values - the user defined lexical values.  
    60.  version - the current version number of topdrawer 
    61.  width - the line width or intensity 
    62.  window - the plotting area (labels may be outside area) 
\end{verbatim}
\subsubsection{Cursor}
\begin{verbatim}
     show cursor 
\end{verbatim}

this  puts  a  cursor  on  the  graph.   after moving the cursor to the
desired location, press any key.  the current location  of  the  cursor
will  be  typed  and  entered into the journal file as a comment.  this
does not work if the device has no cursor.  if you have not plotted any
data this command will not work.  

the  position of the cursor is given in both text and data units.  if a
3-d plot is displayed, the x,y,z given  are  on  a  plane  through  the
origin parallel to the screen.  
\subsubsection{Data}
show data 
\begin{verbatim}
     [sets=[from] n1 [to] [n2]] 
     [select="name"] 
     [points|columns=[from] n1 [to] [n2]] 
     [lines|rows=[from] n1 [to] [n2]] 
     [brief|full|statistics|total[=on|off]] 
     [limited [vlog[=on|off]] [[from]|to [[x=]nx,[[y=]ny[,[z=]nz]]]
          [recursor] [cursor] ] 
     [output=file] 
\end{verbatim}
this  types  the  current  data  and  statistics  on the data.  you may
terminate the listing in the middle by pressing ctrl\_c.  
\paragraph{Brief...}
\begin{verbatim}
     1.  brief=on or full=off displays only the number of data points
         and sets.  
     2.  full=on  or  brief=off  displays  both  data and statistics.
         (default:full) 
     3.  statistics=on  displays  the statistics on the data for each
         set, and the grand total if more than 1 set.  
     4.  statistics=off displays the data without statistics.  
     5.  total=on  displays  only  the  total statistics for all data
         sets.  
     6.  total=off displays the data with statistics, but no total.  
\end{verbatim}
if you do not specify on or off then on is assumed.  
\paragraph{Limited}
limits the data shown to the specified range.  for 3-d data the limit
on the dependent variable fully modifies the statistics.  the  actual
data  shown  my include data outside the z limits, but the statistics
reflects only  the  data  within  all  limits.   if  limits  are  not
specified,    the    default    is    the    current   plot   limits.
see:command set limits.  
\begin{verbatim}
     1.  x - specifies x limit 
     2.  y - specifies y limit 
     3.  z - specifies the z limit 
     4.  cursor - brings up the cursor.  you move it to the x,y value
         you wish then press the space bar to enter both x,y or x  to
         enter x or y to enter y.  
     5.  recursor - the cursor enters both limits.  
              td:show data limited recursor 
                   is the same as...  
              td:show data limited from cursor to cursor 
     6.  vlog - draws a cross when you press the space bar, and draws
         a dotted line around the final limits.  
\end{verbatim}

\begin{verbatim}
                               example
     td:show data limited from 1,1 to 2,5 
          or...  
     td:show data limited from x=1 y=1 to x=2 y=5 
\end{verbatim}
shows  all  points  that with x between 1 and 2 and y between 2 and 5
inclusive.  
\begin{verbatim}
     td:show data limited from y=1 to y=5 
\end{verbatim}
shows all data points with y values between 1 and 5 inclusive.  
\begin{verbatim}
     td:show data limited from cursor to cursor 
\end{verbatim}
shows all data points as defined by the cursor limits.  
\begin{verbatim}
     td:show data limited from x=cursor to x=cursor 
\end{verbatim}
shows all data points according to x values defined by the cursor.  
\paragraph{Lines$|$rows}
for 3-d plots specifies which lines or rows of the mesh data to show. 
\paragraph{Output}
this  specifies  the output file for the data.  if this option is not
used, the default is sys\$output, or your terminal.  if output=none is
specified, then there is not output.  
\paragraph{Points$|$columns}
this  specifies  which  points to show.  for 3-d this is the range of
column numbers.  
\paragraph{Sets}
this specifies which data set to show.  
\paragraph{Statistics}
the  statistics  on  the data are typed at the end of the data.  this
includes the following:  
\begin{verbatim}
     1.  range of data for x and y 
     2.  sum of all y values and average y value per point.  
     3.  error  in  the  sum.   this is only produced if the y values
         have a non zero dy.  to specify errors for existing data see
         the command x.  
     4.  mean x value, and standard deviation of the mean.  
         mean=sum(x*y)/sum(y) 
         std=sqrt(sum((x-mean)*y))/sum(y) 
     5.  error  on  the  mean and standard deviation.  these are only
         produced if either dx or dy are non zero for the  data.   dy
         is  used  if available otherwise dx is used to calculate the
         error.  if the number of data points with y,dx,dy  non  zero
         is less than 3 then dx is also used to calculate the error. 
\end{verbatim}
\paragraph{Example}
you have entered a 3 dim array with 20 by 20 points.  x ranges from 0
to 10, y from 100 to 200.  the  total  array  will  not  fit  on  the
terminal screen so you show a subset of the data:  
\begin{verbatim}
     td:show data points 3 to 8 lines 5 to 10 
     td:show data points 1 5 lines 1 5 (shows 25 points) 
     td:show data line=1 (shows only first 20 points) 
     td:show data limited from .5,100 to .8,120 
\end{verbatim}

\begin{verbatim}
                               example
\end{verbatim}
you  have entered 100 data points x,y,dx,dy etc.  to show only points
number 5 to 20 
\begin{verbatim}
     td:show data points 5 to 20 
\end{verbatim}
\subsubsection{Errors}
this  shows the status of the error reports, and any error messages for
this plot are displayed.  
\subsubsection{Flags}
this  show  the status of all flags.  this is generally only useful for
debugging topdrawer.  there is no set flags command.  
\subsubsection{Histograms}
this  command  shows  the  list  of  histograms  made  available by the
set histogram command.  you may terminate the listing in the middle  by
pressing ctrl\_c.  
show histograms [current] [[ident=] n] [from] [n1] [to] [n2] 
\begin{verbatim}
     [select|name='hist_name'] 
     [exact[=on|off]] 
     [all|full|brief[=on|off]] 
     [contents[=on|off]] 
     [options[=on|off]] 
     [statistics[=on|off]] 
     [area|directory="dir/subdir..."] 
     [search[=on|off]] 
     [tree[=on|off]] 
     [entries[=on|off]] 
     [histogram[=on|off]] 
     [array[=on|off]] 
     [mesh[=on|off]] 
     [ntuples[=on|off]] 
     [nlmit[=n]] 
     [nmask[=n]] 
     [points|columns=[from] n1 [to] [n2]] 
     [lines|rows=[from] n1 [to] [n2]] 
\end{verbatim}
\paragraph{Options}
\begin{verbatim}
     1.  area/directory - specifies the area to list.  
     2.  all, or full - shows everyhing about the histograms.  
     3.  exact  -  histogram  names are treated as exact strings, and
         they are not searched in a case  independent  manner.   this
         must precede the option name="...".  
     4.  brief - shows only a list of the hist idents.  
     5.  contents  -  shows  a  listing  of  the  histogram contents.
         ntuples are shown according  to  the  set  histogram  select
         options.  
     6.  statistics - shows the histogram statistics.  
     7.  points|columns - selects the range of contents to show.  
     8.  search - prints information only if the specified histograms
         are found.  
     9.  tree - looks through the specified directory tree.  
    10.  options - shows the histogram options.  
    11.  area - shows a brief directory of the selected area.  
    12.  current - selects the current hist to show.  
    13.  from, to - selects a range of histograms to show.  
    14.  ident - the histogram number to show.  
    15.  mesh - chooses only mesh data or regular data.  
    16.  ntuple  -  selects  only ntuple data.  off selects both mesh
         and regular data.  
    17.  name - selects hist by name.  
    18.  nlimit - shows the limits on ntuple variables.  
    19.  nmask - shows the hist masks on ntuple variables.  
\end{verbatim}
\paragraph{Area$|$directory}
selects  the area to list histograms.  you may use wild characters to
select the particular areas to show.  see command set histogram.   if
you wish to get a directory of the direct access file:  
\begin{verbatim}
     td:show hist area=//file/subdir/sub-subdir...  
\end{verbatim}
if you wish to get a directory of a global comman section:  
\begin{verbatim}
     td:show hist area=//sect/subdir/sub-subdir...  
                               example
     td:show hist area=/x 
\end{verbatim}
shows all histograms in subdirectory //pawc/x.  
\begin{verbatim}
     td:show hist tree area=/x 
\end{verbatim}
shows all histograms in the directory starting with //pawc/x.  
\begin{verbatim}
     td:show hist area=/x/* 
\end{verbatim}
shows all histograms in subdirectories of area //pawc/x 
\begin{verbatim}
     td:show hist area=/x*/y 
\end{verbatim}
shows  all  histograms in the selecte areas.  for example //pawc/x1/y
and //pawc/x2/y will be shown but  //pawc/n/y  and  //pawc/x1/z  will
not.  
\paragraph{All$|$full}
shows  all  data  about  the  histograms.  this includes the id,name,
size,  and   range   of   values.    for   some   packages   entries,
over/underflows,  sum, mean, and standard deviations are also listed.
for rice histograms the cuts are also listed.  this option  may  slow
the  time  to  show each histogram, especially mesh histograms.  this
will not show extra information about  histograms  in  an  hbook4  rz
file.  all=on is the opposite of brief=off.  
\paragraph{Brief}
shows  only  the  list  of ids available.  all=off is the opposite of
brief=on.  
\paragraph{Contents}
shows  the  contents  of the histograms.  this only works for ntuples
and arrays.  points selects the range of the array or the element  of
the  ntuple.   lines  selects  the ntupl element.  the actual ntuples
shown are modified by previous set histogram  select  commands.   the
number  of significant figures in the output is set by the set digits
command.  
\paragraph{Current}
specifies the current histogram.  
\begin{verbatim}
     see:command set histogram.  
\end{verbatim}
\paragraph{Entries}
if  on  it selects only histograms which have entries (contain data).
if off it selects only histograms without entries.   the  default  is
on.   if the option entries is not used, then no selection by entries
is made.  
\paragraph{Exact}
if  on  and select is used, then the histogram names are searched for
in a case sensative manner.  
\paragraph{From$|$to}
specifies the range of histogram ids to show.  
\paragraph{Ident}
specifies a single histogram to show.  
\paragraph{Histogram}
if histogram=on selects only histograms.  
\paragraph{Ntupl}
selects only ntupl data to show.  
(default:on) 
\paragraph{Nlimits}
shows  you  the  existing limits on ntupl variables.  if the variable
number is negative, it has not been used.  
\paragraph{Nmasks}
shows  you  the  existing masks on ntupl variables.  if the histogram
number is negative, it has not been used.  
\paragraph{Mesh}
if  on  only 3-d or mesh histograms (scatterplots) are listed.  if on
or off are omitted, on is assumed.  
\paragraph{Select}
selects  the  histograms  by name.  all histograms beginning with the
hist\_name  will  be  shown.   percent  '\%'  and  star  '*'  are  wild
characters.  
\paragraph{Search}
information  is  printed  only if the specified histograms are found.
if search=off then the number of histograms and the current area  are
printed.  (default:search=off) 
\paragraph{Statistics}
shows  the statistics on the histograms.  this consists of the number
of entries, the range and the number of over/underflows.   this  will
not show extra information about histograms in an hbook4 rz file.  
\paragraph{Tree}
searches the current or specified directory tree for histograms.  for
example you have directories //pawc/x/a,  //pawc/x/b,  and  //pawc/y.
you wish to list all histograms in directories x/a and x/b:  
\begin{verbatim}
     td:show hist area=//pawc/x tree 
          or ...  
     td:set hist area=//pawc/x 
     td:show hist tree 
\end{verbatim}
\paragraph{Example}
\begin{verbatim}
     td:show hist brief 
\end{verbatim}
lists just the ids of all available histograms.  
\begin{verbatim}
     td:show hist id=5 full 
\end{verbatim}
types all information about histogram id number 5.  
\begin{verbatim}
     td:show hist from 10 select='energy' 
\end{verbatim}
types  the names and ids of all histograms with id above 9 which have
names beginning with 'energy'.  
\begin{verbatim}
     td:show hist select='*energy' 
\end{verbatim}
types  on  your  screen  the  names  of  all  histograms  with  names
containing the word 'energy'.  
\begin{verbatim}
     td:show hist select='*pi*energy' 
\end{verbatim}
types  the  names  of  all histograms containing 'pi...energy' in the
name.  
\begin{verbatim}
     td:show hist select='*energy' mesh 
\end{verbatim}
types  on  your  screen  the  names  of all 3-d histograms with names
containing the word 'energy'.  
\begin{verbatim}
     td:show hist select='*energy' mesh brief 
\end{verbatim}
types  on  your  screen  the  ids  of  all  3-d histograms with names
containing the word 'energy'.  
\subsubsection{Keys}
\begin{verbatim}
     td:show keys [key1,key2....] 
\end{verbatim}
shows you the current key definitions for the selected keys.  if all or
no   keys   are   specified   then   all   keys    will    be    shown.
see:command set key.  

to show kp0-kp9 keys you may specify just kp.  likewise you may specify
pf for pf1-pf4 or f for f1-f20.  

key names are:  
pf1-pf4,  kp0-kp9,  minus,  comma,  period,  enter, f1-f20, insert, do,
prev\_screen, next\_screen.  
\subsubsection{Lexicals}
\begin{verbatim}
     show lexicals [v_lex1] [v_lex2]...  
\end{verbatim}
shows you the values of the specified lexicals.  
\subsubsection{Time}
this  shows  the current cpu and clock time for your process.  there is
no set command corresponding to show time.  
\subsubsection{Example}
\begin{verbatim}
     td:show box circle diamond 
\end{verbatim}
shows the default sizes for drawing boxes etc.  

\begin{verbatim}
     td:show color intensity texture 
\end{verbatim}
shows all the default attributes for lines.  

\begin{verbatim}
     td:show storage 
\end{verbatim}
shows the current storage allocation.  
\subsection{Smooth}
smooth [\{x$|$y$|$z$|$radius$|$theta$|$phi\}] [level=n] 
\begin{verbatim}
     [append[=on|off]] [name="name"] 
     [check[=on|off]] 
     [error[=on|off]] 
     [flat[=on|off]] [average[=on|off]] 
     [sets=[from] n1 [to] [n2]] 
     [select="name"] 
     [points|columns=[from] n1 [to] [n2]] 
     [lines|rows=[from] n1 [to] [n2]] 
     [limited [vlog[=on|off]] [[from]|to [[x=]nx,[[y=]ny[,[z=]nz]]]
          [recursor] [cursor] ] 
     [log[=on|off]] [monitor[=on|off]] 
\end{verbatim}

replace the x,y, or z values by new values that give a smooth curve.  the
data is assumed to be a histogram of equally spaced bins.   an  alternate
way  of  smoothing  data  is  by  convolution.  see command convolute.  a
convolution may be used to both smooth and ``sharpen'' the data.  
\subsubsection{Options}
\begin{verbatim}
     1.  append  -  puts  the  smoothed  data  into  a  new  data  set.
         (default:off) 
     2.  average - modifies flat.  when average=off each channel is the
         sum of n adjacent channels.  (default:on) 
     3.  check  -  checks  the  data to see if it is a proper histogram
         with x or x/y for mesh equally spaced.  if it is not an  error
         message is issued,and no smoothing is done.  (default:on) 
     4.  error  -  modifies the errors on the data assuming statistical
         errors.  if off the errors are not modified.  essentially  the
         smoothing  algorithm  is  used  on  the  square of the errors.
         (default:off) 
     5.  flat - smooths the data by adding together and averaging every
         n points.  the result has every n points  identical.   level=n
         selects the number of points.  for example smooth flat level=2
         averages points 1+2, 3+4, 5+6...  (default:off) 
     6.  level=n - selects approximately the number of points on either
         side of the bin that are used in  determining  the  value  for
         that  bin.  the range of levels is 1 to 5.  1 will smooth over
         3 points, while 5 will smooth over 11 points.  (default:3) 
     7.  lines/rows - selects which rows of a mesh are smoothed.  
     8.  limited - selects the range of x,y,z values to smooth.  
     9.  log - logs or types the results on your terminal.  
    10.  monitor - histograms the original data and joins the result.  
         see:  set mode monitor 
         the  result  is  joined  using  the  secondary attributes.  to
         change them use the set secondary command.  
    11.  name - selects the name of the appended data set 
    12.  points/columns  -  selects  which  data points or columns of a
         mesh are smoothed.  if omitted all are smoothed.  
    13.  select - selects the data set by name 
    14.  sets - selects which data sets to use.  
    15.  x,y,z...  selects whether x, y, or z values are replaced.  for
         mesh data it selects the axis to smooth along.  (default:y) 
\end{verbatim}

\subsubsection{Algorithm}
the  data  is  smoothed  by  repeatedly  transforming  the  points  and
residuals, using running means, quadratic interpolation,  and  hanning.
similar algorithms are described in exploratory data analysis by john w
tukey at slac.  this has been modified slightly by j.  clement to treat
sparse statistical data correctly.  the original alogorithm would throw
away isolated data points which were larger than the surrounding data. 
\subsection{Sort}
sort [symbol$|$x$|$dx$|$y$|$dy$|$z$|$dz$|$$|$theta$|$radius$|$phi$|$dtheta$|$dradius$|$dphi\} 
\begin{verbatim}
     [sets=[from] n1 [to] [n2]] 
     [select="name"] 
     [points=[from] n1 [to] [n2]] 
     [append[=on|off]] [name="name"] 
     [log[=on|off]] 
\end{verbatim}
sorts  the  data  according to the order of the fields (x,y...) selected.
if fields are omitted  then  they  are  sorted  according  to  the  order
symbol,x,dx,y,dy,z,dz.  each data set is sorted independently.  

\begin{verbatim}
                                 example
     td:sort x y z symbol 
\end{verbatim}
sorts  data  so  that  all low x are first.  within groups of identical x
values all low y values are first, and so on.  since dx,dy,dz are omitted
the actual sorting order is:  x,y,z,symbol,dx,dy,dz.  
\subsection{Spawn}
this  command  temporarily  spawns  out  of  topdrawer to vms.  the spawn
command is not journaled unless abort=off mode is selected.  
\begin{verbatim}
     spawn [dcl_command] 
\end{verbatim}

for  example  you  are  trying  to  do a plot, but you have forgotten the
filename with the topdrawer source, so you:  
\begin{verbatim}
     td:spawn  directory you will get a directory listing so you can find
\end{verbatim}
the file name.  

if  you use the spawn command with no parameters, you return to topdrawer
with the vms command:  
\begin{verbatim}
     $ logoff 
\end{verbatim}
\subsection{Stop}
\begin{verbatim}
     stop|end|exit|halt|quit   ['string']   these   commands   all   stop
\end{verbatim}
topdrawer.  any data remaining in the buffers is flushed, and all  output
files are closed.  
this is also equivalent to typing ctrl\_z if ctrl\_z is on.  

the  program  is stopped and the string is typed.  if you wish to stop in
the middle type ctrl\_c.   this  will  abort  the  current  operation  and
restore  the  prompt.   if  you  are in the middle of getting data from a
file, the file will skip to the next clear  or  stop  command.   pressing
ctrl\_c  twice  rapidly  in  succession will abort all file input.  if you
type ctrl\_c twice rapidly again you will  be  prompted  if  you  wish  to
terminate the program completely.  
\subsection{Swap}
swap \{x$|$dx$|$y$|$dy$|$z$|$dz\} [with] \{x$|$dx$|$y$|$dy$|$z$|$dz\} 
\begin{verbatim}
     [points=[from] n1 [to] [n2]] 
     [sets=[from] n1 [to] [n2]] 
     [select="name"] 
     [error[=on|off]] 
\end{verbatim}
this  command swaps the data for 2 variables.  if error is specified then
the corresponding errors are swapped with the data.  this  is  useful  if
the  data in in the wrong location for the type of plot you wish to make.
an example of this is making 3-d plots from binned data.  you  have  read
in  data,  x,y,dy  and  you  have binned it into a frequency distribution
using the bin command.  now the result is in x,y,dx,dy.  you wish to make
a 3-d histogram, but the data must be in x,z,dx,dz.  to fix the problem: 
\begin{verbatim}
     td:swap z with y 
     td:swap dz with dy 
\end{verbatim}
now you can set the y value and perform the histogram.  
\begin{verbatim}
     td:y=10.0 
     td:set three on 
     td:histogram 
\end{verbatim}
you may copy data using the x,y,z,dx,...  commands.  
\subsection{Symbol}
symbol=symbol 
\begin{verbatim}
     [limited [vlog[=on|off]] [[from]|to [[x=]nx,[[y=]ny[,[z=]nz]]]
          [recursor] [cursor] ] 
     [sets=[from] n1 [to] [n2]] 
     [select="name"] 
     [points=[from] n1 [to] [n2]] 
\end{verbatim}
this  resets  the  symbol  for the selected data points or sets.  symbols
should be 0o,1o,...9o, dot or none.  any other symbol may not be properly
centered.  
\subsection{Title}
puts a title on the plot 
title [x, [y, [z]]$|$cursor] 'text\_of\_title' [time] [lexicals] 
\begin{verbatim}
     [top|bottom|right|left|x|y|z|general] 
     [data|xdata|ydata|text] 
     [ljustify|center|rjustify] [lines=n] 
     [size=n] [angle=n] [spaces=n] [index=n] [digits=n] 
     [case 'case modifier string'] 
     [intensity|width=n] [white|red|green|blue|yellow|magenta|cyan] 
\end{verbatim}
the text will be written on the plot.  it must be enclosed in apostrophes
or quotes.  
\subsubsection{Xyz}
x,  y, z - specify the position of left most character in the text.  if
no position is given, and top,bottom...  are not specified the title is
placed  below  the  most  recent title.  cursor produces the cross hair
curson on interactive terminals.  you move the cursor to  the  location
you wish the title to appear, and press the space bar.  the position of
the cursor is written to the journal file so you may  repeat  the  plot
exactly.   if  you  need  titles  at  angles other than horizontal, you
should also use the option angle=n.  
\subsubsection{Angle}
angle=n  -  gives  the  angle  in  degrees  measured  counterclockwise.
angle=90 runs the title from bottom to top.   this  has  no  effect  on
perspective labels for 3-d plots.  
\subsubsection{Ljustify$|$center$|$rjustify}
either left, right or center justifies the title.  (default:ljustify) 
\begin{verbatim}
  *  ljustify - the center of the first character is at x,y,z.  
  *  center  -  the  center  of  the string is at x,y,x.  the center is
     assumed half way between the first and last characters.  
  *  rjustify - the center of the last character is at x,y,z.  
\end{verbatim}
if  top,bottom...   are  specified  then  the  title  is  automatically
centered.  

\begin{verbatim}
                                example
     td:title top ljust 'left' 
     td:title top 'center' 
     td:title top rjust 'right' 
\end{verbatim}
puts 3 titles at the top of the plot on the left,center, and right.  

\begin{verbatim}
                                example
     td:title 10,5 rjust 'location 10,5 -->+' 
\end{verbatim}
puts the title with the plus sign at x=10, y=5.  
\subsubsection{Data$|$xdata$|$ydata$|$text}
specifies  the data or text coordinate frame.  normally x, y are in the
text coordinate system.  if you specify x,y, and z then the data  frame
is assumed.  
data - both x,y are in the data frame.  
text - both x,y are in the text frame.  
xdata - x is in data frame y in text frame.  
ydata - y is in data frame x in text frame.  
\subsubsection{Digits}
digits=n  selects  the  number  of significant figures to reproduce for
lexicals.  if n is omitted it is assumed to be 4.  you may also set the
number of digits with the set digits command.  

\begin{verbatim}
                                example
     td:title top 'mean=',digits=6,t_mean,' err=',digits=2,t_emean 
\end{verbatim}
the top title will be the mean and error for the data.  
\subsubsection{Lexicals}
you  may  include a lexical value in your title.  a title lexical is an
option in the form t\_xxxx, where xxxx is the value to plot.  the number
of  significant digits in the number is normally 4.  you may alter this
with the digits=n option or the set digits command.   for  example  you
wish to show the sum of the data in the top title:  
\begin{verbatim}
     td:title top 'my data - sum=',t_sum 
\end{verbatim}
you may wish to limit the sum to a portion of the data so you first use
show data to calculate the sum for a limited portion, then  put  it  in
the title:  
\begin{verbatim}
     td show data limited from x=5 to x=10 
     td:title bottom 'sum=' t_sum 'x from ' t_xmin ' to ' t_xmax 
\end{verbatim}

\begin{verbatim}
     td:title top 'sum=' t_sum ' '+' case'm' t_esum 
\end{verbatim}
for more information see:topdrawer lexicals.  
\subsubsection{Time}
inserts the current date and time into the title.  for example:  
\begin{verbatim}
     td:title top '(date:  ' time ')' 
\end{verbatim}
assuming  the  current  time  is noon on the jan 1, 1987, the following
title is plotted at the top:  
\begin{verbatim}
     (date: 1-jan-87 12:00:00) 
\end{verbatim}
\subsubsection{Top$|$bottom$|$right$|$left$|$general}
top,bottom,right,left - specify the title position relative to the data
window.  general is a synonym for top.  the title is  centered  at  the
appropriate  position.   the  size  and angle are determined by the set
size command.  if a size, or x,y,z are  specified,  they  override  the
default  position  and  size.   a  top  title  is  generally drawn with
character size  1.5  times  the  default.   if  no  other  options  are
specified, then the character size is reduced by up to a factor of 4 to
keep the title from overflowing the screen.  
\subsubsection{X$|$y$|$z}
for  3-d  plots  the  label  is  drawn  in  perspective parallel to the
selected axis.  for 2-d plots x,y,z are synonyms for left,bottom,top.  
\subsubsection{Lines}
lines=n  -  moves  the  title  up space for n extra lines of text.  the
title is moved up by n-1 lines.  if lines=1 then the title is not moved
at all.  
\subsubsection{Size}
size=n  gives  the  approximate spacing between chars.  in tenths of an
inch.   if  n  is  negative  hardware  characters  will  be  used  when
available.   if  n is positive software characters will be used if mode
vector=on.  if not specified the size is determined by the most  recent
title  command.   once  this parameter has been set, the character size
remains the same unless a bottom,top,right,left,x,y,z or size option is
used.   if  not  specified  the  angle is determined by the most recent
title command.  
\subsubsection{Index}
index=n - specifies the line spacing in multiples of character spacing.
this is used for multiline titles.  
\subsubsection{Spaces}
spaces=n  this  specifies  how  much space the title occupies.  this is
only necessary if variable character spacing, or  sub/superscripts  are
used and you wish to center the title.  
\subsubsection{Intensity}
intensity - determines the intensity or line width of the title.  
\subsubsection{White...}
white....  determines the color of the title.  
\subsubsection{Case}
controls  the  format  of the title.  it must be the next command after
the title or more command or on the same line with the  title  command.
it  modifies only the preceding string.  it must immediately follow the
string either on the same, or the next line.  

\begin{verbatim}
     case 'case_text' 
\end{verbatim}

this  specifies  the  case  or  character type of each character in the
title text.  for the definitions of all possible case characters see:  
\begin{verbatim}
     topdrawer fonts 
\end{verbatim}

it  is  not  necessary to use a case command if you only want upper and
lower case ascii characters.  simply type them into your title  as  you
wish to see them.  

\begin{verbatim}
                                example
     td:title top 'abg' case 'ggg' 
\end{verbatim}
produces a title with 3 lower case greek letters alpha,beta,gamma.  
\begin{verbatim}
     td:title top 'abc' 'abg' case 'ggg' 
\end{verbatim}
produces a title with 3 roman letters followed by 3 greek letters.  
\begin{verbatim}
     td:title top 's=' case 'g' v_sum 
\end{verbatim}
produces a greek ``sigma''.  
\begin{verbatim}
     td:title top 's=' v_sum case 'g' 
\end{verbatim}
does not produce any greek because the case and string are separated by
a lexical.  
\subsubsection{3-d}
if  the  options  x,y,  or  z  are  used,  then the title is plotted in
perspective relative to the default x,y,z axis.  if you set  each  axis
to  be  in  a  particular location, then you should plot the axis title
specifying x,y,z or specify the origin  using  the  set three  command.
the  title  starts  at the origin of the axis.  if the option center is
used the title is centered along the axis.  

if  the  title  position  is  also specified, then the title is plotted
according to the position specified.  

the  axis  angle  determines  the  orientation  of  the title.  see the
command set three.  

if  a  title is drawn in perspective, subsequent title commands without
any position specifications will draw more titles above  or  below  the
first one.  the displacement is picked away from the axis.  
\subsubsection{Consecutive\_titles}
if a title command omits the options top, bottom, right, left, x, y, or
z and no locations (x,y,z) are specified,  then  the  title  is  placed
under the previous one.  the spacing is determined by the index option.
the previous index, angle, size, justification, lines, and spacing  are
retained.  if the previous title was a 3-d title then the current title
may be placed above it, depending on the axes angle and orientation  of
the view.  
\subsection{Transform}
transform [from$|$to cartesian$|$polar$|$spherical] [sets=[from] n1 [to] [n2]]
\begin{verbatim}
          [select="name"] [append[=on|off]] [name="name"] [log[=on|off]] 
\end{verbatim}
transforms  data  coordinates  from  one frame to another.  this will not
work with mesh data.  if  omitted  the  from  or  to  is  assumed  to  be
cartesian coordinates.  
\begin{verbatim}
     1.  append - appends the transformed data as a new data set.  
     2.  log - logs the operation on your terminal.  
     3.  polar - x,y to theta,r.  
     4.  spherical - x,y,z to theta,phi,r.  
                                 example
     td:transform from polar to sperical set=5 
\end{verbatim}
transforms data set 5.  
\begin{verbatim}
     td:transform to polar 
\end{verbatim}
all data sets are transformed from x,y,z to polar coordinates theta,r,z. 
\subsection{Type}
type  [erase]  [top] [bottom] [bell] [bold] [blink] [reverse] [underline]
\begin{verbatim}
          [wide]  [background|foeground  [  [white|red|green|blue|yellow|
          magenta|cyan]] [nn] ["string"] 
\end{verbatim}
the string is typed on your terminal.  this is useful for typing messages
on the users terminal while plots are being produced from an input file. 
\begin{verbatim}
     1.  erase  -  this  erases  the screen (ansi terminal).  this erases
         text from a vt-1xx/2xx compatible terminal.  this does not erase
         tektronix  graphics,  but  regis  graphics will be erased.  this
         must be precede the string.  
     2.  top - the string is at the top of the screen (ansi terminal).  
     3.  bottom  -  the  string  is  at  the  bottom  of the screen (ansi
         terminal).  
     4.  bell - this rings the bell.  
     5.  bold,blink,reverse,underline,wide - sets the video attributes of
         the line.  these must be after the erase,top options and  before
         the  string.  several of these may be used together.  these only
         work for a vt-100 or compatible ansi terminal.  
     6.  background|foreground  control  the color.  (ansi terminal) once
         the color has been set it remains set until another type command
         sets new colors.  
     7.  nn - this inserts a control character (0-255) into the string.  
\end{verbatim}
nn  and string may be repeated to generate control characters.  see also:
set prompt.  
\subsection{X}
put  a  sequence  of  values  into one of the data point or error arrays.
this is useful when the x values are uniformly spaced, and you only  wish
to  enter  the  y  values.   this may also be used for entering constant,
proportional or counting errors.  you may  also  modify  the  data  by  a
constant or an equation.  for example you can normalize the data.  
\{x$|$y$|$z$|$dx$|$dy$|$dz$|$theta$|$radius$|$phi$|$dtheta$|$dradius$|$dphi\}[=] 
\begin{verbatim}
                            generate a range of values
     [bins|values] [[from xlow] [to xhi] [by|width|step xstep] [n=n] 
                                    operations
     "expression" 
     {x|y|z|dx|dy|dz|theta|radius|phi|dtheta|dradius|dphi} [plus|+|minus|
          -|times|*|divide|/] n [[no]error=n] [poisson|sqrt|samples=s] 
                                  data selection
     [append[=on|off]] [name="name"] 
     [sets=[from] n1 [to] [n2]] 
     [select="name"] 
     [points|columns=[from] n1 [to] [n2]] 
     [lines|rows=[from] n1 [to] [n2]] 
     [log[=on|off]] 
\end{verbatim}
\subsubsection{Append}
the  specified  points  are  appended to last data set.  if you wish to
create a new set you should use the data set command.  this option  may
not  be  used  with the sets or points options.  it is also illegal for
mesh data.  along with  append  you  must  specify  3  of  the  options
from,to,by,n.  
\subsubsection{Bins$|$values}
\begin{verbatim}
     bins
\end{verbatim}
creates  values  at  the  centers  of bins whose edge is defined by the
values.  
\begin{verbatim}
     values
\end{verbatim}
specifies  the  values  for the specified coordinate.  if you specify a
range of values or  bins,  you  may  not  specify  operations  such  as
times,divide,minus,plus  or  a  coordinate  such  as  x,y,dx...   or an
equation.  if you generate a new data set you must  specify  3  numbers
for  from,to,by,  and  n.   if  you  modify an existing data set only 2
numbers are spcified.  you may also specify a single value.  
\begin{verbatim}
                               examples
     td:dx=0.5 
\end{verbatim}
sets all dx values to 0.5 
\begin{verbatim}
     td:x=bins from 1 to 10 by 2 
\end{verbatim}
generates a new data set with values x=2,4,6,8 
\begin{verbatim}
     td:x=values from 1 to 10 by 2 
          or...  
     td:x from 1 to 10 by 2 
\end{verbatim}
generates a new data set with values x=1,3,5,7,9 
\begin{verbatim}
     td:x from 1 by 2 set=2 
\end{verbatim}
modifies data set 2 with values x=1,3,5,7,9...  
\begin{verbatim}
     td:x=y times from 1 to 10 by 2 
\end{verbatim}
is illegal 
\begin{verbatim}
     td:x=y times 2 
\end{verbatim}
is legal 
\subsubsection{Error}
this  specifies  the error on the value you have specified.  if you use
noerror the error analysis is turned off.  this only works when you are
modifying  x,y,z  values.   the  specified  error is used to modify the
dx,dy,dz value.  
\begin{verbatim}
                                example
     y=y * 5.0 
\end{verbatim}
will multiply each y and dy value by 5.0.  
\begin{verbatim}
     y=y * 5.0 error=0.1 
\end{verbatim}
will  multiply  each y and dy value by 5.0.  the specified error of 0.1
will be used to compute the actual dy value.  
\begin{verbatim}
     y=y * 5.0 noerror 
\end{verbatim}
will  multiply  each  y  value  by  5.0.   the  dy  values  will remain
unchanged.  
\begin{verbatim}
     y="log(yv)" noerror 
\end{verbatim}
will  take  the  log  to  base  10 of each y value.  the dy values will
remain unchanged.  
\begin{verbatim}
     y="log(yv)" 
\end{verbatim}
will  take  the  log to base 10 of each y value.  the dy values will be
modified as:  dy=abs(log(y+dy)-log(y)).  
\subsubsection{Expression}
you may use an expression inside quotes to actually calculate the value
for an existing data set.  the expression is evaluated for  each  value
required.  the lexicals v\_xvalue,v\_dxvalue,v\_zvalue...  are the current
x,dx,y...  needed to perform the calculation.  if you  do  not  specify
noerror  then the error is also computed when x,y,z are modified by the
expression.  if you use an expression you may not use any other options
except for points, lines, columns, sets, and log and error.  

\begin{verbatim}
                                example
     td:x from 0 to 360 by 5 
     td:y="sine<xv>" 
\end{verbatim}
generates a sine wave between 0 and 360 degrees in 5 degree steps.  
\begin{verbatim}
     td:y=y times 10 
\end{verbatim}
or...  
\begin{verbatim}
     td:y="yv*10" 
\end{verbatim}
both  multiply  y  by  the  constant  value  10.0.  the first method is
faster, and it also multiplies dy by the constant, while the expression
does not alter dy.  
\subsubsection{Lines$|$rows}
selects  the  mesh  rows  for  which values are generated.  if a single
number is specified it is the row number to set.  if both n1 and n2 are
specified they are the first and last row to set.  
\subsubsection{Log}
types on your terminal the results of the operation.  (default:log=off) 
\subsubsection{Name}
the  new  set  will  usually have a name consisting of a transformation
name followed by the old set name.  if you specify a  new  name  it  is
applied to the new data set.  if the name ends in ''\%'' then the old name
is appended to the new name.  see option:append 
\subsubsection{Points$|$columns}
selects the points or mesh data columns for which values are generated.
if a single number is specified it is point number to set.  if both  n1
and n2 are specified they are the first and last point to set.  
\subsubsection{Poisson$|$sqrt}
applies  only to dn and makes dn=sqrt(n).  this is counting statistics.
it is a special case of binomial  statistics  for  a  large  number  of
samples.  
\subsubsection{Samples}
\begin{verbatim}
     samples=s      applies      binomial     statistics     to     dn,
\end{verbatim}
dn=sqrt(max(n,1)*max(s-n,1)/s).  you must supply the number of  samples
(s).   if  you generate a set of data by a monte-carlo method where the
total number of hits s is selected, but the distribution of the hits is
generated by the program then you should use binomial statistics.  
\subsubsection{Sets}
selects the data set to modify.  
\subsubsection{Times}
multiplies  the  selected coordinate by the selected value.  if you are
multiplying x,y, or  z  by  a  constant  the  dx,dy,  or  dz  are  also
multiplied  by  the  constant.  in addition if error was specified then
the   final   dn   will   be   modified   by   the   specified   error.
dy=sqrt(dy**2+(e*n/c)**2)*c where c is the specified constant.  
\begin{verbatim}
                                example
\end{verbatim}
to multiply y by 2+-0.1 
\begin{verbatim}
     y=y times 2 error=.1 
          or...  
     y=y * 2 error=.1 
\end{verbatim}
\subsubsection{Plus}
adds  the  value  to  the coordinate.  if you specify error and you are
adding a constant to x,y, or z the dx,dy,  or  dz  are  also  modified.
dy=sqrt(dy**2+(e)**2) where e is the specified error.  
\begin{verbatim}
                                example
\end{verbatim}
to add 2+-0.1 to y 
\begin{verbatim}
     y=y plus 2 error=.1 
          or...  
     y=y + 2 error=.1 
\end{verbatim}
\subsubsection{Minus}
subtracts  the  value  from the coordinate if you specify error and you
are subtracting a constant to x,y, or z  the  dx,dy,  or  dz  are  also
modified.  dy=sqrt(dy**2+(e)**2) where e is the specified error.  
\begin{verbatim}
                                example
\end{verbatim}
to subtract 2+-0.1 from y 
\begin{verbatim}
     y=y minus 2 error=.1 
          or...  
     y=y - 2 error=.1 
\end{verbatim}
\subsubsection{Divide}
divides  the coordinate by the value if you are dividing x,y, or z by a
constant the dx,dy, or  dz  are  also  divided  by  the  constant.   in
addition  if  error was specified then the final dn will be modified by
the  specified  error.   dy=sqrt(dy**2+(e*n/c)**2)/c  where  c  is  the
specified constant.  
\begin{verbatim}
                                example
\end{verbatim}
to divide y by 2+-0.1 
\begin{verbatim}
     y=y divide 2 error=.1 
          or...  
     y=y / 2 error=.1 
\end{verbatim}
\subsubsection{Warning}
if  you have already plotted the current data, this command will delete
it if just a value, or range of values is specified for x,y, or z.   if
you  wish  to keep the data, you should use the option points, sets, or
append.  
\subsubsection{Examples}
\begin{verbatim}
     td:x values from 0 to 10 by 1 
          creates a data set with 11 values for x 0,1,2,...10 
     td:x bins from 0 to 10 n=10 
          or ...  
     td:x from 0 to 10 by 1 bins 
          create a data set with 10 values for x 0.5,1.5....9.5 
     td:x from 0 to 20 bins points=1 to 10 
\end{verbatim}
generates 1,3,5,...19 for data points 1 to 10.  
\begin{verbatim}
     td:y=2.0 point=5 
\end{verbatim}
data point number 5 is set to 2.0 
\begin{verbatim}
     td:dy =5 
\end{verbatim}
sets all y errors to +-5.  
\begin{verbatim}
     td:dx =.1 * x 
          or ....  
     td:dx =.1 times x 
\end{verbatim}
sets all x errors to 1.  times the x value.  
\begin{verbatim}
     td:x = 1 plus x 
          or ...  
     td:x = 1 plus 
\end{verbatim}
adds 1.0 to all x values.  
\begin{verbatim}
     td:x = 1 + error=0.1 
\end{verbatim}
adds  1.0  to  all  x  values  and  adds  0.1  in quadrature to all dx.
dx=sqrt(dx**2+(0.1)**2) 
\begin{verbatim}
     td:y = divide 5 
\end{verbatim}
both y and dy are divided by 5.  
\begin{verbatim}
     td:y = * 5 error=0.1 
\end{verbatim}
the  y  coordinate is multiplied by 5, and the dy is modified according
to the error.  dy=5*sqrt(dy**2+(0.1*y/5)**2) 
\begin{verbatim}
     td:set order y 
     td:1.5;  3.8;  9.7 
     td:x from 1 to 3 
          or....  
     td:x from 1 by 1 
          is the same as:  
     td:1 1.5;  2 3.8;  3 9.7 
     td:dy samples=100 
\end{verbatim}
each  dy is now equal to sqrt(y*(100-y)/100).  this assumes that each y
represents the number of sucesses out of a sample of 100.  
\begin{verbatim}
     td:y=y / v_sum 
\end{verbatim}
normalizes  the  data  to the total sum over all data.  note:  this may
not have the desired effect if you have  more  than  1  data  set.   to
normalize only one data set:  
\begin{verbatim}
     set statistics set=2 
     td:y=y / v_sum 
\end{verbatim}
normalizes only data set 2.  
\begin{verbatim}
     td:y="log(yv)" 
\end{verbatim}
transforms  y  by  taking the log to base 10.  dy is likewise modified.
y=log10(y) and dy=abs(log10(y+dy)-log10(y)) 
\subsubsection{Mesh\_examples}
assume you have mesh data x,y (row,column) of size 20, 30.  
\begin{verbatim}
     td:x=x minus 5 
\end{verbatim}
shifts the x scale 
\begin{verbatim}
     td:y=y times 2 
\end{verbatim}
expands the y scale 
\begin{verbatim}
     td:z=10 row=5 column=3 
\end{verbatim}
modifies 1 entry 
\begin{verbatim}
     td:z=0 
\end{verbatim}
sets all mesh values to 0.0 
\begin{verbatim}
     td:z=z plus 1 row=5 to 10 column=3 to 5 
\end{verbatim}
adds  1 to all mesh values for the specified range of rows and columns.
15 values are modified.  
\subsection{Y}
see:command x 
\subsection{Z}
see:command x 
\section{Lexical\_values}
you  have access to data via lexicals.  for example you may access the sum,
mean, standard deviation of your data  set  and  put  them  into  a  title.
see:command title.   quantities such as points, sum etc.  are calculated by
a set statistics or show data command.  if you do not use set statistics or
show  data  to  generate  them,  they are automatically calculated from the
entire data set.  whenever you modify  the  data  set,  you  must  use  set
statistics to set the statistics if you wish the lexicals to represent only
a subset of the data.  to use a lexical value you must use a symbol of  the
form:  t\_name, s\_name, or v\_name.  the t\_name inserts the value of the name
into a title string.  v\_name or $<$name$>$ is  used  to  get  the  value  as  a
number.  s\_name gets a value as a string.  you may define lexical values by
the command define value or define string.  

the  lexicals  may  be  abbreviated  to  the  shortest  unique  name.   for
convenience v\_ may be omitted inside an expression.  this will increase the
time  to  parse  an  expression.   for  example  $<$v\_xvalue$>$  is the same as
$<$xvalue$>$ or v\_xv.  
\subsection{List\_of\_names}
\begin{verbatim}
     names - value 
  *  input_level - the number of input files opened by set file input.  
  *  cputime - the current cpu time 
  *  cpulimit - the cpu time remaining 
  *  sets - the number of data sets.  
  *  current_set - the current data set (set by option:sets=) 
  *  xvalue[np,nr,ns] - the x value for point n in set ns.  
  *  dxvalue[np,nr,ns] - the error on x for point np row nr in set ns 
  *  yvalue[np,nr,ns] 
  *  zvalue[np,nr,ns] 
  *  dyvalue[np,nr,ns] 
  *  dzvalue[np,nr,ns] - (data for point n) 
  *  rvalue[np,nr,ns],drvalue[np,nr,ns] - radius in polar or spherical
     coordinates 
  *  thvalue[np,nr,ns],dthvalue[np,nr,ns] - theta in polar or spherical
     coordinates 
  *  phivalue[np,nr,ns],dphivalue[np,nr,ns] - phi in spherical
     coordinates 
  *  plots - the number of plots produced.  
  *  windows - the maximum number of windows 
  *  current_window - the current window number 
  *  xwindows - the maximum number of x windows 
  *  current_xwindow - the current x window number.  
  *  ywindows - the maximum number of y windows 
  *  current_ywindow - the current y window number.  
  *  columns - number of columns in a mesh.  
  *  current_column - current column of a mesh (set by option:point|
     column=) 
  *  rows - number of rows in a mesh.  
  *  current_row - current row in a mesh.  (set by option:line|row=) 
\end{verbatim}

\begin{verbatim}
                      calc by set statistics or show data
  *  points - the points in your data set.  
  *  nmin - the minimum value of n where n=x,y, or z.  
  *  nmax - the maximum value of n where n=x,y, or z.  
  *  sum - the sum over all points.  
  *  average - the average over all points.  
  *  mean - the mean of the data.  
  *  std - the standard deviation of the data from the average.  
  *  esum - the error on the sum.  
  *  eaverage - the error on the average.  
  *  estd - the error in the std.  
  *  emean - the error in the mean.  
\end{verbatim}

\begin{verbatim}
                              calc by fit command
  *  terms - the maximum term fit or constrained.  (0...20) 
  *  npfit - then number of points fit 
  *  ntfit - the number of terms fit.  
  *  chisqr - the chi squared/point.  
  *  ftest - the value of ftest.  
  *  fit[n] - 1 if coefficient n has been fit (n=0-19), 0 if not.  
  *  constrained[n] - 1 if this coefficient has been picked or
     constrained.  
  *  coefficient[n] - value of the coefficient of the fit.  
  *  ecoefficient[n] - error on coefficient n.  
\end{verbatim}

\begin{verbatim}
                            set by set hist command
  *  current_histogram - the current histogram id
     (see:command set histogram) 
  *  hist_name - current hist name (s_ only) 
  *  hist_entries - the number of entries in the hist 
  *  hist_mean - the mean of the hist 
  *  hist_std - the hist standard deviation 
  *  hist_sum[n1,n2] - the sum over this histogram.  
          n1 = 1,2,3 for x under,sum,over 
          n2 = 1,2,3 for y under,sum,over 
          if n1 or n2 are omitted or are 0 then you get the sum of all 3. 
  *  xmarker[n] - array of 8 x markers 
  *  ymarker[n] - array of 8 y markers 
  *  zmarker[n] - array of 8 z markers 
\end{verbatim}

\begin{verbatim}
                                3-d parameters
  *  view_theta - viewing angle 
  *  view_phi - viewing angle 
  *  view_separation - eye separation 
  *  view_distance - distance from observer to object 
  *  view_scrd - distance from observer to screen 
\end{verbatim}

\begin{verbatim}
                                 miscellaneous
  *  repeat[n] - repeat number for loop n.  
\end{verbatim}

\subsection{Values}
v\_xvalue[n,ns]  is the n'th value in set ns.  likewise v\_dxvalue[n,ns] is
the n'th  value  in  set  ns.   you  may  also  access  y,dy,z,  and  dz.
s\_symbol[n,ns]  is the symbol for the the n'th value in set ns.  if ns is
omited then n is the point number.  

for  mesh  plots  v\_xvalue[n1],v\_yvalue[n2],v\_zvalue[n1,n2]  are  used to
access the data point values where n1 is the column number and n2 is  the
row number.  

if  both  n and ns are omitted then it is the value for the current point
in an x=,dx=,y=...  command.  
\subsection{Repeat}
\begin{verbatim}
     v_repeat[n] 
\end{verbatim}
this  is  the  repeat  index  for repeat commands.  it starts at 1 and is
equal to the number of repeats when the loop terminates.  you may specify
the  loop number as n.  if n is omitted it is the current inner loop.  if
n is negative it is the next outer loop.  

\begin{verbatim}
                                 example
     td:repeat   2   "type  ""a"",s_repeat;repeat  3  ""type  s_repeat,'|
\end{verbatim}
's\_repeat[-1]'''''';  
types:  
\begin{verbatim}
     a1.0 
     1.0|1.0 
     2.0|1.0 
     3.0|1.0 
     a1.0 
     1.0|2.0 
     2.0|2.0 
     3.0|2.0 
\end{verbatim}
\subsection{Examples}
\begin{verbatim}
     td:title top 'mean=' t_mean 
\end{verbatim}
writes the mean of our data at the top of the current plot.  
\begin{verbatim}
     td:set order x y dx dy 
     td:data set 
     td:v_mean,v_sum,v_estd,v_esum 
     td:delete set=first 
\end{verbatim}
creates  a new data set containing the y=sum, x=mean, dx=std, dy=error on
sum, then deletes the old data set.  
\begin{verbatim}
     td:if v_fit[1] .gt.  0 title 'coef(1)=',t_coefficient[1] 
\end{verbatim}
writes coefficient 1 if it has been fit.  
\begin{verbatim}
     td:set statistics limited from 0,1 to 5,10 
     td:title top 'sum=',t_sum,' between x=0,5 y=1,10' 
\end{verbatim}
plot a title with the sum over the data between limits on x,y.  
\begin{verbatim}
     td:set statistics set=2 
     td:title top 'sum=',t_sum,' for data set 2' 
\end{verbatim}
plot a title with the sum over data set 2.  
\section{Examples}
there  are  several  files that give examples of topdrawer plots.  they are
contained in the topdrawer\_root:[examples] directory.  
\begin{verbatim}
                      fonts + 2 dimensional graphics
     tdintro.top 
                          3 dimensional graphics
     td3d.top 
                               character set
     font_table.top 
                     color graphics on vt-240 or gigi
     colorgraph.top 
                               test routines
     bench.top 
     resolution_test.top 
     fit_test.top 
          test to make sure coefficients and errors are correct 
     fit_gaus_coef_test.top 
     fit_exp_coef_test.top 
     fit_parabola_coef_test.top 
\end{verbatim}
\section{Subroutines}
the topdrawer package is available as a collection of subroutines.  you can
call them to produce plots from any program.  each subroutine is  the  same
as giving a topdrawer command.  they are available in library 
\begin{verbatim}
                          topdrawer_dir:topdrawer
\end{verbatim}
\subsection{Names}
most  internal  subroutines used by topdrawer begin with t1,t2,td, or tx.
the unified graphics routines all begin with ug.  the exceptions to these
rules  except  for token are documented here.  the cern library histogram
routines used by td do not follow any clear conventions.  
\subsection{Strings}


\begin{verbatim}
                                 warning
\end{verbatim}

\begin{verbatim}
     routines  tdset,tdtitl  and  tdcase  are  not  fully fortran-77
     compatible.  you must use either hollerith data,  byte  arrays,
     literals,  or  character variables sent by reference when using
     those routines.  
\end{verbatim}


if  you use a byte array or a character array sent by reference, you must
terminate it with a semicolon (;) the following are all legal:  
\begin{verbatim}
     character*8 str 
     data str /'title;'/ 
     call tdtitl(%ref(str)) 
          or...  
     call tdtitl('title') 
          or...  
     call tdtitl(6htitle;) 
          or...  
     real*8 str 
     data str /'title;'/ 
     call tdtitl(str) 
          or...  
     byte str(5) 
     data str /'t','i','t','l','e',';'/ 
     call tdtitl(str) 
\end{verbatim}
\subsection{Unit\_numbers}
the following logical unit numbers are used:  
\begin{verbatim}
     unit routine   usage
     1    t2seth    histogram input
     6    all       error message/information output
     7    tdmain    journal file output
     8    all       listing output
     9-19 tdmain    input
     86   all       error message for deferred errors
\end{verbatim}
the  journal  file,  and  listing are not output unless the following set
commands are used:  
\begin{verbatim}
     td:set file log filename 
     td:set file list filename 
\end{verbatim}
\subsection{Linking}
for  more  information on linking see ugs.  the following routines should
be linked to your program.  
\begin{verbatim}
     $ link routine, - 
          topdrawer_dir:topdrawer[/include=td]/library, - 
          packlib_olb/library,mathlib_olb/library, 
          ugs_dir:objlib/library 
\end{verbatim}

if you wish to link a standalone version of topdrawer use /include=td.  

for  example  you  wish to make up a version of topdrawer to do fits with
your own supplied function tdufun.  you  would  link  a  new  version  as
follows:  
\begin{verbatim}
     $ link tdufun, - 
          topdrawer_dir:topdrawer/include=td/library, - 
          packlib_olb/library,mathlib_olb/library, 
          ugs_dir:objlib/library 
\end{verbatim}

\subsection{Demo\_program}
there  are  demo programs in topdrawer\_root:[examples] called td\_demo.for
and td\_demo3.for , with a command file to build them called  td\_demo.com.
this illustrates the usage of the topdrawer subroutines.  
\subsection{List\_of\_subroutines}
\begin{verbatim}
                              miscellaneous routines
     1.  cpulim - gives you the cpu time remaining.  
     2.  cputim - gives you the consumed cpu time.  
     3.  defkey - defines keypad keys.  
     4.  help - types help info in your terminal.  
     5.  intrac - tells you if you are running interactively.  
     6.  spawn - spawns a subprocess 
     7.  trace - generates traceback.  
     8.  readpr - reads input with prompt.  
                                topdrawer routines
     9.  tdarro - draws arrown 
    10.  tdcrcl - draws box, circle, diamond 
    11.  tdend - empties buffers and closes graphics.  
    12.  tdflsh - flushes the buffers 
    13.  tdgetv - gets a lexical value.  
    14.  tdhist - does histograms 
    15.  tdjoin - joins points 
    16.  tdlims - sets plot limits 
    17.  tdnew - same as new frame command (with alias) 
    18.  tdnewp - same as new frame command 
    19.  tdplot - same as plot command 
    20.  tdsets - sets parameters - same as set command 
    21.  tdshow - type parameters on your terminal - same as show command 
    22.  tdtext - draws text - same as title command 
    23.  tdvax_plot - set up default device 
    24.  td3jin - join 3-d data 
    25.  td3hst - histogram 3-d data 
    26.  t2main - the main program for topdrawer 
    27.  t2curs - the cursor routine 
    28.  t2upcs - converts string to upper case 
    29.  t2sqez - squeezes multiple blanks from character strings.  
    30.  t2_trap - sets up to trap control_c 
\end{verbatim}
\subsection{Cputim}
time = cputim(otime) 

\begin{verbatim}
     this  call  will  give  you  the  difference  in cputime between the
\end{verbatim}
current time and the previous time (otime) in seconds.  if otime=0.0 then
the total cpu time since the start of your process is returned.  

this subroutine is independent from the rest of topdrawer 
\subsection{Cpulim}
time = cpulim(otime) 

\begin{verbatim}
     this  call  will  give  you  the  difference  in cputime between the
\end{verbatim}
remaining cpu  time  and  the  previous  time  (otime)  in  seconds.   if
otime=0.0  then  the  total time limit since the start of your process is
returned.  if cpulim(0.0) is zero then there is no  time  limit  on  your
process.  

\begin{verbatim}
                                 example
     time=cpulim(cputim(0.0)) 
\end{verbatim}
this  gives  you the time remaining.  if this is negative, then there was
no limit.  

this subroutine is independent from the rest of topdrawer 
\subsection{Help}
\begin{verbatim}
     call help(string) 
\end{verbatim}
input:  
\begin{verbatim}
     string = string of arguments to pass to help 
\end{verbatim}
this  program  was  copied  from  fortran users guide d-16 it assumes the
primary help library is hlp\$library.  

this subroutine is independent from the rest of topdrawer 
\subsection{Intrac}
\begin{verbatim}
     itest=intrac(i) 
\end{verbatim}
i - a dummy variable 
itest = .true.  if you are running interactively.  
\subsection{Spawn}
\begin{verbatim}
     call spawn('command line') 
\end{verbatim}
a  subprocess  is created, the command line is executed, and then control
returns to your program.  if the command line is blank, the user gets the
dcl prompt, and control returns after the user types log.  
\subsection{Trace}
\begin{verbatim}
     call trace 
\end{verbatim}

generates  an  error message including traceback.  this may be useful for
debugging.  
\subsection{Readpr}
this reads from the sys\$input after prompting 
\begin{verbatim}
     call readpr('prompt',string,length,[itimeout]) 
\end{verbatim}

input:  
\begin{verbatim}
     prompt= string to prompt with 
     itimeout=  timeout  time  in  seconds  (integer) if the time between
\end{verbatim}
typed characters exceeds the timeout then the read is terminated.  
output:  
\begin{verbatim}
     string = input 
     length = length of the input 
\end{verbatim}

you  may  define keypad keys to return a string using defkey.  defkey and
readpr are routines which work together.  

this subroutine is independent from the rest of topdrawer 

\begin{verbatim}
                                 example
     character*20 str 
     call readpr('input:',str,i,60) 
     if (i.gt.0) type *,'str=",str(1:i),'"' 
\end{verbatim}
reads  a  string of 20 characters or less and writes it on your terminal.
you have 60 seconds to begin typing after the prompt.  
\subsection{Defkey}
this defines a key on the keypad.  
\begin{verbatim}
     call defkey(string) 
\end{verbatim}

the  keypad  key  is  defined  in  a manner similar to the define/key vms
command.  all of the ask routines and readpr will  interpret  the  keypad
keys,  according  to  the defkey routine.  defkey and readpr are routines
which work together.  

\begin{verbatim}
                                 example
     call defkey('/key pf2 "help"/terminate') 
\end{verbatim}
this is the default definition of pf2 
\subsection{Tdarro}
this draws an arrow from xyz(1:3,1) to xyz(1:3,2) 
\begin{verbatim}
     call tdarro(xyz,[dxyz,[imode,[itype]]]) 
     1.  xyz(3,2)  is the array of xyz to specify the arrow.  xyz(*,1) is
         the tail while xyz(*,2) is the head.  for 2-d plots the z  value
         is ignored.  
     2.  dxyz(4) specifies the offsets and size.  
         a.  dxyz(1) - the offset for the tail of arrow 
         b.  dxyz(2) - the offset for the head of the arrow 
         c.  dxyz(3) - size of arrow if zero the default is used.  
         d.  dxyz(4) - flare of arrow if zero the default is used.  
     3.  imode(2)  - specifies the frame.  data=0, text=1 imode(1) is for
         tail of arrow.  
     4.  itype - specifies the texture,color,intensity 
\end{verbatim}
for  more  information  see  topdrawer command arrow.  this is similar to
giving the command.  
\begin{verbatim}
     td:arrow    from    xyz(1,1),xyz(2,1),xyz(3,1)   less   dxyz(1)   to
\end{verbatim}
xyz(1,2),xyz(2,2),xyz(3,2)  less  dxyz(2)  size=dxyz(3),dxyz(4)   outputs
extended titles with case formatting.  
this  routine  is not recommended.  it is provided for compatability with
older versions of topdrawer.  use tdtext instead.  
\begin{verbatim}
     call tdcase('title','case'[,x,y]) 
     call tdcase('title','case','position') 
     call tdcase('title') 
\end{verbatim}

see tdtitl for more information 
\subsection{Tdcrcl}
this draws either a box, diamond, or ellipse.  
\begin{verbatim}
     call tdcrcl(xy,dxy,itype,[idata,[mode[,angles]]]);  
     1.  xy(3) is the array of xyz to specify the center of the figure.  
     2.  dxy(3) is the array of xyz to specify the width of the figure.  
     3.  itype 
                1        2       3
              box  diamond ellipse
     4.  idata 
              0 - xyz specified in text coordinates 
              1 - center in data, width in text coordinates.  
              3 - both center and width are in data coordinates.  
     5.  mode      -     specifies     texture,     intensity,     color.
         see:subroutine tdhist.  
     6.  angles(2) - range of angles for circle 
\end{verbatim}

\begin{verbatim}
                                 example
     real xy(3),dxy(3) 
     data xy/1.,1.,1./,dxy/.5,.5,.5/ 
     call tdcrcl(xy,dxy,3,3) 
\end{verbatim}
draws an ellipse centered at 1,1,1 of size 0.5 in the data frame.  
\subsection{Tdend}
this  subroutine  ends  the  graphics  and closes any output files.  this
should not be used to end an individual plot of a series.  it only  needs
to be called once before exiting from your program.  
\subsection{Tdflsh}
normally  the graphics buffer is flushed after the subroutine call unless
flush is set off.  tdflsh will allow you to flush the buffers  only  when
desired.  this is much faster than flushing automatically.  
\subsection{Tdgetv}
\begin{verbatim}
     logical l 
     l=tdgetv('string',value) 
\end{verbatim}
input:  
\begin{verbatim}
     string         = the name of the lexical to get 
\end{verbatim}
output:  
\begin{verbatim}
     value          = the actual value returned 
     l    = .true.  if the lexical was found, .false.  if not found.  
\end{verbatim}

\begin{verbatim}
                                 example
       logical tdgetv 
       if (tdgetv('v_sum',sum)) write(6,*)'sum=',sum 
\end{verbatim}
\subsection{Tdhist}
does histograms 
\begin{verbatim}
     call tdhist(npts,x,y[,dx[,dy[,level[,mode[,z[,dz]]]]]]) 
     1.  npts - then number of points to plot 
     2.  x(npts) - the x for each datum to histogram 
     3.  y(npts) - the y for each datum 
     4.  dx(npts) - the error in x 
     5.  dy(npts) - the error in y 
     6.  level     = 0 histogram 
                   = 1 barchart 
     7.  mode  -  selects the line structure,color,intensiy if zero these
         are set by the set command.  
           intensity*8 (intensity=1 to 5) 
           1*64,   2*64,    3*64,  4*64,  5*64,      6*64,    7*64 
             64,    128,     192,   256,   320,       384,     448 
           dots, dashes, dotdash, solid, funny, patterned, daashes 
           white   red green  blue yellow magenta cyan 
           1*512 2*512 3*512 4*512 5*512  6*512  7*512 
     8.  z(npts) - the z for each datum 
     9.  dz(npts) - the error in z 
\end{verbatim}
\subsubsection{Example}
\begin{verbatim}
     real x(5),y(5),dx(5),dy(5) 
     read (1,'(4f10.0)') (x(i),y(i),dx(i),dy(i),i=1,5) 
     call tdhist(5,x,y,dx,dy,0,64+4*512) 
\end{verbatim}
reads  5  lines  containing x,y,dx,dy on each and histograms the result
with blue dotted lines.  
\subsection{Tdjoin}
joins points 
\begin{verbatim}
     call tdjoin(npts,x,y[,dx[,dy[,level[,mode[,z[,dz]]]]]]) 
     1.  npts - then number of points to plot 
     2.  x(npts) - the x for each datum to histogram 
     3.  y(npts) - the y for each datum 
     4.  dx(npts) - the error in x 
     5.  dy(npts) - the error in y 
     6.  level - the number of segments per/point 
     7.  mode  -  texture,color,intensity,  and  fit parameters are added
         together.  if zero the default is taken  or  the  set  value  is
         used.  
                                     type of fit
                 0      1        2 
           default spline "general" 
                the default is initially "general" 
                                   texture params
           intensity*8 (intensity=1 to 5) 
           1*64,   2*64,    3*64,  4*64,  5*64,      6*64,    7*64 
             64,    128,     192,   256,   320,       384,     448 
           dots, dashes, dotdash, solid, funny, patterned, daashes 
           white   red green  blue yellow magenta cyan 
           1*512 2*512 3*512 4*512 5*512  6*512  7*512 
     8.  z(npts) - the z for each datum 
     9.  dz(npts) - the error in z 
\end{verbatim}
\subsubsection{Example}
\begin{verbatim}
     real x(5),y(5),dx(5),dy(5) 
     read (1,'(4f10.0)') (x(i),y(i),dx(i),dy(i),i=1,5) 
     call tdjoin(5,x,y,dx,dy,1,64+4*512) 
\end{verbatim}
reads  5  lines  containing x,y,dx,dy on each and joins the result with
blue dotted lines.  a single line segment joins each point.  
\begin{verbatim}
     call tdjoin(5,x,y) 
\end{verbatim}
joins  the 5 points with the default line type (solid,white).  the data
will be joined by multiple line segments to form a smooth curve.  
\subsection{Tdlims}
sets limit for data set.  
\begin{verbatim}
     call tdlims(options,npts,values[,errors]) 
\end{verbatim}

\begin{verbatim}
     1.  options:   'xmin',  'xmax',  'x',  'ymin',  'ymax', 'y', 'zmin',
         'zmax', 'z' 
         this specifies what limits are to be set.  for example if 'x' is
         specified, both xmin and xmax are set.  
     2.  npts - the number of data points.  
     3.  values(npts) - the array of data to use in setting the limits.  
     4.  errors(npts) - the corresponding error array 
\end{verbatim}
\subsubsection{Example}
\begin{verbatim}
     real x(2),dx(2) 
     data dx/0.1,1.0/ 
     data x/0.,10./ 
     call tdlims('x',2,x) 
\end{verbatim}
this will set the plot to range from 0 to 10 along the x axis.  
\begin{verbatim}
     call tdlims('xmin',1,0.0) 
     call tdlims('xmax',1,10.0) 
\end{verbatim}
this will set the plot to range from 0 to 10 along the x axis 
\begin{verbatim}
     call tdlims('x',2,x,dx) 
\end{verbatim}
sets the plot to range from -0.1 to 11.0 
\subsection{Tdnew}
this starts a new plot and optionally gives the frame an alias.  
\begin{verbatim}
     call tdnewp('alias') 
\end{verbatim}
if  alias  is  blank (' ') then there is no alias.  alias may either be a
string or a character variable.  
\subsection{Tdnewp}
this starts a new plot.  this differs from the original topdrawer in that
you may not specify an alias.  
\begin{verbatim}
     call tdnewp 
\end{verbatim}
\subsection{Tdplot}
plots points.  
\begin{verbatim}
     call tdplot(npts,x,y[,dx[,dy[,'symbol'[,mode]]]]) 
\end{verbatim}
this plots the x,y data using the specified symbol.  

\begin{verbatim}
     1.  npts - then number of points to plot 
     2.  x(npts) - the x for each datum to histogram 
     3.  y(npts) - the y for each datum 
     4.  dx(npts) - the error in x 
     5.  dy(npts) - the error in y 
     6.  symbol - the symbol to use in plotting 
     7.  mode - determines color,intensity - see tdhist 
\end{verbatim}
\subsubsection{Example}
\begin{verbatim}
     call tdplot(1,100.,200.,0.,0.,2h0o) 
          or...  
     call tdplot(1,100.,200.,0.,0.,'0o') 
\end{verbatim}
puts a cross at 100.,200.  in the data frame.  
\subsection{Tdplt}
plots points.  
\begin{verbatim}
     call tdplt(symbol,npts,x,y[,dx[,dy[,mode[,z[,dz]]]]]]) 
\end{verbatim}
this plots the x,y data using the specified symbol.  

\begin{verbatim}
     1.  symbol  -  the  symbol  to  use in plotting.  it may be either a
         literal or character string.  
     2.  npts - then number of points to plot 
     3.  x(npts) - the x for each datum to histogram 
     4.  y(npts) - the y for each datum 
     5.  dx(npts) - the error in x 
     6.  dy(npts) - the error in y 
     7.  mode - determines color,intensity - see tdhist 
     8.  z(npts) - the z for each datum 
     9.  dz(npts) - the error in z 
\end{verbatim}
\subsubsection{Example}
\begin{verbatim}
     call tdplt('0o',1,100.,200.,0.,0.) 
          or...  
     character*2 symbol 
     data symbol/'0o'/ 
     call tdplt(symbol,1,100.,200.,0.,0.) 
\end{verbatim}
puts a cross at 100.,200.  in the data frame.  
\subsection{Tdset}
sets parameters.  
\begin{verbatim}
     call tdset(options) 
\end{verbatim}

options is a literal, hollerith, or array containing the options.  
\begin{verbatim}
     see:subroutine string 
\end{verbatim}
this  routine  is not recommended.  it is provided for compatability with
older versions of topdrawer.  use tdtext instead.  
the  options  are  the  same as for the set command.  for convenience you
probably should use tdsets which is fortran 77 compatible.  

\begin{verbatim}
                                 example
     call tdset('size 10 by 8') 
\end{verbatim}
this sets the size of the plot.  
\begin{verbatim}
     call tdset('color blue') 
\end{verbatim}
now everything is plotted in blue.  
\begin{verbatim}
          or....  
     byte string(11) 
     data string/'c','o','l','o','r',' ', 
    1 'b','l','u','e',';'/ 
     call tdset(string) 
\end{verbatim}
\subsection{Tdsets}
sets parameters.  
\begin{verbatim}
     call tdsets(options) 
\end{verbatim}

options  is  a  literal  or character string containing the options.  the
options are the same as for the set command.  this subroutine is  fortran
77 compatible so \%ref should not be used.  
\subsubsection{Example}
\begin{verbatim}
                                example
     call tdsets('size 10 by 8') 
\end{verbatim}
this sets the size of the plot.  this is the same as the command:  
\begin{verbatim}
     td:set size 10 by 8 
     call tdsets('color blue') 
\end{verbatim}
now everything is plotted in blue.  
\begin{verbatim}
          or....  
     character*10 string 
     data string/'color blue'/ 
     call tdsets(string) 
\end{verbatim}

\begin{verbatim}
     call tdsets('device versatec') 
\end{verbatim}
sets the current device to be the versatek printer/plotter.  
\subsection{Tdshow}
shows parameters.  
\begin{verbatim}
     call tdshow(options) 
\end{verbatim}

options  is  a literal or character string containing the options to type
on your terminal.  the options are the  same  as  for  the  set  or  show
command.   this subroutine is fortran 77 compatible so \%ref should not be
used.  
\subsubsection{Example}
\begin{verbatim}
     call tdshow('size') 
\end{verbatim}
this types on your terminal the size of the plot.  
\begin{verbatim}
          or...  
     character*4 string 
     data string/'size'/ 
     call tdshow(string) 
\end{verbatim}
\subsection{Tdtext}
outputs titles 
\begin{verbatim}
     call tdtext('options','title','case'[,x[,y[,z]]]) 
\end{verbatim}
this puts the title at the specified x,y,z 

\begin{verbatim}
     1.  options - this is a string specifying the options.  these may be
         any options used by the title command.  some  available  options
         are:  
              top, bottom, right, left, data, angle=n, x, y, z, size=n 
     2.  title - text to put on the screen.  
     3.  case  -  the case string.  if ' ' is used then the text consists
         of normal ascii characters 
     4.  x,y,z  -  specify  the  title  position in text coordinates.  if
         omitted and no options  are  specified,  the  current  title  is
         placed under the last.  
\end{verbatim}

this  is  a  fortran-77  compatible subroutine.  do not use \%ref with the
option, title, or case strings.  options, title, and case must be  either
literals or character variables.  
\subsubsection{Example}
\begin{verbatim}
     call tdtext('top','my graph',' ') 
\end{verbatim}
puts a title at the top of the graph.  
\begin{verbatim}
     call tdtext('bottom','x axes',' ') 
\end{verbatim}
puts a title at the bottom of the graph.  
\begin{verbatim}
     call tdtext(' ','more on the bottom',' ') 
\end{verbatim}
then puts another line ender the previous one.  
\begin{verbatim}
     call tdtext('data','this is at location 5,1.8',' ',5.,1.8) 
\end{verbatim}
puts a title left justified at the data locations 5.0,1.8 
\begin{verbatim}
     call tdtext('data,center','this is at location 5,1.8',' ',5.,1.8) 
\end{verbatim}
puts a title centered at the data locations 5.0,1.8 
\subsection{Tdtitl}
outputs titles 
this  routine  is not recommended.  it is provided for compatability with
older versions of topdrawer.  use tdtext instead.  
\begin{verbatim}
     call tdtitl('title',,x,y) 
\end{verbatim}
this puts the title at the specified x,y.  
\begin{verbatim}
     call tdtitl('title','position') 
\end{verbatim}
this puts the title at the specified position.  
\begin{verbatim}
     call tdtitl('title') 
\end{verbatim}
this puts a title under the last title.  

\begin{verbatim}
     1.  title - title 
     2.  x,y - specify the title position in text coordinates.  
     3.  position - specifies the position relative to the data area 
              parameters:  
              'top', 'bottom', 'right', 'left' 
     4.  if  'title' alone is specified, then the current title is placed
         under the last.  
\end{verbatim}
\subsubsection{Example}
\begin{verbatim}
     call tdtitl('my graph','top') 
     call tdtitl('x axes','bottom') 
\end{verbatim}
\subsection{Tdtset}
this  controls  the  format  of titles.  this modifies only the next call
title or call case.  
this  routine  is not recommended.  it is provided for compatability with
older versions of topdrawer.  use tdtext instead.  
\begin{verbatim}
     call tdtset(size,[angle,[spaces,[idata]]]) 
     1.  size - the character size in tenths of inches.  
     2.  angle - the angle to draw the title at.  
     3.  spaces - the number of spaces the title occupies.  
     4.  idata - non zero if title is specified in data coordinates.  
\end{verbatim}
\subsection{Td3jin}
this joins 3-d data.  
call td3jin(array,nx,ny,[[[[[[ixyz],itxtur],nxl],nyl],nxh],nyh]) 
\begin{verbatim}
     1.  array(nx,ny+1) - contains the 3-d data.  
      *  array(1,1) - n1+n2*4 - n=1,2,3 for x,y,z 
         n1=1 if array(1,1) is x, 2 if y, 3 if z.  
         n2=1 if array(1,1) is x, 2 if y, 3 if z.  
      *  array(2:nx,1) - x values at center of bin 
      *  array(1,2:ny) - y values at center of bin 
      *  array(2:nx,2:ny) - z values 
     2.  ixyz controls type of plot.  
      *  0 = draw x,y,z 
      *  1 = draw x lines only 
      *  2 = draw y lines only 
      *  3 = draw z lines only 
     3.  itxtur - (n*64=texture, n=1:7) 
     4.  nxl - low x chan to plot (0=plot all) 
     5.  nyl - low y chan to plot (0=plot all) 
     6.  nxh - high x chan to plot (0=plot all) 
     7.  nyh - high y chan to plot (0=plot all) 
\end{verbatim}
for more information see the sample program:  topdrawer\_dir:td\_demo3.for 
\subsection{Td3hst}
this histograms 3-d data.  
call td3hst(array,nx,ny[[[[[[[ixyz],itxtur],dxyz],nxl],nyl],nxh],nyh]) 
\begin{verbatim}
     1.  array(nx,ny+1) - contains the 3-d data.  
      *  array(1,1) - n1+n2*4 - n=1,2,3 for x,y,z 
         n1=1 if array(1,1) is x, 2 if y, 3 if z.  
         n2=1 if array(1,1) is x, 2 if y, 3 if z.  
      *  array(2:nx,1) - x values at center of bin 
      *  array(1,2:ny) - y values at center of bin 
      *  array(2:nx,2:ny) - z values 
      *  array(1,ny+1) - lower edge of first x bin 
      *  array(2,ny+1) - upper edge of last x bin 
      *  array(3,ny+1) - lower edge of first y bin 
      *  array(4,ny+1) - upper edge of last y bin 
     2.  ixyz controls type of plot.  
      *  1 = draw x lines 
      *  2 = draw y lines 
      *  4 = draw z lines 
      *  8 = nodepth 
      *  16 = nohide 
      *  32 = noframe 
      *  64 = shade xy face 
      *  128 = shade yz face 
      *  256 = shade zx face 
      *  512 = shade with x's 
      *  1024 = shade with dots.  
     3.  dxyz(3) - distance between lines used to shade plot (dx,dy,dz) 
     4.  itxtur - (n*64=texture, n=1:7) 
     5.  nxl - low x chan to plot (0=plot all) 
     6.  nyl - low y chan to plot (0=plot all) 
     7.  nxh - high x chan to plot (0=plot all) 
     8.  nyh - high y chan to plot (0=plot all) 
\end{verbatim}
for more information see the sample program:  topdrawer\_dir:td\_demo3.for 
\subsection{Tdvax\_plot}
\begin{verbatim}
     call tdvax_plot 
\end{verbatim}
this  will  set  up  your  terminal  as  the default device, provided the
logical name plot\_term is correctly set.  normally it is set when you log
in so you usually do not need to worry about it.  

if this routine is not called, you must call tdset and specify the device
to use in plotting.  
\subsection{T2curs}
\begin{verbatim}
     call t2curs(i,xyz,xyz1,xyz2,lf1,lf2);  
\end{verbatim}
this subroutine returns both text and data coordinates from the cursor:  
\begin{verbatim}
     1.  i = character typed ichar(character) (if 0 no input) 
     2.  xyz(2) = physical device coordinates.  
     3.  xyz1(2) = text coordinates.  
     4.  xyz2(3) = data coordinates.  if the plot is 2-d xyz2(3)=0 
     5.  lf1 = .true.  if text coor.  returned 
     6.  lf2 = .true.  if data coor.  returned 
\end{verbatim}
\subsection{T2main}
\begin{verbatim}
     call t2main(iarg,string) 
\end{verbatim}

this calls the topdrawer main program.  
\begin{verbatim}
     input:  string 
\end{verbatim}
this  is a command to perform.  if it is blank then topdrawer will prompt
you until either a return,stop, or new plot command is typed.  
\begin{verbatim}
     output:  iarg 
          = 1 if a clear command was typed.  
          = 2 if a stop command was typed.  
\end{verbatim}


\begin{verbatim}
                                  note
\end{verbatim}

\begin{verbatim}
     this  subroutine calls all other topdrawer subroutines.  if you
     call it then all other subroutines will  be  included  in  your
     program.  
\end{verbatim}


\subsection{T2upcs}
\begin{verbatim}
     call t2upcs(str) 
\end{verbatim}

converts a character string to upper case.  
\subsection{T2sqez}
\begin{verbatim}
     call t2sqez(str,ilen) 
\end{verbatim}

squeezes  out  multiple  blanks from str, and returns the location of the
last non blank character in ilen.  
\subsection{T2\_trap}

traps  any  control  c interrupts.  while executing topdrawer subroutines
this returns control back to t2main or the user routine.  
\section{Linking}
if  you  wish to link a program so that it may share hbook4 histograms with
topdrawer, you need to do the following:  

\begin{verbatim}
     1.  insert the following lines of code into your main program.  
              integer hcreateg 
              parameter ihsize=100000 
              common /pawc/memory(ihsize) 
                   ......  
              istat=hcreateg('region_name',memory,ihsize) 
              if (istat .le.  0) stop 'can not create region' 
              call hlimit(ihsize) 
         you may adjust the ihsize to the proper value.  
\end{verbatim}

\begin{verbatim}
     2.  when you link you must also include:  
              topdrawer_dir:td_group/opt 
         to give users in your group only read access to your histograms.  
              istat=hcreateg('region_name',memory,-ihsize) 
                   or link using....  
              topdrawer_dir:td/opt 
         td_group  uses  a  modified  version  of  hcreateg rather than the
         standard cern version.  the modified one in addition to setting up
         group  read/write protection on the section also prevents creation
         of the same global section by more than 1 person.   it  creates  a
         new  version  instead.   it  will type out the name of the created
         global section.  the modified version  will  create  a  read  only
         section if -ihsize is passed to it.  
\end{verbatim}

\begin{verbatim}
         to create a global section hcreateg must open a scratch file.  the
         modified  version  attempts  to  do  this  on  sys$scratch:.    by
         reassigning  the  scratch  device  you may put this file on a disk
         with enough quota.  
              $ define sys$scratch usr$scratch:[name] 
\end{verbatim}

for information about using the td subroutines:  
see:subroutines 
\section{Revisions}
the original sources were written by:  
\begin{verbatim}
                             roger b.  chaffee
                        computation research group
                    stanford linear accelerator center
                           stanford, california
\end{verbatim}

modified:  
\begin{verbatim}
                                j.  clement
\end{verbatim}
all syntax has been ``regularized'' and extended for convenience.  
\subsection{Obsolete\_commands}
most  commands  in  the ``original'' version still work.  there are the few
known exceptions.  
\begin{verbatim}
     1.  dump - removed 
     2.  set file debug - removed 
     3.  set file error - removed 
     4.  set  file input n - no longer works the same.  instead of a unit
         number you must specify a file name.  
     5.  set file output n - no longer works the same.  instead of a unit
         number you must specify a file name.  
     6.  time - replaced by show time.  
     7.  set device 4013 - removed.  
     8.  certain undocumented commands were removed.  specifically in the
         original version the option set may be omitted entirely, but  it
         must be used in the bonner lab version for set commands.  
\end{verbatim}
\subsection{86}
\begin{verbatim}
     1.  set color - now accepts full range of ugsys color specs.  
     2.  set  file  -  command rewritten.  it now accepts a file name and
         multiple nesting of set file input is possible.  
     3.  unit numbers modified to allow batch input.  
     4.  set  device  ddname=filename  -  the  limitation  of  8 chars on
         filename is removed.  
     5.  set device - interactive is the default.  
     6.  help - new command added.  
     7.  added prompting for interactive input.  
     8.  plot_term,  plot_device  logical  names  are  used  to determine
         default device.  
     9.  token - tabs are treated as spaces.  
    10.  sixels is now a legal device.  
    11.  set input is a synonym of set card.  
    12.  set errors - new command added.  
    13.  fixed set device command so sideways works properly.  
    14.  intensity/width  3  did not work properly, and would cause other
         options to work improperly.  
    15.  set device accepts lowercase chars.  in string.  
    16.  set structure can now set funny.  
    17.  histogram command now produces a block plot for 3-d data.  
    18.  flush command flushes the buffers.  
    19.  fixed ug errors when no plot make.  
    20.  fixed  failure  to  set  device  in the middle of an interactive
         session.  
    21.  added journal file and set journal command.  
    22.  added show commands as mirror to set commands.  
    23.  rewrote  t23hst  to  add  fast  "lego"  plots as well as minimal
         histograms.  
    24.  title  x|y|z  now  produces titles next to 3-d axes.  the lables
         and titles are always  drawn  on  the  outer  side  of  axes  by
         default.  
    25.  t23hst,t23jin  accept  a  texture  parameter.   t23hst  may plot
         hidden lines with the option hide.  
    26.  t2freq  modified  for  incomplete  storage.   now  only  x,y are
         required.  
    27.  modified  txdot  and  txtsym  to  produce  points  directly from
         unified graphics.  this makes better points  for  hi  resolution
         displays, and faster points on tek4010 displays.  
    28.  t23plt - added to do scatter plots.  
    29.  slice - command added to plot slices of mesh data.  
    30.  fixed t2def2 so limits are properly defined at all times.  
    31.  added  set  hist  command.   this  is  conditionalized for the 3
         different types of hist routines.  you can get rice format,  and
         hbook format hist files to be displayed by topdrawer.  
    32.  show  data  can show you a subset of the data, with calculations
         of statistics on the data.  
    33.  3d is now properly selected for each type of hist.  
    34.  define key - defines keypad keys for ansii terminals.  
    35.  the  previous 20 input lines are stored for recall by ansii mode
         terms.  this is done using smg routines so foreign terminals can
         be defined also.  
    36.  traceback for input files has been added.  
    37.  modified  txline  for  optimized  point  plotting.   essentially
         drawing points with ugmark is faster for  some  protocalls  than
         drawing  zero length lines.  this is automatically done wherever
         possible.  this is especially true for 4010 graphics.  a cluster
         of nearby dots can save a factor of 3 in transfer time.  
    38.  regularized   the   error   messages,  and  added  messages  for
         pen,width,card length out of range.  
    39.  added permanent option to set symbol,grid,width this changed the
         logic of set symbol slightly to conform to other commands.  
    40.  the  card length was changed to conform to standard screen sizes
         (80 cols).  the format string  is  (256a1)  to  make  card  size
         changes easy.  
    41.  fixed automatic flush so it works properly for interactive use. 
    42.  fixed  novector mode so that it works.  make the default vector.
         removed  '$'  as  a  separator.   this  is  necessary  for  some
         filenames!  (txxug77) 
    43.  29-oct-1986     -     added     dsize,dots,...     options    to
         circl,ellipse,diamond,box commands.  (t2main,t2plot) 
    44.  31-oct-1986         -         added        cursor        support
         (t2xfrm,txxug77,t2set1,t2show).  
    45.  5-nov-1986 - fixed the option plot symbol so it works, and fixed
         problem with multiple options after plot.  
    46.  added  color,texture,intensity  control  to  set  axes, outline,
         grid, title.  also color, texture may now be  specifed  for  any
         command that draws something.  plot, hist, box, title....  
    47.  added  support  for  multiple data sets.  option sets picks data
         sets.  
    48.  added mesh data support to smooth command.  
    49.  14-nov-1986  -  fixed  bug  in  t2rdpt.  data after a symbol was
         skipped.  
    50.  modified string handling to fortran-77.  
    51.  26-nov-1986 - added if,else,endif.  
    52.  1-dec-1986  -  added  options scale, ticks, and permanent to set
         scale command.  added option permanent to set  color  and  reset
         color  with  each  new  plot.   fixed a problem with join spline
         dash.  if the points were closely spaced a solid line is  drawn.
         added options blink, proportional, hardtexture.  
    53.  8-dec-1986  -  added sideways option to set size.  added options
         white,...   width=,  dotted  to  set  ticks,labels,title.   show
         command  now  shows  the  default  values as well as the current
         ones.  
    54.  16-dec-1986  -  added options dot/dash/daash/random/space to set
         pattern.  fixed infinite loop when no pattern specified.  
    55.  23-dec-1986  - added option for automatic pattern generation for
         drawing commands.  ie.  join dot dot dash generates the  pattern
         to  do  this.   added  expand  option  to join,hist,plot.  added
         variable, size, norandom options to plot.  
\end{verbatim}
\subsection{87}
\begin{verbatim}
     1.  1-feb-1987 - typing ctrl_c will abort the current plot or print.
         the set histogram command may select histograms by substring  in
         the  title.   next|previous may be combined with name= to select
         the next hist which matches the name.  next|previous  will  wrap
         around  the  end  of  the  histogram  list  unless  wrap  off is
         specified or in batch mode.  
     2.  12-feb-1987  - the journal file will contain coordinates entered
         by  cursor.   the  autoplot  mode  was  added.   the   condition
         interactive  may  be  tested  by the if command.  margins may be
         specified in addition  to  the  screen  size  in  the  set  size
         command.  
     3.  26-feb-1987  -  added  define command and show hist may now show
         more information.  the option current has  been  added  to  show
         hist.  
     4.  10-mar-1987  -  added  option append to set hist.  added options
         points, and sets to bin.  histogramming speed was optimized  for
         cases  with  a  series  of  identical y values.  show data gives
         errors on statistics, and statistics are generated for each data
         set  separately.  fixed problem with histogramming data withou z
         values as 3-d plot.  
     5.  7-apr-1987  -  added  commands delete, fit.  fixed some problems
         with separate data sets.  
     6.  30-apr-1987  - added option add to set hist.  if a mesh was read
         with one axes increasing and the other decreasing, it would  not
         plot properly.  this has been fixed.  
     7.  8-may-1987  -  added  contour plotting, and log_error routine to
         trap arithmetic errors.   the  error  handling  returns  to  the
         t2main program.  
     8.  19-may-1987 - virtual memory added to vax version.  
     9.  5-jun-1987 - fixed hidden line algorithm.  it used wrong average
         x,y to calculate the vertical.  as a result it treated as hidden
         many lines incorrectly.  
    10.  11-jun-1987 - fixed infinite loop in t2msfv.  rewrote t2mesh for
         better hidden line removal.  the old algorithm  would  leave  in
         hidden lines, or omit visible line segments that are bisected by
         sharp peaks.  the new algorith fixes these errors with less than
         a 5% loss in cpu speed.  
    11.  17-jun-1987  - fixed problems with title overflow for small page
         sizes.  added lines parameter to set  title  and  shift  to  set
         labels.   added  numbered  windows.   negative scrd will produce
         constant view of mesh.  dot plots and mesh plots were  optimized
         for  max  speed.   essentially  the user can select windows more
         easily.  
    12.  23-jun-1987 - fixed problems with 3-d text.  the duplex font did
         not work for 3-d text, it caused  array  overflows  or  spurious
         lines.   3-d text processing was somewhat optimized.  the syntax
         of the set grid command  has  been  brought  in  line  with  set
         axes,label,title commands.  
    13.  16-jul-1987 - fixed set three to conform to documentation.  
    14.  23-jul-1987  -  modified  t2tlab so that labels may be generated
         for data with a large range.  some changes  were  also  made  in
         t2cntr.    added   table   generation  for  1/2-d  data.   limit
         generation was modified to keep  the  limits  within  reasonable
         values,  when the data has very small relative variation.  title
         generation was modified to keep long titles on the screen.   the
         character size is reduced by up to a factor of 4 if the title is
         too long to fit.  
    15.  30-jul-1987  - modified year/month scale so that all days occupy
         equal space.  
    16.  30-aug-1987 
         a.  lexicals  were  added  as  input to allow access to internal
             numbers.  
         b.  the set statistics command allows the user to set the actual
             statistical data used by the lexicals.  
         c.  set digits  determines the number of digits used for output.
             digits options in titles  allow  formatting  of  numbers  in
             titles.  
         d.  plot table - plots the data as a table of numbers.  
         e.  plot  axes/grid  -  has  options x,y,z...  to allow absolute
             user control over axes/grid generation.  
         f.  inhibit option has been added to set labels, set axes...  to
             allow inhibiting axes generation while still  leaveing  room
             for the axes.  
    17.  16-sep-1987  -  added  mesh  data handling to x,y,z commands, as
         well as minus and divide options.  multiply and divide  commands
         may  be  used  to  manipulate  histograms or spectra.  the x,y,z
         commands were modified to handle dx,dy, and dz automatically.  
    18.  14-oct-1987  - added lexicals for external histogram parameters.
         modified plot command so that the current texture applies to the
         error bars.  define value and define string commands were added.
         s_ lexicals were added for access to string lexicals.  
    19.  31-oct-1987  -  added  log  option  to bin, swap, x, smooth, set
         hist, define hist, add, multiply, and  divide  commands.   added
         set mode log for automatic logging.  added set window level=n to
         make inset windows easier.  added inside option to set labels to
         make packed windows easy to create.  
\end{verbatim}

\subsection{88}
\begin{verbatim}
     1.  15-jan-1988  -  added  expression  parsing.   this allows you to
         enter an arithmetic expression wherever a number is called  for.
         the  syntax  is <expression>.  see topdrawer data expression for
         more details.  added repeat and endrepeat commands.  the  equals
         sign  "="  may not be replaced by a comma "," or slash "/" where
         it is used in the documentation.  it may be replaced by  spaces.
         plus,  or  minus "+-" signs are not allowed in options except as
         part of a number.  
     2.  4-feb-1988  -  added angle lexicals (view_...).  fixed some bugs
         in mesh plots.  added x="expression" command.  added create mesh
         command  to  create  mesh data sets.  modified plot variable for
         mesh data so the symbol size is automatically set to the maximum
         bin size.  
     3.  apr-1988 - fixed limited option for show data command, and added
         limited option to bin command.  added for option to wait.  added
         packed  option  to  set order.   the  set  order and set storage
         commands no longer delete  data.   the  sort  command  has  been
         added.  
     4.  jun-1988  -  fixed  the  error  calculation  on  mean  for  show
         statistics.  fixed new reset so that scales are  also  reset  to
         the original default.  
     5.  aug-1988  -  added  mesh,normal  options  to bin command.  added
         check option to  add,subtract,multiply,divide  commands.   added
         split      option      to      project      command.       added
         add,subtract,multiply,divide   options    to    define histogram
         commnand.   fixed  an  infinite  loop  when  a  single  point is
         histogrammed.  
     6.  nov-1988  -  added theta,phi,angle specifications to set symbol,
         and set grid.  spiffed up the show formats.  added  excl  driver
         for native mode support of talaris-1590 and ln-03+ printers.  
     7.  dec-1988  - fixed several bugs.  set window 3 of -4 did not work
         as the first window.  only 1 of 4 could be the first window.   a
         call  to  t2main  with  blank string after a call with non blank
         string  would  not  prompt  for  input.   show   window   worked
         incorrectly   for   -   windows.    calls  to  t2join,t2hist...,
         incorrectly reset permanent limits.  
     8.  dec-1988  -  added  ntfit,npfit  lexicals,  and  extended tdplt,
         t2join,t2hist for 3-d data.  
     9.  28-dec-1988              -             added             options
         area/share/save/read/write/input/output    to    set/define/show
         histogram commands.  created a compatible version with hbook4.  
\end{verbatim}
\subsection{89}
\begin{verbatim}
     1.  jan-1989  -  fixed error calculations for fitting routines.  the
         quoted errors  were  too  large.   added  error  option  to  fit
         command.   fitted curves are generated with error bars now.  add
         fit to data set now also adds the fitted errors unless error=off
         is   specified.   fixed  equation  evaluation.   exponents  were
         evaluated incorrectly ie.  <1.0e-3>-->1000.  and <1.0e+3>-->.001 
     2.  feb-1989  -  added  file spec/list=/command=/journal= options to
         topdrawer command line.  
     3.  mar-1989 - added merge command to merge data sets.  
     4.  apr-1989 - added mixed data sets.  mesh + regular data may exist
         as separate data sets.  added error display to join command  for
         mesh  data.   add/sub/mul/div  commands  now  may add mesh data.
         added errors to mesh data.  show stat now shows mean/std...  for
         mesh data.  new options:  
              error had been added to command read mesh,join,create mesh. 
              append has been added to read mesh,create mesh 
         modified options:  
              sets,   points|columns,   lines|rows  are  now  independent
         options.  
         sets - specifies the data set only.  
         points - specifies the point/column number within the set.  
         when  the  limited  option is used, the z value is used to limit
         the data.  
         created netwindow files to use td over the decnet.  
     5.  june-1989  -  added  tek4207  support to tek driver.  speeded up
         regis/tek drivers.  added vaxstation (uguisd driver) support.  
     6.  nov-1989  - added support for ntuples to ntopdawer.  removed the
         slash as a separator.  it may be put back by the command:  
              set character separator "/" 
     7.  dec-1989   -   a  number  of  enhancements  have  been  made  to
         ntopdrawer.   added  the  options   tree   to   save/restore/set
         histogram  commands  and added support for "wild" characters for
         histogram area specification.  added the monitor histogram,  and
         delete  histogram.   removed  support  for  rice histograms from
         ntopdrawer.  added  support  for  a  global  section  containing
         hbook4 histograms.  ntopdrawer is now capable of handling hbook4
         histograms in a fairly simple fashion.  
\end{verbatim}

\begin{verbatim}
         both  ntopdrawer  and  topdrawer  have been enhanced as follows.
         the driver for tek4010 has been modified to allow you to specify
         the necessary escape sequences for entering and exiting tek-4010
         mode as  well  as  for  reproducing  colors.   support  for  the
         lsi-7107  color terminal has been added.  support for postscript
         and am imagen printer has been added.   the  options  space  has
         been added in addition to dots,dash...  for line structure.  
\end{verbatim}
\subsection{90}
\begin{verbatim}
     1.  jan-1990  - optimized expression evaluation to achieve a 30 fold
         speedup.  added error  handling  to  smoothing  routine.   added
         support for a postscript printer.  
     2.  feb-1990  -  added  equation  option to fit.  for general linear
         fits.  optimized the auto label generation for logarithmic axes.
         removed  restrictions  on  order of decades, subticks in the set
         scale command.  added device uis to be able to  dump  screen  of
         vaxstation  to  disk in full resolution.  made vaxstation driver
         work for  color  station.   added  color  mappings  for  decgigi
         driver.  
     3.  july-1990  - fixed project command in topdrawer.  it was correct
         in ntopdrawer.  it would sometimes give  incorrect  results  for
         hbook histograms.  
         added  options fetch and read to topdrawer command restore hist.
         (version 3.5) 
\end{verbatim}

\begin{verbatim}
         note:   some  versions  may  have  had problems with the fitting
         lexicals.  vCoefficient...  as some modules were  compiled  with
         /gfloat  and  some  were  not.   the standard is /gFloat for all
         double precision operations if it is supported  by  the  machine
         that it is compiled on.  
\end{verbatim}

\begin{verbatim}
         fixed lexicals v...  inside expressions.  y="vXvalue**2" yielded
         incorrect results while y="xvalue**2" was correct.  this problem
         dated from jan-1990 when optimization was introduced.  
     4.  dec-1990     -     added     options    weight,    eweight    to
         add,subtract,multiply,divide commands.  these allow weighting of
         data when doing the operation.  
\end{verbatim}
\subsection{91}
\begin{verbatim}
     1.  jan-1991  -  added documentation to command set device on how to
         create separate plot and command windows for tek-4010 and  regis
         graphics.  
         added procedures:  freeze.top and freeze22.top to topdrawer_dir: 
     2.  18-jan-1991  - fixed bug in module txline that produced spurious
         lines by incorrect scissoring.  
     3.  added  set  cycle,  show  cycle  commands  and  cycle  option to
         contour.  
     4.  21-jan-1991 - fixed bug in plot,join,histogram commands.  slices
         did not work properly when three was off.  revision level 3.6.  
\end{verbatim}

\begin{verbatim}
         added options offset,data,absolute,value,cursor to plot table to
         move the table to more convenient locations.  
\end{verbatim}

\begin{verbatim}
         add option vector to plot command, to plot a vector field, where
         dx,dy,dz is the length of the vector.  
     5.  29-jan-1991  -  fixed  security  hole  to conceal passwords when
         using  files  over   decnet.    fixed   some   minor   bugs   in
         add/subtract/multiply commands.  
\end{verbatim}

\begin{verbatim}
         add  option  vector  to  add/subtract/multiply  commands  to  do
         operations on vector fields.  
     6.  8-feb-1991  -  added  set mode check,confirm,tree.  these select
         whether checking, confirmation,  or  hbook  tree  searching  are
         defaulted   to   on.    added   options  points,rows,limited  to
         add/sub/mult/div commands this allows  compete  selection  of  a
         subset of data to operate on.  added option check to fft command
         to allow turning off data checking.  
     7.  11-feb-1991  -  fixed problem with short ticks being produced in
         set scale when option label=n preceded ticks=m.  version 3.7 
     8.  28-feb-1991 - added options fill,hide to most commands that draw
         something.  this means you  may  fill  and  hide  boxes,circles,
         diamonds,  histograms,  arrows.   in  addition the set/show fill
         commands allow you to manipulate the fill patterns.   added  the
         command  set/show  shield  to  shield  a  section of a plot from
         further plotting.  added a  gks  (dec)  for  those  devices  not
         availble  through  ugs.   fixed  a  minor bug that switched line
         texture unexpectedly in uis driver.  
     9.  4-mar-1991  -  added  set/show  units to allow you to reset both
         text and character units.  
\end{verbatim}

\begin{verbatim}
         fixed  some  problems  with  3d histograms.  conditionalized the
         code and documentation to allow cm instead of inches throughout.
         this  means  that  the defaults are all in cm instead of inches.
         you may use these conditionals  to  build  a  completely  metric
         verstion of td.  
    10.  15-mar-1991   -   version   4.0   added  options  select="name",
         name="name" to add names to data sets.  select selects data sets
         by  name,  while  name  sets the data set name.  hbook histogram
         names are automatically propagated  to  the  data  set.   option
         title  added to join, plot, hist, contour to add title when data
         set is plotted.  
    11.  25-mar-1991  -  added  nonlinear  option to fit to use minuit to
         perform fits on nonlinear functions.  fixed problem in arbitrary
         linear fit functions.  
         version 4.0.  
    12.  2-may-1991  -  updated  hbSearchId and t2seth4 to be hbook v4.10
         compatible.  added more units to set units.  
    13.  14-jun-1991 - add options width, height to set device.  made the
         default the new version of topdrawer.  fixed some bugs with  set
         units.  
    14.  28-jun-1991  -  modified  the x,y,z commands so that dx,dy or dz
         are automatically recomputed when an expression is used.  
    15.  22-jul-1991  -  fixed  a  bug  in store hist command.  the wrong
         default  recl  was  used.   removed  calls  to  obsolete  hfnext
         routine.  
    16.  23-aug-1991  -  added  option mesh to fit command for mesh fits.
         added option append to list command to  append  listing  on  old
         file.  fixed error handling for minuit fits.  now the errors are
         the same for linear and nonlinear fits.  added functiond  dgauss
         and  dpoly  to  list of functions, as fits of z vs x,y.  fixed a
         bug in t2mesh that  messed  up  histograms  of  mesh  data  with
         errors.  gminut has been added to ntopdrawer.  
    17.  sept-1991  -  added  option  save to contour to save the contour
         lines in data sets.  fixed an 2 outrageous bugs in t2smooth  and
         smctrl.   one bug incorrectly treated the end points of a curve.
         the other would delete isolated data points that  were  adjacent
         to a flat region in the data.  
\end{verbatim}

\begin{verbatim}
         added  interpolate  command  to  produce  interpolated data sets
         using local fitting procedures.  this uses the  same  procedures
         as  the  join  command.   added monitor option to bin, fit, fft,
         convolute, smooth commands to plot before/after pictures.  added
         set mode monitor/append to enable automatic monitor, append.  
\end{verbatim}

\begin{verbatim}
         fixed  equations  inside fit command.  arbitrary equations would
         sometimes give incorrect results.  it has been fixed.  
         version 4.31 
    18.  nov-1991  -  fixed fatal error in fit that prevented producing a
         curve from set coefficients.  fixed some bugs  in  all  commands
         that use the option check.  
    19.  version 4.4 added orientation=n to set size/device commands.  
         version 4.32 
    20.  apr-1992  added  option  portrait/landscape to set device.  also
         added device eps.  
\end{verbatim}
\section{Installation}
to install topdrawer on your system:  
\begin{verbatim}
     1.  copy  the  topdrawer*.exe to a suitable directory.  you may either
         put it into sys$system or any other directory  accessable  to  all
         users.  there are currently versions for topdrawer.  
         a.  topdrawer.exe - the version with hcuis, but no gks routines.  
         b.  topdrawer_nohcuis.exe  -  a  version with dummy hcuis routines
             this version should work on virtually any vax.  
         c.  topdrawer_gks  -  a  version  using  the dec gks routines, and
             dummy hcuis routines.  
              $ assign /system disk:[directory] topdrawer_dir 
              $ copy *topdrawer*.exe topdrawer_dir:  
         the assign command should be added to sys$manager:systartup.com.  
     2.  copy the sample files to topdrawer_dir:  tdintro.top, td3d etc.  
     3.  create  the  symbol  topdrawer.  this line should be placed in the
         system wide login file.  
              $ topd*rawer :== $topdrawer_dir:topdrawerxxx 
         where xxx are selected for the proper version.  
     4.  insert the help into a suitable help library.  
         a.  if you wish to have it in the system help:  
                  $ libr /help sys$help:helplib topdrawer 
         b.  or if you wish to have it in an auxiliary help library:  
                  $ libr /help/create user_help:userhelp topdrawer 
             then make the user help library available to all users 
                  $ assign /system user_help:userhelp hlp$library 
             this line needs to be added to sys$manager:systartup.com.  
     5.  if  you  wish  to  use this program, but you do are not the system
         manager, follow the same steps outlined above,  but  omit  /system
         from the assign command.  the symbol creation and the help library
         assignment will go in your login.com file.  
\end{verbatim}
\section{Documentation}
for a printed copy of this help file:  
\begin{verbatim}
     $ document topdrawer 
     $ print topdrawer.doc 
     $ delete topdrawer.doc;* 
\end{verbatim}
\section{Author}
this version is heavily modified from the original written by:  
\begin{verbatim}
     roger b. chaffee
     computation research group
     stanford linear accelerator center
     stanford, california
     u.s.a.
\end{verbatim}

the original docuementation is:
\begin{verbatim}
     cgtm 178 - november 1980
\end{verbatim}

this version was written by:
\begin{verbatim}
     john clement
     bonner nuclear lab
     rice university
     p.o. box 1892
     houston, tx, 77251
     (713) 527-8101 x 2037
     internet: clement@physics.rice.edu
     bitnet:   clement@ricevm1
     hepnet:   fnbit::riphys::clement
          or...
               riphys::clement
\end{verbatim}
\end{document}
